\subsection{Power Flow}
In order to run the OPA simulation of cascading failures, the ability to calculate power flows on the system is critical.  To model a balanced, three phase power flows, in full resolution we need to use the concept of complex power.  The alternating current of the power system can be represented by sine or cosine waves.  A three phase power system has three lines for each transmission element and each line has a wave that is out of phase with the other two.  Using one wave as a reference, the other phases will attempt to be 120 degrees behind reference and 120 degrees ahead of refrence.  This improves the efficiency and quality of power for loads over a 2 line system as well as not requiring an excessive amount of lines for each transmission element.  

After going into details about complex power and electrical parameters of transmission elements, the balanced three phase power flow equations will be shown.  These non-linear equations model the net power and reactive power injects as well as voltage and phase angle at each vertex of the power system.  A few simplifying assumptions will be made to allow the equations to become linear for the DC (decoupled) power flow equations.  Then, using the dc power flow model, basic economic dispatch and unit commitment models will be shown.  Finally, some pseudo-topological measures will be reviewed that will be useful in the optimization process.

\subsubsection{Electrical Information}
Complex power has both real and reactive parts.  Alternating currents  on a circuit affect components of energy storage such as inductors (changing the current is opposed by a voltage)  or capacitors (store electrical charge).  Over one full cycle of the electricity changing direction, across any individual element there can be real power transfered.  However, there is also power which is stored and released within one cycle and the net energy transfer of this power is 0.  This is called reactive power and is modelled as the imaginary component of complex power.  Let $S_i$ be the complex power inject at some bus on the grid, that is
\begin{equation}
S_i = P_i + j Q_i = V_i I_i^*
\end{equation}
where $P$ is real power, $Q$ is reactive power, $V_i$ is complex voltage, and $I_i^*$ is the complex conjugate of current.  
	
To model a transmission line, the characteristic impedence is used.  At any point, there is complex current I, in each individual line in the transmission element.  Also, there is a complex voltage V difference between the lines.  The characteristic impedence (generalization of Ohm's Law) is then
\begin{equation} \label{impedence}
V/I = Z_0
\end{equation}

Here is a phasor representation of complex voltage $V_i = | V_i | e^{i (\omega t + \delta_i) }$, where the real voltage would be Re[$V_i$]$=\cos (\omega t + \delta_i)$.  Now, by modeling \ref{impedence} in phasor notation, we have 
\begin{equation}
 | V_i | e^{j (\omega t + \delta_V) } = | I_i | e^{j (\omega t + \delta_I) }  | Z_i | e^{j ( \delta_Z) } = |I||Z|e^{j (\omega t + \delta_I + \delta_Z)}
\end{equation}
For this equation to hold at all times $t$, we know that this must hold $\delta_V = \delta_I + \delta_Z$.  When an element has $\delta_Z = 0$, the current and voltage are exactly in phase and it is a purely resistive load with no reactive production or absorption. 

The lines can be modelled using a pi model (schematic diagram looks like $\pi$), which uses line parameters of resistance $R$, inductance $L$, conductance $G$, and capacitance $C$. 
\begin{equation}
Z_0 = \sqrt{  \frac{R+j \omega L}{G+j\omega C} }
\end{equation}
The system angular frequency is $\omega = 2 \pi f$, where $f$ is frequency here.  

If a transmission element is loaded at its surge impedence loading, it neither creates nor consumes reactive power.  This level is defined by 
\begin{equation}
SIL = V_{LL}^2 / Z_0
\end{equation}
Where $V_{LL}$ is the line to line voltage.  When a transmission line is at a level below this, it will supply reactive power and raise system voltages.  When the loading is above this level, the transmission line consumes reactive power, depressing voltages.

\subsubsection{Balanced Three Phase Power Flow}
In a balanced three phase system, at every vertex $i$, we have
\begin{equation}
S_i = P_i + j Q_i = V_i I_i^*
\end{equation}
In addition, with $KCL$, we have $I_i = \sum_{k=1}^n Y_{ik} V_k$, where $Y$ is the admittance bus matrix.  The admittance is the inverse of imdedence, that is
\begin{equation}
Y = G + j B = 1/Z = 1/(R + jX) 
\end{equation}
where $B$, the imaginary part of admittance, is susceptance and $X$, the imaginary part of impedence, is reactance.  Now we have that the complex power at every vertex $i$ is
\begin{equation}
S_i^* = P_i - j Q_i = V_i^* \sum_{k=1}^n Y_{ik} V_k
\end{equation}
By converting these into rectangular coordinates, we get two equations for each bus
\begin{align} \label{ac-pf}
P_i = \sum_{k=1}^n |V_i| |V_k| \left[ G_{ik} \cos (\delta_i - \delta_k) + B_{ik} \sin (\delta_i - \delta_k) \right]  \\
Q_i = \sum_{k=1}^n |V_i| |V_k| \left[ G_{ik} \sin (\delta_i - \delta_k) - B_{ik} \cos (\delta_i - \delta_k) \right]  
\end{align}

For each bus, in addition to $P_i$ and $Q_i$,  we have its voltage $|V_i|$ and its phase angle $\delta_i$.  So, we have $4 n_v$ variables and $2 n_v$ equations.  By supplying the value to $2 n_v$ variables and defining a slack bus, we can find unique values for the remaining $2n_v - 1 $ variables.  Depending on the type of bus, different variables are supplied.  If it is a generator, both $P$  and $V$ are supplied.  A load is defined by a $P$ and $Q$.  The slack bus is a generator in which the phase angle is set to 0.  The phase angle $\delta$ and reactive power production $Q$ are found for each generator and the phase angle $\delta$ and the voltage $|V|$ are found for the loads.

\subsubsection{Decoupled Power Flow}
This model makes assumptions such as lossless lines, small voltage angle differences, and a flat voltage profile.  This is a common simplifing model which is used routinely in economic and reliability analysis of power systems.  A flat voltage profile implies that $\forall i \in \cV$, we have $V_i = 1$.  Small voltage angle difference in conjunction with sine and cosine give the following approximations
\begin{align}
\cos(\delta_i - \delta_j) &\approx 1	\\
\sin(\delta_i-\delta_j) & \approx \delta_i - \delta_j
\end{align}
In addition, since these lines are lossless, $R$ and $G$ are 0.  

The following equations represent the necessary constraints on the power flow in this lossless system.  The first equation represents the conservation of energy and the second equation represents Kirchoffs Current Law.  Conservation of energy implies that the sum of generation and demand is equal to 0 at every point in time.
\begin{align}	\displaystyle
\sum_{j | e=(i,j),  e \in \cE}{f_{e}} &= g_{i} -d_{i} \hspace{17px}   \forall i \in \cV   \label{pf1}
\\
\theta_{i} - \theta_{j} &= X_{e} f_{e}			\hspace{27px}	\forall e=(i,j) \mbox{ s.t. } e \in \cE   \label{pf2}
\end{align}
The value $p_{i} = g_{i} - d_{i}$ represents the net power inject for that node.  A reliability focused model would seek to maximize demand served at all points in time.

When a line is outaged, the power flow has obviously dropped to 0, that is $f_e = 0$.  In addition, the constraint \ref{pf2} to the system is no longer there and must be removed from the formulation.

\subsubsection{Economic Dispatch}
To clear the electricity market at a single point in time, a least cost dispatch model is used.  This model takes bids from generators, a known demand, as well as transmission and ramping constraints and finds a set of generator outputs which meet the demand at least cost.  Using a quadratic cost function for the generators (this cost function can be thought of as a bid from generators which includes the profit the generator would like to make for each marginal unit of production), the least cost dispatch model is as follows. 

The following model is a quadratic program with linear constraints.  The objective function is to minimize the cost of generation.  Typical least cost dispatch and unit commitment models will make various assumptions to allow for linear constraints versus the physical nonlinear constraints to which the power system is subject.  This is the most basic least cost dispatch model, which could be used for clearing the real-time market.  Models in use can have extensions such as ''$N-1$ constraints", which are reliability and security requirements.

\begin{subequations}
\begin{align}
 \min \sum_i \alpha_i g_i &+ \beta_i g_i^2	+ W_i(\tilde{d} - d_i)&	\\
g_i - d_i &= \sum_j f_{ij}	&	\forall i \in N 	\\
X_{ij} f_{ij} &= \delta_i - \delta_j & \forall (i,j) \in M \\
g_i  &\in \left[ g_i^- , g_i^+ \right]		&	\forall i \in N 	\\
f_{ij} &\in \left[ -U_{ij}, U_{ij} \right]	&	\forall (i,j) \in M 
\end{align}
\label{leastcostdispatch}
\end{subequations}

Here, $\alpha_i g_i + \beta_i g_i^2$ can be seen as the cost function for the generator at node $i$ and $W_i(\tilde{d} - d_i)$ is the cost for shedding load.  The generator is bounded between a maximum $g_i^+$ and minimum $g_i^-$ that represent its ramping rate over a specific time interval.  

The day-ahead market uses unit commitment models.  This model will have power flow equations for each $t \in \cT$.  These are integer programs due to the introduction of binary variables $y_{it}$, which take on the value 1 if the generator is switch on and 0 if the generator is off.  The following logical constraint enforces this by
\begin{equation}
y_{it} g_i^- \le g_{it} \le y_{it} g_i^+
\end{equation}
The cost function can then include a fixed cost for operating a generator, such as increased staff during operation.  This cost is not dependent on the level of production but rather if the plant is in an on or off state.  The cost function for each node $i$ and time period $t$ is
\begin{equation}
\alpha_i g_it + \beta_i g_{it}^2 + c_i y_{it}
\end{equation}
This subproblem will be used in a full day model in which the power flow problem is solved for each time period, while remaining feasible with respect to ramping rates and on and off times for generators.



\subsubsection{Pseudo-Topological Measures}
Cotilla et. al. \cite{cotilla_2012} use the fact that voltage phase angles between areas as measure of stress in power networks (47).  Using the power flow Jacobian matricies,
\begin{equation}
 \Delta P = \frac{ \partial P }{ \partial \theta} \Delta \theta + \frac{ \partial P }{ \partial | V | } \Delta | V |
\end{equation}
and assuming voltages are held constant, then $\frac{ \partial P }{ \partial \theta} $ is a Laplacian matrix.
Set the conductance matrix $G = \frac{ \partial P }{ \partial \theta}$, then with
\begin{equation}
e_{ij} = g_{ii}^{-1} + g_{jj}^{-1} - g_{ij}^{-1} - g_{ji}^{-1}
\end{equation}
$E$ satisfies properties of distance matrix under dc power flow assumption, and empiraclly held otherwise.  $E$ is weighted and fully connected with $n_v(n_v-1)$ links.

We then have a quantity that is analogous to node degree, $k_i$, called electrical distance
\begin{equation}
e_i = \sum_{j=1}^n \frac{e_{ij}}{n-1}
\end{equation}
with the inverse representing centrality
\begin{equation}
c_i = e_i^{-1}
\end{equation}

A unweighted graph can represent these distances.  Let $R$ be an adjencency matrix and by defining $r_{ij} = 1 $ if $e_{ij} < t$ and adjusting $t$ so that there are  $n_e$ links.  $R$ has lots of nodes with no connection, with the interpretation that few nodes have disproportionate influence on a large portion of the nework.

Comparing the topological and electrical measures, Cotilla et. al. \cite{cotilla_2012} have that the topological distances have exponential tail and the electrical distances have power-law tails.  Also, there is weak correlation between the two types of distances.  Electrical centrality seem to point out very well the importance of each node to grid stability.  This may be used to find areas to improve the network.