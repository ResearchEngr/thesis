%\documentclass{standalone}
%\begin{document}
\chapter{Derivative Free Optimization}

\begin{itemize}
\item Develop an efficient evaluation of cascading power failures for use in optimization procedure
\begin{itemize}
\item Computational implementation
\item Variance reduction
\item Large scale problems
\end{itemize}
\item Investigate the behavior of this model
\begin{itemize}
\item Lack of convexity for topology changes
\item Non-monotonic behavior for line capacity additions
\item Poor 1st stage approximators of total load shed
\end{itemize}
\item Create algorithm to find good solutions to design problems
\begin{itemize}
\item Finding good search directions
\item Line search procedure for search directions
\item Incorporation of problem specific information
\end{itemize}
\end{itemize}


\section{Computational Techniques}

This is how the simulation is done

\subsection{Implementation}
Parallel computation - condor
C++ simulation code
Python code for analysis

\subsection{Variance Reduction}
Important for reducing computation time
\subsubsection{Common Random Numbers}
very cool
\subsubsection{Sampling Techniques}
havent done yet


\section{Model Behavior}
To investigate the behavior of the model, we will begin by looking at the results from a single contingency.  Once this is better understood, the effects of multiple possible contingencies will be looked at.
\subsection{Topology Information}
anything can happen
\subsubsection{Clustering}
Lines tend to behave in groups which relate to underlying topology.
Take advantage by looking at correlation between load shed and line outages


\begin{figure}
\begin{center}
%\documentclass{standalone}
%\usepackage{tikz}
%\usepackage{pgfplots}
%\begin{document}
\foreach \i in {n3}
{
	\begin{tikzpicture}[scale=1,line width=1]
	\begin{axis}[scale=1.23 , ylabel=Load Shed, xlabel=Capacity , legend style={at={(1.5,.97)},anchor=north east}]

		\addplot+[opacity=1,mark size=2,solid,thick,black] table[x=run, y=ex] {./data/clusters/\i.dat};
	\addlegendentry{Expect}
%		\addlegendentryexpanded{$\i$ - St. Dv.}	
	\end{axis}
	\begin{axis}[ axis y line*=right,title=\mbox{  Line $\i$ },scale=1.23, legend style={at={(1.5,.85)},anchor=north east} ,ylabel=Frequency, y label style={at={(1.25,.5)}} ]	
	\foreach \j in {163,164,165,166,168,170,171,173,175,176}{
		\addplot+[opacity=.4, mark size=.4, solid] table[x=name, y=\j] {./data/clusters/\i.ldat};
		\addlegendentryexpanded{Line $\j$}
	}

	\end{axis}	
	\end{tikzpicture}
}
%\end{document}

\end{center}
\caption{A cluster of a lines responsible for reducing system performance}
\label{fig:cluster}
\end{figure}

This information can be used to aid in the simulation process.  By supplementing this information with topology information, interesting conclusions can be drawn.

\subsection{Power-Law Distribution}
risk measures
The load shed from power outages follows a power law distribution.


\subsection{Capacity Addition}
To begin, we will look at what happens when you double capacity on each line.  (\ref{fig:grad})  Out of the 186 lines, 51 lines resulted in an improved system with respect to the cascading process, 47 were within the margin of error of the nominal system, and 88 reduced the system performance.
\begin{figure}
\begin{center}
	\begin{tikzpicture}
	\begin{axis}[ title=\mbox{ Load Shed for Capacity Addition on Coordinate Directions},legend pos=north east ,scale=1.12, xlabel=Line Number, ylabel=Load Shed (MW)]	
			\addplot+[opacity=.6775, mark size=.915, only marks,error bars/.cd, y dir=both, y explicit] table[x=name, y=ex, y error=se] {./data/grad.dat};
\addlegendentry{Coordinate Directions};
			\addplot+[opacity=.1175, mark size=.915, red, only marks,error bars/.cd, y dir=both, y explicit] table[x=name, y=ex, y error=se] {./data/base.dat};
\addlegendentry{Nominal System}
	\end{axis}	
	\end{tikzpicture}
\caption{Comparing the total load shed for doubling capacity along the coordinate direction.}
\label{fig:grad}
\end{center}
\end{figure}

Lets look along a line in each group.
\begin{figure}

\begin{subfigure}[b]{0.3\textwidth}
\centering
	\begin{tikzpicture}
	\begin{axis}[title=\mbox{Line 1},scale=.4568]			\addplot+[opacity=.5, mark size=.5, smooth,solid] table[x=name, y=ex, y error=se] {./data/done.dat};
%		\addlegendentryexpanded{$\i$ - Ex}
	\end{axis}	
	\end{tikzpicture}
\end{subfigure}
\begin{subfigure}[b]{0.3\textwidth}
\centering
	\begin{tikzpicture}
	\begin{axis}[title=\mbox{Line 2}, scale=.4568]			\addplot+[opacity=.5, mark size=.5, smooth,solid] table[x=name, y=ex, y error=se] {./data/done.dat};
%		\addlegendentryexpanded{$\i$ - Ex}
	\end{axis}	
	\end{tikzpicture}
\end{subfigure}
\begin{subfigure}[b]{0.3\textwidth}
\begin{center}
	\begin{tikzpicture}
	\begin{axis}[title=\mbox{Line 3},scale=.4568]			\addplot+[opacity=.5, mark size=.5, smooth,solid] table[x=name, y=ex, y error=se] {./data/done.dat};
%		\addlegendentryexpanded{$\i$ - Ex}
	\end{axis}	
	\end{tikzpicture}
\end{center}
\end{subfigure}
\begin{subfigure}[b]{0.3\textwidth}
\centering
	\begin{tikzpicture}
	\begin{axis}[title=\mbox{Line 1},scale=.4568]			\addplot+[opacity=.5, mark size=.5, smooth,solid] table[x=name, y=ex, y error=se] {./data/done.dat};
%		\addlegendentryexpanded{$\i$ - Ex}
	\end{axis}	
	\end{tikzpicture}
\end{subfigure}
\begin{subfigure}[b]{0.3\textwidth}
\centering
	\begin{tikzpicture}
	\begin{axis}[title=\mbox{Line 2}, scale=.4568]			\addplot+[opacity=.5, mark size=.5, smooth,solid] table[x=name, y=ex, y error=se] {./data/done.dat};
%		\addlegendentryexpanded{$\i$ - Ex}
	\end{axis}	
	\end{tikzpicture}
\end{subfigure}
\begin{subfigure}[b]{0.3\textwidth}
\begin{center}
	\begin{tikzpicture}
	\begin{axis}[title=\mbox{Line 3},scale=.4568]			\addplot+[opacity=.5, mark size=.5, smooth,solid] table[x=name, y=ex, y error=se] {./data/done.dat};
%		\addlegendentryexpanded{$\i$ - Ex}
	\end{axis}	
	\end{tikzpicture}
\end{center}
\end{subfigure}
	\caption{Plotting expected load shed for capacity additions along coordinate directions.  }


\end{figure}


\subsubsection{Risk Measures}
\begin{figure}

	\begin{tikzpicture}
	\begin{axis}[ title=\mbox{ Risk Measures },legend pos=outer north east, scale=1.6374568, xlabel=Capacity (MW), ylabel=Load Shed (MW)]			\addplot+[black,opacity=.5, mark size=.5,only marks] table[x=name, y=ex] {./data/done.dat};
\addlegendentry{Expect}
	\addplot+[blue,opacity=.5, mark size=.5,solid,error bars/.cd, y dir=both, y explicit] table[x=name, y=ex, y error=se] {./data/done.dat};
\addlegendentry{St. Error}
	\addplot+[red,opacity=.15, mark size=.5,solid,error bars/.cd, y dir=both, y explicit] table[x=name, y=ex, y error=st] {./data/done.dat};
\addlegendentry{St. Dev}
	\addplot+[purple,opacity=.715, mark size=.5,solid] table[x=name, y=var] {./data/done.dat};
\addlegendentry{5\% V@R}
	\addplot+[purple,opacity=.815, mark size=.5,solid] table[x=name, y=cvar] {./data/done.dat};
\addlegendentry{5\% CV@R}
	\addplot+[purple,opacity=.8515, mark size=.5,solid] table[x=name, y=max] {./data/done.dat};
\addlegendentry{Maximum}
%		\addlegendentryexpanded{$\i$ - Ex}
	\end{axis}	
	\end{tikzpicture}
\caption{Value at Risk, Conditional Value at Risk, and Maximum risk measures for load shed distribution}
\end{figure}

The value at risk and conditional value at risk tend to track in a similar manner to the expect value.  The maximum can display much more erratic behavior and is unreasonable in a realistic setting with cost constraints.

results
reserve allocations?


k
