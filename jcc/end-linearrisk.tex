\textbf{Linearization of Line Risk}

To find the linearization for small $t$, the taylor expansion $f(t)=f(t) + \grad f (a)(t-a) +...$ around $a=0$ will be useful.  Applying the taylor expansion to our risk measure \ref{nofail}, we end with the linear approximation
\begin{equation}\label{linear}
h(z) \approx 1 - \sum_e z_e
\end{equation}
%by using the partial derivative of $H(z)$    %%  + \frac{1}{2} (t-a)^T \grad^2 f(a)(t-a)  <---- Quadratic term
%\[ \frac{\partial}{\partial z_{e'}} H ( z ) = \prod_{e \neq e'} (1-z_e) ( -1 )\]
%and $\grad H(0) = -1$.
It is important to note that the linear approximation underestimates system risk and the quadratic approximation overestimates risk for small risk $h(z)$.

The risk can also be bounded using the Bonferroni Inequalities, which is a generalization of Boole's Inequality, or the union bound.
