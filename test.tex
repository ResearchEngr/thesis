
\documentclass[12pt,leqno]{book}
\usepackage{amsmath,amssymb,amsfonts} % Typical maths resource packages
\usepackage{graphics}                 % Packages to allow inclusion of graphics
\usepackage{color}                    % For creating coloured text and background
\usepackage{hyperref}                 % For creating hyperlinks in cross references

\usepackage{algorithm}
\usepackage{algorithmic}
    
\usepackage{amssymb}

\parindent 1cm
\parskip 0.2cm
\topmargin 0.2cm
\oddsidemargin 1cm
\evensidemargin 0.5cm
\textwidth 15cm
\textheight 21cm
\newtheorem{theorem}{Theorem}[section]
\newtheorem{proposition}[theorem]{Proposition}
\newtheorem{corollary}[theorem]{Corollary}
\newtheorem{lemma}[theorem]{Lemma}
\newtheorem{remark}[theorem]{Remark}
\newtheorem{definition}[theorem]{Definition}


\def\R{\mathbb{ R}}
\def\S{\mathbb{ S}}
\def\I{\mathbb{ I}}
\makeindex


\title{ Analysis of Power Grids }

\author{Eric Anderson \\
{\small\em \copyright \  Draft date \today }}

 \date{ }
\begin{document}
\maketitle
 \addcontentsline{toc}{chapter}{Contents}
\pagenumbering{roman}
\tableofcontents
\listoffigures
\listoftables
\chapter*{Preface}\normalsize
  \addcontentsline{toc}{chapter}{Preface}
\pagestyle{plain}
The book root file {\tt bookex.tex} gives a basic example of how to
use \LaTeX \ for preparation of a book. Note that all
\LaTeX \ commands begin with a
backslash.

If an exogenous event impacts the topology or operating characteristics of an electric power grid, the currents reroute themselves according to the laws of physics.  The change in flows may result in power lines reaching capacity, overloading, and eventually failing.  This in turn will re-route the flows and may set off another round of line failures.  A sequence of failures in stages is known as a cascade.  We discuss a simulation-optimization-based approach for designing power grids to minimize the impact of cascades caused by line outages.
 \ldots

Each
Chapter, Appendix and the Index is made as a {\tt *.tex} file and is
called in by the {\tt include} command---thus {\tt ch1.tex} is
the name here of the file containing Chapter~1. The inclusion of any
particular file can be suppressed by prefixing the line by a
percent sign.


 Do not put an {\tt end{document}} command at the end of chapter files;
just one such command is needed at the end of the book.

Note the tag used to make an index entry. You may need to consult Lamport's
book~\cite{lamport} for details of the procedure to make the index input
file; \LaTeX \ will create a pre-index by listing all the tagged
items in the file {\tt bookex.idx} then you edit this into
a {\tt theindex} environment, as {\tt index.tex}.





\pagestyle{headings}
\pagenumbering{arabic}

\include{cascade_simulation_model}


\include{ch2}

\begin{thebibliography}{99}
  \addcontentsline{toc}{chapter}{Bibliography}
\bibitem{lamport} L. Lamport. {\bf \LaTeX \ A Document Preparation System}
Addison-Wesley, California 1986.
\end{thebibliography}

\include{index}
  \addcontentsline{toc}{chapter}{Index}
\end{document}
