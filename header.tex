\documentclass[11pt]{report}
\usepackage{standalone}
\usepackage{amsmath,amssymb,amsfonts} % Typical maths resource packages
\usepackage{graphicx}                 % Packages to allow inclusion of graphics
\usepackage{color}                    % For creating coloured text and background
\usepackage[table]{xcolor}
\usepackage{hyperref}                 % For creating hyperlinks in cross references
\usepackage{algorithm}
%\usepackage{algorithmic}
\usepackage[noend]{algpseudocode}
\usepackage{longtable}
\usepackage{import}
\usepackage{lscape}
\usepackage{booktabs,colortbl}
\usepackage{tikz}
\usetikzlibrary{patterns}
\usepackage{pgfplots}
\usepackage{pgfplotstable}
\usepackage{endnotes}
\usepackage{cleveref}
\usepackage{caption}
\usepackage{subcaption}
\usepackage{comment}    
\graphicspath{{./img/}}

%\usetikzlibrary{external} 
%\tikzexternalize[prefix=tikz/]
%\tikzset{external/system call={lualatex --shell-escape -halt-on-error -interaction=batchmode -jobname ``\image'' ``\def\tikzexternalrealjob{main}\documentclass[11pt]{report}
\usepackage{standalone}
\usepackage{amsmath,amssymb,amsfonts} % Typical maths resource packages
\usepackage{graphicx}                 % Packages to allow inclusion of graphics
\usepackage{color}                    % For creating coloured text and background
\usepackage[table]{xcolor}
\usepackage{hyperref}                 % For creating hyperlinks in cross references
\usepackage{algorithm}
\usepackage{algorithmic}
\usepackage{longtable}
\usepackage{import}
\usepackage{lscape}
\usepackage{booktabs,colortbl}
\usepackage{tikz}
\usetikzlibrary{patterns}
\usepackage{pgfplots}
\usepackage{pgfplotstable}

\usepackage{cleveref}
\usepackage{caption}
\usepackage{subcaption}
    
\graphicspath{{./img/}}



%\setlength{\textwidth}{15cm}
\parindent 1cm
\parskip 0.2cm
\topmargin 0.2cm
\oddsidemargin 1cm
\evensidemargin 0.5cm
\textwidth 15cm
\textheight 21cm
\newtheorem{theorem}{Theorem}[section]
\newtheorem{proposition}[theorem]{Proposition}
\newtheorem{corollary}[theorem]{Corollary}
\newtheorem{lemma}[theorem]{Lemma}
\newtheorem{remark}[theorem]{Remark}
\newtheorem{definition}[theorem]{Definition}


%\newcommand{\mag}[1][mag]{ \left| {#1} \right| }
\newcommand{\magf}{ \left| f \right| }
\newcommand{\magp}{ \left| p \right| }
\newcommand{\magr}{ \left| r \right| }
\newcommand{\magomega}{ \left|  \omega  \right| }
\newcommand{\magE}{ \left| \mathcal{E} \right| }
\newcommand{\magG}{ \left| \mathcal{G} \right| }
\newcommand{\magL}{ \left| \mathcal{L} \right| }
\newcommand{\magM}{ \left| \mathcal{M} \right| }
\newcommand{\magT}{ \left| \mathcal{T} \right| }
\newcommand{\magV}{ \left| \mathcal{V} \right| }

\newcommand{\R}[1]{\mathbb{R}^{#1}}
\def\S{\mathbb{ S}}
\def\I{\mathbb{ I}}


\newcommand{\cA}{\mathcal{A}}
\newcommand{\cB}{\mathcal{B}}
\newcommand{\cD}{\mathcal{D}}
\newcommand{\cCALI}{\mathcal{CALI}}
\newcommand{\cDC}{\mathcal{DC}}
\newcommand{\cGR}{\mathcal{GR}}
\newcommand{\cE}{\mathcal{E}}
\newcommand{\cF}{\mathcal{F}}
\newcommand{\cG}{\mathcal{G}}
\newcommand{\cL}{\mathcal{L}}
\newcommand{\cN}{\mathcal{N}}
\newcommand{\cM}{\mathcal{M}}
\newcommand{\cO}{\mathcal{O}}
\newcommand{\cS}{\mathcal{S}}
\newcommand{\cT}{\mathcal{T}}
\newcommand{\cV}{\mathcal{V}}
\newcommand{\cX}{\mathcal{X}}
\newcommand{\cU}{\mathcal{U}}

\newcommand{\Expect}{\mathbb{E}}
\newcommand{\Prob}{\mathbb{P}}
\newcommand{\cvar}{\mathbb{CV}{\mathsf @}\mathbb{R}}
\newcommand{\grad}{\bigtriangledown}
\newcommand{\lb}{\left\{}
\newcommand{\rb}{\right\}}

\newcommand{\defeq}{\stackrel{\rm def}{=}}




\tikzset{
        hatch distance/.store in=\hatchdistance,
        hatch distance=10pt,
        hatch thickness/.store in=\hatchthickness,
        hatch thickness=2pt
    }

\makeatletter
    \pgfdeclarepatternformonly[\hatchdistance,\hatchthickness]{flexible hatch}
    {\pgfqpoint{0pt}{0pt}}
    {\pgfqpoint{\hatchdistance}{\hatchdistance}}
    {\pgfpoint{\hatchdistance-1pt}{\hatchdistance-1pt}}%
    {
        \pgfsetcolor{\tikz@pattern@color}
        \pgfsetlinewidth{\hatchthickness}
        \pgfpathmoveto{\pgfqpoint{0pt}{0pt}}
        \pgfpathlineto{\pgfqpoint{\hatchdistance}{\hatchdistance}}
        \pgfusepath{stroke}
    }

\makeindex


\title{Computational Models for Risk and Reliability on Bulk Power Systems}

\author{Eric Anderson \\
{\small\em \copyright \  Draft date \today }}

 \date{\today }

\begin{document}

\clearpage\pagenumbering{roman}

\begin{titlepage}
    \begin{center}
        
Computational~Models for 
Line~Failure~Risk and Cascading~Power~Failures 
on~Bulk~Power~Systems
              
\vfill

        By

        Eric Anderson

        \vfill

        A dissertation submitted in partial fulfillment of

        the requirements for the degree of

        \vfill

        Doctor of Philosophy

        (Industrial and Systems Engineering)

        \vfill 

        at the

        UNIVERSITY OF WISCONSIN-MADISON

        2015

        \vfill

    \end{center}

    Date of final oral examination: August 25th, 2015


\singlespacing        
        The dissertation is approved by the following members of the Final Oral Committee:

        \hspace{20pt} Jeff Linderoth, Professor, Industrial and Systems Engineering

\hspace{20pt}         James Luedtke, Professor, Industrial and Systems Engineering

\hspace{20pt}         Bernard Lesieutre, Professor, Electrical and Computer Engineering

\hspace{20pt}         Thomas Rutherford, Professor, Agriculture and Applied Economics

\hspace{20pt}         Stephen Wright, Professor, Computer Sciences

\doublespacing
        

\end{titlepage}


%\clearpage

%\afterpage{\blankpage}


\chapter*{Preface}\normalsize
  \addcontentsline{toc}{chapter}{Preface}

The primary objective of this research effort is to develop desicion making framework in the design and operation of Power Systems.  The methodology involved will be derived from the Industrial Engineering discipline of Operations Research.  Thanks to all those who came before to develop the tools and methods I could never dream of.

Thanks to UW-Madison for providing the opportunity to learn and grow.  In particular, to the many Professors I have had the privelidge to work with and who have tought me the tools of Operations Research.   Thanks to Jeffrey Linderoth for advising me along the way and helping me find a path (as well as all the knowledge about things like Stochastic Programming and slowly converting me to the glory of linux workflows); Stephen Robinson for somehow always being able to explain theoretical concepts so they make sense and combining topics such as economics, probability, and optimization in applied settings seemlessly; Jim Luedtke for helping me find a place in the department and wisdom in integer programming; Oguzhan Alagoz for knowledge in Markov Decision Processes; and Vicki Bier for topics relating to risk analysis.  Thanks to Bernard Lesieutre for helping me understand power systems and Gregory Nemet for issues around energy policy.

Thanks to Jeff Twidwell, Kevin Rockwell, my Parents, and others for helping me edit this document to make it coherent and powerful.

Thanks to my Mom (Sue), Dad (Tim), and Sister (Steph) for always giving me a supportive environment to develop as a person.  Thanks to all my extended family and friends for helping me enjoy life.



 \addcontentsline{toc}{chapter}{Contents}
\setcounter{tocdepth}{2}

\clearpage
\tableofcontents

\clearpage
\listoffigures

\clearpage
\listoftables

\clearpage
\listofalgorithms

\clearpage
\lstlistoflistings

\clearpage\pagenumbering{arabic}

\newcommand{\mypath}{../thesis}
\chapter{Introduction}
This thesis focuses on reliability issues for the electricity grid that powers the United States.  Electricity is a critical service used by almost every person and company within our country.  Reliability issues cost industry billions of dollars every year.  Large scale power outages are national disasters due to the loss of services such as cell phones, city wide water pumping, gasoline stations, trains, subways, and cooling which inevitably lead to economic loss and loss of life.  

The introduction starts with an overview of the electrical infrastructure of the United States.  Following are the basics of how the power system operates on a day-to-day basis and the organizational structure that operates it. Then, the events of August 2003 are explained where part of the Northeast electricity grid collapsed in a cascading power failure leading to millions of people without power, billions in economic loss and the loss of life.  This is just one example of a large scale power outage, which is rare, but extremely costly to society. After this, general reliability issues of the power grid will be discussed which cost society billions annually.
 
After defining the problem, this thesis explains the tools and metholodgy for attempting to make our system more robust against such failures.  First, a model of the cascading process is made.  This model captures the effects of the cascading process while remaining simple enough to solve in a reasonable time.  After a model is developed, design problems are formulated in order to improve either the infrastructure or the operation of the system.  These design problems fit into broad families of optimization problems.  Algorithms for these types of problems are used and modified for our particular problem.  Finally, with the help of computation resources, solutions are found to these design problems in order to gain insight into the problem and help protect our system from potential future outages.

\section{Power Systems Introduction}
The United States electrical infrastructure is complex physical structure which connects the consumers of electricity with generating assets over a large geographical area.  The nature of electricity makes the operation of this infrastructure extremely difficult and is accomplished through numerous organizations and people working together to supply electricty at the least cost while maintaining a given level of reliability.

	As of 2004, the electrical infrastructure was comprised of more than \$1 trillion in asset value.  This includes over 200,000 miles of high voltage transmission lines, 950,000 MW of generation, and 3500 utility organizations serving 283 million people.\cite{northeast_2003}  The introduction briefly looks at generation and the transmission system which efficiently moves the energy over long distances.

\subsubsection{Electricity Generation}
	Electricity is generated using a variety of fuels and procceses.  The most common method of electricity generation creates steam by heating water, which spins a turbine producing an alternating current of electricity.  The United States operates its power grid at 60 hz, that is, the direction of electrical flow switches 60 times every second.  Since the generators are tied to the grid, when the grid is stable, the generators are rotating synchronously with the power grid.  The water can be heated by different fuels such as coal, natural gas, fissioning heavy elements such as uranium, or even using geothermal temperature differentials.  The Rankine cycle is a model of converting heat into mechanical energy for steam engines.  The efficiency of the cycle is limited by the difference between the turbine entry temperature and the condenser temperature.  This means that steam cycle power plants need an external cooling source which removes the waste heat from the working fluid before it begins the cycle again.  

	The first example of a steam cycle power plant is a coal plant.  The chemical energy in coal is converted into thermal energy and byproducts such as carbon dioxide.  Power from coal provides around 40\% (for the year 2013 \cite{eia_gov}) of the electricty generation in the United States. These plants have more thermal inertia making changing the output level a slower process.  Modern coal plants achieve efficiencies of 30-40\%, that is the percentage of input energy which is injected into the grid as electricity.

	Nuclear power plants (19\% of electricity) also operate on the steam cycle, producing heat by fissioning heavy elements such as Uranium-235.  Nuclear plants have relatively slow ramping rates and cheap fuel, which lead them to be dispatched at high output rates continuously.  These plants, along with coal plants, provide the majority of baseload power production. Baseload power is the minimum amount of power that needs to be produced continuously throughout the day.  Nuclear fuel has a desirable aspect of being extremely energy dense.  The energy density of Uranium-235 is roughly 3 million times denser than coal.  This means that the waste products of this process are much less than other types of power plants and are also captured completely without being released to the environment.

	Natural gas plants (27\% of electricty) can capture, in addition to the heat energy, the combustion energy and use it to spin a gas turbine.  Gas turbines, while having a lower thermal efficiency, have the desirable trait to quickly throttle production to the desired level in contrast with steam cycles.  Additionally, by using a combination of both combustion and heat energy, combined-cycle natural gas plants can reach efficiencies of around 60\%.  These fast ramping rates and more expensive fuel lead natural gas plants to provide peaking power matching electrical demand over baseload.  Recently, with relatively cheaper natural gas and increased efficiencies from combined-cycle plants, natural gas plants are playing more of a role for electrical demand between baseload and peaking levels.    

	Hydroelectric power (6\% of electricity) has many desirable traits and it is the largest source of renewable energy generation installed around the world.  By creating a reservoir to hold a large amount of water at a high level, potential energy can be stored.  When this water is released, it becomes kinetic energy, which can be captured by a turbine and used to generate electricity.  By controlling the flow into the turbine or the amount of turbines spinning, hydro power is capable of not only storing energy, but also quickly adjusting its output rate.  However, they have the additional constraint of needing to maintain given reservoir levels throughout the year.

	In the past decade, we have introduced a sizeable amount of electricity production from renewable sources such as wind turbines and solar panels (both total 6\% of electricity, with wind comprising the majority).  These generation sources have a cheap fuel (kinetic energy from the wind and solar radiation), but are unable to control output.  

	The different characteristics of generators give them different roles to play in the operation of the power system in order to meet demand.  These various generation assets are owned by utilities, independent power producers, large industrial customers themselves, and, more recently, residential consumers.  

\subsubsection{Transmission Network}
	The production from the majority of large generation plants is at lower voltages (10kV - 25kV).  Electricity traveling in power lines lose energy due to resistive losses, which primarily goes into heating up the power line.  The resistive losses are proportional to the current.  In order to reduce losses, the voltages are stepped up to between 230 kV to 765kV for long distance transmission elements, which has the effect of reducing the current for a given amount of power.  At the demand side, there are radial tree-like distribution networks operating at low voltages (less than 1kV) which connect every demand node to the power grid.  The United States power grid is broken into three distinct power grids: Western Interconnection, Eastern Interconnection, and Texas Interconnection. Each has a network of transmission lines connecting all of the generators with all of the loads.  

	Power flows, according to the laws of physics, along ``paths of least resistance", which are modeled with Kirchoff Voltage Laws (KVL) and Kirchoff Current Laws (KCL).  This means that electricity flow can't be controlled like many other complex networks such as cell phone and internet traffic, but instead follows laws of physics much like water or gas in pipe networks.  In addition, electricity flows at close to the speed of light and currently is hard to store economically, unlike water and gas.   There needs to be an instantaneous balance of generation and demand.  

\subsubsection{Organizational Structure}
The primary reliability organization which develops operating and planning standards is North American Relaibility Council (NERC) and ten regional councils. 
\begin{description}
\item[NERC] develops standards for reliable operation and planning of the bulk electric system and then monitors and assesses compliance.  Also, they provide education and training, while coordinating critical infrastructure protection, such as information exchange between reiliability service organization.  Finally, they assess, analyze, and report performance and system adequacy.
\end{description}
 The primary focus of the relability and planning standards is to be able to serve all demand reliability both today and in the future.  There are 7 primary tasks:
\begin{itemize}
\item Balance power generation and demand continuously;
\item Balance reactive power supply and demand to maintain scheduled voltages;
\item Monitor flows over transmission lines and other facilities to ensure thermal limits not exceeded;
\item Keep system in stable condition;
\item Operate so that it remains in reliable condition even if contingency occurs (N-1 critiera) and when a contingency does occur, maneuver to new stable N-1 position. The N-1 criteria states that the system must be robust against any single component of the power grid failing;
\item Plan, design, and maintain the system to operate reliably; and
\item Prepare for emergencies.
\end{itemize}

Within the United States, Federal Energy Regulatory Commission (FERC) is the federal agency with control over electricity sales, natural gas and oil pipelines, and hydropower projects and has increased power after the  Energy Policy Act of 2005.  This differentiates it from NERC in that it can impose mandatory standards.

\begin{description}
\item[FERC] imposes mandatory reliability standards for the bulk transmission system and imposes penalties on those that manipulate the markets.  Its primary tasks are:
\end{description}
\begin{itemize}
\item Regulate transmission and wholesale electricitity sales in interstate commerce;
\item License non-federal hydroelectric projects;
\item Ensure reliability of high voltage interstate transmission system;
\item Monitor and investigate energy markets;
\item Penalize violating entities, through civil penalties or other mean; and
\item Oversee environmental matters relating to major policy initiatives.
\end{itemize}

The restructuring of the power grid has decoupled utilities from the responsibility of maintaining a control area.  The 140 control areas are operated by the 10 regional Independent System Operators (ISOs) or Regional Transmission Organizations (RTOs).  
\begin{description}
\item[ISO/RTO] are tasked with providing least cost energy to everyone within its territory while maintaining a given level of reliability.  In order to do this, they create wholesale electricity markets in order to balance generation and load in real time at least cost.  Their tasks are:
\end{description}
\begin{itemize}
\item Manage the wholesale electricity markets while maintaining the reliability of the system; 
\item Direct the operation of the assets owned by their members; and
\item Can encompass multiple control areas within the territory that they operate.
\end{itemize}

\subsubsection{Wholesale Electricity Markets}
The ISO or RTO use 3 primary markets for determining how much each generator should produce at any given time.  The first market is bi-lateral contracts.  These are long term contracts between producers and suppliers to exchange a certain quantity of energy at a given price.  These contracts help provide reliable, long term forecasting for supply and load. This reduces exposure to the volatility of the day ahead and real time markets.  

The day ahead market takes bids from generators which contain an array of costs to produce a certain amount of electricity.  The generators also provide additional constraints which they need to operate within, such as ramping rates and start-up and shut-down times and costs.  In addition to generators' physical constraints, forecasts for demand of the following day will be used to get an estimate of the amount of generation needed.  Using the transmission constraints for the given network, the ISO will clear this market during the previous day.  This process usually takes around four hours and is done with specialized optimization tools solving the optimal power flow problem with security constraints.  The outcome of this process is hourly locational marginal prices (LMP) at each location in the network for the following day.  These prices are guaranteed and any deviation from the schedule will need to be made up in the real-time market.	

The real-time market makes up for errors in forecasting as well as possible outages in generation or transmission.  This market operates on five minute intervals and can be extremely volatile.  Similar methods are used to clear this market, although simplified due to the time constraint.  Price spikes can occur during times of peak demand and a congested grid in which the price of electricity can often become over 10-20 times as expensive.  The reverse is also possible.  During times of low demand it is possible for negative prices on the real time market.  This occurs because the generators are constrained by ramping rates and there is too much production on the grid. In figure (\ref{fig:lmpmiso}), there are negative prices across the Midwest and, in particular, Iowa, with much higher prices in Illinois.  This LMP snapshot is taken at night when there is low demand for electricity as well as increased production from wind farms located in the Midwest.  The differences in prices suggest transmission constraints between the Upper Midwest and Illinois.   


\begin{figure}
\centering
\includegraphics[width=5in]{lmp}  
\caption{Location Marginal Prices (LMP) for Midwest ISO's territory}
\label{fig:lmpmiso}
\end{figure}


\subsubsection{Operating Constraints}

In order for the power grid to operate and maintain stability, a number of things must hold.  First and foremost, there must be a continuous balance of power generation and demand.  The US grids are all operated with a target frequency of 60 hz, but the actual system frequency varies.  When there is excess generation on the grid, the frequency will increase and with a lack of generation, frequency will drop.  Since the generators are synchronized with the grid, when the frequency deviates from normal, the generators can move out of their operating limits and cause damage.  Generators will have internal protections to identify bad operating points and trip offline in order to protect themselves.  In order to protect the grid, there is automated tripping of load at certain frequency points to take customers off line and prevent total collapse.  

In addition to real power balance, reactive power supply and demand must be used to maintain scheduled voltages.  Low voltage can cause system instability and damage to motors and electrical equipment.  Also, high voltage can exceed insulation capability and cause dangerous arcs.  Reactive power is supplied through capacitor banks and generator output.

In order to protect the transmission elements, the flows over transmission lines and other facilities must be monitored to ensure thermal limits are not exceeded.  Lines are heated by electricity flow and equipment can be damaged, such as conductors sagging from stretch and expansion due to high temperatures.  These problems are affected by ambient temperature and wind conditions.  The flow is limited so the line does not sag into obstructions such as trees and telephones.



\section{Cascading Power Failures}
A cascading power failure is a set of failures of individual components of the transmission system which leads to the redistribution of power over the new topology.
 When the system is in a stable operating position, individual components will often have a negligable effect on system-wide distribution.  However, after several failures, the transmission network may not have enough capacity to distribute the power from the generation to the load.  This can cause a series of fast acting trips in which a large portion of the grid may fail.  While rare, these events are extremely costly.


\subsection{Northeast 2003 Blackout} 

%This account of the Northeast Blackout is drawn from the \textit{Final Report on the August 14, 2003 Blackout in the United States and Canada} by the U.S.-Canada Power System Outage Task Force.\cite{northeast_2003}  

\subsubsection{Historical Problems}
The Northeast blackout of 2003 began in the Cleveland-Akron area of the Eastern Interconnection.  This area had a history of problems of low voltages due to the relatively high amount of imported power.  In 1998, the system was becoming unstable and everything other than load shed was done to fix the problem.  It was shown to not be within the reliability standards, and regulations were loosened instead of addressing the problem.  A transformer problem led American Electric Power (AEP), a control area entity, to perform a reliability study on the neighboring control area operated by First Energy (FE).  The study again showed voltage instability problems.

The summer of 2003 was fairly typical with less than historical peak load.  While the load was less than historal peak, it was consistently greater than the forecasted load.  High temperatures, creating a greater air conditioning load, contributed to reactive power shortages.  The voltages were low in Cleveland-Akron during the week, but within the new operating limits (92\% to 105\%).  A group of shunt capacitors, out of service for planned maintenance, further reduce reactive power supply.  Also, a large nuclear reactor, Davis-Besse, was in a long term forced outage state.  This plant is normally able to provide a large amount of real power as well as reactive power support.

\subsubsection{Normal Day Turns Bad}
A series of initiating events, as well as the failure of FE's control server, led to an unstable system and the inability for operators to do anything about it.  In addition, neighboring regional reliability coordinators (Midwest Independent System Operator, MISO, and Pennsylvania-New Jersey-Maryland Interconnect, PJM) had bad information on the state of the system and were unable to provide support.  The operators had concern about voltage levels and were trying to get additional reactive power online.  The shunt capacitors were unable to return to operation.

The following initiating events didn't cause the system to collapse, but moved it into an unreliable state, which allowed the collapse to take place. The Stuart-Atlanta 345 kV line tripped due to contact with surrounding vegetation (area voltage at 98.5\%) at less than 100\% capacity.  In addition to the loss of the line, MISO was unaware, which led to unusable output and the inability to provide support.  An important generating unit, Eastlake 5, attempts to increase reactive power output, but the internal protection trips the generator offline (area voltage 97.3\%).  This led to real power imports rising, which increased the need for reactive power as well.  Another transmission line loss of Harding-Chamberlin 345 kV continued to depress the voltage (95.9\%).  Around this time, the underlying 138 kV network began to fail.   MISO was unable to perform N-1 contingency analysis and the FE reliability charts flatlined due to the server problems.

Ultimately, it was in reliable state before 15:05, however, after all of these events, the system was no longer N-1 stable.  In addition, reliability coordinators were using separate state information due to poor communication.  

\subsubsection{System Becomes Unreliable}

A total of 3 lines tripped at below operating capacity due to tree contact starting with Harding-Chamberlin (44\% of capacity) at 15:05.  Then the Hanna-Juniper line tripped (88\% normal and emergency rating) at 15:32.  The Star-South Canton line had multiple tree contacts (55\%) starting at 14:27, and finally tripped off for good (93\% emergency rating)  at 15:41.  There was no prior sustained outages from tree contact in the previous years and FE used vegetation management practices consistent with the industry.  When designing line limits, the thermal ratings are based on many variables including ambient temperatures and wind speeds.  A combination of higher than nominal ambient temperature and lower wind speeds reduced the cooling capacity of lines.

Perry plant, the largest local generator, was getting voltage spikes at levels close to tripping the generator.  They notified FE of the problems, but the operators were unable to recover from this precarious situation.  The task force notes that load shed here of 1500MW may have saved the system by increasing voltages and stopping the ultimate trip of Sammis Star line.

\subsubsection{High Speed Cascading Failures}
At 16:05 the final straw was pulled by the tripping of the Sammis-Star 345kV line seperating geographical regions of the grid.  Unplanned power shifts across regions caused phase 3 operations, which trip lines far away from the problem area.  The grid continued to stabilize after each one of these individual outages.  However, by 16:10, the north and south separated (AEP separates from FE), and the high generation area was no longer directly connected to the high load center.  This caused a massive power surge from PJM through New York and Ontario counter clockwise around the great lake to load centers in Michigan. This surge caused the Northeast to separate from the rest of Eastern Interconnect (EI).

There was insufficient generation in the newly formed islands to support the load.  The frequency in remaining parts of EI rose to 60.3\%, representing 3700 MW of excess generation.  The Northeast grid kept breaking apart until islands were formed in which an equillibrium between load and generation could be made.  Under-frequency and under-voltage load shedding helped to stabilize the system within the islands.  New York dropped 10,648 MW through automatic load shedding and Ontario dropped 7800 MW.  Some generators tripped off at unreasonable levels making island stabilization more difficult.  Thousands of events occured between 16:10 and 16:13.  

The cascade spread, not due to voltage problems, but dyanmic power swings which caused system instability.  The voltage instability was only companion to the angle instability (represents large real power flow swings) which tripped generators and lines.  The large power swings come from imbalance between generation and load across regions and the electrical separation of these two areas.  The inherent weak points are lines with the highest impedence, which trip off relatively early due to protective relay settings.  These are typically long over-head lines with high loadings.


\subsubsection{The Blackout Results}
In the United States, 45 million people lost power totaling 61,800 MW in Ohio, Michigan, Pennsylvania, New York, Vermont, Massachusetts, Connecticut, and New Jersey.  The US loss estimate was between \$4 billion and \$10 billion.  Another 10 million people from Canada lost power leading to an estimated 18.9 million lost man hours and \$2.3 billion in lost manufacturing shipments.   There were at least 11 fatalities, power took up to 4 days to return, and rolling blackouts continued in Ontario for the following week.\cite{northeast_2003}

The formation of a large island, based off of hydro plants in western New York and Canada, was the basis for system restoration.
 

\section{Reliability in Power Systems}
The electricity grid is held to a higher reliability standard than other services since it is critical infrastructure for society.  Even though the power system maintains a very high level of reliability, power interuptions still have large costs to consumers.  After a discussion of the economic costs, the trends in power outages for the North American grid will be looked at.

The economic loss caused by power interruptions to electricity consumers in the United States for 2001 was estimated at \$79 billion.\cite{lacommare_2006}  This can be compared to the total revenue of electrical sales in the year 2001 of \$247 billion.\cite{eia_sales}  The single event of the Northeast blackout in 2003 was estimated to have a total impact on US workers, consumers, and taxpayers as a loss of approximately \$6.4 billion.\cite{anderson_2003}  This cost is hidden from the system, but may entertain the notion that the true cost of electricity is higher than the current prices.  

The current power system has evolved throughout the last century due to economic and reliability issues and the responses of the operating entities to these forces.  The power system has self-organized into a dynamic equilibrium where blackouts of all sizes occur \cite{dobson_2001}.  The average frequency of blackouts in the United States is 13 days and has been the same for 30 years \cite{carreras_2004}, which represents this equilibrium.  
	
NERC data shows that distribution of blackouts for the last 15 years (1988-2003) follows a power tail curve with an exponent of around $-1.3\pm.2$ \cite{carreras_2004}. This is strengthened further where Hines et. al. showed that show the frequency of blackouts in the United States is not decreasing, changes seasonally and with the time of day, and follows a power-law distribution.\cite{hines_2008} \cite{hines_2009}

Some emerging conditions on the grid may make protection more important and more difficult (table \ref{tab:change1}).  Electric vehicles adopted in masse have a potential to offer useful services to the power grid through the use of a sizeable amount of energy storage in aggregate.  However, they also represent a considerable stress on the system in different spots and ways than it is used to.  The ideal car battery system would have a quick charge, that is, the ability to transfer the energy from the grid to the battery extremely quickly.  This would allow consumers to use charging stations similar to the current gasoline stations for combustion engines.  A charging station capable of charging many cars would have extremely high and unpredictable volatility that the system will need to protect against.

The increased penetration of renewable energy production also increases stress on the power system.  Wind and solar are both variable generation devices which do not actively control the amount of electricity being produced.  The system needs to maintain ample ramping capability in order to maintain stability.  These stresses can be reduced by improving the quality of forecasting efforts on various timescales.  In addition, these technologies are geographically constrained by their fuel source availability, wind speeds and solar irradition.  However, these natural fuel sources do not align with large demand centers, so the transmission system is needed to efficiently distribute the energy produced.

\begin{table}
\centering
\footnotesize
\begin{tabular}{| p{7cm} | p{7cm} |}
\hline
\bf Previous Conditions 	&	\bf Emerging Conditions 		\\	\hline  	\hline
Fewer, relatively large resources	&	Smaller, more numerous resources		\\	\hline
Long-term, firm contracts 	&	Contracts shorter in duration, more non-firm transactions, fewer long-term firm transactions	\\	\hline
Bulk power transactions relatively stable and predictable	&	Bulk power transactions relatively variable and less predictable	\\	\hline
Assessment of system reliability made from stable base (narrower, more predictable range of potential operating states)	&	Assessment of system reliability made from variable base (wider, less predicatable range of potential operating states)		\\		\hline
Limited and knowledgable set of utility players 	&	More players making more transactions, some with less interconnected operation experience; increasing with retail access	\\	\hline
Unused transmission capacity and high security margins	&	High transmission utilization and operation closer to security limits	\\	\hline
Limited competition, little incentive for reducing reliability investments	&	Utilities less willing to make investments in transmission reliability that do not increase revenues	\\	\hline
Market rules and reliability rules developed together	&	Market rules undergoing transition, reliability rules developed separately	\\	\hline
Limited wheeling	&	More system throughput	\\	\hline
\end{tabular}
\caption{\small Changing conditions that affect system reliability (from the Northeast outage report \cite{northeast_2003}).}
 \label{tab:change1}
\end{table}

\section{Thus We Care!}
So this is how we will do something about it.

Ultimately, we need to be able to give operators good information and as much time as possible to be able to react to the contingency and avert collapse.  In order to give the operators more time, we must make the system more robust to contingencies.  The primary goal of this research is to be able to optimize various design problems with respect to cascading power failures. 

\subsection{Report Outline}

\subsubsection{Modeling Cascades}
The topological model of the power grid is composed of the various elements in the transmission network as well as their parameters.  While these types of models are simple, they are incapable of capturing effects which are loading dependent. 

The OPA type model is a sequential process in which the power flows for the entire network are calculated in order to determine the loading on the various transmission elements.  This in addition to a deterministic or probablistic failure model is used to outage individual components at each stage in the cascade.  The cascade concludes when no elements are outaged.

Even in its simplest form, this type of model has been shown to capture the criticality effects of blackouts (higher than expected large blackouts) we see in real power systems in aggregate.  As more complexity is added into this model, this type of model can produce reasonable sequences of outages for the system, while remaining accurate in aggregate. 

Building this model requires calculating the power flows.  While the system is not necessarily stable or balanced throughout the cascade process, the balanced, steady state 3 phase power flow equations will be used as a basis for the power flow analysis.  This keeps the calculations tractable and gives valuable information about the various topologies and loading patterns.  In the simplest form of the OPA Model, the power flows will be approximated using the DC (Decoupled) power flow model, which is a linear approximation to the AC power flow equations.

In addition to the OPA model, accessory information can be used to aid in the optimization process.  This includes things such as pseudo-topological statistics (electrical distances and centrality), line outage patterns and recognition, and finally, economic information.  Additionally, making a link between reliability and economics requires development of an economic dispatch model.

Design problems will be formed that change aspects of the topology, component parameters, or operation of the system.  Using the OPA model to gauge the response of the system to these changes, we will have formulated a mathematical optimization problem.  These design problems can range from transmission expansions to setting operational reserve levels or even producing protective relay settings for individual components.

\subsubsection{Optimizing Design Problems}
This problem offers several difficulties which make optimization challenging.  

A primary difficulty is the sequential process of the OPA model in which operators make decisions under uncertainty and the decisions effect the future stages and progression of the process.  In addition, the decision dependent outcomes only have a probabilistic effect on the progression of the cascade.  That is, an overloaded line may not fail in all scenarios, so that at each stage a range of outcomes is needed to capture the probabilistic effects.  In order to properly describe the outcomes of the process under a range of scenarios, a multi-stage tree with even a modest number of stages and outcomes explodes quickly.

Another main challenge is that the change of topology of the power system creates nonlinear and nonconvex effects.  It is possible to reduce the cost of electricity by removing transmission elements.  On the other hand, by adding transmission lines, a system capable of meeting demand can become infeasible.  Since the system is constrained by KVL and KCL, each transmission line, while providing a path for electricity flow, further constrains the system.  It is well known that nonconvex behavior is not desirable for optimization problems, and in this case we can have nonconvex behavior between each stage of our process, since by definition, each stage before the final has topology changes.

\subsubsection{Multi-Stage Integer Program}
In order to better understand the structure of the problem, it will be formulated as a multi-stage stochastic program with decision-dependent uncertainty.  In order to do this, logical connections between the stages will be made using binary variables, that is variables that can only take the value 0 or 1.  In this way, the decision dependent uncertainty can be modeled explicitly and the probalistic effects can be modeled by sampling the random variables a prior.  As the number of outcomes per stage increases, the better the probablistic effects can be modeled.  Using this type of model, the number of stages has to be decided a prior as well.  This is a major shortcoming, as we are concerned with the effects of the worst cascades, which can take many stages to complete.  As the number of nodes, $N$, is related to the number of outcomes, $a$, and number of stages, $b$, by $N \propto a^{b}$, the model quickly becomes intractable. 

 However, if a crude approximation to the cascade process can get similar aggregate effects from the outcome, we will be able to use this formulation as a subproblem.  This can then be embedded in design master problem, an example being transmission expansion, and then solved with an out of the box mixed-integer solver.

\subsubsection{Derivative Free Optimization}
The computational complexity of the first approach grows too fast with respect to increases in model accuracy.  In this approach, the OPA model will be simulated instead of more rigoroursly mathematically defined.  In this we lose some structure of the problem but by decoupling the stages of the OPA model from the master problem, we have a fairly simple monte carlo simulation.  Using variance reduction techniques of common random numbers we can resolve the outcome of any configuration of the design problems extremely quickly by running a large number of simulations.  In addition, we can parallelize the process in order to evaluate multiple configurations simultaneously.  The outcome of these simulations will be statistics about the magnitude of the blackout for the various contingencies as well as possible sequence of outages.

The optimization field has many algorithms that can use a zeroth order oracle, that is for each configuration the oracle returns a real valued number and may include stochastic variation or numerical error.  In our case the simulation is the oracle and the real valued number could be the expected load shed or the value-at-risk which attempts to capture large tail effects of distributions.  The main types of derivative free optimization are pattern search, model based, and stochastic approximations.  

A pattern search method doesn't attempt to understand anything about the underlying structure but instead tries to search in a specific way such that the function value improves.  Certain patterns can offer local convergences guarantees, that is the solution is at least better than all of those in the neighborhood.  This, in combination with a grid search in which the whole space is partitioned and sampled, can get you good global solutions, but global convergence will not be provable.  These search methods can be improved by using well positioned exploratory steps and search directions that conform to local topology.

  The model based method takes a set of sampled points and builds a polynomial model, typically linear or quadratic, and then optimizes this model exactly.  This type of model can have faster local convergence properties, although in some cases it performs much worse than even a simple pattern search.

Stochastic models assume an underlying probability distribution about the function.  In addition, assumptions are made such that by increasing sampling infinitely, the sample approximation converges to the true value.  These models can have characteristics of both pattern search and model based methods, but under the assumption of some stochastic error being present.

\subsubsection{Changing the Market Model}
The third approach will attempt to adjust the current market model in order to put a price on destabilizing effects.  In order to illustrate some of the shortcomings of the current market, two examples will be used.  The first example will look at generators and highlight the fact that they are not getting paid according to their value to the system.  The second will look at demand and highlight that all demands are not equal and thus should not pay the same marginal price.

The first example is of 2 generators at the same bus of a power system and they have equal marginal cost at some point in time.  Generator A is outputting at its maximum output level but generator B is below its maximum output level.  Generator A is only able to ramp down but generator B is able to ramp up or ramp down.  Generator B provides additional flexibility to the system in order to find a least cost power flow and also increase stability of the system.  In today's market, this generator could bid this capacity into regulation services and get paid.  However, if these generators are not used for ancillary services, they get paid at an equal rate despite offering unequal services.   Furthermore, the regulation and reserve markets are set up so that you can bid into the markets as long as you meet some minimum requirements.  This puts unequal services on the same level and when you are finding a least cost solution, you are likely to deploy inferior quality services.  The market should be able to place the correct value on the true varying levels of service.

The second example is of two different demand profiles at the same node and their demand does not affect the locational marginal price (LMP).  Demand A is a fixed demand and doesn't fluctuate at all.  Demand B has an average demand equal to demand A, but half the time is at twice A's level and half the time at 0.  Since they both consume the same amount of power, these demands both pay the same rate.  But these demands place unequal stress on the system.  If all demand was type B, additional resources would need to be put on the grid to meet demand.  Type B demand also increases the regulation needed to ensure reliability of the grid.  Currently, the regulation services bought to support type B demand is paid for by all demand equally.  The market should be able to place the correct price on the varying levels of stress demand puts on the grid.     

The market model needs to account for the cost of volatility to the system.  Currently, this cost is socialized to the whole system through the cost of ancillary services such as reserves and regulation support instead of being bore by those that cause the instabilities.  In order to properly account for these costs and reward those offering these services, a new market model will be made.  The strengths and weakness of this model will be showed, in particular the effects it has on the likelihood of  cascading power failures.

A key tool in building the market model will be critical points. The reliability of the electricity grid can be controlled by varying the distance between the operating positions and various critical points.  Using a list of contingencies, costs associated with each contingency can be measured by its new distance to bad operating points.  Once we have a measure, we will have a value to place on entities causing stress to the system and reward those providing the services.  This market will be able to take over both the regulation and reserve markets.  Regulation and reserve markets are incapable of properly pricing reliability issues, which is exactly what they are designed to do.  Instead, due to having set points for how much reserve and regulation they have, there is no ability to toggle how reliable the system is.  However, we know that the reliability of power systems fluctuates seasonally and daily, but the level of support the system is getting is not.  In addition, with the ability to have a tangible measure for reliability, the consumers of electricity will have a better quality product.  The power systems will now have the ability to adjust the cost of electricity for a trade-off in reliability.  This will link the economic and reliability issues of the power grid.  It will allow a uniform system to apply to various tough problems in electricity markets such as demand response and energy storage.

If we look at an uncertain demand and note it has some average consumption rate over a time interval and a certain amount of variance from its mean, we can see that a load with zero variance will be the cheapest possible load.  Any positive amount of variance will move the system closer to critical points at some point in time over the given interval.  Loads with extremely high variance will have to pay even more as they will move the system even closer to the bad operating points.  Energy storage devices and demand response will be able to play an important role in these markets as well.  It is unlikely that we stumbled onto the reliability frontier and this model should help the system move closer.  



\newcommand{\mypathmsip}{../thesis/msip}
\newcommand{\mypathmsipdata}{../thesis/msip/data}
\chapter{Modeling Cascading Power Failures}\label{msip-model}
An ideal model is complex enough to capture the important features and effects of the process, while remaining simple enough to work with and remain believable.  Arbitrarily complex models can be made to produce the output matching nearly any process, but often there is no reason to believe that the individual mechanisms being modeled are important or represent a real part of the process.

\section{Literature Review}
Historically, cascading power failures have been a hard problem to understand, and since they are rare, the process was not studied much.  However, after the Northeast Blackout of 2003, more focus has been devoted to this problem.  

The simplest models of the cascading process are strictly topological models with some mechanism to fail components such as a failed node leads to its neighbors to fail with a given probability.  While these models can capture interesting effects due to degree distribution and clustering, they lack important information about the electrical parameters of the topology as well as loading patterns.   

After discussing historical topological models, Hines, Cotilla. et. al. \cite{hines_2010}, \cite{cotilla_2012} provide a rebuttal to strictly topological models.  Here, topological measures will be looked at in addition to the electrical information behind the topology of the grid. 
%Then, pseudo-topological measures such as electrical distance and centrality will be developed and their importance discussed.

Finally, the Oak Ridge National Laboratory, Power System Engineering Research Center of Wisconsin University, and Alaska University (OPA) type models will be developed.  These models use the electrical parameters as well as demand and generation information to solve the power flow problem.  The results of this are loading patterns on the topology of the power grid.  Using these loading patterns as well as a deterministic or probabilitic failure mechanism, another level of complexity can be added to the cascading failure model.  The following quote from Hines et. al. \cite{hines_2011} supports the use of this type of model.

\begin{quote}
While a perfect model of cascading failure would accurately represent the continuous dynamics of rotating machines, the discrete dynamics associated with relays that disconnect stressed components from the network, the non-linear algebraic equations that govern flows in the network, and the social dynamics of operators working to mitigate the effects of system stress, all power system models simplify these dynamics to some extent. Unlike simple topological metrics, our model does capture the effects of Ohm's and Kirchhoff's laws, by using linear approximations of the non-linear power flow equations \cite{bergen_1986}. Similar models have been used to study cascading failure in a number of recent papers \cite{carreras_2004}, \cite{dobson_2007}, \cite{mei_2009}.
\end{quote}

In addition to the fast time scale cascading model, the OPA work included a slow time scale model, which responded to these cascading failures by engineering improvements.  It is the dynamic interactions of these two forces that leads to an equilibrium in which blackouts of all sizes occur and the size follows a power law distribution.

In order to use OPA models, power flows need to be calculated and to reduce the computational complexity of the model, a Decoupled (DC) power flow model is used by making a few simplifying assumptions.  Using the power flow model, an economic dispatch model can be made to dispatch the generators at least cost, while remaining within its operating constraints.  This dispatch model will be used to connect the reliability issues of the power grid with economic ones.
%First, the balanced three phase AC power flow equations will be discussed along with its strengths and weaknesses. 

\subsection{Topologic Model}

This section will first discuss historical topological models and then describe common topological measures.  These measures will be used to compare to other network structures.  Using these measures, it is shown that power grids differ from many other common network structures and thus need to be analyzed on their own.  

\subsubsection{Historical Topological Models}

These topological models used flow estimates instead of power flow calculations.

In 2004, Albert et. al. \cite{albert_2004} worked on large blackouts in response to August 2003 and developed a deterministic model of failures based on topological measures.  They used 4 methods of removing nodes from the grid one at a time, randomly, highest degree, highest load, and cascading.  The main simplifying assumption is that if a generator is connected to a load, its power is available.  In addition, power is routed along the shortest path from generation to load.  Then, in order to monitor the effects of the failure patterns, connectivity loss is recorded, which represent the average decrease in number of generators connected to a substation.  They conclude by noting possible solutions of increasing redundancy and capacity of the system or decreasing reliance on transmission by using more generation at the distribution substation level.

Kinney and others developed another method for estimating power flows on a given topology \cite{kinney_2005}.  They introduce the concept of efficiency for power lines and use the harmonic composition of the efficiency of lines to calculate an efficiency measure for any given path.  Now, the electricity is distributed to a load from a generator along the most efficient path.  Then, they modeled an efficiency degradation based on loading through time as well as tolerance measures to outage lines probabilistically.

A handful of these topological models with flow estimates were done in between 2003 and 2010.  Some were predicated on behaving like other networks such as scale-free networks \cite{zhao_2004}, \cite{wang_2009} or small-world networks \cite{ding_2006}.  Others used matching models with a profit function to protect against cascading failures \cite{sun_2008} or novel recourse strategies such as deliberate weak lines for network islanding \cite{duenas-osorio_2009}. 

\subsubsection{Topological Measures}

The topology of a power grid can be described as an unweighted, undirected graph $\cG$ with verticies $\cV$ and edges $\cE \subset (\cV \times \cV)$ which connect the verticies.  A particular grid will be denoted by a subscript, such as $\cG_{EI} = \left\{ \cV_{EI}, \cE_{EI} \right\}$, which would be the graph that represents the Eastern Interconnect.  The verticies $\cV$ on the graph can represent demand nodes, generator nodes, and buses in the transmission network.  The edges $\cE$ represent elements such as transmission lines and transformers.  For convenience, we will define $n_v = \magV$ and $n_e = \magE$.

A useful tool for describing topological measures on graphs is the adjacency matrix, $A$.  The elements $a_{ij}$ of $A$ represent whether nodes $i$ and $j$ are connected, such that if $(i,j) \in \cE$ then $a_{ij} = 1$, else $a_{ij}=0$.  The degree $k_i$ of a node measures how connected it is to the rest of the network, with
\begin{equation}
k_i = \sum_{j=1}^{n_v} a_{ij}
\end{equation}
A common measure to compare different graphs is the average degree $\hat{k} = 2 n_e/n_v$.  Cotilla et. al. \cite{cotilla_2012}, using data for EI from NERC in 2012 and data for WI and TI from FERC in 2005, found the average degree of the networks, which is tabulated in \ref{tab:topo_info}.  This tells us that our power grids are sparsely connected, with around 2.5 transmission elements connecting each vertex, noting that parallel lines are counted as one.  The following statistical analysis of topological measures was done by Cotilla et. al. \cite{cotilla_2012} in order to show that power grids are neither small-world networks nor scale free networks.

\begin{table}
\centering
\begin{tabular}{| c | c c c c c c c|}
\hline
Grid & Verticies & Edges & $\hat{k}$  &  $k_{max}$ & $C$ & $L$ & $d_{max}$ \\
\hline
$\cG_{EI}$	& 41,228	&	52,075	&	2.53	&	29	&	0.068	&	31.9	&	94	\\
$\cG_{WI}$	& 11,432	&	13,734	&	2.4	&	22	&	0.073	&	26.1	&	61	\\
$\cG_{TI}$ 	& 4,513	&	5,532		&	2.45	&	18	&	0.031	&	14.9	&	37	\\
\hline
\end{tabular}
\caption{Topological measures for the three US power grids}
\label{tab:topo_info}
\end{table}


There are many statistical measures used to compare our power grid graphs to other common graph structure.  The first measure is the distribution of the degree, $k_i$, of all the nodes.  One type of network to compare to is a scale-free network which have a power-law degree distribution.  These networks have highly connected central hubs, which are inherent weak points to the network.  However, high-degree nodes are far less common in power grids than would be expected with a scale-free network.    

Two additional measures, which are distance metrics, are diameter and characteristic path length.  The distance $d_{ij}$ between $i$ and $j$ is the minimum number of links needed to traverse from vertex $i$ to vertex $j$.  The diameter is then 
\begin{equation}
d_{max} = \max_{ij} d_{ij}
\end{equation}
 and the characteristic path length is 
\begin{equation}
L = \frac{1}{n_v (n_v -1)} \sum_{\forall i,j | i \neq j} d_{ij}
\end{equation}
In addition, the average nodal distance $\hat{d} = \sum_{j=1}^{n_v} d_{ij}$ can be used.  As the size of small-world networks increase, the characteristic path length increases roughly with $\ln n_v$, which means the distances between verticies grows slowly.  However, the power grid's path length always grows faster than $\ln n_v$ and falls between small-world networks and regular grids, which scale linearly with $n_v$.

Another useful measure will be the cluster coefficient which gives insight into neighborhoods of nodes.  Let $e_i$ be the number of edges connected to vertex $i$ and its immediate neighbors $N_i$ by the following $e_i =\sum_{\forall j,k \in \left\{ N_i \cup i \right\}} a_{jk}/2$.  Then the clustering of node $i$ is
\begin{equation}
c_i = \frac{e_i}{(k_i(k_i-1))/2}
\end{equation}
and the cluster coefficient of the graph is $C = \frac{1}{n} \sum_{i=1}^{n_v} c_i$.  Power networks were found to have less clustering than small world networks but much larger than random grids, which may be due to relatively few long distance lines.

The final measure used was degree assortativity, which is the correlation of the degree of two connected nodes.  Power networks were found to have small, negative degree assortativity.  This was due to distribution feeders, which have a large number of radial lines connecting single loads to the substation.  This behavior was not found in small-world networks.

Hines et. al. \cite{hines_2010} conclude that while these topological measures are useful for understanding the structure and perhaps indicating general vulnerabilities, they can lead to erroneous conclusions.  For example, Kinney et. al. \cite{kinney_2005} and Albert et. al. \cite{albert_2004} draw different conclusions about such things as the effects of single outages using similar data.  Power flow based models are more realistic and thus more useful for vulnerability analysis.  However, analogous measures with electrical topological information can be very useful. 
%\endnote{\textbf{Pseudo-Topological Measures}

Cotilla et. al. \cite{cotilla_2012} use the fact that voltage phase angles between areas as measure of stress in power networks (47).  Using the power flow Jacobian matricies,
\begin{equation}
 \Delta P = \frac{ \partial P }{ \partial \theta} \Delta \theta + \frac{ \partial P }{ \partial | V | } \Delta | V |
\end{equation}
and assuming voltages are held constant, then $\frac{ \partial P }{ \partial \theta} $ is a Laplacian matrix.
Set the conductance matrix $G = \frac{ \partial P }{ \partial \theta}$, then with
\begin{equation}
e_{ij} = g_{ii}^{-1} + g_{jj}^{-1} - g_{ij}^{-1} - g_{ji}^{-1}
\end{equation}
$E$ satisfies properties of distance matrix under dc power flow assumption, and empiraclly held otherwise.  $E$ is weighted and fully connected with $n_v(n_v-1)$ links.

We then have a quantity that is analogous to node degree, $k_i$, called electrical distance
\begin{equation}
e_i = \sum_{j=1}^n \frac{e_{ij}}{n-1}
\end{equation}
with the inverse representing centrality
\begin{equation}
c_i = e_i^{-1}
\end{equation}

An unweighted graph can represent these distances.  Let $R$ be an adjencency matrix and by defining $r_{ij} = 1 $ if $e_{ij} < t$ and adjusting $t$ so that there are  $n_e$ links.  $R$ has lots of nodes with no connection, with the interpretation that few nodes have disproportionate influence on a large portion of the nework.

Comparing the topological and electrical measures, Cotilla et. al. \cite{cotilla_2012} have that the topological distances have exponential tail and the electrical distances have power-law tails.  Also, there is weak correlation between the two types of distances.  Electrical centrality seems to point out very well the importance of each node to grid stability.  This may be used to find areas to improve the network.

}


\subsection{OPA Model}\label{opa_section}

The next level of complexity to add is to use electrical information about the grid as well as loading patterns to determine the power flows.  The loadings on particular elements have a large effect on the failure probability of the given element.  This type of model was done by three groups, Oak Ridge National Laboratory, Power System Engineering Research Center of Wisconsin University, and Alaska University.  This class of models will be called OPA models.  They look at the power transmission system and consider engineering and physical aspects, as well as economic, regulatory, and political responses to blackouts and increases in load.

In 2001, Dobson et. al \cite{dobson_2001} found that the opposition of the slow time scale force of growth in load and system capacity and fast time scale of cascading power failures and outages produced a dynamic equilibrium that can be seen in real world data.  Many real world complex systems can be seen to have this self-organized criticality property.  This criticality means that the blackout size distribution follows a power-law distribution, $f(x) = ax^k$ with an exponent of $-1.3 \pm 0.2$,  making large blackouts more likely.  In addition to criticality, it also represents an equilibrium.  The distribution of blackout size has not changed in the past 30 years.  They argue that you can't study large blackouts by looking at initial triggering events only, but you must look at the root cause, and deeper, long-term forces that drive the evolution of power system.

On a slow time scale, there are several things that happen to the electricity grid.  The first main force is the slow growth in load (around 0.7\% growth per year for first decade of 21st century \cite{eia_gov}).  This has the effect of reducing the available capacity margins on power lines and increasing the likelihood of failures as well as possibly further constraining economic dispatch.  While the slow time scale is progressing, random exogenous events, acts of nature, happen to outage individual components.  These possibly initiate large cascading failures and blackout portions of the system.  The engineering response to blackouts in operating policies, maintenance, equipment and controls have the effect of increasing margins on the slow time scale.  These forces push against each other and settle in an equilibrium.

The following parts go through the details of these forces mathematically.  These are drawn from several OPA papers \cite{dobson_2001}, \cite{carreras_2004},\cite{dobson_2007}.  

\subsubsection{Slow Time Scale}
The slow time scale is simplified by using days as the time step, represented by index $t$.  There are three main components to the slow dynamics.
\begin{enumerate}
\item The demand grows at the beginning of each day.  We have $d_{it} = \lambda d_{i \rho(t)}$ where $\rho(t)$ represents the preceding time period and $i$ is a vertex with a load demanding $d_{it}$ on day $t$ at peak load.  They used $\lambda = 1.00005$, which corresponds to a yearly growth rate of $1.8\%$ (the yearly average for 1980-2000).  To represent daily load fluctuations, all loads are multiplied by a random number $r$, such that $2-\gamma \le r \le \gamma$ with $1 \le \gamma \le 2$.
\item The response to blackouts is to upgrade the transmission system by increasing the maximum capacity of transmission lines.  If a line has overloaded in a blackout, the response is to increase its capacity so that $U_{et} = \mu U_{e \rho(t)}$ with $e$ being an edge with capacity $U_{et}$ on day t.  They varied $\mu$ between $1.01$ and  $1.1$.  This parameter simplifies all of the efforts that go into these responses including increasing the frequency of maintenance, changing operating policy, installing new equipment, and adjusting or adding alarms and controls.  These responses are modeled as happening before the next day, but in reality can take place over many different time scales.
\item The response to increased demand is to increase generator power so that all demand can be met.  First, they assume that increases in power is quantized and not continuous.  The quantity $\Delta G_t = \kappa ( D_t / n_g )$ represents the amount of power increase for a generator with $\kappa$ being a few percent, $n_g$ being the number of generators, and $D_t = \sum_{i \in \cV} d_{it}$ is the total demand for time period $t$.  In order to increase generation at a node $i$, we need $g_{it}^+ + \Delta G_t \le \sum_{e=(i,j) | e \in \cE} U_{et}$, that is, the increased power needs to be able to flow out of its neighborhood.  $g_{it}^+$ represents the maximum power generation at node $i$ on day $t$.  Power is continued to be added to eligible nodes, $g_{it}^+ \leftarrow g_{it}^ + + \Delta G_t$, until the generator capacity margin has risen above a prescribed level.  The generator capacity margin is defined by
\begin{equation}
\left(\frac{G-D}{D}\right)_t = \frac{\sum_i g_{it}^+ - D_0 e^{(\lambda-1)t} }{D_0 e^{(\lambda-1)t}}
\end{equation}
with $D_0 e^{(\lambda-1)t}$ being the average power demanded, not including daily fluctuations.  The generator capacity margin is used to deal with daily fluctuations in demand.  The generator capacity margin of the U.S. has an estimated mean value between 15\% and 30\% \cite{carreras_2004}.
\end{enumerate}

These forces balance against system outages throughout time.  The outages were modeled as being possible to take place every day and begin with random events with a given probability.  The next section will go into detail about the cascading process that is possible after the random events take place.

\subsubsection{Fast Time Scale}

Individual blackouts triggered by random events (equipment failure, weather, vandalism, attack) can become widespread through a series of cascades. 
The initial goal of building this cascading failure model was to produce a list of lines that could plausibly be involved in cascading event.  It simplifies the process of cascading failures considerable, but is still able to capture important effects of topology changes throughout the process.  Figure \ref{fig:cascade} gives a quick overview of the fast time scale simulation used to model cascading power failures.

\begin{figure}
\centering
\begin{tikzpicture}
\draw [->,thick] (1,1) node[anchor=east, circle, draw]{$\xi$ \scriptsize occurs}
-- (1.35,1) node(TC)[anchor=west,text width=1.2cm,text centered,  rectangle, draw]{\scriptsize Topology changes};
\draw[->,thick] (TC)
--(3.3,1) node(PF)[anchor=west,text width=1.6cm,text centered, rectangle,draw]{\scriptsize Power flow calculated};
\draw[->,thick] (PF)
--(5.7,1) node(F)[anchor=west,text width=1.95cm, text centered, rectangle, draw]{\scriptsize Overloaded lines fail, probability $p_1$};
\draw[->,thick] (F)
--(8.3,1) node(DN)[anchor=west,text width=1.75cm, text centered,rectangle,draw,line width=1.5pt]{\scriptsize No failures, cascade done};
\draw[->,thick] (F.south)
--(5.2,-.4) node(LF)[anchor=east,text width=1.6cm, text centered, rectangle, draw]{\scriptsize Line failures };
\draw[->,thick] (LF.west)
-- (TC.south);

\end{tikzpicture} 
\caption{OPA simulation of a cascading power failure}
  \label{fig:cascade}
\end{figure}


The OPA model of the cascade process begins with an exogenous event, $\xi$, that effects the topology of the grid.  In their initial version, $\xi$ were the branches that randomly failed in an independent and identically distributed (I.I.D) manner, a Bernoulli trial with probability $p_0$.  Moments after the outage, the power flows are rerouted through the new topology based on the laws of physics.  In a longer time frame, it is possible for operators to take actions such as load shed and generator redispatch.  The resulting loading on the transmission lines is evaluated.  Their model then failed overloaded transmission lines with a Bernoulli trial with probability $p_1$ between  0.1 and 1.  After all overloaded lines are evaluated, a transition is made to the next stage.  Either there are no more outages, in which case the cascade is over, or more branch elements are outaged, the topology changes, and the operator is allowed to take recourse.  This process repeats until the system is stable and no outages occur.  Figure \ref{fig:cascade-example} shows a visual example of the cascading process.

For a given grid $\cG$ and initial demand $d_0$
\begin{description}\label{fast_opa}
\item[ Initial $ \xi $ ]  For $e=1,...,n_e$, draw $\omega$ from $U[0,1]$ and if $\omega \le p_0$, line $e$ is outaged.
\item[ ] \hspace{35px} \vdots 
\item[ Stage $m$ ]  Calculate the power flows $f_{m}$, a column containing the branch flows for all edges in $\cE$, by using DC power flow equations with demand vector $d_m$. 

For $e=1,...,n_e$, if branch flow $f_{em} \ge U_{em}$, draw $\omega$ from $U[0,1]]$ and if $\omega \le p_1$, outage line $e$.
\end{description}


%\documentclass{standalone}
%\begin{document}

\def \sc     { .65 }
\def \lw     { 1.25pt }

\begin{figure}
\centering
\begin{subfigure}[b]{.46\linewidth}
\begin{tikzpicture}[line width=\lw,scale=\sc]

\node[circle,fill=blue!20] (one) at (3,0) {\small 1};
\node[circle,fill=red!20] (two) at (5,0) {\small 2};
\node[rectangle,fill=green!20] (three) at (7,1) {\small 3};
\node[rectangle,fill=green!20] (four) at (5,1.75) {\small 4};
\node[rectangle,fill=green!20] (five) at (3,1.75) {\small 5};
\node[circle,fill=red!20] (six) at (2,3) {\small 6};
\node[rectangle,fill=green!20] (seven) at (5,3) {\small 7};
\node[circle,fill=red!20] (eight) at (3.85,3.45) {\small 8};
\node[rectangle,fill=green!20] (nine) at (7,3) {\small 9};
\node[rectangle,fill=green!20] (ten) at (5.85,4.1) {\small 10};
\node[rectangle,fill=green!20] (eleven) at (3.75,5) {\small 11};
\node[rectangle,fill=green!20] (twelve) at (2.35,5) {\small 12};
\node[rectangle,fill=green!20] (thirteen) at (.81,4.2) {\small 13};
\node[rectangle,fill=green!20] (fourteen) at (4,6) {\small 14};


\draw (one) -- (two) ;
\draw (one) -- (five);
\draw (two) -- (three) ; 
\draw (two) -- (four) ; 
\draw (two) -- (five) ; 
\draw (three) -- (four) ; 
\draw[red] (four) -- (five) ; 
\draw (four) -- (seven);
\draw (four) -- (nine);
\draw (five) -- (six) ; 
\draw (six) -- (eleven) ; 
\draw (six) -- (twelve) ; 
\draw (six) -- (thirteen) ; 
\draw (seven) -- (eight) ; 
\draw (seven) -- (nine) ; 
\draw (nine) -- (ten) ; 
\draw (nine) .. controls +(up:1.2cm) .. (fourteen) ;
\draw[red] (ten) -- (eleven); 
%\draw[red] (eleven) -- (twelve); 
\draw[red] (twelve) -- (thirteen);
\draw (thirteen) .. controls +(up:1.2cm) .. (fourteen) ; 
\end{tikzpicture}
\caption{Initial Event: The three red lines are outaged and the power flow redistributes.}
\end{subfigure}
\begin{subfigure}[b]{.46\linewidth}
\begin{tikzpicture}[line width=\lw,scale=\sc]

\node[circle,fill=blue!20] (one) at (3,0) {\small 1};
\node[circle,fill=red!20] (two) at (5,0) {\small 2};
\node[rectangle,fill=green!20] (three) at (7,1) {\small 3};
\node[rectangle,fill=green!20] (four) at (5,1.75) {\small 4};
\node[rectangle,fill=green!20] (five) at (3,1.75) {\small 5};
\node[circle,fill=red!20] (six) at (2,3) {\small 6};
\node[rectangle,fill=green!20] (seven) at (5,3) {\small 7};
\node[circle,fill=red!20] (eight) at (3.85,3.45) {\small 8};
\node[rectangle,fill=green!20] (nine) at (7,3) {\small 9};
\node[rectangle,fill=green!20] (ten) at (5.85,4.1) {\small 10};
\node[rectangle,fill=green!20] (eleven) at (3.75,5) {\small 11};
\node[rectangle,fill=green!20] (twelve) at (2.35,5) {\small 12};
\node[rectangle,fill=green!20] (thirteen) at (.81,4.2) {\small 13};
\node[rectangle,fill=green!20] (fourteen) at (4,6) {\small 14};

\draw[red] (one) -- (two) ;
\draw (one) -- (five);
\draw (two) -- (three) ; 
\draw[red] (two) -- (four) ; 
\draw (two) -- (five) ; 
\draw[red] (three) -- (four) ; 
%\draw (four) -- (five) ; 
\draw (four) -- (seven);
\draw (four) -- (nine);
\draw (five) -- (six) ; 
\draw (six) -- (eleven) ; 
\draw (six) -- (twelve) ; 
\draw (six) -- (thirteen) ; 
\draw (seven) -- (eight) ; 
\draw (seven) -- (nine) ; 
\draw (nine) -- (ten) ; 
\draw (nine) .. controls +(up:1.2cm) .. (fourteen) ;
%\draw (ten) -- (eleven); 
%\draw (twelve) -- (thirteen);
\draw (thirteen) .. controls +(up:1.2cm) .. (fourteen) ; 
\end{tikzpicture}
\caption{Stage 1: Lines 1-2, 2-4, and 3-4 are overloaded. Lines 1-2 and 3-4 fail, but line 2-4 remains in operation at an overloaded state. }

\end{subfigure}
\begin{subfigure}[b]{.46\linewidth}

\begin{tikzpicture}[line width=\lw,scale=\sc]

\node[circle,fill=blue!20] (one) at (3,0) {\small 1};
\node[circle,fill=red!20] (two) at (5,0) {\small 2};
\node[rectangle,fill=green!20] (three) at (7,1) {\small 3};
\node[rectangle,fill=green!20] (four) at (5,1.75) {\small 4};
\node[rectangle,fill=green!20] (five) at (3,1.75) {\small 5};
\node[circle,fill=red!20] (six) at (2,3) {\small 6};
\node[rectangle,fill=green!20] (seven) at (5,3) {\small 7};
\node[circle,fill=red!20] (eight) at (3.85,3.45) {\small 8};
\node[rectangle,fill=green!20] (nine) at (7,3) {\small 9};
\node[rectangle,fill=green!20] (ten) at (5.85,4.1) {\small 10};
\node[rectangle,fill=green!20] (eleven) at (3.75,5) {\small 11};
\node[rectangle,fill=green!20] (twelve) at (2.35,5) {\small 12};
\node[rectangle,fill=green!20] (thirteen) at (.81,4.2) {\small 13};
\node[rectangle,fill=green!20] (fourteen) at (4,6) {\small 14};

%\draw[red] (one) -- (two) ;
\draw (one) -- (five);
\draw (two) -- (three) ; 
\draw[red] (two) -- (four) ; 
\draw[red] (two) -- (five) ; 
%\draw[red] (three) -- (four) ; 
%\draw (four) -- (five) ; 
\draw (four) -- (seven);
\draw (four) -- (nine);
\draw (five) -- (six) ; 
\draw (six) -- (eleven) ; 
\draw (six) -- (twelve) ; 
\draw[red] (six) -- (thirteen) ; 
\draw (seven) -- (eight) ; 
\draw[red] (seven) -- (nine) ; 
\draw (nine) -- (ten) ; 
\draw (nine) .. controls +(up:1.2cm) .. (fourteen) ;
%\draw (ten) -- (eleven); 
%\draw (twelve) -- (thirteen);
\draw (thirteen) .. controls +(up:1.2cm) .. (fourteen) ; 
\end{tikzpicture}
\caption{Stage 2: On the new topology, lines 2-5, 6-13, and 7-9 become overloaded.  The cascade progresses by outaging lines 2-4 and 7-9.}
\end{subfigure}
\begin{subfigure}[b]{.46\linewidth}
\begin{tikzpicture}[line width=\lw,scale=\sc]


\node[circle,fill=blue!20] (one) at (3,0) {\small 1};
\node[circle,fill=red!20] (two) at (5,0) {\small 2};
\node[rectangle,fill=green!20] (three) at (7,1) {\small 3};
\node[rectangle,fill=green!20] (four) at (5,1.75) {\small 4};
\node[rectangle,fill=green!20] (five) at (3,1.75) {\small 5};
\node[circle,fill=red!20] (six) at (2,3) {\small 6};
\node[rectangle,fill=green!20] (seven) at (5,3) {\small 7};
\node[circle,fill=red!20] (eight) at (3.85,3.45) {\small 8};
\node[rectangle,fill=green!20] (nine) at (7,3) {\small 9};
\node[rectangle,fill=green!20] (ten) at (5.85,4.1) {\small 10};
\node[rectangle,fill=green!20] (eleven) at (3.75,5) {\small 11};
\node[rectangle,fill=green!20] (twelve) at (2.35,5) {\small 12};
\node[rectangle,fill=green!20] (thirteen) at (.81,4.2) {\small 13};
\node[rectangle,fill=green!20] (fourteen) at (4,6) {\small 14};

%\draw[red] (one) -- (two) ;
\draw (one) -- (five);
\draw (two) -- (three) ; 
%\draw[red] (two) -- (four) ; 
\draw[red] (two) -- (five) ; 
%\draw[red] (three) -- (four) ; 
%\draw (four) -- (five) ; 
\draw[red] (four) -- (seven);
\draw[red] (four) -- (nine);
\draw (five) -- (six) ; 
\draw (six) -- (eleven) ; 
\draw (six) -- (twelve) ; 
\draw[red] (six) -- (thirteen) ; 
\draw (seven) -- (eight) ; 
%\draw[red] (seven) -- (nine) ; 
\draw (nine) -- (ten) ; 
\draw (nine) .. controls +(up:1.2cm) .. (fourteen) ;
%\draw (ten) -- (eleven); 
%\draw (twelve) -- (thirteen);
\draw[red] (thirteen) .. controls +(up:1.2cm) .. (fourteen) ; 
\end{tikzpicture}
\caption{Stage 3: This has the effect of routing all power destined for load 8 through the north passage. Lines 13-14, 4-7, and 4-9 are outaged along the path. }
\end{subfigure}

\begin{subfigure}[b]{.46\linewidth}
\begin{tikzpicture}[line width=\lw,scale=\sc]


\node[circle,fill=blue!20] (one) at (3,0) {\small 1};
\node[circle,fill=red!20] (two) at (5,0) {\small 2};
\node[rectangle,fill=green!20] (three) at (7,1) {\small 3};
\node[rectangle,fill=green!20] (four) at (5,1.75) {\small 4};
\node[rectangle,fill=green!20] (five) at (3,1.75) {\small 5};
\node[circle,fill=red!20] (six) at (2,3) {\small 6};
\node[rectangle,fill=green!20] (seven) at (5,3) {\small 7};
\node[circle,fill=red!20] (eight) at (3.85,3.45) {\small 8};
\node[rectangle,fill=green!20] (nine) at (7,3) {\small 9};
\node[rectangle,fill=green!20] (ten) at (5.85,4.1) {\small 10};
\node[rectangle,fill=green!20] (eleven) at (3.75,5) {\small 11};
\node[rectangle,fill=green!20] (twelve) at (2.35,5) {\small 12};
\node[rectangle,fill=green!20] (thirteen) at (.81,4.2) {\small 13};
\node[rectangle,fill=green!20] (fourteen) at (4,6) {\small 14};


%\draw[red] (one) -- (two) ;
\draw (one) -- (five);
\draw (two) -- (three) ; 
%\draw[red] (two) -- (four) ; 
\draw[red] (two) -- (five) ; 
%\draw[red] (three) -- (four) ; 
%\draw (four) -- (five) ; 
%\draw (four) -- (seven);
%\draw (four) -- (nine);
\draw (five) -- (six) ; 
\draw (six) -- (eleven) ; 
\draw (six) -- (twelve) ; 
\draw (six) -- (thirteen) ; 
\draw (seven) -- (eight) ; 
%\draw[red] (seven) -- (nine) ; 
\draw (nine) -- (ten) ; 
\draw (nine) .. controls +(up:1.2cm) .. (fourteen) ;
%\draw (ten) -- (eleven); 
%\draw (twelve) -- (thirteen);
%\draw (thirteen) .. controls +(up:1.2cm) .. (fourteen) ; 
\end{tikzpicture}
\caption{Stage 4:  Finally, line 2-5 that is still overloaded is outaged. }
\end{subfigure}
\begin{subfigure}[b]{.46\linewidth}
\begin{tikzpicture}[line width=\lw,scale=\sc]


\node[circle,fill=blue!20] (one) at (3,0) {\small 1};
\node[circle,fill=red!20] (two) at (5,0) {\small 2};
\node[rectangle,fill=green!20] (three) at (7,1) {\small 3};
\node[rectangle,fill=green!20] (four) at (5,1.75) {\small 4};
\node[rectangle,fill=green!20] (five) at (3,1.75) {\small 5};
\node[circle,fill=red!20] (six) at (2,3) {\small 6};
\node[rectangle,fill=green!20] (seven) at (5,3) {\small 7};
\node[circle,fill=red!20] (eight) at (3.85,3.45) {\small 8};
\node[rectangle,fill=green!20] (nine) at (7,3) {\small 9};
\node[rectangle,fill=green!20] (ten) at (5.85,4.1) {\small 10};
\node[rectangle,fill=green!20] (eleven) at (3.75,5) {\small 11};
\node[rectangle,fill=green!20] (twelve) at (2.35,5) {\small 12};
\node[rectangle,fill=green!20] (thirteen) at (.81,4.2) {\small 13};
\node[rectangle,fill=green!20] (fourteen) at (4,6) {\small 14};


%\draw[red] (one) -- (two) ;
\draw (one) -- (five);
\draw (two) -- (three) ; 
%\draw (two) -- (four) ; 
%\draw[red] (two) -- (five) ; 
%\draw[red] (three) -- (four) ; 
%\draw (four) -- (five) ; 
%\draw (four) -- (seven);
%\draw (four) -- (nine);
\draw (five) -- (six) ; 
\draw (six) -- (eleven) ; 
\draw (six) -- (twelve) ; 
\draw (six) -- (thirteen) ; 
\draw (seven) -- (eight) ; 
%\draw[red] (seven) -- (nine) ; 
\draw (nine) -- (ten) ; 
\draw (nine) .. controls +(up:1.2cm) .. (fourteen) ;
%\draw (ten) -- (eleven); 
%\draw (twelve) -- (thirteen);
%\draw (thirteen) .. controls +(up:1.2cm) .. (fourteen) ; 
\end{tikzpicture}
\caption{Stage 5:  The system stabilizes into islands with generator 1 serving load 6.  However, loads 2 and 8 are out of service.}
\end{subfigure}
\caption[An example of the OPA cascading sequency]{ \label{fig:cascade-example} \small An example of a cascading power failure. Node 1 is a generator and nodes 2, 6, and 8 are loads.}
\end{figure}

%\end{document}


\subsubsection{Dynamic Equilibrium}

These opposing forces eventually find an equilibrium.  The equilibrium tends to be near critical points, which are points that have maximum power flow through the network for the nominal system capacity.  The system self-organizes toward these points, which maximize efficiency of its assets.  When the system approaches these critical points, the power flow becomes limited due to two possible causes:
\begin{itemize}
\item The power flows are limited due to transmission line constraints.  This type of critical point has larger blackouts, but happen less frequently.  In addition, the blackouts typically have  multiple lines tripping.
\item The power flows are limited due to generation capacity.  This results in frequent blackouts, but of smaller size.
\end{itemize}
However, it is also possible to be in an operating regime which is close to both types of critical points simultaneously.  When this is the case, blackouts of all sizes occur.  While this operating regime may not be good from a reliability standpoint, it has the desirable characteristic of being able to deliver the maximum power for the system.   That is, when these two points are balanced, the system is capable of maximum throughput from its generators to its loads with minimal excess capacity.  This is important from an economic perspective and a critical reason the system self-organizes to these types of points.  This type of point tends to be the cheapest way to supply all the loads with power, while statisfying the minimum system reliability standards.

\subsubsection{Additional Complexities}

The OPA model can be extended to include additional complexities for many different reasons.  It is always a balance of how much resources you have to solve the problem and the amount of resolution you need in the solution.  The base OPA model is easy and fast to replicate, however by trying to gain increased resolution in the output, the model becomes increasingly complex and difficult to solve in a timely manner.

In related work, Chen et. al. \cite{chen_2005} find many similar conclusions to the OPA model by using an extension which included a hidden failure model.  A hidden failure is undectable in normal operations, but as the system becomes disturbed, a relay may incorrectly trip.  These protective relays are in operation to protect the line from overloads and disconnect it from the system.  This work introduces a loading dependent failure model that trips neighboring lines to outaged components.  This hidden failure is exposed the first time a disturbance nearby occurs and if it doesn't fail then, it is assumed to be properly operating and future disturbances will not undergo this hidden failure mechanism.  The probability of these happening in the real world are non-negligble according to NERC data \cite{nerc_dawg}.  

This work displays many similar results to the OPA papers.  First, the power law behavior near critical loading can be seen by varying the system loading levels.  They find that by increasing spinning reserves the risk of big blackouts is greatly reduced.  The blackout size distribution tends toward an exponential distribution as reserve levels are increased.  By lowering the hidden failure rate, the system becomes more robust and larger blackouts become less likely.  Finally, they note that prompt control actions can reduce the risk of big disturbances.  While all of these results seem fairly straightforward, it points to the fact that the important aspects of the cascading process are being modeled and the effects are similar to what would happen in the real world.

Bienstock made several modifications \cite{bienstock_2011} to the original OPA model in order to remove some undesirable effects of the simulation, notably the erratic behavior of its output under small changes in the input.  To do this, he introduced the concept of memory to the system.  In order to see if a line is in or near an overloaded state, it uses a running time average of the current state and previous states.
\begin{equation}
\tilde{f}_{et} = \alpha f_{et} + (1-\alpha) \tilde{f}_{e\rho(t)}
\end{equation} 
Here, $f_{et}$ represents the power flow on edge $e$ at time $t$.  Then, $\tilde{f}_{et}$ is used in the overload and outage calculations.

In addition to including a memory, he also smoothed out the definition of an overloaded line by creating a step in between normal and overloaded states in which the failure probability was more than nominal but less than in the overloaded state.  Using $0 \le \epsilon \le 1$, for edge $e$, the following failure model smooths the effects of overloaded lines failing.
\begin{align}
\tilde{f}_{et} \ge (1+\epsilon) U_{et}			&	\hspace{10pt} \mbox{The line outages with certainty} 	\\
(1-\epsilon) U_{et} < \tilde{f}_{et} < (1+\epsilon) U_{et}	&	\hspace{10pt} \mbox{The line outages with probability }\frac{1}{2} 	\\
\tilde{f}_{et} \le (1-\epsilon) U_{et}			&	\hspace{10pt} \mbox{The line remains in operation} 
\end{align}

An AC power blackout model was developed at the University of Manchester (Manchester Model) by Nedic \cite{nedic_2006} and the original OPA authors.  This model is able to represent real world disturbances more accurately by using the full non-linear model that describes power flow.  This gives resolution into areas for generator instability, under-frequency load shedding, redispatch of active and reactive sources, and emergency load shedding.  The model has both automated control systems and operator recourse.  This model was used in OPA criticality works by Mei et. al. \cite{mei_2006}, including one with mechanisms using voltage stability margins \cite{mei_2008}.  However, the additional complexity comes at the cost of having to solve a nonlinear program and the loss of convergence gaurantees.  This is all done in order to represent something that is a companion to, but not the main driver of the cascading process (according to the Northeast outage working group, the main driver was angle instability not voltage instability).

Mei and others have worked to improve the accuracy of OPA by increasing the amount of details modeled \cite{mei_2009}.  In the fast dynamics, along with protective relays being modeled as hidden failures, they included a failure mechanism for the operational dispatch center that is responsible for generator redispatch.  In addition, they used a failure model where an underloaded line is failed with probability $p_1 \left( f_e/U_e \right)^N$, with $N=10$.  For the slow dynamics, they added a step that models a planning department by increasing the capacity of lines which have a loading rate $\left( f_{e}/ U_{e}  \right)$ greater than a set point.  

In 2013, Qi et. al. extended this model to include slow dynamics of vegetation growth and management as well as differential equations representing line heating.  Neglecting spatial variation and heating/cooling effects from   the environment, they model line temperature by a differential equation
\begin{equation}
\frac{dT(t)}{dt} = \alpha I^2 - \nu (T(t) - T_0)
\end{equation}
where $T(t)$ is the line temperature at $t$, $I$ current, $T_0$ initial temperature, and $\alpha$ and $\nu$ are calculated parameters.  This can be solved by assuming constant branch flow, $f$, to calculate the temperature over time
\begin{equation}
T(t) = e^{-\nu t}\left( T_0 - T_e(f) \right) + T_e(f)
\end{equation}
which can be used to find the final equilibrium temperature, $T_e(f)$ (a function of its constant power flow $f$ on the line), and time until a given temperature.  Using the calculated temperature, the horizontal span, and an elongation parameter, the line sag distance can be calculated.  When the minimum distance between lines and vegetation passes a breakpoint based on transmission line characteristics such as operational voltage, the line will be outaged.  They model the vegetation with a daily growth rate model and include a managment simulation in which they identify future potential hazards and cut down trees over time.  The statistical analysis of their model agreed well with historical data in China.


\subsection{Power Flow Review}
In order to run the OPA simulation of cascading failures, the ability to calculate power flows on the system is critical.  To model a balanced, three phase power flows, in full resolution we need to model complex power.  The alternating current of the power system can be represented by sine or cosine waves.  A three phase power system has three lines for each transmission element and each line has a wave that is out of phase with the other two.  Using one wave as a reference, the other phases will attempt to be 120 degrees behind reference and 120 degrees ahead of reference.  This improves the efficiency and quality of power for loads over a 2 line system as well as not requiring an excessive amount of lines for each transmission element.  

The details about complex power and electrical parameters of transmission elements to model balanced three phase power flow equations will be left to the appendix.
%\endnote{\textbf{Electrical Information}

Complex power has both real and reactive parts.  Alternating currents  on a circuit affect components of energy storage such as inductors (changing the current as opposed by a voltage)  or capacitors (store electrical charge).  Over one full cycle of the electricity changing direction, across any individual element there can be real power transferred.  However, there is also power which is stored and released within one cycle and the net energy transfer of this power is 0.  This is called reactive power and is modelled as the imaginary component of complex power.  Let $S_i$ be the complex power inject at some bus on the grid, that is
\begin{equation}
S_i = P_i + j Q_i = V_i I_i^*
\end{equation}
where $P$ is real power, $Q$ is reactive power, $V_i$ is complex voltage, and $I_i^*$ is the complex conjugate of current.  
	
To model a transmission line, the characteristic impedence is used.  At any point, there is complex current I, in each individual line in the transmission element.  Also, there is a complex voltage V difference between the lines.  The characteristic impedence (generalization of Ohm's Law) is then
\begin{equation} \label{impedence}
V/I = Z_0
\end{equation}

Here is a phasor representation of complex voltage $V_i = | V_i | e^{i (\omega t + \delta_i) }$, where the real voltage would be Re[$V_i$]$=\cos (\omega t + \delta_i)$.  Now, by modeling \ref{impedence} in phasor notation, we have 
\begin{equation}
 | V_i | e^{j (\omega t + \delta_V) } = | I_i | e^{j (\omega t + \delta_I) }  | Z_i | e^{j ( \delta_Z) } = |I||Z|e^{j (\omega t + \delta_I + \delta_Z)}
\end{equation}
For this equation to hold at all times $t$, we know that this must hold $\delta_V = \delta_I + \delta_Z$.  When an element has $\delta_Z = 0$, the current and voltage are exactly in phase and it is a purely resistive load with no reactive production or absorption. 

The lines can be modelled using a pi model (schematic diagram looks like $\pi$), which uses line parameters of resistance $R$, inductance $L$, conductance $G$, and capacitance $C$. 
\begin{equation}
Z_0 = \sqrt{  \frac{R+j \omega L}{G+j\omega C} }
\end{equation}
The system angular frequency is $\omega = 2 \pi f$, where $f$ is frequency here.  

If a transmission element is loaded at its surge impedence loading, it neither creates nor consumes reactive power.  This level is defined by 
\begin{equation}
SIL = V_{LL}^2 / Z_0
\end{equation}
Where $V_{LL}$ is the line to line voltage.  When a transmission line is at a level below this, it will supply reactive power and raise system voltages.  When the loading is above this level, the transmission line consumes reactive power, depressing voltages.
} \endnote{\textbf{Balanced Three Phase Power Flow}

In a balanced three phase system, at every vertex $i$, we have
\begin{equation}
S_i = P_i + j Q_i = V_i I_i^*
\end{equation}
In addition, with $KCL$, we have $I_i = \sum_{k=1}^n Y_{ik} V_k$, where $Y$ is the admittance bus matrix.  The admittance is the inverse of imdedence, that is
\begin{equation}
Y = G + j B = 1/Z = 1/(R + jX) 
\end{equation}
where $B$, the imaginary part of admittance, is susceptance and $X$, the imaginary part of impedence, is reactance.  Now we have that the complex power at every vertex $i$ is
\begin{equation}
S_i^* = P_i - j Q_i = V_i^* \sum_{k=1}^n Y_{ik} V_k
\end{equation}
By converting these into rectangular coordinates, we get two equations for each bus
\begin{align} \label{ac-pf}
P_i = \sum_{k=1}^n |V_i| |V_k| \left[ G_{ik} \cos (\delta_i - \delta_k) + B_{ik} \sin (\delta_i - \delta_k) \right]  \\
Q_i = \sum_{k=1}^n |V_i| |V_k| \left[ G_{ik} \sin (\delta_i - \delta_k) - B_{ik} \cos (\delta_i - \delta_k) \right]  
\end{align}

For each bus, in addition to $P_i$ and $Q_i$,  we have its voltage $|V_i|$ and its phase angle $\delta_i$.  So, we have $4 n_v$ variables and $2 n_v$ equations.  By supplying the value to $2 n_v$ variables and defining a slack bus, we can find unique values for the remaining $2n_v - 1 $ variables.  Depending on the type of bus, different variables are supplied.  If it is a generator, both $P$  and $V$ are supplied.  A load is defined by a $P$ and $Q$.  The slack bus is a generator in which the phase angle is set to 0.  The phase angle $\delta$ and reactive power production $Q$ are found for each generator and the phase angle $\delta$ and the voltage $|V|$ are found for the loads.}  
These non-linear equations model the net power and reactive power injects as well as voltage and phase angle at each vertex of the power system.  A few simplifying assumptions will be made to allow the equations to become linear for the DC (decoupled) power flow equations.  Then, using the dc power flow model, basic economic dispatch and unit commitment models will be shown.  
%Finally, some pseudo-topological measures will be reviewed that will be useful in the optimization process.

\subsubsection{Decoupled Power Flow}
This model makes assumptions such as lossless lines, small voltage angle differences, and a flat voltage profile.  This is a common simplifying model which is used routinely in economic and reliability analysis of power systems.  A flat voltage profile implies that $\forall i \in \cV$, we have $V_i = 1$.  Small voltage angle difference in conjunction with sine and cosine give the following approximations
\begin{align}
\cos(\delta_i - \delta_j) &\approx 1	\\
\sin(\delta_i-\delta_j) & \approx \delta_i - \delta_j
\end{align}
In addition, since these lines are lossless, $R$ and $G$ are 0.  

The following equations represent the necessary constraints on the power flow in this lossless system.  The first equation represents the conservation of energy and the second equation represents Kirchoff's Current Law.  Conservation of energy implies that the sum of generation and demand is equal to 0 at every point in time.
\begin{align}	\displaystyle
\sum_{j | e=(i,j),  e \in \cE}{f_{e}} &= g_{i} -d_{i} \hspace{17px}   \forall i \in \cV   \label{pf1}
\\
\theta_{i} - \theta_{j} &= X_{e} f_{e}			\hspace{27px}	\forall e=(i,j) \mbox{ s.t. } e \in \cE   \label{pf2}
\end{align}
The value $p_{i} = g_{i} - d_{i}$ represents the net power inject for that node.  A reliability focused model would seek to maximize demand served at all points in time.

When a line is outaged, the power flow has obviously dropped to 0, that is $f_e = 0$.  In addition, the constraint \ref{pf2} to the system is no longer there and must be removed from the formulation.

\subsubsection{Economic Dispatch}
To clear the electricity market at a single point in time, a least cost dispatch model is used.  This model takes bids from generators, a known demand, as well as transmission and ramping constraints and finds a set of generator outputs which meet the demand at least cost.  Using a quadratic cost function for the generators (this cost function can be thought of as a bid from generators which includes the profit the generator would like to make for each marginal unit of production), the least cost dispatch model is as follows. 

The following model is a quadratic program with linear constraints.  The objective function is to minimize the cost of generation.  Typical least cost dispatch and unit commitment models will make various assumptions to allow for linear constraints versus the physical nonlinear constraints to which the power system is subject.  This is the most basic least cost dispatch model, which could be used for clearing the real-time market.  Models in use can have extensions such as ''$N-1$ constraints", which are reliability and security requirements.

\begin{subequations}
\begin{align}
 \min \sum_i \alpha_i g_i &+ \beta_i g_i^2	+ W_i(\tilde{d} - d_i)&	\\
g_i - d_i &= \sum_j f_{ij}	&	\forall i \in N 	\\
X_{ij} f_{ij} &= \delta_i - \delta_j & \forall (i,j) \in M \\
g_i  &\in \left[ g_i^- , g_i^+ \right]		&	\forall i \in N 	\\
f_{ij} &\in \left[ -U_{ij}, U_{ij} \right]	&	\forall (i,j) \in M 
\end{align}
\label{leastcostdispatch}
\end{subequations}

Here, $\alpha_i g_i + \beta_i g_i^2$ can be seen as the cost function for the generator at node $i$ and $W_i(\tilde{d} - d_i)$ is the cost for shedding load.  The generator is bounded between a maximum $g_i^+$ and minimum $g_i^-$ that represent its ramping rate over a specific time interval.  

The day-ahead market uses unit commitment models.  This model will have power flow equations for each $t \in \cT$.  These are integer programs due to the introduction of binary variables $y_{it}$, which take on the value 1 if the generator is switched on and 0 if the generator is off.  The following logical constraint enforces this by
\begin{equation}
y_{it} g_i^- \le g_{it} \le y_{it} g_i^+
\end{equation}
The cost function can then include a fixed cost for operating a generator, such as increased staff during operation.  This cost is not dependent on the level of production but rather if the plant is in an on or off state.  The cost function for each node $i$ and time period $t$ is
\begin{equation}
\alpha_i g_it + \beta_i g_{it}^2 + c_i y_{it}
\end{equation}
This subproblem will be used in a full day model in which the power flow problem is solved for each time period, while remaining feasible with respect to ramping rates and on and off times for generators.





\section{Multi-Stage Stochastic Programming}

We will use the framework of stochastic programming to model the dynamics of cascading power failure.  Each stage will represent an epoch at which one or more lines fail based upon their loading.  In order to do this, we will use a large mixed-integer linear program.

\subsection*{Mixed-Integer Program}
The mixed-integer formulation of cascading power failures uses binary variables to model line availability for stages in the cascade.  In addition to these extra variables, some input parameters are needed to formulate the problem.  First, the number of stages, $n_T$, for the cascade needs to be decided.  If large blackouts are of interest, this number should be large enough to not exclude the worst case scenarios.  Also, the number of outcomes at each node needs to be decided.  These outcomes represent the decision dependent uncertainty.  As the number of outcomes increases, the resolution of the uncertainty increases as well.  The size of the problem is related to the size of the stochastic tree, which is $n_o^{ n_T }$, where $n_o$ is the number of outcomes at each node.  As the subproblem is difficult, the computational complexity of this problem increases rapidly with the number of outcomes and stages.

\begin{figure}
\centering
\input{\mypathmsip/fig-stochtree}
\caption{Stochastic tree for Mixed-Integer Formulation}
  \label{fig:mip}
\end{figure}


\subsection{Decision Dependent Uncertainty}\label{decisiondependentuncertainty}
In order to model the decision dependent uncertainty, a cumulative distribution function (cdf) for line failures based on loaded is needed.  This model uses a parameter, $R$, to represent the effective capacity of a line.  This is found by sampling from the cdf before formulating the problem.  This effective capacity represents the loading of the transmission line that will cause its failure.  This failure is represented in the binary variable, $z$. \\

In Mixed-Integer Programming, there have been standard equations developed to model logical conditions.  This problem has two logical conditions that need to be modeled in order to represent this decision dependent uncertainty.  First, the condition that the line will fail if it has more power flow than its effective capacity.
\begin{align*}
\mbox{If }
		\hspace{30px}&\left| f_{e\rho(n)} \right| > R_{en}  \\
\mbox{Then }
		\hspace{30px}&z_{en} = 0
\end{align*}

To model this logic, a Big-M constraint can be used, where M represents a large number.
\begin{align}
	f_{e\rho(n)} - R_{en} &\le M^R_e (1-z_{en})	\label{r1}\\
	f_{e\rho(n)} + R_{en} &\ge - M^R_e (1-z_{en})	\label{r2}
\end{align}
with $M^R_e = U_e - R_e$. \\

Now, when the line is available in stage $n$, that is $z_{en} = 1$, then the line flow in the predecessor node is within the effective capacity, -$R_{en} \le f_{e\rho(n)} \le R_{en}$.		\\

The second logical condition is that when the line is unavailable, the power flow on that branch is zero and the phase angles between the two nodes are not constrained. 
\begin{align*}
\mbox{If }
		\hspace{20px}&z_{en} = 0	\\
\mbox{Then }
		\hspace{20px}&f_{en} = 0  \hspace{10px} \mbox{ and}\\
				&\theta_{in} - \theta_{jn} - X_{e}f_{en} \mbox{ is arbitrary}
\end{align*}	

This can be achieved through the following equations.
\begin{align}
-U_{e} z_{en} \le f_{en} &\le U_{e} z_{en}	\label{lf1}\\
\theta_{in} - \theta_{jn} + X_e f_{en} &\ge -M^\theta_e(1-z_{en}) \label{lf2} \\
\theta_{in} - \theta_{jn} + X_e f_{en} &\le M^\theta_e(1-z_{en})  \label{lf3}
\end{align}
with $M^\theta_e = 2 \theta_{max} + X_e U_e$.


\subsection*{Failure Density Function}
The OPA simulation uses a step function for the failure probability of a line.  A line will fail if it $f_e \ge \alpha U_e$ with probability $p$.    To model that scenario here, a sample from the distribution can be used to form the effective capacity of the lines.  Let $\omega_n \equiv\left[ 0, 0, 1, 0, \cdots, 1\right]$  be a vector sampled from the probability distribution.  Now, a line will fail if $f_e \ge \alpha U_e$ and $w_{ne} =1$.  With this sampling, the effective capacity can be designed to incorporate this information.    
\begin{equation}
 R_{ne} = 
 \left\{ 
	\begin{array}{lr}
				\alpha U_e & \mbox{if } \omega_{ne}=1\\
			  U_e + \epsilon & \mbox{if } \omega_{ne}=0
	\end{array}
 \right. \label{r}
\end{equation}

The resulting set of effective capacities $R$ can be represented by a cumulative distribution function. The distribution in Figure \ref{cdf} is one example of a viable line failure distribution input.  It is the result of a uniform distribution for $\alpha$ between $L$ and 1 and Bernoulli trial with probability $p$ for $\omega$.  


\begin{figure}
\centering
\pgfplotsset{every axis plot/.append style={line width=2pt}}
\tikzset{
every pin/.style={fill=yellow!50!white,rectangle,rounded corners=3pt,font=\tiny},
small dot/.style={fill=black,circle,scale=0.3}
}

\begin{tikzpicture}[scale=.85]
\begin{axis}[ 
	xlabel=$r$,
	ylabel=$P( R < r )$,
	title = Effective Capacity Cumulative Distribution Function,
	unbounded coords = jump,
	xtick= {0},
	ytick= {0, 1},
	extra y ticks={.5},
	extra y tick style={grid=major},
	extra y tick labels={$p$},
	extra x ticks={.575,1},
	extra x tick labels={$L$, $U$},
scatter/classes={
 		 a={mark=o,line width=3.5pt},%
 		 b={mark=o,line width=1pt,scale=1.75}%
		}]
	\addplot[black] coordinates { 
		(0,0)		
		(.575,0)	
		(.985,.485) 		
		(1,inf)		
		(1,1)		
		};	
	\addplot[ 
		scatter,only marks,
		scatter src=explicit symbolic]
		coordinates {  
		(0,0)		[a]
		(1,.5)		[b]
		(1,1)		[a]
		};


	

\end{axis}
\end{tikzpicture}

\caption{ Example Effective Capacity Distribution for equation (\ref{r}) }
\label{cdf}
\end{figure}



\subsection{Feasible Cascades}\label{feasiblecascades}
The cascading process begins with an intial exogenous event, $\xi \in \left\{ 0, 1 \right\}^{\magE}$.  The following equation enforces a line outage throughout the whole cascade for all $\xi = 1$.

\begin{equation}
z_{en} \le 1- \xi_e  \label{xi}
\end{equation}

Now, the set of all feasible cascades can be described as follows, 
\begin{equation}
\Omega(\xi) \equiv \left\{ \left(d, p, f, z \right)  |  \mbox{ \cref{pf1,pf2,r1,r2,lf1,lf2,lf3,r,xi}  hold } \right\} 
\end{equation}
where $d = \left[ d_0, d_1, \cdots, d_N \right]$ is the set of vectors of demand served for each node in the scenario tree.  Now, this representation of the cascading process can be used in many ways.  One example is a chance constrained model, in which the operator requires a line to be available in a certain percentage of scenarios.  Another example is to use this as a subproblem in a larger planning model. 

\subsection{Least Cost Dispatch}
The following model is a least cost dispatch model that includes the effects from cascading failures due to a set of contingencies $\xi\in\Xi$.  
\begin{subequations}
\label{leastcost}
\begin{align} \displaystyle
	{\large \mbox{min}} \hspace{10px} &  \Expect_\Xi \sum_{in} \left[ C_{in}  p_{in,m}  + W_{in} (\hat{d}_{i} - d_{in,m}) \right]	\\
	&(d,p,f,z)_m  \in \Omega(\xi_m)    \hspace{20px}   \forall \xi_m \in \Xi	
\end{align}
\end{subequations}
This can be adapted to several different problems in power systems.  The main difficulty with this multi stage stochastic program is computational complexity.  The program can be calibrated in order to produce similiar load shed distributions to the OPA model, however the number of outcomes and stages in the model needs to be small or the computational hurdle becomes too large.
\textbf{Model Calibration}

In order to get reliable output from the optimization routine, this model needs to be calibrated against the OPA simulation.  The primary calibration parameters are $L$ and $p$ in the line failure distribution as well as the costs for load shed, $W$, which depend upon the stage in the cascade.  The OPA simulation is a greedy algorithm, attempting to maximize demand served in the current stage without regard to the line failure consequences.  In order to capture this effect, more weight was placed on demand served in earlier stages of the cascade. \\

 The root problem used is comprised of 4 intitial outages and each outage has a scenario tree that is 4 stages long with 2 outcomes at each node.  The power system modeled is the IEEE 118 bus grid with a nominal demand of 3668 MW and around 29,600 MW in branch capacities.  The parameters chosen for the MSIP model were $p=.5$, $L =.575$.  The weight for the stages of the cascade were $[500, 10, .05, .0001]$, which seemed to capture the greedy behavior of the OPA simulation.  The OPA simulation used the step function failure model with $\alpha=.99$ and $p=.5$.  The output of the simulation and MSIP formulation are shown in Table (\ref{compare}) along with the load shed distribution for $\xi = [12,14,34,111]$ in Figure (\ref{dist}). 


% Calibrate ---------------------------------------------------------

\input{\mypathmsip/load_shed_distribution}


\input{\mypathmsip/comp_sim_msip}

%\input{line_outage}


.


\section{Model Flexibility}
The primary strength of this formulation is the flexibility it has in using additional constraints and objectives to inform decision making about power systems.  All of these models can then be solved with commercial solver software without the need to develop additional specialized routines.  The first example is a chance constrained model, which enforces a probabilistic constraint on the number of line outages in any scenario. 
%\endnote{\textbf{Chance Constraint on N-k Criteria}

In this model, constraints are used to express the probability that less than a given number of lines are outaged is close to 1.  A typical case in power systems would be that the system should not go beyond 1 contingency in a large percentage of scenarios.
\begin{equation}
P\left\{\text{Line Outage} \le k \right\} \ge 1 - \epsilon
\end{equation}
This can be done by adding another binary variable for each scenario that represents whether or not that scenario has less than $k$ outages.  Then these new binary variables will be summed and constrained by the given probability.
\begin{subequations}
\begin{align}
\sum_i z_{is} &\le	M_s \hat{z_s} + | \xi_s | + k \\
\sum_s \hat{z_s} &\le \epsilon | \cS |
\end{align}
\end{subequations}
where $M_s = | \cE | - | \xi_s | - k$.  The objective of this program would be to minimize load shed as in (\ref{leastcost}) such that the power system does not lose more than $k$ branches.
}  
In this model, constraints are used to express the probability that less than a given number of lines are outaged is close to 1.  A typical case in power systems would be that the system should not go beyond 1 contingency in a large percentage of scenarios.
\begin{equation}
P\left\{\text{Line Outage} \le k \right\} \ge 1 - \epsilon
\end{equation}
This can be done by adding another binary variable for each scenario that represents whether or not that scenario has less than $k$ outages.  Then these new binary variables will be summed and constrained by the given probability.
\begin{subequations}
\begin{align}
\sum_i z_{is} &\le	M_s \hat{z_s} + | \xi_s | + k \\
\sum_s \hat{z_s} &\le \epsilon | \cS |
\end{align}
\end{subequations}
where $M_s = | \cE | - | \xi_s | - k$.  The objective of this program would be to minimize load shed as in (\ref{leastcost}) such that the power system does not lose more than $k$ branches.


The second example is a redispatch model that moves to a generator configuration that is close to the original while minimizing the expected size of the cascade.
%\endnote{\textbf{Generator Redispatch}

This model tries to find an operating configuration that is within a given distance from the current operating regime and minimizes the worst case scenario outage.  The input for this model is the current operating configuration $(p_0, d_0)$ as well as a distance vector $\delta$ that represents how far each generator can move from its current output levels.  In this case, the root node of the stochastic tree will have variables that represent initial generator output levels and the child trees will be constrained from how far they can move from this.  Also, a continuous variable will be added that represents the amount of load shed in the worst case scenario.  The objective will be to minimize this worst case scenario.  
\begin{equation}
p - p_0 \le \delta
\end{equation}
\begin{equation}
l \ge \sum_i \hat{d}_i - d_{is}
\end{equation}
aim to min 
This program could be used when the system is becoming close to unstable and cost concerns become less of a priority than system stability.
} 
This model tries to find an operating configuration that is within a given distance from the current operating regime and minimizes the worst case scenario outage.  The input for this model is the current operating configuration $(p_0, d_0)$ as well as a distance vector $\delta$ that represents how far each generator can move from its current output levels.  In this case, the root node of the stochastic tree will have variables that represent initial generator output levels and the child trees will be constrained from how far they can move from this.  Also, a continuous variable will be added that represents the amount of load shed in the worst case scenario.  The objective will be to minimize this worst case scenario.  
\begin{equation}
p - p_0 \le \delta
\end{equation}
\begin{equation}
l \ge \sum_i \hat{d}_i - d_{is}
\end{equation}
aim to min 
This program could be used when the system is becoming close to unstable and cost concerns become less of a priority than system stability.


 Another exampe is a reserve planning model that allocates reserves among the generators in such a way as to minimize the worst case failure.
%\endnote{\textbf{Operating Reserves}

In this case an operating configuration need not be given.  The program can either search for an operating configuration as well as reserves or given an operating configuration, allocate the reserves.  These operating reserves determine where the system is able to relieve congestion in a contingency.
\begin{align}
p_{in} + r_{in} &\le \overline{P}_i \\
p_{in} &\le p_{i\rho (n)} + r_{i\rho (n)} \\
\sum_i r_{in} &\le \beta \sum_i \hat{d}_i 
\end{align}
where $\beta$ is the level of operating reserves alotted for the system.
}  
In this case an operating configuration need not be given.  The program can either search for an operating configuration as well as reserves or given an operating configuration, allocate the reserves.  These operating reserves determine where the system is able to relieve congestion in a contingency.
\begin{align}
p_{in} + r_{in} &\le \overline{P}_i \\
p_{in} &\le p_{i\rho (n)} + r_{i\rho (n)} \\
\sum_i r_{in} &\le \beta \sum_i \hat{d}_i 
\end{align}
where $\beta$ is the level of operating reserves alotted for the system.


Finally, the example we will use is a transmission expansion model that allocates a budget for capacity expansion on the grid in such a way as to minimize the expected size of the cascade.  This will have computation results and compare them with the OPA simulation to a simple heuristic which allocates the expansion budget.




\subsection{Transmission Expansion}
Using the MIP formulation, a stochastic program can be developed to model planning decisions with respect to cascading power failures.  Consider the problem of transmission expansion.  A set of contingencies, $\xi \in \Xi$, has been identified as the primary risk for initiating a cascading event.  There is a budget to use for expansion and the objective is to allocate the budget in such a way as to minimze some risk measure of load shed for the cascading power failures. \\

Let $x$ be the design variable, which is decided at the root node and represents additional capacity on power lines.  This affects the constraint set such that equations (\ref{lf1}) and (\ref{r}) become  
\begin{align}
-(U_{e}+x_e) z_{en} \le f_{en} \le (U_{e}+x_e) z_{en} &	\label{lfm}	\\
 R_{ne} = 
 \left\{ 
	\begin{array}{lr}
				\alpha (U_e + x_e) & \mbox{if } \omega_{ne}=1\\
			  (U_e + x_e) + \epsilon & \mbox{if } \omega_{ne}=0
	\end{array}
 \right. \label{rm}
\end{align}



\begin{figure}
\centering
\begin{tikzpicture}

\draw (.5,.5) node(ROOT)[circle,draw]{\small $x$ };
\draw (3,3) node(ONE)[circle,draw]{ \small $\Omega_1$ };
\draw (3.25,1.5) node(TWO)[circle,draw]{ \small $\Omega_2$ };
\draw (3.25,.85) node(DOTONE){ \large $\vdots$ };
\draw (3.25,0) node(N)[circle,dashed,draw]{ \small $\Omega$ };
\draw (3.15,-.65) node(DOTTWO)[rotate=-15]{ \large $\vdots$ };
\draw (3,-1.5) node(S)[circle,draw]{ \small $\Omega_M$ };

\draw (1.85,2.15) node(OMG1){ \small $\xi_1$ };
\draw (2.05,1.25) node(OMG2){ \small $\xi_2$ };
\draw (2.1,.4) node(OMG){ \small $\xi$ };
\draw (1.9,-.35) node(OMGS){ \small $\xi_M$ };

\draw[thick, ->] (ROOT) -- (ONE) ;
\draw[thick,->] (ROOT) -- (TWO);
\draw[thick,dashed, ->] (ROOT) -- (N);
\draw[thick,->] (ROOT) -- (S);

%\draw[thick,dashed] (TWO) -- (N);
%\draw[thick,dashed] (N) -- (S);

\end{tikzpicture} 
\caption{Scenario Tree for Two Stage Problem}
  \label{fig:mip}
\end{figure}



The objective is to allocate a budget for capacity additions in order to reduce the expected blackout size.  $W_{in}$ is the cost of load shed at bus $i$ in node $n$ of the scenario tree.  $B$ is the budget for capacity expansion.
\begin{subequations}

\begin{align} \displaystyle
	{\large \mbox{min}} \hspace{10px} &  \Expect_\Xi \sum_{in} \left[ C_{in}  p_{in,m}  + W_{in} (\hat{d}_{i} - d_{in,m}) \right]	\\
	&(d,p,f,z)_m  \in \Omega(\xi_m)    \hspace{20px}   \forall \xi_m \in \Xi	\\
	& \underline{X}_e y_e \le x_e \le \overline{X}_e y_e \hspace{35px} \forall e \in \cE\\
	&\sum_e x_e \le B_x 	\\
	& \sum_e y_e \le B_y  
\end{align}
\label{TX}
\end{subequations}



\subsection{Computational Implementation}
The model outlined in (\ref{TX}) is solved using Gurobi 4.5.  The stopping criteria was either a 40\% optimality gap or 10,000s.  The program was run for several different expansion budgets, with both the total budget and the maximum number of lines being changed.  The output of this model is a vector $x$ which represents the amount of capacity to add to each power line on the grid.  This is used to modify the initial grid and run the OPA simulation to find the effects it has on the system.  In order to find out whether these solutions are reasonable or not, a heuristic was developed to compare against.

The heuristic used is based on a large number of OPA simulation runs in which the power lines were ranked in descending order based on the percentage of runs in which that given power line was outaged.  Then, for a given total budget $B_x$ and maximum number of changed lines $B_y$, the heuristic picks the top $B_y$ lines in the list and then distributes the budget $B_x$ evenly over the lines.  The OPA simulation is used to compare the results of the two models.  Since the MSIP was calibrated based on the 4 contingencies, a second set of 4 random initial contingencies to start the OPA process to evaluate the effects.




\begin{figure}
 \centering
	\begin{tikzpicture}
		\begin{axis}[xlabel=$LS$ (MW), ylabel=$P($Load Shed $> LS )$
				,legend pos=north east
				,grid=major,
				,xmin=-25,xmax=1300
				,title=\mbox{Load Shed Distribution} ]


 	\addplot[red,line width=2pt] table[x=ls, y=prob] {./data/d25k20.dat};
	\addlegendentry{design}
	
	\addplot[blue,line width=2pt] table[x=ls, y=prob,mark=square] {./data/dumb25k20.dat};
	\addlegendentry{heuristic}

	\addplot[black,line width=2pt] table[x=ls, y=prob,mark=square] {./data/sim.dat};
	\addlegendentry{nom}




		\end{axis}	
	\end{tikzpicture}
  \caption{Load Shed Distribution for the OPA simulation and MSIP formulation}
 \label{dist}
\end{figure}

% Calibrate ---------------------------------------------------------

\section{Results and Conclusion}
After several trials, we found this solution methodology to be inadequate.  Due to the computational difficulty, the stochastic tree could only have around 4 stages with 3 outcomes per node in ideal settings by trial and error.  When using 4 initial contingencies, only sometimes were solutions found within a 40\% optimality gap.  This course representation of the underlying uncertainty made it a poor representation of the effects of the decision variables on the OPA simulation.  However, this approach may be successful if more analysis could determine highly probable failure sequences which need to be protected against.

Using the OPA simulation as a reference for how the grid may respond to cascading power failures, a mixed integer model was developed which represents the cascading effects over a fixed number of outcomes and stages.  While this model can be difficult computationally due to the decision dependent uncertainty, it is extremely flexible.  The model can be used to include cascading effects in a wide range of power system problems with optimality criteria ranging from least cost to minimize worst case problems.  A computation example was done on transmission expansion, which showed the model was able to better than a reasonable heuristic, even though it was only solved to a 40\% optimality gap.  Future work can be done in order to improve the solve time for these types of models based on techniques developed in stochastic and mixed integer optimization.

%\theendnotes
%\setcounter{endnote}{0}


\newcommand{\mypathdfo}{../thesis/dfo}
\newcommand{\mypathdfodata}{../thesis/dfo/data}
\newcommand{\mypathdfocode}{../thesis/dfo/code}
\newcommand{\scd}{\cD_\oplus}
\newcommand{\btu}{\bigtriangleup}
\newcommand{\wrp}[1]{ $<$#1$>$}

\usetikzlibrary{shapes.geometric,arrows,shadows,backgrounds}
\usetikzlibrary{positioning-plus,node-families,calc}


\definecolor{codegreen}{rgb}{0,0.6,0}
\definecolor{codegray}{rgb}{0.5,0.5,0.5}
\definecolor{codepurple}{rgb}{0.58,0,0.82}
\definecolor{backcolour}{rgb}{0.95,0.95,0.92}


\lstdefinestyle{mystyle}{
    backgroundcolor=\color{backcolour},   
    commentstyle=\color{codegreen},
    keywordstyle=\color{magenta},
    numberstyle=\tiny\color{codegray},
    stringstyle=\color{codepurple},
    basicstyle=\scriptsize,
    breakatwhitespace=false,         
    breaklines=true,                 
    captionpos=b,                    
    keepspaces=true,                 
    numbers=left,                    
    numbersep=5pt,                  
    showspaces=false,                
    showstringspaces=false,
    showtabs=false,                  
    tabsize=2
}

\lstset{style=mystyle}


\tikzset{
  basic box/.style = {
    shape = rectangle,
    align = center,
    draw  = #1,
    fill  = #1!25,
    rounded corners},
  header node/.style = {
    Minimum Width = header nodes,
    font          = \strut\ttfamily,
    text depth    = +0pt,
    fill          = white,
    draw},
  header/.style = {%
    inner ysep = +1.5em,
    append after command = {
      \pgfextra{\let\TikZlastnode\tikzlastnode}
      node [header node] (header-\TikZlastnode) at (\TikZlastnode.north) {#1}
      node [span = (\TikZlastnode)(header-\TikZlastnode)]
        at (fit bounding box) (h-\TikZlastnode) {}
    }
  },
  hv/.style = {to path = {-|(\tikztotarget)\tikztonodes}},
  vh/.style = {to path = {|-(\tikztotarget)\tikztonodes}},
  fat blue line/.style = {ultra thick, blue}
}

\tikzstyle{process} = [rectangle, minimum width=3cm, minimum height=1cm, text centered, text width=3cm, draw=black, fill=orange!30]
\tikzstyle{sensor}=[draw, fill=blue!20,
    text centered, minimum height=2em,drop shadow]
\tikzstyle{sensor2}=[draw, fill=blue!20, 
    text centered, minimum height=2.5em,drop shadow]
\tikzstyle{ann} = [above, text width=5em, text centered]
\tikzstyle{wa} = [sensor, text width=10em, fill=red!20, 
    minimum height=6em, rounded corners, drop shadow]
\tikzstyle{sc} = [sensor, text width=13em, fill=red!20, 
    minimum height=10em, rounded corners, drop shadow]
\tikzstyle{data} = [sensor2,cloud, fill=red!20, rounded corners, drop shadow]
\tikzstyle{decision} = [diamond, minimum width=3cm, minimum height=1cm, text centered, draw=black, fill=green!30]

\chapter{Exploring the OPA~Simulation Through Transmission Expansion}\label{dfo-chapter}
%\section{Solving Design Problems}

The cascading model built in Chapter \ref{msip-model} is too hard to optimize at the scale we need to model the uncertainty in the cascading process.  In order to work around this difficulty, we will decompose the multi-stage structure of this problem and instead simulate the cascading process.  The primary benefit is removing the temporal linked binary variables for line outages from the master problem allowing sequential evaluation of the decision dependent uncertainty.  This allows us to parallelize the OPA evaluations to increase the computational effort.

We will explore the long term design problem of transmission capacity expansion using the OPA simulation to evaluate rare event stress on the transmission infrastructure.  We begin by laying out the OPA simulation which involves sequential evaluations of a linear program modeling economic dispatch with a weighted load shed term in the objective.  The lines will fail in the same manner described in the previous chapter in \cref{decisiondependentuncertainty} and the failure density function can be seen as the line having a random effective capacity that when exceeded the line will fail with certainty.  As the simulation proceeds, the topology will change and a new linear program will be solved.  The first primary benefit of this sequenctial procedure is the ability to hot start the LP solver.  The constraint for a line that has failed which links the connected nodes phase angle will be relaxed.  The LP solver can use the previous basis and perform a small number of pivots to find an optimal solution to the new problem representing the changed topology.

Following the discussion on the LP subproblem and the failure density function will be the algorithm used to evaluate an OPA simulation trial for a fixed demand and contingency.  This single trial will sample from the cascade evolution distribution $\Omega$ and is representing the same uncertainty as the presampled $\Omega$ from the multi-stage stochastic program in \cref{feasiblecascades}.  In this section we are going to explore the uncertainty in load shed distribution due to the uncertainty caused by the cascade evolution uncertainty in $\Omega$ and a small subset of initial contingencies in $\Xi$. We describe the statistical measures of load shed distribution we are interested in \cref{riskmeasures} and focus primarily on the function describing expected value of load shed for the simulation exploration and design problem optimization.

We then look at whether we can create a surrogate function for the OPA cascade simulation in order to develop a computationally cheap evaluation that is correlated with the function representing expected load shed.  We plot the values of load shed by stage in the cascade and see that the cascades with large load shed do not necessarily have large load sheds in the first stage.  Additionally, the number of lines that fail in the first stage of the cascade is not correlated with the load shed at the final stage.  While we certainly have not explored all possible surrogate functions, we proceed by using the full cascade simulation instead of a surrogate function.

Since we need to use the full cascade simulation and there is a high standard deviation in the load shed distribution, it is important to use variance reduction techniques to reduce the computational effort needed to achieve a small confidence interval on the expected value of load shed.  Common random numbers significantly improves the ability to compare different systems and reduce the variance of the difference of the performance of two systems.  For design problems, this is exactly what we need.  We would like to evaluate how systems perform relative to each other. 

After explaining the common random number scheme used, we proceed to explore the function describing the expected value of load shed.  This OPA simulation and the expected value of load shed over many trials provides fundamental difficulties to the optimization process.  The function is neither convex nor continuously differentiable.  The load shed is inherently noisy and its  distribution is wide, often characterized by a power law distribution.  Some of these effects can be smoothed away by employing a wide range of potential initiating events, however it may be important to optimize against a small subset of events that have a higher likelihood of occurrence and are known to be risky.  In this case, the non-convexity and discontinuity are most apparent. 

After learning about the function characteristics, we begin to explore how we can optimize the design problem of transmission expansion.  We look to the class of derivative free optimization techniques of direct search methods which are simple, flexible, and powerful.  Altough we do not have the nice function properties that guarantee global or local convergence, they have been found to work well for these types of black box optimization routines.  By using a grid based pattern search method that describes a spanning set, we can be assured of local convergence in regions of the functions with nice properties.  Additionally, we can use exploratory steps that can be based off of accessory information that can speed up the search routine and not ruin the local convergence properties.

In order to tackle this problem, we will use massive computational effort through parallelization of the OPA simulations.  We describe this parallelization process using Condor on the UW-Madison Center for High Throughput Computing.  In order to be a friendly load on the Condor system, we employed Condor DAGMan to manage the job submission procedure for each iteration in order to smooth job submissions and ensure a small number of idle jobs at any given time to reduce the stress on their system.  The file and submission structure will be described as well as the scripts needed to manage this process.  Years of computational time can be done within a short time horizon.  This enables a fine mesh search pattern on the function in order to improve on a completely brute force search procedure.  Finally, we show the results from implementation of this simple DFO method and the improvements in expected load shed from the OPA simulation.


\section{OPA Cascade Simulation}

The load shed distribution of the fast time scale OPA model (given in \cref{fast_opa}) has been shown to have the same power-law distribution seen in real-world blackout data.  This simulation can be seen as a surrogate model for the response of the power system to rare-event stress.  As such, it is useful to explore the effects different parameters can have on the distribution and even optimize over them to find any characteristics or trends there may be. To begin, we need to develop an efficient evaluation of the cascading power failure simulation for eventual use in an optimization procedure. 

A brief description of the parameters, variables, indicies and sets will be given and the LP subproblem for the cascade simulation will follow.  Variable $x_j$ represents the generation from generator $j$ in the set of all generators $\cJ$.  The LP will have a quadratic cost function with $c_j^2, c^1_j, c^0_j$ representing the cost parameters for generator $j$.  Branch flow is represented by $y_e$ for branch $e$ in the set of all branches $\cE$.  The parameter $b_e$ represents the susceptance of branch $e$ and is used to describe the branch flow based on the difference of phase angles between the two connected nodes.  The variable $l_i$ represents load shed at node $i$ in the set of all nodes $\cI$, the parameter $W$ representing the cost of load shed, and the parameter $d_i$ is the nominal demand.  Finally, $c^x_{ij}$ is an incidence matrix which takes a 1 when generator $j$ is located at node $i$ and $c^y_{ie}$ is a -1 when branch $e$ is from nodes $i$ and a 1 when branch $e$ is to node $i$. The following linear program (\ref{dcopf_program}) is a load shedding version of the standard DC OPF economic dispatch model.  
\begin{subequations}
\label{dcopf_program}
\begin{alignat}{3}
\min_{\left(x,l;\theta,y\right)} && \displaystyle\sum_{j \in \cJ} \left[  c_j^2 x_j^2  + c_j^1 x_j + c_j^0 \right] &+ W z &  \label{jcc_obj}\\
                        && \sum_{j \in \cJ} c^x_{ij} x_j - \sum_{j \in \cJ} c^b_{ie} y_e   +l_i       &=d_i       && \forall i \in \cI \label{opf_cons}\\ 
                 && y_e - b_e \sum_{i \in \cI} c^b_{ie} \theta_i          &=0         && \forall e \in \cE \label{opf_kcl}\\
&& z - \sum_{i \in \cI} l_e &=0 &&  \label{opf_loadshed} \\
                 && x_j &\in \left[ G^{min}_j, G^{max}_j \right] && \forall j \in \cJ \label{opf_gen}  \\
                 && y_e &\in \left[ -U_e, U_e \right] && \forall e \in \cE \label{opf_limit}\\
                 && l_i &\in \left[ 0, d_i \right] && \forall e \in \cE \label{opf_loadshed}
\end{alignat}
\end{subequations}
where $z$ is the total load shed for a particular dispatch point with vectors $(x,y,l)$ representing generation, branch flows, and nodal load shed.


We will look to understand the effect of adding capacity $u$ to the system prior to an initiating event for the OPA cascading procedure.  The subproblem (eqs. \ref{dcopf_program}) is modified to allow the branch flows to attain their new capacity level, that is
\begin{equation}
 y_e  \in \left[ - U_e - u_e, U_e + u_e \right]    \hspace{15px} \forall e \in \cE
\end{equation}
In addition, the failure density function from the previous section shown in \cref{cdf} will also move to account for the additional capacity.  This means that not only does the capacity level change, the point $L$, in which the line becomes risky also moves, so that our line risk function is now
\begin{equation}\label{linefaildensity}
g_e(y_e,u_e) = \left\{ \begin{array}{ll}
1 & y_e > U_e + u_e \\
(y_e-u_e)U_e^{-1}p - L_e p & U_e + u_e \geq y_e > L_e U_e + u_e \\
0 & \mbox{o/w}
\end{array}
\right. 
\end{equation}

\begin{figure}
\centering
\includegraphics{\mypathdfo/fig-effectivecapacity}
\caption{ Line failure density function from eq. (\ref{linefaildensity}) }
\label{cdf}
\end{figure}


We will use the subproblem from \cref{dcopf_program} and the failure density function to simulation the OPA cascading process.  The algorithm requires sequential solves of the DC OPF with changes to the topology. We can take advantage of this using LP hot starts where the sequential solves typically require only a small number of pivots to find the new optimal solution.  The cascading algorithm is given in \cref{opa_alg}.  The algorithm is run for a fixed design vector $u$, demand vector $d$, and initial contingency $\xi$ representing a set of lines that has failed to initiate the cascading sequence.  The topology is changed according to the contingency $\xi$ and the new DC OPF problem is solved.  The algorithm proceeds by using the failure density function to determine the probability that a line will fail and then samples a Bernoulli variable with this probability to determine the outcome.  If no lines fail, the cascade is considered over.  If lines fail, the topology is changed, the new LP is solved, and the process repeats until no lines fail.

\begin{algorithm}
\caption{OPA cascading algorithm simulating the evolution of the cascading process governed by uncertainty space $\Omega$}\label{opa_alg}
\begin{algorithmic}
\Procedure{OPA}{$u, d, \xi$}
\State Solve ( \ref{dcopf_program} ) to find base case load shed $z_0$
\State $\xi$ occurs and corresponding changes to the grid are made
\State Stage $s \gets 1$
\While{ Not DONE }
\State Solve ( \ref{dcopf_program} ) to find power injects and branch flows for adjusted grid
\State $\mathbb{O}_s \gets \emptyset$
\For{$\forall e \in \cE $}
\State $\mathbb{O}_s = 
\left\{ 
\begin{array}{lr}
  \mathbb{O}_s + \left\{ e \right\} & \mbox{w/ prob. } g_e(y_e,u_e), \mbox{ draw } \omega_{es} \\
  \mathbb{O}_s & \mbox{o/w }
\end{array}
\right. $ 
\EndFor
\If{ $\mathbb{O}_s \neq \emptyset$ }
\State Modify Grid with $\mathbb{O}_s$
\State $s\gets s+1$
\Else
\State $s^* \gets s,$ calculate $z_{s^*}$, DONE
\EndIf
\EndWhile
\State \label{done}Load Shed $  \lambda\left(u,d,\xi,\omega\right)  = z_{s^*} - z_0$
\EndProcedure
\end{algorithmic}
\end{algorithm}

\subsection{Risk Measures}\label{riskmeasures}
The OPA algorithm simulates one outcome for a fixed $u, d,$ and $\xi$.  Let $\omega$ represent this sampled uncertainty and $\lambda\left(u,d,\xi,\omega\right)$ be the load shed from this single OPA trial.  Repeating this process for $N^t$ trials will give a distribution of load shed values.  We will primarily be interested in the expected value of load shed over the initial contingencies $\Xi$ and the cascade evolution $\Omega$.  Let function $f (u) $  for design $u$, fixed demand $d$ and contingency $\xi$ be calculated as 
\begin{equation}
f(u) = \Expect_{\Xi \Omega} \left[  \lambda\left(u,d,\rxi,\romega\right)  \right]
\end{equation}

\newcommand{\tabheight}{11pt}
\begin{table}
\centering
\begin{tabular}{| c | c | c|}
\hline
& & \\[1pt]
Sample Mean & $f$ &$ \hat{Z}(N^t)=\sum_n \frac{Z_n}{N^t} $\\[\tabheight]
Sample Variance & $s^2_d$ &$ S^2(N^t)=\sum_n \frac{\left(Z_j - f\right)^2}{N^t-1} $\\[\tabheight]
Standard Error & $s^2_e$ &$ \Var{f}= \frac{s^2_d}{N^t}$\\[\tabheight]
Confidence Interval& $CI(1-\eta)$  & $f \pm z_{1-\eta/2} \sqrt{s^2_e}$ \\[\tabheight]
Value at Risk & $ VaR(\eta)$& $ \lb Z_{a} |  a = \floor{ \eta N^t }\rb $\\[\tabheight]  %$\inf\left\{l \in \R{} | F_L(l) \geq \eta \right\}$\\[\tabheight]
Conditional Value at Risk & $ CVaR(\eta)$& $\Exx{Z_n | Z_n \geq VaR(\eta)}$ \\[\tabheight]
\hline
\end{tabular}
\caption{Risk Measures}\label{tab:risk}
\end{table}

For $N^t$ trials with load shed $Z_n = \lambda (u,d,\xi^n,\omega^n )$, we can describe the distribution of load shed by a set of risk measures.  We just looked at the expected load shed and the sample mean for $N^t$ trials can be given in \cref{tab:risk}.  In addition to the mean, we have the sample variance, which is a measure of the width of the distribution and the standard error, which measures how accurately we have calculated the mean based on how many samples we have taken.  With the mean and standard error we can describe $1-\eta$ confidence intervals, which says that the mean will be within a given range at least $1-\eta$ percent of the time.  This confidence interval is calculated using the inverse of the standard normal distribution.  The value at risk is a measure of the percentile of the distribution, so that the $\eta$ value at risk is the value of the load shed at percentile $\eta$.  The conditional value at risk is the expectation of the load shed conditioned on it being above a percentile $\eta$.  This finds the area underneath the tail of a distribution.  All of the risk measures are shown along with the formulas to calculate them for a set of $N^t$ samples in \cref{tab:risk}.  Additionally, these statistical measures are plotted for the load shed distribution for a sequence of design points $u$ in which capacity is being added to one line in \cref{fig:riskmeasure}

\begin{figure}
\centering
\includegraphics{\mypathdfo/fig-riskmeasure}
\caption[Risk Measures]{Value at Risk, Conditional Value at Risk, and Maximum risk measures for load shed distribution}\label{fig:riskmeasure}
\end{figure}



\subsection{Parameters and Simulation Inputs and Outputs}
We will now go into the details involving the OPA simulation, in particular, the parameters that are used in the algorithms and the input and output files used in the computational implementation. As with all of the models in this thesis, the dispatch model is done on a network topology with parameters defining susceptance for branch flow, generators and their cost coefficients, and the nominal demand of the system.  These paramters are stored in a \wrp{.gr} file and is one of the inputs to the simulation routine.  All of the other parameters, which are used in the cascade \cref{opa_alg} are defined in a \wrp{.in} file and an example is showed in \cref{paramdef}.  This includes the line failure distribution parameters $L$ and $p$ as well as a scenario tree definition, which connects this simulation to the MSIP model in the previous chapter.  The scenario tree is built by defining the number of stages and how many outcomes per node.  The number of outcomes per node is allowed to vary depending on the stage of the simulation.  Since this is a simulation and we are decomposing the scenario tree structure, we typically use a large number of outcomes at the root node and then resolve these completely.  The number of initial contingencies is the parameter called scenarios and for each scenario the initial outages are defined in the Outage section.  Finally, trials is how many times the scenario tree is solved, so that a 100 child node tree with 500 trials will result in 50,000 samples of a full cascade simulation.  Finally, a seed is given for the random number generators so that a common random number scheme can be used if desired.

The solve methodology takes information from the parent node in the scenario tree about the new topology required, relaxes the required branch flow constraints, solves the problem, and stores the results in a tree data structure.  The process repeats until all nodes in the scenario tree are resolved.  If no lines fail in a node of the scenario tree, the child nodes are truncated since the OPA cascade algorithm has ended.  After a scenario tree is solved, the final nodes of the tree are looped over and the load shed of that sample is output into a \wrp{.dem} file and each line number that has filed is output into a \wrp{.lin} file.  These raw data files can be large and there are python scripts to do a load shed analysis on the \wrp{.dem} file to give the risk measures associated with the load shed distribution in a \wrp{.lsa} file.  The \wrp{.lin} file is counted with a python script and output in a \wrp{.lao} file in order to find out how many times a given line has failed over all the trials.  These input and output files are tabulated in \cref{tab:files}.

\begin{table}
\centering
\begin{tabular}{| l  r | l | m{10cm} |}
\hline
File && Size & Description \\
\hline
\wrp{.gr} & & 8K &  Defines bus, branch, and generator parameters (grid2.gr~$\equiv$~IEEE~118~bus,8K)\\
\wrp{.in} &\cref{paramdef} & 4K & Simulation parameter definitions \\
\wrp{.cap} &\cref{optimalpointcap} & 4K & Capacity addition file \\
\wrp{.dem} & & 80K & The load shed from every trial cascade simulation \\
\wrp{.lin} & & 2.1M & The line number for every time a line has failed in a cascade during all the trials \\
\wrp{.lsa} &\cref{optimalpointlsa} & 4K & Statistical analysis of load shed from cascade trials in \wrp{.dem} \\
\wrp{.lao} &\cref{optimalpointlao} & 4K & Count of line outages from cascade simulations output in \wrp{.lin} \\
\hline
\end{tabular}
\caption{Data files used in parallelization routine}
\label{tab:files}
\end{table}



\linespread{1}
\lstinputlisting[language=bash,label={paramdef},caption={simcplex.in: Parameter definition file}]{\mypathdfocode/Powerin/shared/simcplex.in}
\linespread{2}




\subsection{Surrogate Functions}
The OPA cascade simulation requires many LP solves in order to simulate the cascading process for many trials.  One potential way to reduce the computational effort needed would be to develop a surrogate function which is correlated with the expected load shed function we are trying to minimizing.  If we could find a surrogate that was cheap to compute and by minimizing the surrogate function, the expected load shed function would also decrease, this function could be used in the optimization procedure in many ways.   For our surrogate function, we consider using information resolved in the first stage of the cascade algorithm.

\begin{figure}
\centering
\includegraphics{\mypathdfo/fig-firststage}
\caption[Lack of correlation between load shed and first stage line failures]{The poor relationship between load shed and the number of lines that failed in the first stage.}
\label{fig:first}
\end{figure}


We continue exploring the OPA simulation by looking at potential first stage approximators of the OPA cascading process. The two potential approximators we look at are the load shed after the first stage in the cascade and the number of lines that have failed after the first stage of the cascade.  We begin by looking at the number of lines that have failed after the first stage of the cascade. Figure \ref{fig:first} shows a scatter plot of load shed in the 1st stage with load shed in the final stage.  Through visual inspection, we see that there is little to no correlation between the number of failed lines and the final load served.  This may be due to the difference of importance of transmission lines in the network; we explore this idea in the final chapter \cref{ch:jccow}.


\begin{figure}
\centering
\includegraphics{\mypathdfo/fig-loadserve}
 \caption{Load Served at Different Stages}\label{fig:loadserve}
\end{figure}


Next, we explore whether or not the load served at intermediate stages of the cascade has any correlation with the final load served. In \cref{fig:loadserve} we see the progression of the load shed over the course of the cascade.  These scatter plots chart the load served, out of a potential 3,668 MW.  At stage zero the load served is 3,668 for all scenarios and we see the distribution of the final load served between about 1,500 and 2,500 MW.  After the first stage, the load served for some scenarios drops below the nominal load.  However, the amount of load served after the first stage does not appear to be well correlated with the final load served.  As the stages progress up to 20, we see that the load served in intermediate stages of the cascade give little insight into whether it is a bad cascade or not.  


\subsection{Common Random Numbers}
The stochastic uncertainty of the OPA process leads to large standard errors for the risk measures.  Variance reduction techniques are important to reduce the computational burden and a common random number scheme was employed.  Common random numbers essentially give each test system the same set of experimental conditions.  By doing so, the variance in the difference between two systems is reduced and less computational resources will be needed. \cite{law_2007}

The OPA simulation uses effective capacities to determine whether a line fails for a particular stage and branch flow loading.  In order for the alternative configurations to be under similar experimental conditions, a random number seeding strategy was used to ensure alternative configurations would recieve the exact same effective capacity for that particular stage in that particular node of the scenario tree.   Formally, we can see why this works in some cases.  Suppose we have two systems with expected load shed $L_{ij}$ for systems $i=1,2$ and trials $j=1,2,...N$.  Now lets study the metric $Z_j = L_{1j} - L_{2j}$ and let the true comparison be $\Exx{Z}=\mu_Z$.  In order to make decisions about this, we need to be fairly confident in our estimation $\hat{Z}(n)=\sum_j \frac{Z_j}{n}$ of $\mu_z$.  The standard error of our sample mean $\hat{Z}(n)$ is
\begin{equation}
\Var{ \hat{Z}(n) } = \frac{ \Var{X_1} + \Var{X_2} - \CoVar{X_1}{X_2} }{n}
\end{equation}

In order to reduce the variance, we need $\CoVar{X_1}{X_2}>0$.  Be using a common random number scheme giving the alternative system configuration the same experimental conditions, we end up with a postive covariance.  This makes sense, as a sample path with consistently low effective capacities should do worse in all systems.  In practice, this common random number scheme is very successful for our problem and makes comparing systems less costly.


\subsection{Capacity Addition}
We now explore the landscape of the function with respect to capacity additions by looking at the results from simple changes to the design variables.  We start by looking at what happens when you add load shed along the coordinate directions.  We restrict ourselves to looking at lines that already exist and adding capacity to these lines.  We will not impose any restrictions on the amount of capacity to add to lines.  Thermal line limits are often constrained in order to maintain a minimum clearance between the lines and vegetation or other infrastructure.  So, in addition to being able to build a parallel line, it is also feasible to increase this clearance by installing taller poles or increasing vigilence in vegetation management.

Our first figure \cref{fig:double} shows what happens if you simply double the capacity on each line individually and do nothing to the rest of the system.  The red points are the nominal system and its standard error plotted along each direction and the blue points correspond to the line number in which the capacity was doubled. Perhaps contrary to what would be expected or hoped for, the OPA simulation does not respond favorably in the majority of the coordinate directions and less than 40\% of lines lead to a reduction.  There may be a few reasons for this.  First, by increasing the limit of one line, it may now be possible to overload neighboring lines and increase their likelihood of failure. 

\begin{figure}
\centering
\includegraphics{\mypathdfo/fig-double}
\caption{Comparing the total load shed for doubling capacity along the coordinate direction.}
\label{fig:double}
\end{figure}

Additionally, by looking along just one direction and slowly increasing capacity, we see that the function can change dramatically.  Even when the number of trials are increased so that the standard error is low, the function evaluations corresponding to adding capacity on one transmission element are variable and can have large discontinuities.  We have plotted three transmission elements in which the capacity is varied from zero to three times its nominal capacity in \cref{fig:capadd}.  The point in which the capacity is zero is still enforcing the phase angle linkage so that when the branch flow is zero, the phase angles of the two connecting nodes must be equivalent.


\begin{figure}
\begin{subfigure}[b]{0.3\textwidth}
\centering
	\begin{tikzpicture}
	\begin{axis}[title=\mbox{Line A},scale=.4568]			\addplot+[opacity=.5, mark size=.5, smooth,solid] table[x=run, y=ex, y error=se] {./data/clusters/m-n1.dat};
%		\addlegendentryexpanded{$\i$ - Ex}
	\end{axis}	
	\end{tikzpicture}
\end{subfigure}
\begin{subfigure}[b]{0.3\textwidth}
\centering
	\begin{tikzpicture}
	\begin{axis}[title=\mbox{Line B}, scale=.4568]			\addplot+[opacity=.5, mark size=.5, smooth,solid] table[x=run, y=ex, y error=se] {./data/clusters/m-n2.dat};
%		\addlegendentryexpanded{$\i$ - Ex}
	\end{axis}	
	\end{tikzpicture}
\end{subfigure}
\begin{subfigure}[b]{0.3\textwidth}
\begin{center}
	\begin{tikzpicture}
	\begin{axis}[title=\mbox{Line C},scale=.4568]			\addplot+[opacity=.5, mark size=.5, smooth,solid] table[x=run, y=ex, y error=se] {./data/clusters/m-n3.dat};
%		\addlegendentryexpanded{$\i$ - Ex}
	\end{axis}	
	\end{tikzpicture}
\end{center}
\end{subfigure}

	\caption{Plotting expected load shed for capacity additions along coordinate directions.  }
\label{fig:capadd}
\end{figure}

Looking further at the ability of increased capacity to influence neighboring lines, we focus on line C from \cref{fig:capadd}.  Line C highlights the difficulty of optimizing over this function by showing the discontinuities of the OPA simulation with respect to capacity additions.  In figure \cref{fig:cluster}, we visually show that these discontinuities can be associated with changes in the frequency of line failures for lines in the given cascade simulations and additionally the frequency of line failures may be correlated with nearby lines.  Here we see multiple jumps in the expected value of load shed, and during the last jump, we also see a spike in frequency of line failures for a subset of the transmission lines that are in the same region of the transmission network.  The frequency of line failures can often be lined up with corresponding changes in expected load shed.

\begin{figure}
\centering
\includegraphics{\mypathdfo/fig-linecluster}
\caption{A cluster of lines correlated with reduced system performance}
\label{fig:cluster}
\end{figure}

Finally, we look at a surface plot of adding capacity on separate branches in \cref{fig:heatmap}.  Exploring the effects of adding capacity to lines 67 and 79, we vary the amount of capacity from zero to a very large number.  We then truncate the pictures at the point where the additional capacity has no more effect on the OPA simulation and show only the part where changes occur.  We see that the surface is bumpy, that is it has some higher frequency effects with low amplitude.  Additionally, we note there are larger trends that have much higher amplitude effects.  It will be important for our optimization method to be robust to these high frequency bumps so that they do not get trapped in local minima.  Additionally, the conditional value at risk is plotted as well for this same set of trial points in \cref{fig:heatmap}. The CVaR has some similar trends and minima as the expected value, however it differs in some notable ways as well.  The expected value has a large region in which the value is near minima.  In the CVaR plot, the minima is a smaller region.  Additionally, the CVaR also has more high frequency bumps creating local minima and maxima.  This may be true of the actual function, however, since this is measuring the tail of the distribution, it may be related to sample size as well.

\begin{figure}
\centering
\includegraphics{\mypathdfo/fig-heatmap}
\includegraphics{\mypathdfo/fig-heatmapcvar}
\caption[Expected load shed and conditional value at risk for capacity expansion]{2d heat map of expected load shed and conditional value at risk for adding capacity to two different lines, 67 and 79}\label{fig:heatmap}
\end{figure}



\section{Optimizing Design Problems using Pattern Search}

This section will give an overview of derivative free optimization.  The foundation for direct search and model based search of DFO techniques will be explored to be used in solution methods for the OPA optimization procedure.  The DFO field has been around for some time now and has seen a resurgence over the last decade and a half with the only textbook at less than 5 years old (Conn, Scheinberg, Vicente) \cite{conn_2009}.  Kolda et. al. provide an extensive work on direct search methods and its extensions in \cite{kolda_2003}. 

This class of optimization strategies covers a wide-range of problems and techniques.  In general, the problem has a function $f: \R{n} \rightarrow \R{}$ that takes a decision variable $x \in \R{n}$ and returns a scalar. 
\begin{equation}
\min_{x \in \R{n}} f(x)
\end{equation}
The primary problem attribute that makes DFO a good choice for solution method is that standard gradient or Newton based method will not work.  This can be due to a variety of reasons, a common one being simulation-based optimization.  Here, the derivative is unavailable symbolically (through hard work or automatic differentiation schemes) and, perhaps due to stochastic or numerical noise, is unable to be calculated with finite difference methods.  Even if the underlying function is smooth, a costly function evaluation may make the finite difference approach undesirable due to the considerable time to calculate full gradients.

While still useful for smooth functions with a Lipschitz continuous derivatives, these methods can shine in nonsmooth and even non-convex application.  Direct search methods work by searching in many directions in order to guarantee a descent direction is chosen.  This has the ability to provide robustness against noise that may mislead gradient based methods using a single search direction.  In addition, by using relatively large step size the trial points provide a smoothing affect to the function which allow it to ignore high-frequency noise until it is close to a lower-frequency, higher amplitude minimum.  Contrary to problem application, we will assume a smooth function with a Lipschitz continuous derivative in order to find convergence results for different algorithms.  No guarantees can be made for the nonsmooth problems, however in practice these techniques are relatively successful for this class of problems.

\begin{figure}
\centering
\begin{tikzpicture}[
  level 1/.style={sibling distance=40mm},
  edge from parent/.style={->,draw},
  >=latex]


% root of the the initial tree, level 1
\node[root] (root) {\large \textbf{ Derivative Free Optimization}}
% The first level, as children of the initial tree
  child {node[level 2] (c1) {Direct Search}}
  child {node[level 2] (c2) {Model Based}}
  child {node[level 2] (c3) {Heuristics}}
  child {node[level 2] (c4) {Global}};

\node[oracle] (fo) at (5.1,2.1){Function Oracle};
\node[level 2 oracle,below of = fo,xshift=5pt] (o1) {Dimension Reduction};
\node[level 2 oracle,below of = fo,xshift=80pt, yshift=-20pt] (o2) {Variance Reduction};
\node[level 2 oracle,below of = o1,yshift=-10pt] (o3) {Accessory Information};
  \draw[->] (fo.190) |- (o1.west);
  \draw[->] (fo.190) |- (o2.west);
  \draw[->] (fo.190) |- (o3.west);

\draw[<->,very thick,dashed] (root) -- (fo.west);

% The second level, relatively positioned nodes
\begin{scope}[every node/.style={level 3}]
%\node [below of = c1, xshift=15pt,yshift=-5pt] (c11) {Generalized Pattern Search};
\node [below of = c1,xshift=15pt,yshift=-5pt] (c11) {Generating Set Search}; %with bound constraints
\node [below of = c11,yshift=0pt] (c12) {Mesh Adaptive};
\node [below of = c12] (c13) {Geometric, Simplex};

\node [below of = c2, xshift=15pt,yshift=2pt] (c21) {Trust Regions};
\node [below of = c21] (c22) {Implicit \newline Filtering};
\node [below of = c22,yshift=-5pt] (c23) {Stochastic \newline Approximation};

\node [below of = c3,xshift=15pt,yshift=2pt] (c31) {Particle Swarm};
\node [below of = c31] (c32) {Simulated \newline Annealing};
\node [below of = c32,yshift=-5pt] (c33) {Genetic \newline Algorithms};

\node [below of = c4, xshift=15pt,yshift=-5pt] (c41) {Nested \newline Partitioning};
\node [below of = c41, yshift=-2pt] (c42) {Space Filling};
\node [below of = c42,yshift=7pt ] (c43) {Global Model}; %Response Surface, radial basis functions, kriging
\end{scope}

% lines from each level 1 node to every one of its ``children''
\foreach \value in {1,2,3}
  \draw[->] (c1.195) |- (c1\value.west);

\foreach \value in {1,...,3}
  \draw[->] (c2.195) |- (c2\value.west);

\foreach \value in {1,...,3}
  \draw[->] (c3.195) |- (c3\value.west);

\foreach \value in {1,...,3}
  \draw[->] (c4.195) |- (c4\value.west);
\end{tikzpicture}

\caption{DFO methods for continuous variable optimization}
\end{figure}

\subsection{Direct Search}
An overview of the convergence results for a standard direct search method will be given. Positive spanning sets as well as the cosine measure are used to bound the angle between the polling directions and the negative gradient.  Then, using the subsequence of unsuccessful iterates, the trial steps become arbitrarily small and $x$ will approach a limit point.  We will need the assumption that the gradient is Lipschitz continuous with constant $M$.
%\begin{equation}
%\norm{\grad f(y) - \grad f(x)}  \leq M \norm{ y - x }
%\end{equation}

\begin{figure}
\centering
\begin{tabular}{c c}
\begin{tikzpicture}[ 
   scale=5,
    axis/.style={very thick, ->, >=stealth'},
    important line/.style={thick, color=red, ->},
    dashed line/.style={dashed, thin},
    pile/.style={thick, ->, >=stealth', shorten <=2pt, shorten
    >=2pt},
    every node/.style={color=black}
    ]
    % axis
    \draw[axis] (-0.1,0)  -- (1.1,0) node(xline)[right]
        {$x_1$};
    \draw[axis] (0,-0.1) -- (0,1.1) node(yline)[above] {$x_2$};
    % Lines
    \draw[pile] (.5,.5) coordinate (x) -- (.775,.5)
        coordinate (d2) node[right, text width=5em] {$d_2$};
    \draw[pile] (.5,.5) coordinate (x) -- (.3,.3)
        coordinate (d1) node[right, text width=5em] {$d_3$};
    \draw[pile] (.5,.5) coordinate (x) -- (.3,.7)
        coordinate (d3) node[right, text width=5em] {$d_1$};
    \fill[red]  (x) +(0,0) coordinate (.5,.5) circle (.4pt)
        node[above] {$x$};

    \draw[important line] (.5,.5) coordinate (x) -- (.78,.925)
        coordinate (gf) node[right, text width=5em] {$-\grad f(x)$};

     \draw[dashed line] (d2) -- (.78,.5);
     \draw[dashed line] (gf) -- (.78,.5);

     \path[clip] (.5,.5) -- (.78,.5) -- (.78,.925);

     \node[circle,draw=blue,minimum size=25pt] at (.5,.5) (circ) {};
     \node at(.62,.55) {$\theta$};


\end{tikzpicture}
&
\begin{tikzpicture}[ 
   scale=5,
    axis/.style={very thick, ->, >=stealth'},
    important line/.style={thick, color=red, ->},
    dashed line/.style={dashed, thin},
    pile/.style={thick, ->, >=stealth', shorten <=2pt, shorten
    >=2pt},
    every node/.style={color=black}
    ]
    % axis
    \draw[axis] (-0.1,0)  -- (1.1,0) node(xline)[right]
        {$x_1$};
    \draw[axis] (0,-0.1) -- (0,1.1) node(yline)[above] {$x_2$};
    % Lines
    \draw[pile] (.5,.5) coordinate (x) -- (.75,.5)
        coordinate (d2) node[right, text width=5em] {$d_2$};
    \draw[pile] (.5,.5) coordinate (x) -- (.5,.25)
        coordinate (d3) node[right, text width=5em] {$d_3$};
    \draw[pile] (.5,.5) coordinate (x) -- (.5,.75)
        coordinate (d1) node[ text width=3em] {$d_1$};
    \draw[pile] (.5,.5) coordinate (x) -- (.25,.5)
        coordinate (d4) node[above right, text width=5em] {$d_4$};
    \fill[red]  (x) +(0,0) coordinate (.5,.5) circle (.4pt)
        node[above left] {$x$};

    \draw[important line] (.5,.5) coordinate (x) -- (.78,.925)
        coordinate (gf) node[right, text width=5em] {$-\grad f(x)$};

     \draw[dashed line] (d1) -- (.5,.925);
     \draw[dashed line] (gf) -- (.5,.925);

     \path[clip] (.5,.5) -- (.78,.925) -- (.5,.925);

     \node[circle,draw=blue,minimum size=25pt] at (.5,.5) (circ) {};
     \node at(.535,.625) {$\theta$};
\end{tikzpicture}
\\
$N+1$ points & $2N$ points 
\end{tabular}
%    \draw[important line] (.15,.85) coordinate (C) -- (.85,.15)
%        coordinate (D) node[right, text width=5em] {$\mathit{NX}=x$};
    % Intersection of lines
 %   \fill[red] (intersection cs:
%       first line={(A) -- (B)},
%       second line={(C) -- (D)}) coordinate (E) circle (.4pt)
%       node[above,] {$A$};
    % The E point is placed more or less randomly
%    \fill[red]  (E) +(-.075cm,-.2cm) coordinate (out) circle (.4pt)
%        node[below left] {$B$};
    % Line connecting out and ext balances
%    \draw [pile] (out) -- (intersection of A--B and out--[shift={(0:1pt)}]out)
%        coordinate (extbal);
%    \fill[red] (extbal) circle (.4pt) node[above] {$C$};
    % line connecting  out and int balances
%    \draw [pile] (out) -- (intersection of C--D and out--[shift={(0:1pt)}]out)
%        coordinate (intbal);
%    \fill[red] (intbal) circle (.4pt) node[above] {$D$};
%    % line between out og all balanced out :)
 %   \draw[pile] (out) -- (E);

\caption{Positive spanning sets for $\R{2}$}
\end{figure}

A positively spanning set $\cG$ of $\R{n}$ can write any vector $v \in \R{n}$ as a positive combination of points $d_i \in \cG$,  $\beta_i \geq 0 \forall i$
\begin{equation}
v = \sum_i \beta_i d_i
\end{equation}
Kolda, Lewis, and Torczon \cite{kolda_2003} call this a generating set of $\R{n}$ which makes the foundation for their class of generating set search methods.  This is a large class of problems which generalize lattice methods of Berman \cite{berman_1966} \cite{berman_1969} as well as their own older methods \cite{torczon_1997} \cite{lewis_2000} and include the original Hookes and Jeeves method\cite{hooke_1961}.  Generating sets can be adapted for explicit linear constraints to conform to local topology.

These constraints can be dealt with in the framework of generating set search.  Here, one needs to concern themselves only with the nearby active constraints and a satisfactory set of polling directions would be a positive spanning set of the tangent cone of the nearby active constraints. 
\begin{figure}
\centering
\begin{tabular}{m{.625\textwidth} m{.3\textwidth}}
\begin{tikzpicture}[ 
   scale=7,
    axis/.style={very thick, ->, >=stealth'},
    locset/.style={very thick},
    important line/.style={thick, color=red, ->},
    dashed line/.style={dashed, thin},
    pile/.style={thick, ->, >=stealth', shorten <=2pt, shorten
    >=2pt},
    every node/.style={color=black}
    ]
    % axis
%  \draw[locset] (.45,0)  -- (.75,.3) -- (.75,.8) -- (.7,.9) --  (0,.9);
\filldraw[fill=green!20,draw=green!50!black] (0,0) -- (.45,0) -- (.75,.3) -- (.75,.7) -- (.55,.9) --  (0,.9) -- cycle; 

    \draw[axis] (-0.1,0)  -- (1.1,0) node(xline)[right]
        {$x_1$};
    \draw[axis] (0,-0.1) -- (0,1.1) node(yline)[above] {$x_2$};


    \fill[red]  (.65,.35) coordinate (x) circle (.4pt) ;

    \node[above left] at (x) {$x$};
     \node[circle,draw=blue,minimum size=75pt] at (x) (circ) {};

%    \draw[important line] (x) -- (1,.725)
%        coordinate (gf) node[right, text width=5em] {$-\grad f(x)$};  %% gradient
     \draw[thin,<->,color=blue] (x) -- node[above right]{$\epsilon$} +(5.38pt,0);  %% radius of circle
     
     \draw[very thick,color=blue!50!black] (.75,.3) -- (.75,.7);
     \draw[very thick,color=blue!50!black] (.45,0) -- (.75,.3);

    \draw[pile]  (x) -- +(0,.25)
        coordinate (d1) node[right, text width=5em] {$d_1$};

    \draw[pile]  (x) -- +(-.15,-.175)
        coordinate (d1) node[right, text width=5em] {$d_2$};


    \draw[pile,color=blue]  (.75,.575) -- +(.17,0)
        coordinate (n1) node[right, text width=5em] {};
    \draw[pile,color=blue]  (.55,.1) -- +(.12,-.12)
        coordinate (n2) node[right, text width=5em] {};

    \draw[thin,dashed,color=blue] (.55,.1) -- +(.04,-.04) -- +(.08,0) -- +(.04,.04);
    \draw[thin,dashed,color=blue] (.75,.575) -- +(.06,0) -- +(.06,.06) -- +(0,.06);

\end{tikzpicture}
&
\begin{tikzpicture}[
    scale=2,
    axis/.style={very thick, <->, >=stealth'}]

    \draw[thin] (-1.1,-1.1) -- (-1.1,1.1) -- (1.1,1.1) -- (1.1,-1.1) -- cycle;
    \filldraw[fill=blue!15!white,draw=blue!50!black] (0,0) -- (1.1,0) --  (1.1,-1.1) --   cycle; 
    \filldraw[fill=green!20,draw=green!50!black] (0,0) -- (0,1.1) -- (-1.1,1.1) --  (-1.1,-1.1) -- cycle; 
    \draw[axis] (-1.1,0)  -- (1.1,0) node(xline)[right] {$x_1$};
    \draw[axis] (0,-1.1) -- (0,1.1) node(yline)[above] {$x_2$};
    
    \node at (-.5,.5) {$\cK^o(x,\epsilon)$};
    \node at (.7,-.325) {$\cK(x,\epsilon)$};



\end{tikzpicture}
\end{tabular}

\caption{Generating set for polar cone to conform to local explicit constraints}
\end{figure}
The compass search for standard bound constraints conforms to the constraints ideally and no modifications need to be made other than letting $f = \inf $ for $x$ out of bounds and not waste the time on the function evaluation.


By searching in all directions of a generating set of $\R{n}$, we are guaranteed to have a direction which is somewhat aligned with the descent direction $f$, if it is smooth and has a Lipschitz continous derivative.  To make this matter concrete, the cosine measure of a set is the worst case scenario for the descent direction aligning with any direction of the generating set $\cG$.
\begin{equation}\label{kappa}
\kappa ( \cG) \equiv \min_{v \in \R{n}} \max_{d \in \cG} \frac{v^t d}{\norm{v} \norm{d}}
\end{equation}
This can be calculated for various generating sets, for example the compass search generating set gives $\kappa(\scd) = \frac{1}{\sqrt{n}}$ where $n$ is the dimension of the problem.  This begins to show why this method will struggle as the dimension of the problem increases.  The cosine measure of the search directions must be bounded below to ensure the search directions do not deteriorate.

\begin{algorithm}
\caption{Compass search, a generating set search}\label{dfo_genset}
\begin{algorithmic}
\Procedure{CS}{$f:\R{n} \rightarrow \R{}$}
\State $x_0 \in \R{n}$ Initial guess
\State $\bigtriangleup_{tol} >0$ Termination criteria
\State $\bigtriangleup_0 > \bigtriangleup_{tol} >0$ Initial neighborhood
\BState For each $k=1,2,...$

\State Let $\scd =\left\{ \pm e_i| i=1,...,n\right\}$ be the set of coordinate directions
\If{ $\exists d_k \in \scd$ such that $f(x_k + \btu_k d_k ) < f(x_k)$}
\State $x_{k+1} \gets x_k + \btu_k d_x$
\State $\btu_{k+1} = \btu_k$
\Else
\State $x_{k+1} \gets x_k$
\State $\btu_{k+1} = \frac{1}{2} \btu_k$
\If{$\btu_{k+1} < \btu_{tol}$} terminate \EndIf
\EndIf
\EndProcedure
\end{algorithmic}
\end{algorithm}

As long as our step size goes to zero $\Delta_k \rightarrow 0$ as our work effort increases $k\rightarrow \infty$, we will converge to a stationary point.  There are two primary ways to ensure this happens.  The first is to use a forcing function that constrains the trial step to have more than simple decrease.  Another way is to ensure all the trial points lie on a rational lattice.  If all the trial points are then integer combinations, all future trials will lie on this lattice.  As it is rational, only a finite number of points will be evaluated before there is an unsuccessful iteration.  As there are only a finite number of evaluations in between unsuccessful steps and at each unsuccessful step the step size is reduced, that step size will converge to 0.  In \cref{fig:explore} we see an example of a polling step with exploratory trial points on a rational lattice. For a more detailed convergence proof for standard compass search and the more general generating set search, Kolda et. al. \cite{kolda_2003}.



\begin{figure}
\centering
\begin{tikzpicture}[ 
   scale=5,
    axis/.style={very thick, ->, >=stealth'},
    important line/.style={thick, color=red, ->},
    dashed line/.style={dashed, thin},
    pile/.style={thick, ->, >=stealth', shorten <=2pt, shorten
    >=2pt},
    every node/.style={color=black}
    ]
%  \draw [step=1.0,blue, very thick] (0.5,0.5) grid (5.5,4.5);
\draw [color=blue!40,opacity=.7, step=.1,xshift=0, yshift=0] (0,0) grid +(1.1,1.1);

    % axis
    \draw[axis] (-0.1,0)  -- (1.1,0) node(xline)[right]
        {$x_1$};
    \draw[axis] (0,-0.1) -- (0,1.1) node(yline)[above] {$x_2$};
    % Lines

        
    \fill[red]  (.4,.4) coordinate (x) circle (.4pt)
        node[above left] {$x_k$};

    \fill[blue]  (.4,.4) coordinate (x)+(0,.2) circle (.4pt)
        node[above left] {$d_k^1$};
    \fill[blue]  (.4,.4) coordinate (x)+(.2,0) circle (.4pt)
        node[above left] {$d_k^2$};
    \fill[blue]  (.4,.4) coordinate (x)+(0,-.2) circle (.4pt)
        node[above left] {$d_k^3$};
    \fill[blue]  (.4,.4) coordinate (x)+(-.2,0) circle (.4pt)
        node[above left] {$d_k^4$};

    \fill[brown]  (.4,.4) coordinate (x)+(.6,0) circle (.4pt)
        node[above left] {$h_k^1$};
    \fill[brown]  (.4,.4) coordinate (x)+(.3,.3) circle (.4pt)
        node[above left] {$h_k^2$};


%    \fill[red]  (.6,.4) coordinate (x) circle (.4pt)
%        node[above left] {$x_1$};

%    \fill[red]  (.8,.4) coordinate (x) circle (.4pt)
%        node[above left] {$x_2$};

\end{tikzpicture}

\caption{Polling step with exploratory trial points on a rational lattice}\label{fig:explore}
\end{figure}




\subsubsection{Flexibility}
This class of direct search methods was chosen because of the flexibility in its framework.   As long as the trial points lie on a rational lattice, the method will converge to a local stationary point.  This means that the lattice can be rotated to conform to local topology, it will admit exploratory points, and can even use model based methods to improve its search direction and exploratory steps.  This includes aligning search directions with approximate gradients that can be calculated using simplex directions or previous trial points. This framework will allow us to test using accessory information to speed up the optimization procedure and compare its effects to model based methods.
%\textbf{Model Based Methods}



To get an idea of the strengths and weaknesses of direct search on the OPA simulation, the standard compass search algorithm was implemented on a reduced 2 dimensional subspace.  These figures show the rough search space and the need for filtering higher frequency effects.  Similar properties are seen for conditional value at risk.

\begin{figure}
\centering
\includegraphics{\mypathdfo/fig-trialmap}
\caption{Trial exploration using standard compass search with initial step size 50}
\end{figure}


Here we can see the compass search gets trapped in a local minima.  If a small step size is used, the search will get trapped in nearby minima when more progress can be made elsewhere.   We ran 4 different compass searches with initial trial steps of 1, 25, 50, and 75.  Large step sizes tend to improve final solution results and it is important to note that they all found very different solutions. This highlights the need for a strategy to find local minima that are nearer to the global minimum. 


\begin{figure}
\centering
\includegraphics{\mypathdfo/fig-functionvalue}
\caption{Resulting points from compass search with different initial step size values}
\end{figure}











\subsection{Line Search Breakpoints}

We can actually find more information in the actual line flow statistics for different stages in the cascade.  Changes with respect to 1st stage line flows can identify changes in final stage load shed, although not whether it will improve or degrade the performance.


We need an algorithm that can handle the noisy output of the OPA model and uses the accessory information in the simulation.  This information includes the line with the most failures, clustering, topology information, electrical properties, and correlations between lines and load shed. Using direct search as the foundation, we will modify the exploratory steps to take advantage of this information.  This will provide local convergence guarantees while giving more robustness to the solution methodology.


\begin{figure}
\centering
\includegraphics[scale=.84]{\mypathdfo/fig-lineflowscene}
\includegraphics[scale=.84]{\mypathdfo/fig-failprobscene}
 \caption{Line flows for each scenario and  their averages}
\label{fig:flows}
\end{figure}


Additionally, we can find the point in which adding additional capacity won't have any effect.  
\begin{equation}
\nu_e = \max_{\begin{array}{c}\Xi,\Omega\\s=0,1,....,s^*\end{array}} \left\{ \ry_{es} \right\}
\end{equation}
Here $\nu_e (u)$ is the maximum flow on a particular line.  By choosing $u$ such that the line has no chance to fail, any additional capacity will not change the outcome.



\begin{figure}
\centering
\includegraphics{\mypathdfo/fig-breakpoint}
  \caption{Approximation of function by finding breakpoints}
\label{fig:break}
\end{figure}

We also want to make our search process more global.  In order to do that, we want to take exploratory trial points along specific directions.  We would like to maximize how much we learn with each trial poin by looking at how different two systems may be.  In order to compare the systems we use first stage line failure probabilities.
\begin{equation}
\tau (x_i, x_j) = \norm{ p_e(x_i) - p_e(x_j) }^2
\end{equation}
This information can be used to tell if two operating points have different cascading properties.  By looking at the distance between two operating points, breakpoints can be found where the distance between operating points risk characteristics are extremely different from another, close, operating point.







\subsection{Implementing Parallelization in Condor}

The Center for High Throughput Computing provides computation resources for UW and affiliated researchers.  Jobs can be submitted through Condor \cite{beowulfbook-condor} which manages the collective pool of around 1 million cpu hours per day.  Users submit jobs to the cluster, which assigns resources that process the job.  Requirements can be given to ensure that the resource is capable of performing the job.  In order to get access to a larger portion of the cluster, low memory and disk requirements help.  Overhead associated with the process, such as being assigned resources and data transfers can be minimized but not removed. As such, the workflow was designed for job times of around 5-30 minutes and perform all analysis locally with the raw data to reduce network data exchanges.

The majority of the Condor cluster and UW-Madison uses the Scientific Linux distrubution with version 5 and 6.  The submit node I was using had SL6 installed, and I compiled the C++ binary directly on that machine.  In order to access the portion of the cluster which uses SL5, I had to find a remote resource with SL5 installed and build the binary remotely on that machine.  Condor gives a way to have an interactive session the remote resource.  In the Condor submit file (\cref{interactivesession}) I request an interactive session and restrict the remote resource to be a MatlabBuildJob, which ensures an SL5 remote resource.  I transfer the source code and associated libraries as a tar file, unpack and build on the remote resource and close the session, which initiates a transfer of all newly created files, including the SL5 binary for the C++ code.

\linespread{1}
\lstinputlisting[language=bash,label={interactivesession},caption={inter-sl5.cmd: Condor submit file to run an interactive session on an SL5 machine}]{\mypathdfocode/Powerin/intersl5.cmd}
\linespread{2}


After having the main C++ program compiled for two primary linux kernals, we can also access  other Condor networks through the Open Science Grid and a simple flag WantFlocking (\cite{condor_flock}).  This allows us to have a large base of computational resources to request to do our job.  However, many of these resources are memory and storage constrained.  In order to tap into these resources, we restrict the memory requirements  of our computational process to 500MBs and storage at 3GBs, which gave plenty of buffer room for program operation and still allowed the capture of the majority of the resources.  The large data output from the OPA simulation includes stage by stage details of net power injects, branch flows, and load shed.  This data is then analyzed on the remote resource using python script files to find the risk metrics of the load shed as well as any accessory information needed in the optimization algorithm.  The output of the analysis is less than 8K for load shed and outage data that is needed for most of the optimization routines.  
\begin{table}
\centering
\begin{tabular}{| l r | c | m{6cm} |}
\hline
File && Diagram &   Description \\
\hline
proc.py &\cref{procpy} & Condor Queue Reader & Condor queue reader and command instructor \\
runit &\cref{runit}& Runit Daemon &  Principal process flow manager \\
allocate.py &\cref{allocatepy}& Pattern Search Logic & Pattern search logic \\
consub.py &\cref{consubpy}& Dag File Structure &Create condor submit structure based on given search points from pattern search logic\\
%power.py &\cref{powerpy}& &Classes for grid, bus, and branches \\
%tools.py &\cref{toolspy}& &Common functions for analysis \\
CONTROL &\cref{controlfile} & CONTROL: job t & A condor file of bash commands to run on the remote resource \\
%REMOVE && & A list of files to remove before transfering created files back to master computer \\
countLines.py &\cref{countlinespy}& Data Analysis & Take \wrp{.lin} file and count number of line outages \\
loadShed.py &\cref{loadshedpy}& Data Analysis & Take \wrp{.dem} file and do statistical analysis of load shed \\
\hline
\end{tabular}
\caption{Scripts and command files used in parallelization routine}\label{tab:condorscript}
\end{table}

Now, we begin to outline the parallelization routine used with condor to optimize the capacity expansion problem using the OPA simulation to evaluate rare event stress.  In \cref{fig:parallel}, the main process flows are represented.  The blue boxes represent processes or scripts that create or modify data.  The red clouds represent data used in these processes, however some files are omitted from this diagram in exchange for overall process clarity.  Dashed lines are initiating events or creation of files from the processes and dashed lines are data transfers, inputs, or structural relationships between data.  The green decision box represents the daemon that keeps the overall parallelization algorithm running.  The principal scripts used in these processes are tabulated in \cref{tab:condorscript}.

\linespread{1}
\begin{figure}
\centering
\footnotesize
\input{\mypathdfo/fig-parallel}
\caption[Process flow for parallel OPA evaluations]{Process flow for parallel OPA evaluations and a simple pattern search DFO method.  Red clouds represent data and blue boxes represent processes.  Dashed lines are for data input or transfers and solid lines are processes or data creation. }
\label{fig:parallel}
\end{figure}
\linespread{2}

The runit daemon begins by determining whether the optimization routine is currently running.  This allows the process to be started and stopped at will and will ensure robustness to loss of network connection.  The daemon queries the proc.py script to check the condor queue and the current folder structure to see if any jobs are currently running or there has been steps taken in the optimization routine.  If there is, it continues where it left off, otherwise it initiates the first step in the optimization algorithm.  It has the option to give a set of initial lines to search from function improvement, otherwise it evaluates the nominal system.  The condor file structure is made, the job is submitted, and the daemon waits for the results.
%\linespread{1}
%\lstinputlisting[language=bash,label={runit},caption={runit: Main Process Flow}]{\mypathdfocode/runit.sh}
%\linespread{2}

After the process set up, we begin a standard iteration.  The daemon waits for the Condor queue to become empty and when it does it begins to initiate processes.  It starts by summarizing the output to be used in the pattern search logic.  After this is done, the pattern search logic determines which lines should be searched for improvement and the condor file submit structure is created based on these search directions.  The daemon submits the condor jobs and then begins waiting again for their completion.

Condor DAGman was used for job submission in the intra-iteration process flow.  DAGman allows jobs to be described in a directed acyclic graph which gives control over the order in which jobs are submitted and their dependencies.  It also has additional features to be a friendly load on the Condor system.  In various iterations, there may be over 1000-2000 potential trial points that need to be evaluated.  Instead of submitting these all instaneously, these jobs are processed sequentially and have a small delay in between job submission as to not overload the system.  In addition, it limits the number of jobs that are sitting idle in the queue, which does not slow down the optimization routine but reduces the demand on the condor job management system.  DAGman does add some overhead to the process, however, due to the immense process capabilities of the Condor system, these are dwarfed by the sheer amount of computational power you gain.  Being a friendly load on the system is a small price to pay.

DAGman uses a very specific file structure in order to submit jobs that have shared resources.  The file structure for the input to DAGman is given in \cref{fig:filedagman}.  To make the DAG submission job, Powerin is given as the input directory.  Powerin contains a folder called shared that holds the resources needed for every job.  DAGman will create a job for each folder inside Powerin and that folder will hold the resources necessary for that particular job.  In our case, we include a \wrp{.cap} file which defines the design decision $u$ for that particular trial point.  In the shared folder, there is a file called CONTROL (\cref{controlfile}) that DAGman uses to process each job.  CONTROL is a list of shell command to initiate on the host resource. 
\linespread{1}
\lstinputlisting[language=bash,label={controlfile},caption={CONTROL: commands to run on remote resource}]{\mypathdfocode/Powerin/shared/CONTROL.f}
\linespread{2}
 Additionally, the grid definition file, as well as the simulation parameters, are stored there as they do not change from job to job.  Packed in sl5.tar.gz and sl6bin.tar.gz are the binaries for those particular instances, and DAGman will only bring the correct binary depending on the host resource being used.  SLIBS.tar.gz contains the common libraries that are used by the programs and python scripts.  The python data analysis scripts are also used at the remote resource to process the raw data files in order to reduce the network transfers.  The REMOVE file will clean up the host resource before the job ends and every newly created file that is not removed is transferred back to the submit node.  The other files contained in Powerin are scripts used to create the job folders and run the pattern search logic.  After the folder structure has been created, a condor dag job can be created by running the following command.
\begin{lstlisting}
mkdag --data=Powerin --outputdir=step$iteration --cmdtorun=arrive.py --pattern=want.lao --pattern=want.lsa --type=Other --maxidle=500
\end{lstlisting}
DAGman will use the Powerin directory to create the DAG and create an output directory called stepN depending on which iteration the optimization algorithm is on.  It will create a job for every folder inside Powerin, except shared.  It will begin with the script called arrive.py and upon completion, it will check to see if want.lao and want.lsa have been created.  If they have not, it will be assumed the job has failed and DAGman will resubmit the job.  Additionally, it limits the maximum number of idle jobs in the queue to 500.

\begin{figure}
\linespread{1}
\begin{forest}
  dir tree
  [ Powerin 
    [allocate.py (search logic for allocating trial points)]
    [consub.py (construct condor submit file structure for search pattern)]
    [power.py (power classes for python scripts)]
    [tools.py (common functions for python scripts)]
    [cap.py (class to analysis capacity files for opt routine)]
    [shared
      [CONTROL (commands for host computer)]
      [REMOVE (files to remove before transfer)] 
      [countLines.py (data analysis before network transfer)]
      [loadShed.py (data analysis before network transfer)]
      [grid.gr (grid definition file)]
      [simcplex.in (simulation parameters)]
      [sl5bin.tar.gz (binary for sl5 instances)]
      [sl6bin.tar.gz (binary for sl6 instances)]
      [SLIBS.tar.gz (common libraries)]
    ]
    [ pt1/add.cap    ]
    [ $\vdots$ ]
    [ ptT/add.cap]
  ]
\end{forest}
\linespread{2}
\caption{Folder structure for DAG input in parallelization routine}\label{fig:filedagman}
\end{figure}

The output folder for DAGman will contain subfolders for each job that contain log files as well as the output files that were transferred back to the submit node.  After the daemon sees that all the condor jobs are done, it will initiate summarization and then proceed to the next step in the algorithm.  The overall folder structure for this optimization procedure is given in \cref{fig:optroutine}.  The Powerin folder is cleaned of all job subfolders in between each iteration and new job subfolders are created depending on the new trial points created from the pattern search logic.  Important files about the search points used are copied over to the output folders before they are cleaned from the input folder.  The output folders can be used to trace what has happened up to the current point in the algorithm in order to allow the daemon to continue off where it last was if it was restarted, either intentionally or unintentionally.

\begin{figure}
\linespread{1}
\begin{forest}
  dir tree
  [ChtcRun
    [proc.py (condor queue reader and develop commands)]
    [runit (parallelization daemon)]
    [ Powerin 
      [ python scripts ]
      [shared (files for every trial point) ]
      [ pt1/add.cap (file defining trial point $u^1$ for current iteration)]
      [ $\vdots$ ]
      [ ptT/add.cap (file defining trial point $u^T$ in unique folder)]
    ]
    [ step0  (output directory for first iteration)
      [ pt1  (output for first trial point of first iteration)]
      [ $\vdots$ ]
      [ ptT  (output for Tth trial point)]
    ]
    [ $\vdots$ ]
    [ stepN  (output directory for Nth iteration)
      [ $\vdots$ ]
    ]
  ]
\end{forest}
\linespread{2}
\caption{Folder structure for using DAGs in parallelization routine}\label{fig:optroutine}
\end{figure}

Using a relatively naive pattern search with a fine mesh grid aligned on the coordinate directions, the brute force power of Condor was able to achieve significant improvement in the function value.  The function value and associated risk measure are plotted in \cref{fig:opt1}.  The majority of the improvement was made over the initial iterations of the algorithm.  The algorithm stopped when the function value made no improvement over the last iteration.  The final design point is shown in \cref{optimalpointcap} and the risk measures tabulated in \cref{optimalpointlsa}.


\linespread{1}
\lstinputlisting[language=bash,label={optimalpointcap},caption={point.cap: Transmission expansion design from pattern search method}]{\mypathdfocode/opt/point3.cap}
\lstinputlisting[language=bash,label={optimalpointlsa},caption={point.lsa: Load shed analysis from chosen design}]{\mypathdfocode/opt/point.lsa}
\linespread{2}


\begin{figure}
\begin{center}
\includegraphics{\mypathdfo/fig-optroute}
 \caption{Pattern search procedure for transmission expansion design problem}
 \label{fig:opt1}
\end{center}
\end{figure}






\section{Conclusion}
We began by decomposing the scenarios developed in the previous chapter in order to parallelize the computational effort needed in order to evaluate the wide load shed distribution.  We continued by exploring the function defined by the expected value of load shed over a small subset of initial contingencies as well as the cascade evolution.  We saw that doubling the capacity along coordinate directions does not lead to improvements in the OPA simulation the majority of the times.  We traced this to overloading neighbors when additional capacity was added to one line but not others.  We also saw that discontinuities of load shed could be correlated with the frequency of line failures in the OPA cascade simulation.  We used this to develop a line search procedure that finds these discontinuities. Finally, we implemented the parallelization in Condor using the DAGman functionality for job submission.  This allowed for job submission smoothing as well as restarting hung jobs or jobs that have failed.  We used this parallelization and a naive pattern search with a fine mesh grid to achieve improvement in the function value for the transmission expansion design problem.



%\theendnotes
%\setcounter{endnote}{0}



%\subsection{MCS Implementation}
%Negative - No local convergence properties

%\endnote{\textbf{Brute Force Serach}

%Old problem parameters, which is why the scales are so different

%}



%This chapter gives a brief overview of DFO techniques.  Direct search was chosen due to its ability to filter high frequency noise and its flexibility in implementation while still guaranteeing local convergence under mild assumption.  This flexibility allows us to use accessory information in the exploratory steps to speed up solve times as well as find better global solutions.
%Several indicators were found that can be used to improve both search directions as well as exploratory trial points.  This will be combined with a direct search method in order to see the effects of these ideas as well as compare them to model based methods and their effects.  Additionally, the OPA simulation was implemented with a common random number scheme to reduce variance and low system requirements to parallelize the computational effort.  This allowed for large test cases to be evaluated and optimized using the OPA simulation as a subproblem.





\newcommand{\mypathjcc}{../thesis/jcc}
\newcommand{\mypathjccdata}{../thesis/jcc/data}
\newcommand{\bE}{\mathbb{E}}
\newcommand{\bD}{\mathbb{D}}
\newcommand{\E}[2]{\mathbb{E}_{#1} \left[ #2 \right]}
\newcommand{\bP}[2]{\mathbb{P}_{#1} \left[ #2 \right]}
\newcommand{\dpO}{d\mathbb{P}_\Omega}
\newcommand{\feb}{f_e\left(\beta\right)}
\newcommand{\fenb}{f_{en}\left(\beta\right)}
\newcommand{\fehb}{f_e\left(\hat{\beta}\right)}
\newcommand{\pfeb}{\frac{\partial}{\partial \beta_j}\feb}
\newcommand{\pfenb}{\frac{\partial f_{en}\left(\beta \right)}{\partial \beta_j}}
\newcommand{\pfehb}{\frac{\partial f_e\left(\hat{\beta} \right)}{\partial \beta_j}}
\newcommand{\hye}{\hat{y}_e}
\newcommand{\hse}{\hat{s}_e}
\newcommand{\pe}{\pi_e}
\newcommand{\pn}{\pi_n}
\newcommand{\psen}{\psi_{en}}
\newcommand{\rand}{\boldsymbol{\delta}}
\newcommand{\ri}{\pmb{\delta^m}}
\newcommand{\rf}{\pmb{\delta^y}}
\newcommand{\rx}{\pmb{x}}
\newcommand{\rX}{\pmb{X}}
\newcommand{\rD}{\pmb{\Delta}}
\newcommand{\rdm}{\pmb{\delta^m}}
\newcommand{\rdmk}{\pmb{\delta^m_k}}
\newcommand{\rdy}{\pmb{\delta^y}}
\newcommand{\hry}{\pmb{y}}
\newcommand{\hy}{y}
\newcommand{\rz}{\pmb{z}}
\newcommand{\sD}{\sigma_\Delta^2}
\newcommand{\se}{\sigma_e^2}
\newcommand{\seone}{\sigma_{e}^2}
\newcommand{\setwo}{\sigma_{n}^2}
\newcommand{\sn}{\sigma_n^2}
\newcommand{\sen}{\sigma_{en}^2}
\newcommand{\see}{\sigma_{ee}^2}
\newcommand{\snn}{\sigma_{nn}^2}
\newcommand{\seealone}{\sigma_{e n}^2}
\newcommand{\sko}{\sum_{k_1 \in \cM}}
\newcommand{\skt}{\sum_{k_2 \in \cM}}
\newcommand{\sigot}{\Sigma_{k_1,k_2}}
\newcommand{\irzp}{\E{\Omega}{g(\hry_e)}}
%\newcommand{\rw}{{w(\omega)}} %\newcommand{\rx}{{x(\omega)}} %\newcommand{\ry}{{y(\omega)}}
\newcommand{\Ery}{\E{\Omega}{\ry}}
\newcommand{\Erx}{\E{\Omega}{\rx}}
\newcommand{\Erdm}{\E{\Omega}{\rdm}}
\newcommand{\ErD}{\E{\Omega}{\rD}}
\newcommand{\tB}{\tilde{B}}
\newcommand{\tx}{\tilde{x}}
\newcommand{\ttheta}{\tilde{\theta}}

\chapter{Line~Failure~Risk~Models for Real-Time Dispatch}\label{jcc-chapter}

\section{Introduction}

The use of optimization models for operation of bulk power systems has been critical in creating efficient markets for wholesale power generation over the last few decades.  Optimization models are solved to clear multiple markets, including day ahead and real time markets, while maintaining physical constraints related to generator characteristics and power flow constraints on the high voltage transmission system. The nonlinear and nonconvex equations for balanced three phase power flow make these problems particularly difficult to solve and approximations, such as decoupled (DC) power flow, are commonly used in economic models. 
%The full nonlinear models are seen in reliability focused problems and design problems such as capacitor placement and can solved via local methods or convex relaxations \cite{hijazi_2015,lavaei_2012}.  
These optimization models and solution methodologies have become standard tools for independent system operators tasked with operating the bulk power system.

In this chapter, we focus on the operation of the real time dispatch market (five minutes and less) and the associated reliability issues.  Reliability issues have a large impact on the economy, estimated at \$79 billion in 2001 \cite{lacommare_2006}, relative to the total cost of electricity, \$247 billion in 2001 \cite{eia_gov}.  Additionally, large power outages are disruptive to society, are costly (the 2003 Northeast blackout was an estimated loss of \$6.4 billion to the economy), and can lead to the loss of life \cite{northeast_2003}.  Blackout frequency changes seasonally and with the time of day.  The load shed from blackout events follows a power-law distribution where large blackout are more likely than expected \cite{hines_2009}.  This blackout distribution has been stable over the last thirty years and represents a dynamic equilibrium \cite{dobson_2007,hines_2009}.  Small frequent blackouts are associated with small reserve margins in generation and large infrequent blackouts are associated with a highly utilized transmission system \cite{dobson_2007}.  We develop a risk measure that captures the utilization of transmission elements in a systems perspective.  This is similar to the use of severity measures to capture line loading risks from a systems perspective \cite{vrakopoulou_2013c,wang_2014,wang_2013}.  Our model is equivalent to their system risk measure when there is fixed demand and we use a linear approximation of the risk measure.  However, we solve our original log-convex risk measure exactly via nonlinear programming. We also extend our model to the case of a multivariate Gaussian distribution for net injection uncertainty where we make a linear approximation to retain a tractability.

There has been increased importance placed on models that account for the growing generation uncertainty due to renewables such as wind and solar.  This uncertainty is being  handled for the traditional line threshold model of economic dispatch using chance constraints (CC).  Several groups have developed chance constrained models to ensure that the power flow on any transmission element is less than its capacity for a large percentage of scenarios.  Several extensions are made which involve arbitrary slack distribution \cite{bienstock_2012}, application to large penetration of wind farms \cite{vrakopoulou_2013b}, with HVDC lines \cite{vrakopoulou_2013}, and others \cite{roald_2013,vrakopoulou_2013c} which will be discussed in section \ref{chanceconstraints}.  Instead of relying on the traditional line threshold, we will directly enforce a constraint on the quantity of interest, the probability that a line will fail. The nominal capacity of transmission elements is based on thermal characteristics of the line as well as a set of environmental characteristics, which represent the worst case scenario for different operating times.  Typically, there may be a summer rating and a winter rating for each transmission element, where the winter rating has colder ambient temperature (less heating and less sagging) thus a higher capacity.  The limiting constraint determining the line capacities are typically due to the acceptable sagging level derived from a predetermined risk level of fault \cite{seppa_2007}.  Dynamic line limits are being explored to account for real time environmental conditions \cite{bucher_2013,wang_2011,yip_2009,zhang_2002}.  

In this chapter, we quantify the endogenous risk of line failure due to line loading.  Exogenous failure events, primarily caused by weather, such as falling tree limbs taking out circuits, are accounted for through the N-1 security requirements.  In order to quantify the endogenous risk, we make several assumptions on the failure density function, which are used elsewhere in literature and have a physical interpretation.  Our failure density function assumes that below some loading level there is no endogenous risk associated with the loading and that above a critical level the risk is monotonically increasing.  We use a piecewise linear function to model the risk and this model has been used in cascading power failure research in papers \cite{carreras_2002,chen_2005,dobson_2007,hines_2011,newman_2011}.  While there is a hard limit on transmission lines that causes them to trip with certainty (due to protective relay elements), the system is always operated far away from these points.  If the system was operating close, random fluctuation in power injects would cause the system to trip much more regularly than is seen.  Our model is built on the assumption that increasing power flow increases the risk of line failure, and we can only make probabilistic statements about these failure rates. 


We combine individual line risks to form a system risk measure that represents the probability that one or more lines fail (due to loading), which we constrain to be small depending on the operators' risk tolerance.  The combination of a system risk measure and demand uncertainty leads to our joint chance constraint (JCC) model.  When there is no uncertainty in demand and generation, this  system can be solved exactly using nonlinear programming and its linear approximation reduces to other system risk models that have apeared in literature \cite{vrakopoulou_2013c,wang_2013}.  Under demand uncertainty, the line risk's failure becomes dependent on both the mean and standard deviation of line flow. In order to solve this model, a linearization of the system risk measure is made.  Once this approximation is made, the problem becomes solvable for large test instances even when exogenous N-1 contingencies are considered.  
%%
In section \ref{economicdispatch} we look at the standard economic dispatch model, referred throughout as optimal power flow (OPF).  Using the DC power flow approximation, we show the linear program used in the real time market with a quadratic cost function. In the following section \ref{chanceconstraints}, we begin by exploring the uncertainty introduced into the system by wind generation and variable load.  We then show the chance constraints (CC) used in other work to ensure the reliability constraints are met a large percentage of the time.  In section \ref{jointchanceconstraint}, we develop a system risk measure defined by the probability that one or more lines fail by assuming the probability a line fails is related to line loading and can be represented by a piecewise linear function.  Using this risk measure and the assumption of fixed generation and demand, we can solve this model exactly using nonlinear programming.  Finally, we look to combine uncertain wind and load with our system risk measure.  Under the assumption of a multivariate Gaussian distribution for wind and load, we find the covariance matrix to enforce the resulting variation in branch flows.  The risk constraints are convex with respect to the mean and standard deviation of branch flow.  We then describe the solution methodology in section \ref{solutionmethodology} which uses an iterative algorithm and cutting planes to describe the convex risk constraints.  Finally, we explore the computational results in section \ref{computationalresults} to understand the characteristics of the standard economic dispatch model, chance constraint model, and our joint chance constraint model.


In order to solve our problem, a cutting plane algorithm is used to underestimate the line risk function (convex, not analytic) as well as the branch flow standard deviations (second order cone).  This multiobjective problem of cost and risk form a frontier that is shown in the computational section.  The OPF and CC models give dispatch points on the interior of this frontier, which are inefficient according to our system risk measure.  The CC and JCC models take around the same amount of time to solve and are an order of magnitude slower than OPF.  Our JCC model can reduce the expected number of lines over their hard threshold compared to the CC model, which is what the chance constraint model aims to do on an individual line level.  Finally, our JCC model is robust to deviations in line risk parameters as well as other failure density function models. 




%%%%%%%%%%%%%%%%%%%Section 2, DC dispatch introduction
\section{Economic Dispatch Models}\label{economicdispatch}
Bulk power systems rely on dispatch models to clear the markets for power in a timely manner.  These models are critical to ensure that the market is cleared minimizing cost while maintaining a set of reliability constraints. The DC power flow model is typically used to clear economic markets for both real-time and day-ahead operation, where the day-ahead operation has the additional complexity of committing slow ramping resources and is solved with mixed-integer programming. 


\subsection{Optimal Power Flow using DC approximation} 
The DC power flow model is a simplification of the AC power flow model, which more accurately represents the true physics of the balanced three phase electrical system.  The DC model makes the assumptions that the power lines are lossless, voltages are equal to nominal, and the phase angle differences are small.  While losing some accuracy and important information about the voltages, the problem becomes more tractable.  Methods can be used to find approximate voltages, which can be important indicators for system stability.  Additional complexities are  ignored for clarity such as shunt elements which consume power \cite{matpower}.

The topology of the power grid, with $N_l$ lines and $N_b$ buses, can be represented using an incidence matrix, $C \in \R{N_l \times N_b}$, where $c_{ei}=1$ if line $e$ begins at node $i$ and $-1$ if line $e$ ends at node $i$.  The set of all edges is denoted as $\cE$ and the set of all nodes as $\cI$.  In the DC power flow model, the line susceptance is the constant of proportionaility between bus voltage and liv, so let $D=diag\left(b_1,b_2,...b_{N_l}\right)$ be the diagonal susceptance matrix.  The branch flows $y \in \R{N_l}$ for given a set of phase angles $\theta \in \R{N_b}$ at each bus are determined by Kirchoff's Current Law and can be written as
\begin{equation}\label{kcl}
y=D C \theta
\end{equation}
for the DC power flow model.  This results in the typical DC line constraints $y_{e} = b_{e} (\theta_i - \theta_j)$ for each line $e$ connecting from node $i$ to node $j$.  Applying $C^T$ to the branch flows $y$ in \cref{kcl} give the net injects for a given set of branch flows and represents conservation of energy at each node.
  The system matrix $B = C^T D C$ can be used to write the DC power flow equations for the net injects $x \in \R{N_b}$
\begin{equation}\label{dcpow}
x = C^T y = B \theta
\end{equation}
This equation has one degree of freedom and by removing the slack bus inject (row) and phase angle (column), then the following system
\begin{equation}
\tx = \tB \ttheta
\end{equation}
has a unique solution.   Here, $\tx \in \R{N_b-1}$ has the slack node removed and $\tilde{B}$ is a $(N_b-1) \times (N_b-1)$ matrix with the row for the slack bus inject and the column for the slack bus angle has been removed.  The OPF model uses the DC simplifications from the full AC model to ensure the model can be solved reliably in a timely manner.  A standard OPF model is shown here and used in the computational section.

The following program \cref{opf_program} is minimizing a quadratic cost function of generation $x_j$ for each generator $j$ in the set of all generators $\cJ$ with cost coefficients $c^2_j, c^1_j$ and $c^0_j$.  There are two incidence matricies, $c_{ij}^g$ for generators and $c_{ie}^b$ for branches.  $c_{ij}^g$ takes the value 1 when generator $j$ is connected to node $i$ in the set of all nodes $\cI$.  $c_{ie}^b$ takes the value 1 if the edge $e$, in the set of all edges $\cE$, originates in node $i$ and a -1 if edge $e$ terminates in node $i$.  We also have nominal demand $d_i$ for each node $i$, branch capacities $U_e$ for each branch $e$, and generator limits $G^{min}_j$ and $G^{max}_j$.
\begin{subequations}
\label{opf_program}
\begin{alignat}{3}
\textbf{OPF:= }\min_{\left(x;\theta,y\right)} && \multispan{2}{$\displaystyle\sum_{j \in \cJ} \left[  c^2_j x_j^2 + c_j^1 x_j + c_j^0 \right]$}  & \label{jcc_obj}\\
%\min_{\left(x,\beta,;\theta,y\right)} && \multispan{2}{$\displaystyle\sum_j \left[  c_2 \left(x_j^2 + \beta_j^2 \sD \right) + c_1 x_j + c_0 \right]$}  & \label{jcc_obj}\\
                        &&  \sum_{j \in \cJ} c^g_{ij} x_j - \sum_{j \in \cJ} c^b_{ie} y_e          &=d_i       && \forall i \in \cI \label{opf_cons}\\ 
                 && y_e - b_e \sum_{i \in \cI} c^b_{ie} \theta_i          &=0         && \forall e \in \cE \label{opf_kcl}\\
                 && y_e &\in \left[ -U_e, U_e \right] && \forall e \in \cE  \label{opf_limit}\\
                 && x_j &\in \left[ G^{min}_j, G^{max}_j \right] && \forall j  \in \cJ \label{opf_gen}  
\end{alignat}
\end{subequations}


\section{Chance Constraints for Random Branch Flows}\label{chanceconstraints}



As the penetration of renewables increases, there is an increasing interest in the inclusion of uncertainty in dispatch models.  The primary approach has been to treat wind generation and load variables as random variables and enforce probabilistic constraints replacing the standard deterministic reliability constraints.  Probabilistic constraints, or chance constraints, enforce that a constraint involving a random variable must be satisfied a large percentage of the time.  Chance constraints have been used in a variety of power system problems, from a unified model including large wind penetration and HVDC lines \cite{vrakopoulou_2013c}, to models using an arbitrary slack distribution \cite{bienstock_2012}.  Additionally, their are models that have different assumptions on the probability distributions of the random variables, from the common independent Gaussian assumption \cite{bienstock_2012,roald_2013} to no assumption on the uncertainty except for the ability to sample from it \cite{vrakopoulou_2013,vrakopoulou_2013c,vrakopoulou_2013b}.



We look first at chance constraints for branch flows, which begins by looking at the root uncertainty in net injections due to wind and load. The deviation of net injections from forecast leads to a response by automatic generator control to ensure that supply and demand are constantly in balance.  Under the DC power flow model, the branch sensitivities to changes in net injections are a linear relationship.  This allows us to calculate the uncertainty in branch flows dependent on the uncertainty in wind and load.  A common assumption is that the uncertainty in deviation from forecast is a Gaussian distribution,  and that its mean and covariance is known.  This assumption allows for a tractable formulation of the chance constraint without sampling from the uncertainty distribution. 

\subsection{Random Power Flows}
The power injections at each node are subject to random fluctuation over varying time intervals.  The variation over the 5 minute time scale are dealt with using ancillary services such as regulation and reserve.  The sources of the fluctuations can be due to random demand or generation.  These fluctuations affect the power flows on transmission lines and cause a linear shift in the approximated DC power flow model.  The risk of the system is dependent on the characteristics of these resulting random power flows.  In greater time intervals, the generators can redispatch for economic or reliability reasons.


Over the course of the 5 minute interval dispatch period, demand  fluctuates from its expectation.  This  fluctuation can be due to the behaviors of aggregated residential load, commercial or industrial processes, or power production for renewable sources such as wind and solar.  Assuming these random injects are Gaussian, the linear approximations made in the DC power flow leads to the branch flows also being Gaussian random variables.  The mean vector and covariance matrix of branch flows can be calculated if the mean and covariance of the random injections are known.

Let $\ri \in \R{N_m}$ be a random vector describing the deviation from forecast on a subset of nodes $k \in \cM$ which are random ($| \cM | = N_m$).  Bold lettering $\ri$ represents a random variable with a probability distribution supported on $\bD$.  Define an incidence matrix $C_M \in \R{N_b \times N_m}$ where $c_{im}=1$ if random inject $m$ is connected to node $i$.  Applying \ref{kcl},\ref{dcpow}, the random power injects for the network are
\begin{equation}\label{rand_inj}
 \rx = C_g\left(x_0+\beta \rD\right) - (d + C_M \ri) 
\end{equation}
 where the left term is generation/controllable and the right term is load/uncontrolled.  Both terms have a fixed component due to the forecasted system and a random component due to the random injections.  The random injects $\ri$ are assumed to be a multivariate Gaussian distribution with known mean $\mu^m$ and covariance $\Sigma^m$.  


Now, we want to understand small perturbations to the system.
The random injects cause loading on the transmission lines to fluctuate and the flows are random variables themselves. Taking the derivative of \ref{dcpow}, we get $d \tx = \tilde{B} d \ttheta $.  The changes to the phase angles given a vector of changes to net injects is $ d \ttheta = \tilde{B}^{-1} d \tx $.  This can be carried through to find the sensitivity of branch flows to net injects 
\begin{equation}\label{isf}
 d y = A d x 
\end{equation}
where $A$ is the injection shift factor $A = D C \left[\begin{array}{cc} 0 & \tilde{B}^{-1} \end{array} \right]$ and $0\in \R{N_b}$ is a column of zeros.  For a general slack distribution $\beta$ with $\sum_{j \in \cJ} \beta_j=1$ and $\Delta = \sum_{i \in \cI} dx_i$, the net injects also change by $- \beta \Delta$ so that $dy = A\left( dx - C_g \beta \Delta \right)$.  The slack distribution is responsible for making up the mismatch between forecasted demand and realized demand and a subset of the generators is capable of performing this role.  The random power flows $\ry$ are
\begin{equation}\label{rand_flow}
 \ry = y_0+ A C_G \beta \rD  - A C_M \rdm 
\end{equation}
with $\rf = A C_G \beta \rD - AC_M \ri$ being the random component of branch flows.
Without loss of generality, we can assume that the random injects have zero mean.  If we know them to have a nonzero mean, we can shift the forecasted system to account for it. 
The forecasted system  $\left(x_0, y_0, \theta_0\right)$ is the expectation of \ref{rand_inj} and \ref{rand_flow}
\begin{align*}
C_G x_0 - d &= C^T y_0 \\
y_0 &= B^T C \theta_0 
\end{align*}
with zero mean deviation $\Erdm=\left[0 \cdots 0\right]^T$ and $\ErD=0$. 
From \ref{isf}, we have that the random component of the flow is made part from the random injects and part the generators response.   The random branch flows $\rf$ are a linear function of the random injects $\ri$, thus $\ry$ has a multivariate Gaussian distribution as well.  The branch covariance matrix can be calculated
\begin{equation*}
\E{\bD}{ \left( \ry - \mu^y \right) \left( \ry - \mu^y \right)^T } = \E{\bD}{\left( A(C_G\beta 1^T - C_M) \ri \right) \left( A(C_G\beta 1^T - C_M )\rdm \right)^T } 
\end{equation*}
Since the expectation is linear and the covariance $\Sigma^m$ of $\ri$ is known, we 
have power flow covariance matrix
\begin{equation}\label{branch_cov}
\Sigma^y = A(C_G\beta 1^T - C_M) \Sigma^m (C_G\beta 1^T - C_M)^T A^T
\end{equation}
using the covariance identity $\Sigma (A \rX ) = A \Sigma (\rX) A^T$ for matrix $A$ acting on random vector $\rX$.

This equation can be found piecewise by taking the variance of the random component of the branch flow for a given branch $e$ and substitution $\pi_e = \sum_{j \in \cJ} A_{ej} \beta_j$ (the effect on branch $e$ from aggregate deviation from forecast $\Delta$), we have
\begin{align*}
  \rdfe &= \sum_{j \in \cJ} A_{ej} \beta_j \rD - \sum_{k \in \cM} A_{ek} \rik\\
&= \pi_e \sum_{k \in \cM} \rik - \sum_{k \in \cM} A_{ek} \rik\\
&= \sum_{k \in \cM} \left[ \pi_e  - A_{ek} \right] \rik 
\end{align*}
for the random component of branch flow. The covariance between branch $e$ and $n$ is
\begin{subequations}\label{covar_branch}
\begin{align}
\operatorname{CoVar}[ \rdfee,\rdfeee ] &=  \sum_{k_1 \in \cM} \sum_{k_2 \in \cM} (\pi_{e} - A_{e k_1}) (\pi_{n} - A_{n k_2}) \Sigma_{k_1 k_2} \\
\operatorname{CoVar}[ \rdfee,\rdfeee ] &=  \pi_{e} \pi_{n} \sum_{k_1 \in \cM} \sum_{k_2 \in \cM} \Sigma_{k_1 k_2} \\
& -  \pi_{e} \sum_{k_1 \in \cM} \sum_{k_2 \in \cM} A_{n k_1} \Sigma_{k_1 k_2} - \pi_{n} \sum_{k_1 \in \cM} \sum_{k_2 \in \cM} A_{e k_1} \Sigma_{k_1 k_2}   \\
&+ \sum_{k_1 \in \cM} \sum_{k_2 \in \cM} A_{e k_1} A_{n k_2} \Sigma_{k_1 k_2}  \\
\operatorname{CoVar}[ \rdfee,\rdfeee ] &=  \pi_{e} \pi_{n} \sD -  \pi_{e} \setwo - \pi_{n} \seone   + \seealone
\end{align}
\end{subequations}
where $\Sigma_{k_1 k_2} = \operatorname{CoVar}[ \rikk ,\rikkk ]$ and  $\sD, \se,$ and $\sen$ are all pre-computable parameters based on the covariance matrix of the uncertain net injections given by the following equations
\begin{subequations}
\begin{align}
 \sD &= \sko \skt \sigot  \\
 \se &= \sko \skt A_{ek_1} \sigot   \hspace{15px} \forall e \in \cE\\
 \seealone &= \sko \skt A_{e k_1} A_{n k_2} \sigot \hspace{15px} \forall e,n \in \cE
\end{align}
\end{subequations}


There are two primary parts to the variance in line flow, that from the uncertain random injects and that from the aggregate deviation from forecast at the slack distributions response and it can be described as follows  
\begin{equation}
\operatorname{Var}[ \rdfe ] =  \pi_e^2  \sD  - 2 \pi_e \se  + \see  \hspace{15px} \forall e \in \cE
\end{equation}



When the net injections are uncertain, the branch flows are also uncertain and represented by $\ry_e$ for branch $e$, where the bold letter denotes a random variable throughout this chapter.  Let $\epsilon_l$ be the percentage of time we will allow the constraint to be violated for each line $l$, so that the following constraint 
\begin{equation}\label{eq_cc}
 \bP{\Omega}{  -U_ \leq  \ry_e \leq U } \geq 1- \epsilon_l \hspace{10px} \forall e \in \cE
\end{equation}
represents the chance constraint version of the nominal line capacity limit.  Assuming $\ry_e$ is a Gaussian random variable with mean $\mu^y_e$ and standard deviation $\sigma^y_e$. Given a line risk preference $\epsilon_l$, let $\eta_l = \Phi^{-1}\left(1-\epsilon_l\right)$ where $\Phi^{-1} ( \cdot )$ is the inverse of the standard normal distribution. The following constraints
\begin{align*}
 \mu^y_e + \eta_l \sigma^y_e \leq U_e  \hspace{10px} \forall e \in \cE \\
 \mu^y_e - \eta_l \sigma^y_e \geq -U_e  \hspace{10px} \forall e \in \cE
\end{align*}
are deterministic constraints for \ref{eq_cc}

In addition to chance constraints on branch flows, the output of generators may be uncertain.  Let $\rD$ represent the aggregate deviation from forecast that the generators must respond to.   The slack distribution is a subset of generators who are able to respond to deviations in forecast in a fast time scale.  They typically follow a signal such as the area control error (ACE).  Let $\beta_j$ define this slack distribution, where $\beta_j$ is the portion of $\rD$ that generator $j$ compensates for.  We know that $\sum_{j \in \cJ} \beta_j =1$ for demand to be satisfied exactly.  Additional constraints can be put on the slack variables $\beta$ depending on generator characteristics and their ability to respond to variations in load.  The total output for generator $j$ will be $x_j + \beta_j \rD $
and we will need to enforce its constraints probabilistically since it is a random variable.  Assuming that the individual variations in wind and load are Gaussian, their aggregate will also be Gaussian.   Let $\epsilon_j$ be the probability that the constraint can be violated for generator $j$ and we would like a constraint to enforce
\begin{equation}\label{eq_cc_gen}
\bP{\Omega}{ G^{min}_e \leq x_j + \beta_j \rD  \leq G^{max}_e } \geq 1 - \epsilon_j \hspace{10px} \forall j \in \cJ 
\end{equation}
where $G^{min}_e, G^{max}_e$ represent the minimum and maximum output of the generator over the given time interval and accounts for ramping constraints and already made commitment decisions. We can assume that $\E{\Omega}{\rD} = 0$, otherwise we can shift the forecasted system to account for the mean deviation.  Additionally, assume the standard deviation of $\rD$ is known and takes the value $\sigma_\Delta$.  The following constraints to enforce \cref{eq_cc_gen} are
\begin{align*}
x_j + \beta_j \sigma_\Delta \eta_g &\leq G^{max}_j \hspace{10px} \forall j \in \cJ \\
x_j - \beta_j \sigma_\Delta \eta_g &\geq G^{min}_j \hspace{10px} \forall j \in \cJ 
\end{align*}

By allowing the slack distribution to be variables in the optimization procedure, the cost to meet load is now a random variable.  Assuming we would like to minimize the expected cost of generation, we would like to minimize
\begin{equation*}
\E{\Omega}{ \sum_{j \in \cJ} \left[ c^j_2 (x_j+\beta_j \rD)^2 + c^j_1 (x_j + \beta_j \rD) + c_0^j \right] }
\end{equation*}
Since $\rD$ is a Gaussian and $\E{\Omega}{\rD} = 0$, we have that $\E{\Omega}{\rD} = \sigma_\Delta^2$.  Our new objective will now be
\begin{equation*}
\sum_{j \in \cJ} \left[  c_2^j \left(x_j^2 + \sD \beta_j^2 \right) + c^j_1 x_j + c^j_0 \right]
\end{equation*}

We defer derivation of the standard deviation of branch flow, $\sigma_e$,  \cref{cc_pi,cc_var} in the following optimization problem, to the following section with a complete discussion on random power flows assuming a multivariate Gaussian distribution on net injects.  The following optimization problems highlight the difference between OPF hard line thresholds for branch flow limits and generator constraints and the probabilistic versions in the CC version.  The CC models make use of both mean and standard deviation for the random variables in question.  The parameters $\epsilon_l$ and $\epsilon_g$ can be changed to reflect the preference or aversion of a constraint being violated.  Of primary note is that with $\epsilon_l=\epsilon_g=.5$ the standard OPF model is recovered.  As these epsilon parameters are reduced, constraints are tightened so that the optimal value of OPF \cref{opf_program} will be less than or equal to the optimal value of CC \cref{cc_program}.
\begin{subequations}
\label{cc_program}
\begin{alignat}{3}
\textbf{CC:= }\min_{\left(x,\beta,;\theta,y\right)} && \multispan{2}{$\displaystyle\sum_{j \in \cJ} \left[  c_2 \left(x_j^2 + \beta_j^2 \sD \right) + c_1 x_j + c_0 \right]$}  & \label{cc_obj}\\
                        &&  \sum_{j \in \cJ} c^g_{ij} x_j - \sum_{j \in \cJ} c^b_{ie} y_e          &=d_i       && \forall i \in \cI \label{cc_cons}\\ 
                 && y_e - b_e  \sum_{i \in \cI} c^b_{ie} \theta_i          &=0         && \forall e \in \cE \label{cc_kcl}\\
                 && \mu^y_e + \eta_l \sigma^y_e &\leq U && \forall e  \in \cE\\
                 &&        \mu^y_e - \eta_l \sigma^y_e &\geq -U && \forall e \in \cE\\
                 && x_j + \beta_j \sigma_\Delta \eta_g &\leq G^{max}_j   && \forall j  \in \cJ  \\
                 && x_j - \beta_j \sigma_\Delta \eta_g &\geq G^{min}_j && \forall j   \in \cJ \\
                 && \pe - \sum_{j \in \cJ} A_{ej} \beta_j   &=0 &&\forall e \in \cE \label{cc_pi}\\ 
                 && s^2_e - \pi_e^2 \sD + 2 \pi_e \se      &\geq\see &&\forall e  \in \cE \label{cc_var}\\
                 &&  \sum_{j \in \cJ} \beta_j &=1 && \label{cc_slack}
\end{alignat}
\end{subequations}

%%%%%%%% Section 4
\section{Joint Chance Constraint for System Risk Measure} \label{jointchanceconstraint}
In order to ensure reliability of the bulk power system, the system risk needs to be quantified and constrained. In this chapter, we define system risk as the probability that no lines fail.  We are primarily concerned with what we term endogeneous system risk which is the likelihood of failure induced by the branch flows.  We start with our system risk analysis under a fixed generation and demand scenario.  We can solve this problem exactly with nonlinear programming.  Following, we will perform the analysis when the demand and generation are a multivariate Gaussian distribution in which the mean and covariance are known.  In this system we make a linearization to approximate our system risk function.   Let $h(y) : \R{M}_+ \rightarrow \left[ 0, 1 \right]$ be the risk function dependent on line flows $y$, let $\epsilon$ be the risk tolerance and define the system risk constraint as
\begin{equation}\label{systemriskmeasure}
h(y) = P_\Xi \left[ \mbox{no line fails} | \mbox{line flows } y\right] \leq \epsilon
\end{equation}
where the probability space $\Xi$ is orthogonal to that of demand $\bD$ and can be thought of as an effective capacity.  That is, the line fails if it has flow above its effective capacity.  The space $\Xi$ represents the transmission elements' effective capacity distribution.  
This risk measure \cref{systemriskmeasure} defines a Bernoulli random variable that takes the value 1 with probability $h(y)$ if no lines fail, the complement of all lines succeeding.  Assuming that the failure probabilities of individual lines are independent given the flow, we can find $h(y)$ by multiplying the probabilities that individual lines succeed $(1-g_e(y_e))$, we have the probability that all lines succeed
\begin{equation}  \label{nofail}
h(y) = \prod_{e \in \cE} \left( 1 - g_e(y_e) \right)
\end{equation}  
where $g_e(y_e)$ is the probability that line $e$ fails given line flow $y_e$.   The line risk function $g : \R{+} \rightarrow [0,1]$ takes the line flow $\hy_e$ and returns the probability $g_e(y_e)$ that the line will fail.   
\begin{equation}
 g_e(\hy_e) = \bP{\Xi}{\text{Line }e\text{ fails} | \hy_e}  \hspace{15px} \forall e \in \cE
\end{equation}
We make note that the probability a line fails is only dependent on the flow $y_e$.  While the flow $y_e$ certainly covaries with the other line flows $y$, we assume the failure process is independent of other line flows.

\begin{figure}
 \centering                    
   \includegraphics[scale=1.35]{\mypathjcc/fig-linefail}   
   \caption{Line failure density function for normalized line flow $y'$} \label{fig:linefaildensity}
\end{figure}


A piecewise linear function captures the important features of the endogenous risk of the transmission element associated with loading.  Up to a certain point, $L_e$, the power flow does not add any risk above that of its normal outage rate.  Above that point, the additional risk is proportional to the additional flow on the line.  Once a critical capacity $U_e^c$ is reached, a protective element is tripped and the line fails with certainty (ignoring hidden failures).  
\begin{equation}\label{pwl_risk}
g_e(\hy_e) = \left\{ \begin{array}{l l}
  0 & \hy_e \leq L_e \\
  a_e + b_e \hy_e & L_e \leq \hy_e < U_e^c \\
  1 & U_e^c \leq \hy_e 
\end{array}
\right.
\end{equation}
Using as reference the probability $p_e$ that the line fails at nominal capacity $U_e$, the piecewise linear paramters $a_e$ and $b_e$ can be calculated as $a_e = -p_eL_e (1-L_e)^{-1}$ and  $b_e = p_e(1-L_e)^{-1}$.  This piecewise linear failure density function is shown in \cref{fig:linefaildensity}.







%Also, we restrict $z_e \in [0,1)$ since when $z_e = 1$ for any line, we know that the probability no line fails is 0.  
%This implies that an individual line has a hard line limit (denoted $U^\epsilon$) related to the system risk $\epsilon$.

\begin{lemma}
$h(y)$ is log-concave in domain $\left\{ y_e \in \R{\cE} | 0 \leq y_e < U^c_e \hspace{5px} \forall e \in \cE \right\}$
\end{lemma}
\begin{proof}
  First we note that for $\left\{ y_e | y_e < U^c_e \forall e \right\}$,  $g_e(y_e) \in \left[ 0, 1\right)$ and is the max of two convex functions and is thus convex. To show that $\ln h(y)$ is concave, we  take the log of both sides of \cref{nofail}, giving
\begin{align*}
\ln  h(y) & = \ln \left[ \prod_{e \in \cE} \left( 1 - g_e(y_e) \right) \right] \\
&= \sum_e \ln \left[ \left( 1 - g_e(y_e) \right)\right]
\end{align*}
For $x_e = g_e(y_e)$, let $H(x) = \ln h(x) = \sum_e \ln \left[ 1- x_e \right]$, we have
\begin{align*}
\frac{\partial H(x)}{\partial x_e} &=  -\frac{1}{1-x_e}  \\
\frac{\partial^2 H(x)}{\partial x_n^2} &=  - \frac{1}{(1-x_e)^2} \\
\frac{\partial^2 H(x)}{\partial x_n \partial x_e} &=  0 
\end{align*}
which shows that $H(x)$ is a decreasing and concave function on $\left\{x_e | x_e \in \left[0,1\right) \forall e \right\}$, thus $-H(x)$ is convex and non-decreasing.  Composing a convex, non-decreasing function with a convex function is convex, so that $-\ln h(y)$ is convex and $\ln h(y)$ is concave.  Thus, $h(y)$ is log concave. \qed
\end{proof}


This function can be solved exactly for fixed demand using nonlinear programming by taking a log transform of the line risk.  The system risk constraint \cref{nofail} can be written
\begin{equation*}  \prod_{e \in \cE} \left( 1 - g_e(y_e) \right) \geq 1 - \epsilon
  \end{equation*}  
where $1 - g_e(y_e)$ is the probability that line $e$ doesn't fail given flow $y_e$.  Taking the product of all these events gives the probability that no lines fail, we have
\begin{align*}
  \ln \left( \prod_{e \in \cE}\left( 1 - g_e(y_e)\right) \right) &\geq \ln \left(  1 - \epsilon \right)  & g_e(y_e) \in [0,1) \forall e  \in \cE & \Leftrightarrow \\
  \sum_{e \in \cE} \ln \left( 1-g_e(y_e) \right)  &\geq \ln \left( 1- \epsilon \right) & g_e(y_e) \in [0,1) \forall e  \in \cE. &
\end{align*}
If we enforce $w_e \leq \ln (1-g_e(y_e)) \forall e \in \cE$, then we have our no line failure system risk constraint \cref{nofail} using the following constraint
\begin{equation}\label{cceq}
\sum_{e \in \cE} w_e  \geq \ln \left( 1-\epsilon \right)
\end{equation}
To enforce $w_e \leq \ln \left( 1 -g_e(y_e) \right)$ we manipulate to
\begin{align*}
w_e &\leq \ln \left( 1 - g_e(y_e)\right)  & \forall e \in \cE \\
\exp(w_e) &\leq 1-g_e(y_e)  & \forall e \in \cE \\
g_e(x_e) + \exp(w_e) &\leq 1  & \forall e \in \cE .
\end{align*}
Equation (\ref{cceq}) implies that there are fixed line limits related to the risk tolerance $\epsilon$.  Since $\ln \left(1 - x \right) \leq 0 $  for all $x \in [0,1)$ shown in \cref{fig:linerisksub}, this immplies that each term in the sum also satisfies the constraint  $\ln \left(1-g_e(y_e)\right) \geq \ln \left( 1-\epsilon \right)$, which implies that the upper limit on flow $y_e$  defined by
\begin{equation}
U_e^\epsilon = g_e^{-1}(y_e) \hspace{15px} \forall e \in \cE
\end{equation}


\begin{figure}
\begin{center} 
\input{\mypathjcc/fig-riskconst}
\end{center}
\caption{Line risk substitution for individual branches and possible approximations}\label{fig:linerisksub}
\end{figure}




\subsection{System Risk for Multivariate Gaussian Branch Flows}
Applying the line risk function \cref{pwl_risk} to the system risk measure \cref{nofail} and taking the expectation over the random injection space $\bD$, we have
\begin{align}\label{nofailuncertainty}
  \E{\bD}{h(\ry)}  &= \E{\bD}{P_\Xi \left[ \mbox{no lines fail} | \mbox{line flows } \ry\right]}\\
  &= \E{\bD}{ \prod_{e \in \cE} (1 - g_e(\hry_e) )}
\end{align}
since $ 1- g_e(\ry_e) = \bP{\Xi}{\text{Line }e\text{ succeeds} | \ry_e} $ and multiplying over all lines to find the probability that no line fails. The function \cref{nofailuncertainty} can be approximated by using a linearization of $h(\ry)$.  Let $r$ be the expectated value of the probability that one or more lines fail (the complement of $\E{\bD}{h(\ry)}$). Then, we have
\begin{align*}\label{linear}
r &= \E{\bD}{1 - h(\ry)}  \\
 &= 1 - \E{\bD}{\prod_{e \in \cE}\left( 1 - g_e(\hry_e)\right) } \\
  &= 1 - \E{\bD}{ 1 - \sum_e g_e(\ry_e) + \sum_{e_1,e_2|e_1\neq e_2} g_{e_1} (\ry_{e_1}) g_{e_2}(\ry_{e_2}) - \sum_{e_1,e_2,e_3|e_1\neq e_2 \neq e_3} g_{e_1} (\ry_{e_1}) g_{e_2}(\ry_{e_2}) g_{e_3} (\ry_{e_3}) + \cdots } \\
  & \approx 1 - \E{\bD}{ 1 - \sum_{e} g_e(\ry_e) + \frac{1}{2} \sum_{e_1,e_2|e_1\neq e_2} g_{e_1} (\ry_{e_1}) g_{e_2}(\ry_{e_2}) - \cdots } \\
 &\approx \sum_{e \in \cE} \E{\bD}{g_e(\hry_e)}
\end{align*}
where the approximation derived above comes from taking a taylor expansions of $h(\cdot)$ at $y=0$ and second order and higher terms.  This approximation is good for small risk values of $h(y)$.
%It is important to note that the linear approximation underestimates system risk and the quadratic approximation overestimates risk for small risk $h(y)$. The risk can also be bounded using the Bonferroni Inequalities, which is a generalization of Boole's Inequality, or the union bound. 
The system risk contributed by line $e$ is integrated over the space $\bD$, which is orthogonal to effective capacity, or likelihood of failure due to loading, $\Xi$.  This system risk measure $r$ captures the endogenous system risk of line failures due to loading under uncertainty in branch flows.

% -----  Picture ----- Gaussian PDF and line failure function ---------------------------------%
\begin{figure}
 \centering                    
  \begin{subfigure}[b]{0.4\textwidth}
   \includegraphics{\mypathjcc/fig-pdfflow}   
   \caption{Probability density of power flow}
  \end{subfigure}
  \begin{subfigure}[b]{.4\textwidth}
    \includegraphics{\mypathjcc/fig-linefail}
    \caption{Line failure density function}
  \end{subfigure}
  \caption{Random Power Flows and the Failure Density Function}
\end{figure}


% -------------------- Linearization for Small Epsilon ENDNOTE -------------------
%To find the linearization for small $t$, the taylor expansion $f(t)=f(t) + \grad f (a)(t-a) +...$ around $a=0$ is useful.  Applying the taylor expansion to our risk measure \ref{nofail}, we end with the linear approximation


Let $ z_e = \E{\bD}{g_e(\ry_e)} $ be the individual line risk. We can break $z_e$ into three segments based upon the piecewise linear function \ref{pwl_risk}.  The first segment for $\hry_e \leq L$ takes the value zero, the second segment, for $L \leq \hry_e \leq U^c$, takes an expected value of a truncated normal distribution, and the last segment, $U^c \leq \hry_e$, takes one with a CDF evaluation of a normal distribution.  That is
\[ z_e = \E{\bD}{ a+b\hry_e | L \leq \hry_e \leq U^c } \bP{\bD}{L\leq \hry_e \leq U^c}  + \bP{\bD}{ U^c \leq \hry_e } \] 

 The truncation from the critical capacity $U^c$, the level at which the line fails with certainty, is approximately 0 for choices of $\epsilon$ around a few percent.  This approximation can be checked during the iterative procedure and depends on the value of $ (U^c - U^{\epsilon} )/\sigma $, where $U^\epsilon$ is the level at which line risk is equal to the system risk tolerance.  Then, we have
\[ z_e = \E{\bD}{ a+b\hry_e | L \leq \hry_e } \bP{\bD}{L\leq \hry_e} \]
The branch flows are truncated Gaussian, which have known mean and variances. Truncated Gaussian's expectation and tail probability are
\[  \E{\bD}{ \hry_e | L \leq \hry_e } = \mu^y_e + \frac{\phi( \alpha_L )}{ 1- \Phi(\alpha_L)} \sigma^y_e \]
\[ \bP{\bD}{ L \leq \hry_e } = 1 - \Phi(\alpha_L) \]
where $\alpha_L = \frac{L - \mu^y_e}{\sigma^y_e}$, the PDF $\phi(\cdot)$ , and the CDF $\Phi(\cdot)$ are for the standard normal distribution.  The line risk is a function of the mean and covariance of the branch flows, given as
\begin{equation}\label{line_risk}
z_e(\mu^y_e,\sigma^y_e) = (a + b \mu^y_e)\left[ 1 - \Phi(\alpha_L) \right]  + b \sigma^y_e \phi(\alpha_L) 
\end{equation}







%%%%%%%% Section 3
\subsection{Joint Chance Constraint Model}%WITH ARBITRARY SLACK!!
The JCC model departs from traditional power flow models in that it allows for a trade-off between line risk, system risk, and cost. Since the cost and risk are both related to the slack distribution, the model is given with the slack distribution as a variable.  In this section, the full joint chance constraint model follows with a brief explanation of some of the constraints.  The relevant parameters derived from the injection covariance matrix is also given.  Then a cutting plane algorithm is shown to solve the full JCC model.
\begin{subequations}
\label{jcc_program}
\begin{alignat}{3}
\textbf{JCC:= }\min_{\left(x,\beta,;\theta,y,\pi,s,z\right)} && \displaystyle\sum_{j \in \cJ} \left[  c_2 \left(x_j^2 + \beta_j^2 \sD \right) + c_1 x_j + c_0 \right]  && \label{jcc_obj}\\
                        &&  \sum_{j \in \cJ} c^g_{ij} x_j - \sum_{j \in \cJ} c^b_{ie} y_e          &=d_i       && \forall i \in \cI \label{jcc_cons}\\ 
                 && y_e - b_e  \sum_{i \in \cI} c^b_{ie} \theta_i          &=0         && \forall e \in \cE \label{jcc_kcl}\\
                 && y_e &\in \left[ -U_e^\epsilon, U^\epsilon_e \right] && \forall e \in \cE \label{jcc_limit}\\
                 && x_j + \beta_j \sigma_\Delta \eta_g &\leq G^{max}_j   && \forall j \in \cJ \label{jcc_gen1} \\
                 && x_j - \beta_j \sigma_\Delta \eta_g &\geq G^{min}_j && \forall j  \in \cJ \label{jcc_gen2}\\ 
                 &&  \sum_{j \in \cJ} \beta_j &=1 && \label{jcc_slack}\\
                 && \pe -  \sum_{j \in \cJ} A_{ej} \beta_j   &=0 &&\forall e \in \cE \label{jcc_pi}\\ 
                 && s^2_e - \pi_e^2 \sD + 2 \pi_e \se      &\geq\see &&\forall e \in \cE \label{jcc_var}\\
                 && z_e - g_e(|y_e|,s_e)  &\geq 0 && \forall e \in \cE \label{jcc_lr}\\
                 &&  \sum_e z_e &\leq \epsilon && \label{jcc_risk}
\end{alignat}
\end{subequations}


The objective \ref{jcc_obj} for the JCC model is the typical quadratic objective for the OPF model plus a contribution from the slack distribution due to the uncertainty in the random injects.  This objective is the expected cost of meeting the realized demand.  We use a chance constraint to deal with the uncertainty in demand for \cref{jcc_gen1,jcc_gen2} with $\eta_g = \Phi^{-1}(1-\epsilon_g)$ being its tolerance.  The upper and lower bounds $[G^{min},G^{max}]$ is dependent on many things, most importantly the time frame used.  This is affected by the status of the generator (on,off,starting up,shutting down) and its ramping rate as well as any physical limits it may have.  

The first equations \cref{jcc_obj,jcc_cons,jcc_kcl,jcc_slack,jcc_limit} are your typical DC power flow, \cref{jcc_gen1,jcc_gen2} are chance constraints on generators, and the last set \cref{jcc_pi,jcc_var,jcc_lr,jcc_risk} describe system risk.  The branch variance equations \ref{jcc_var} are a second order cone and the line risk equations \ref{jcc_lr} involve the CDF of a normal distribution.  These equations are solved via a cutting plane approach so that the individual subproblems are linear programs.


\subsection{Solution Methodology}\label{solutionmethodology}
Due to the inclusion of non-analytic functions (the line risk function), a commerical solver would not work.  Since the risk function is convex, we choose cutting planes to approximate the line risk constraints \ref{jcc_lr}.  In addition, there were many second order cone constraints (number of lines).  Instead of using these constraints explicitly, they were added through cutting planes as well.  This kept the program to a manageable size and allowed for fast solve times.  When exogenous contingencies are added, cutting planes are extremely efficient since no constraints are added for lines and scenarios without risk.

\begin{lemma}
The line risk function \ref{line_risk} is convex with respect to $\mu^y_e$ and $\sigma^y_e$
\end{lemma}
\begin{proof}
Starting with the line risk function \ref{line_risk}
\begin{equation*}
z_e(\mu^y_e,\sigma^y_e) = (a + b \mu^y_e)\left[ 1 - \Phi(\alpha_L) \right]  + b \sigma^y_e \phi(\alpha_L) 
\end{equation*}
First, we calculate $\frac{\partial \Phi(\alpha_L)}{\partial \mu^y_e} = - \frac{1}{\sigma^y_e} \phi(\alpha_L)$ and $\frac{\partial \phi(\alpha_L)}{\partial \mu^y_e} =  \frac{1}{\sigma^y_e} \alpha_L \phi(\alpha_L)$ using the chain rule.  Now note that from our piecewise linear function \ref{pwl_risk}, we have the identity $L = -\frac{a}{b}$. Then, taking the derivative of $z$ with respect to $\mu^y_e$, we get  
\begin{align*}
\frac{\partial z_e}{\partial \mu}(\mu^y_e,\sigma^y_e) &= (a + b \mu^y_e) \frac{1}{\sigma^y_e} \phi(\alpha_L) + b \left[ 1 - \Phi(\alpha_L) \right] + b \alpha_L \phi(\alpha_L)\\
&= \frac{a + b \mu^y_e}{\sigma^y_e}\phi(\alpha_L)  + b \left[ 1 - \Phi(\alpha_L) \right] +  \frac{bL - b\mu^y_e}{\sigma^y_e}\phi(\alpha_L) \\
&= \frac{a + b \mu^y_e}{\sigma^y_e}\phi(\alpha_L)  + b \left[ 1 - \Phi(\alpha_L) \right] -  \frac{a + b\mu^y_e}{\sigma^y_e}\phi(\alpha_L)\\
& = b\left[ 1 - \Phi(\alpha_L) \right]
\end{align*}
For our derivative with respect to $\sigma^y_e$, we calculate $\frac{\partial \Phi(\alpha_L)}{\partial \sigma^y_e} = - \frac{1}{\sigma^y_e} \alpha_L \phi(\alpha_L)$ and $\frac{\partial \phi(\alpha_L)}{\partial \sigma^y_e} =  \frac{1}{\sigma^y_e} \alpha_L^2 \phi(\alpha_L)$.  Then, taking the derivative of $z$ with respect to $\sigma^y_e$, we have
\begin{align*}
\frac{\partial z_e}{\partial \sigma}(\mu^y_e,\sigma^y_e) & = (a+b\mu^y_e)\frac{1}{\sigma^y_e} \alpha_L  \phi(\alpha_L) + b \sigma^y_e \frac{1}{\sigma^y_e} \alpha_L^2  \phi(\alpha_L) + b \phi(\alpha_L)\\
& = -b\frac{L-\mu^y_e}{\sigma^y_e} \alpha_L  \phi(\alpha_L) + b \alpha_L^2  \phi(\alpha_L) + b \phi(\alpha_L)\\
 & = b \phi(\alpha_L) 
\end{align*} 
where the identities $a=-bL$ and $\alpha_L = \frac{L-\mu^y_e}{\sigma^y_e}$ were used.

The Hessian is found with the second derivatives and is given by
\begin{equation}
\cH_z(\mu^y_e,\sigma^y_e) = \frac{b \phi(\alpha_L)}{\sigma}
\left[ 
\begin{array}{c c}
1 & \alpha_L \\
\alpha_L & \alpha_L^2
\end{array}
\right]
\end{equation}
Then, we note that the determinant is 0 and the diagonal elements are positive so that there is a positive eigenvalue and a zero eigenvalue, thus the line risk is convex with respect to $\mu^y_e, \sigma^y_e$.  Since system risk is approximated by the sum of line risks, this system risk measure is convex. \qed
\end{proof}

The line risk constraint \ref{jcc_lr} involving the CDF function for a normal distribution which require look up tables. Since the equations are convex, we can solve this using a cutting plane approach to describe the line risk in terms of the mean branch flow $y_e$ and the standard deviation of branch flow $s_e$.  While there are infinitely many cuts, this program can be solved for a given error tolerance with a finite, and typically small, set of cuts.
\begin{subequations}
\label{line_risk_cuts}
\begin{align}
z_e \geq g_e(\hye,\hse) &+ \frac{\partial g}{\partial y_e}(\hye,\hse) \left(y_e - \hye \right) 
+ \frac{\partial g}{\partial s_e}(\hye,\hse) \left(s_e - \hse \right) \\
g_e(y_e,s_e) &= (a + b y_e)\left[ 1 - \Phi(\alpha_L) \right]  + b s_e \phi(\alpha_L)  \\
 \frac{\partial}{\partial y_e}g_e(y_e,s_e) &= b\left[ 1 - \Phi(\alpha_L) \right]\\
\frac{\partial}{\partial s_e}g_e(y_e,s_e) &= b \phi(\alpha_L) 
\end{align}
\end{subequations}

The standard deviation of branch flow can be formulated as a second order cone with respect to the slack distribution variables.  While this could be solved with a commercial solver, we proceed with a cutting plane approach to speed up solve times.  As the solution approach is already iteratively improving system risk, it adds only a small amount of work to add these cuts as well and reduces the subproblem to a linear program.
\begin{subequations}
\label{branch_var_cuts}
\begin{align}
s_e \geq \fehb &+ \sum_{j \in \cJ} \pfehb \left( \beta_j - \hat{\beta_j} \right)\\
  \feb &= \sqrt{\pe^2 \sD - 2 \pe \se  + \see }\\
  \pfeb &= \frac{A_{ej} \left( \pe \sD - \se \right)}{\sqrt{\pe^2 \sD - 2 \pe \se  + \see }}
\end{align}
\end{subequations}




\subsubsection*{Cutting Plane Algorithm}
Now, we can describe the cutting plane algorithm for JCC at a high level and give pseudo-code \ref{jcc_alg} for implementation.  The main subproblem the algorithm solves is the standard DC power flow.  After solving, the generator injects $x$, slack distribution $\beta$, and branch flows $y$ are used to calculate risk information about the dispatch point.  To get the risk information, the branch standard deviations $s$ need to be calculated.  With the mean flow $y$ and standard deviation $s$, the line risk $z$ can be calculated.  The sum of line risk is system risk $r$.  If $r$ is less than the required system risk $\epsilon$, the problem is solved.  Otherwise, the algorithm adds cuts for all lines with a positive risk $z$.  The cuts describe how $z$ is related to $y,s$ and how $s$ is related to $\beta$.  Then the power flow subproblem is solved with the addition of the cuts and this repeats until it is infeasible or the risk constraint is satisfied.
\begin{algorithm}
\caption[Cutting plane algorithm for joint chance constraint model]{This cutting plane algorithm solves JCC \ref{jcc_program} via linear programs and cutting planes}\label{jcc_alg}
\begin{algorithmic}
\Procedure{JCC}{d,$\Sigma^m$,$\epsilon$,$\epsilon_g$,L,p}
\State $L \gets \emptyset$  (Set of Lines with potential risk)
\State $S \gets \emptyset$  (Set of Cuts)
\State $r \gets 0$ (Risk)
\BState \emph{solve}:
\State $(\hat{x},\hat{\beta},\hat{y}) \gets $Solve DC Power Flow, \cref{jcc_obj,jcc_cons,jcc_kcl,jcc_slack,jcc_limit,jcc_gen1,jcc_gen2}, with cuts $S$, risk $r\leq\epsilon$
\If {Infeasible} \Return Problem Infeasible 
\EndIf
\State Calculate $\hat{s},\hat{z},\hat{r}$ using $(\hat{x},\hat{\beta},\hat{y})$ and \cref{branch_cov,line_risk}
\If {$\hat{r} \leq \epsilon + tol$} \Return Optimal $(\hat{x},\hat{\beta},\hat{y},\hat{s},\hat{z},\hat{r})$
\EndIf
\For{$\forall e$}
\If {$\hat{z_e} \geq tol$}
    \If {$e \notin L$}
            \State $L \gets \left\{L,e\right\}$
            \State Initialize $s_e,z_e$
            \State $r \gets r + z_e$
    \EndIf            
    \State $S \gets$ line risk cuts \ref{line_risk_cuts} for $z_e,y_e,s_e$ dependent on $\hat{z}_e,\hat{y}_e,\hat{s}_e$
    \State $S \gets$ branch variance cuts \ref{branch_var_cuts} for $s_e,\beta_e$ dependent on $\hat{s}_e,\hat{\beta}_e$
\EndIf
\EndFor
\State \textbf{goto} \emph{solve}
\EndProcedure
\end{algorithmic}
\end{algorithm}





%%%%%%%% Section 5
\section{Computational Experiments}\label{computationalresults}
This section explains the computation setup, the test cases, the experiments performed, and the intuition to gain from the comparison of OPF, CC, and JCC models.
\subsection{Implementation of the JCC Model}
All of the experiments are run on a laptop with an Intel i7-3537U processor with 2 cores,4 threads at 2.00GHz.  The laptop has 8GiB of memory, but even with the large case files (2383 nodes), memory is not an issue.  The laptop is running Linux Mint Petra 16.  The OPF, CC, and JCC models are all solved within a C++ program.  The program uses Armadillo\cite{armadillo} for linear algebra and Concert and CPLEX to solve the linear programs.  CPLEX is run with default settings and the dual simplex algorithm is used to solve the linear programs (quadratic objective for the small case).  All time comparisons are using the same system environment with a fixed clocked speed and few programs in background.  The primary DC OPF solver in the C++ program has been developed to output nearly identical results to Matpower\cite{matpower}.
\subsection{Single Instance}
The test cases are all pulled from Matpower test cases.  The two test cases used are the 30 bus test case as well as the 2383wp test case.  The small test case is used to show some properties of the different dispatch points and the cost-risk frontier.  The large test case is used for time trials as well as a cost-risk scatter plot.  

\subsubsection*{30 Bus Case}
This case from Matpower has 30 buses, 41 branches, and 6 generators.  The branches have a single capacity rating and in the given demand scenario does not have active line constraints.  A capacity factor $M$ is used to uniformly scale the branch capacities.  All of the generators are active and have quadratic cost functions.  Ramping constraints for generators are not considered and the generators are allowed to take any value in its given range (the probabilistic generator constraints are ignored to focus on branch constraints).  In addition, all generators are allowed to participate in the slack distribution.  Each bus with a demand is considered random, with mean equal to the demand.  For this example, the injections are independent of each other but this need not be the case.  The variance of each injection is equal to 5\% of the demand at the node times a budget factor $B$.

\begin{table}
\centering
\begin{tabular}{ |c| c c c |}
\hline
& OPF & CC & JCC \\
\hline
\hline
Cost & 567.1 & 574.0 & 565.2\\
$r$ & 0.0171 & 0.0179 & 0.0080\\
\hline
\end{tabular}
\caption{Cost and risk results for OPF,CC, and JCC models on the small test case}\label{solve_results}
\end{table}

\begin{table}
\centering
\small
\begin{tabular}{| c| c |c c c c c c |}
\hline
 & Mod & g1 & g2 & g3 & g4 & g5 & g6 \\
\hline
\hline
$\left[ g_{min}, g_{max} \right]$& & [0,80]&[0,80] &[0,50] &[0,55] &[0,30] &[0,40]  \\
$\left\{ c_2, c_1 \right\}$ && $\left\{0.02,2\right\}$  &$\left\{0.0175,1.75\right\}$ &$\left\{0.0625,1\right\}$ &$\left\{0.00834,3.25\right\}$ &$\left\{0.025,3\right\}$ &$\left\{0.025,3\right\}$  \\
\hline
\hline
$x_g$ &OPF& 39.7  &  52.3  &  24.2  &  35.7  &  19  &  18.3   \\
$x_g$ &CC& 35.9  &  47.9  &  25.7  &  37.2  &  19.3  &  23.1    \\
$x_g$ &JCC& 41.9  &  55.0  &  23.2  &  34.0  &  18.6  &  16.5    \\
\hline
$\beta_g$ &OPF& 0.1548  &  0.1769  &  0.0495  &  0.3712  &  0.1238  &  0.1238    \\
$\beta_g$ &CC& 0  &  0  &  0.4411  &  0.2986  &  0  &  0.2602   \\
$\beta_g$ &JCC& 0.2456  &  0.2795  &  0.0846  &  0.0646  &  0.0597  &  0.2659   \\
\hline
\end{tabular}
\caption{Generator results using OPF, CC, and JCC models on the small test case.}\label{solve_one}
\end{table}

The first example has risk parameters $L=0.9$, that is a line  begins taking on risk after it is at 90\% of its rated capacity.  The factor $p=0.005$ means that at nominal capacity, the line has a $0.5\%$ chance of failing.  The line capacities are scaled by $M=0.745$. The system risk constraint is $\epsilon=0.008$, that is a $0.8\%$ chance of one or more lines failing.  The chance constraint version of OPF is defined by $\eta_L=0.05$, that is the line capacities can not be violated more than $5\%$ of the time.  Finally, the covariance matrix is scaled by $B=0.025$ so that the standard deviation of aggregate demand is $0.6$, a small fraction of total demand $189.2$.  The results for $(x,b)$ are tabulated in \cref{solve_results,solve_one}.  Increasing $B\rightarrow 0.25$ by 10 fold led to the standard deviation of aggregate demand to be $1.345$, around 1\% of total load.  The cost of OPF and JCC had negligible change, whereas CC cost increased to \$597, an increase of \$30 or an increase of 5\% over the OPF solution. When the OPF model has transmission congestion, the cost of the CC model is highly sensitive to the uncertainty in demand  and  often infeasible solutions.



The CC model is always more conservative than the OPF model as it tightens the line constraints.  This means that the CC version is always at least as expensive as the OPF model.  In this case, the total cost rose by \$7 (to \$574, or a 1.2 \% increase) to ensure the line constraints were met probabilistically (95\% of the time).  The JCC model was able to lower the cost because it removed the branch capacity constraints (or increased them by 6\% to match the system risk level).  In addition, JCC knows and constrains the system risk measure so that it is able to find a cheaper point with a system risk level half that of the OPF and CC models.



Another interesting feature is that the slack would like to be distributed as much as possible to reduce cost.  This can be seen from the objective \ref{jcc_obj} due to the squared beta term and the OPF results show it spreading when there is no concerns of line or system risk.  The CC model slightly changes its generator position as well as removing 3 generators from the slack distribution.  It is removing generators which has an effect on the probabilistically constrained transmission lines in order to ensure that the constraints are met $95\%$ of the time.  The JCC model has moved in almost exactly the opposite direction and has kept a distributed slack (as seen in \cref{solve_one}).




\subsubsection*{Cost-Risk Frontier}
  \begin{figure} % redo with new operating models
\centering
\includegraphics{\mypathjcc/fig-costrisk}
\caption{Reliability frontier for the small test case}\label{costriskfront}
\end{figure}
%pow case/30.db  .775 1.1 1 .00195 .4 .00275 .5 .9 .005 .4 1000 2> cost-risk.dat
%m0   m1          mg   eps epsN     pL pG     L  p  B    T
%0.775            1.1  1   0.00195  0.4       0.00275    0.5     0.9     0.005   0.4     1000
Now we use this same example case to draw a cost-risk frontier.  An efficient operating point would be at that boundary of the feasible points, that is, neither cost nor risk can be improved without a loss to the other.  Since neither the OPF or CC model uses or constrains our risk measure, it would be unlikely for them to be on the boundary.  In figure \ref{costriskfront} we see that this is the case.  The OPF model is in the interior and is equivalent to the CC model when $\eta_l=.5$.  Since the branch flows are Gaussian, if the mean flow is at its threshold, it has a 50\% chance of being over its threshold.  As $\eta_l$ is reduced, the CC line is drawn.  The JCC line is started by finding the point when the system risk level is not constrained, for this case $r=.01$.  As $r$ is decreased, the system risk becomes constrained and the cost begins to rise as the risk level is reduced.  We used parameters $M=0.775$ for line capacity scaling and $B=0.4$ with standard deviation of aggregate load being around $1\%$ of total demand.  The left plot uses $L=.9$ for calculating the system risk measure whereas the right plot uses $L=.99$ to calculate the system risk measure.  In practice, it was found that as $L$ gets closer to 1, the model behaves more like the CC model.


It is important to note that the cost-risk frontier is entirely dependent on how system risk is measured.  In our case, we use a piecewise linear failure density function with parameters $L$ and $p$.  As $L$ and $p$ are varied, the shape of these frontiers changes.  Holding $p$ fixed and increases $L$ towards 1, the CC model, while still on the interior of the frontier, does a better job of reducing system risk.  With $L$ around .98 (this depends on the variance of aggregated injects), the CC and the JCC behave similar.  Both programs try to reduce the flow of lines that are at their capacity, with the exception that JCC allows a small number, typically one, of these lines to increase.  More discussion on this behavior is given in the sensitivity analysis section \ref{senseanal}


\subsubsection*{2383wp Bus Case}
This case from Matpower has 2383 buses, 2896 branches, and 327 generators.  The case is similar to the small case in many respects, such as only a single line rating capacity being given.  The generators have the biggest difference in that there are many which are nearly fixed and have no cost.  The larger flexible generators are only given a linear cost so that there is no quadratic objective.  The covariance matrix is developed the same as in the small case.

Running this case with parameters $M=1.03, \epsilon=0.03, \eta_L=0.05, L=0.85, p=0.005$, and $B=1$ and recording the standard cost and risk information as well as solve time.  The case was repeated 10 times and the mean time is reported in \ref{solve_time}.  CC and JCC take a similar amount of time, the majority of the work at each step being the calculation of the branch covariance matrix.  The total time is largely dependent on how many iterations the algorithms take, which is typically around 5-7. 

In addition, we solved 100 instances with random demand and the same covariance matrix to show the speed-up you can achieve with repeated solves.  By having the same covariance matrix, all of the cuts for previous solves are still valid.  So, in addition to using a warm start on the LP, after the first few solves the algorithm typically needs no additional cuts.  In practice, this could mean that the important lines and scenarios from the risk perspective are known before hand and cuts are added based on the assumed covariance matrix.  Instead of taking up to 5-7 iterations, it may solve in only one and perform a simple check to ensure that any lines not included in the analysis are not violated.

\begin{table}
\centering
\begin{tabular}{| c| c c c| }
\hline
Time ($\eta_L,\epsilon$) & OPF & CC & JCC \\
\hline
\hline
Time (0.05,0.03)& 0.37  & 8.7 & 7.5 \\
%Time (0.10,0.05)& 0.37 & 8.6  & 7.4  \\
Time (0.20,0.07)& 0.37  & 8.4 & 8.0 \\
%Time (0.30,0.10)& 0.37  & 8.4 & 7.6 \\
Time (0.40,0.20)& 0.37  & 8.3 & 6.1 \\
%Time (0.45,0.30)& 0.37  & 8.3 & 6.1 \\
Time (0.48,0.30)& 0.37  & 7.7 & 6.1 \\
\hline
\hline
Avg Time (100 trials)& 0.34  & 2.2 & 2.3 \\
\hline
\end{tabular}
\caption{Time comparison, in seconds, for OPF,CC, and JCC on the large test instances.}\label{solve_time}
\end{table}

\subsubsection*{Line Threshold Comparison}
Here we explore what the JCC model is doing to lower its cost or risk compared to the traditional models.  For this experiment, the parameters are as follows, $M=1.03, \epsilon=0.015, \eta_L=0.05, L=0.95, p=0.005$, and $B=1$.  Solving this program, we see that the majority of the lines are less than 50\% utilized.  There are a small number of lines that are at or near their nominal capacity.


\begin{figure}
\centering
\begin{subfigure}[b]{0.4\textwidth}
\includegraphics{\mypathjcc/fig-normflow}
\caption{All lines that have mean flow (in OPF) above 95\% of nominal capacity}\label{solve_shadow}
\end{subfigure}
\hspace{15px}
\begin{subfigure}[b]{0.4\textwidth}
\centering
\begin{tikzpicture}[scale=.8]
\begin{axis}[title=Histogram of Line Flow for 1000 realizations, 
    xlabel=Line Flow Bin, 
    ylabel=Number of Lines,
x tick label style={
/pgf/number format/1000 sep=},
  ybar, bar width=5pt ,
  enlargelimits=0.1]
\addplot coordinates { ( 0.9,4450)  ( 0.925,1729)  ( 0.95,1949)  ( 0.975,4471)  ( 1,1770)  ( 1.025,3) };
\addplot coordinates { ( 0.9,4219)  ( 0.925,1678)  ( 0.95,1834)  ( 0.975,5077)  ( 1,53) };
\addplot coordinates { ( 0.9,5931)  ( 0.925,2218)  ( 0.95,711)  ( 0.975,70)  ( 1,784)  ( 1.025,221) };
\legend{opf,cc,jcc}
\end{axis}
\end{tikzpicture}

\caption{Histogram to compare high flow lines of OPF,CC, and JCC}\label{histogram}
\end{subfigure}
\end{figure}

Now, we look at the trade-off that the JCC model is able to make due to it increasing the nominal line limit by 10\% to match the given system risk level.  In figure \ref{solve_shadow}, 8 lines are shown, which are lines above 95\% of their nominal capacity in the OPF model.  The mean line flows are plotted as well as the standard deviation based on the slack distribution for the solution.   There are 8 lines above 95\% of their capacity in the OPF model, indexed as lines 23,291,320,1380,1381,1815,2108, and 2109. In the OPF model, there are 3 lines at the nominal capacity.  Line 23 (index 1 in figure) has a dual price of 872, line 291 (index 2) has a dual price of 32, and line 2108 (index 8) has a dual price of 294.  Ideally, line 23 constraint should be relaxed due to the higher shadow price and perhaps other lines further constrained so that the system risk level is not increased. However, in the CC model, these constraints are tightened further to ensure that these thresholds are met at least 95\% of the time.  The JCC model on the other hand relaxes these constraints and imposes a system risk constraint instead.  The final solution for the JCC model has line 23 exceed its threshold with certainty but is rewarded with savings related to the high shadow price.  On the other hand, other lines with positive line risk are be restricted to ensure the system risk levels are met.

To give another perspective, the line flows were sampled and binned to create a histogram of line flows to show the trade-off between models.  For the JCC model, the line risk parameter $L$ was chosen to be 0.9.  The results can bee seen in figure \ref{histogram} where the bin size is $0.025$, for example the point $1$ is in reference to bin $1$ to $1.025$.  We can see that the normalized line flows pile up in bin 0.975 for the CC model and there are a number of lines in OPF that are in bin $1-1.025$, thus exceeding their capacity a significant amount of the time.  The JCC also has a line that exceeds its capacity, almost with certainty.  However, the JCC has far fewer lines that are at or near their capacity.



The primary strength of the CC model is to meet the line threshold constraints probabilistically when the demands are not known with certainty.  Let's look at how well it does from the system perspective, that is the expected number of lines to be over their nominal capacity, tabulated in \ref{solve_risk}.  First we note the costs, both CC and JCC are around 5\% higher and in return for the higher cost, have better risk characteristics.  In the standard OPF model, the expected number of lines over their threshold is 20 times that of the CC model.  The JCC halves the CC model for roughly the same cost.  The CC model directly constraints these probabilities, but does so on the individual line level.  By having a system constraint, the JCC model is better suited to optimize risk characteristics related to system level transmission utilization.


\begin{table}
\centering

 \begin{tabular}{ |c| c c c |}
\hline
& OPF & CC .01 & JCC .98 \\
\hline
\hline
Cost & 565.2 & 590.2 & 589.6 \\
$r$ & 0.00504 & 0.00150 & 0.00031 \\
\hline
$Ex$ & 0.3875 & 0.0225 & 0.0125 \\
\hline
\end{tabular}
\caption{Risk comparison for OPF,CC, and JCC on the small test instance.}\label{solve_risk}
\end{table}


\subsection{Sensitivity Analysis}\label{senseanal}
Now, we want to look at the sensitivity of the JCC model to its input parameters and demand uncertainty.  First we look at how the risk of the different dispatch points from the OPF, CC, and JCC models respond to changes in the failure density function parameters $L$ and $p$.  To this end, we solved the OPF, CC ($\eta_L=.01$), and JCC ($\epsilon=0.0003$, $L=.98$).  These three dispatch points are used to find the risk of the branch flows for varying risk parameters $L,p$. The cost of the dispatch points are tabulated in \ref{tabsense} and the sensitivity analysis is shown in figure \ref{figsense}. This figure shows that for the given JCC solution, the system risk stays under that of the CC model.  The OPF model has a lower cost so it is reasonable that the risk is much higher.  The same results were seen when polynomial risk functions were evaluated, such as $x^2,x^3,$ and $x^4$.  These parameters will be very difficult to estimate in the real world and as such it is very important that our model has shown to be robust to changing both the parameters in the piecewise linear model as well as testing against different polynomial models for failure density functions.  This sensitivity analysis shows that while our parameters may be off, this model will still do well for the assumptions that at some point, lines begin to take on additional risk due to congestion and that risk is monotonically increasing.

\begin{table}
\centering
 \begin{tabular}{ |c| c c c |}
\hline
& OPF & CC & JCC \\
\hline
\hline
Cost & 567.1 & 574.0 & 572.2\\
\hline
\end{tabular}
\caption{Table of cost for the three dispatch points used in the sensitivity analysis}\label{tabsense}
\end{table}


\begin{figure}
\centering
\includegraphics{\mypathjcc/fig-sensitivity}
\caption{A comparison of the three dispatch points for varying risk function parameters}\label{figsense}
\end{figure}






%%%%%%%% Section 6
\section{Conclusion}


The primary strength of this model is the addition of a system risk constraint and the relaxation of line thresholds. Line thresholds are hard constraints that subject the system to price spikes.  The hard line constraints are somewhat arbitrary as the line does not fail when it is exceeded by a small amount.  Instead, an economic trade-off should be made between individual lines to ensure the system risk is constrained at an adequate level.  The system risk measure allows for direct comparison of different dispatch points with respect to risk.

The JCC model is computationally efficient in its full form.  It allows for creation of a cost-risk frontier that finds dispatch points that are better in both terms of risk and cost.  Under uncertainty, this model should be compared against the CC model, which is a probabilistic interpretation of line thresholds.  Probabilistic line thresholds exacerbate the economic problems of hard constraints by further tightening the constraints (compared to when uncertainty is ignored).  In the computational section, we saw the extremely high sensitivity of cost to uncertainty in load when the transmission system is congested.  In both of these models, the variable slack distribution plays a small direct role in cost through the objective, however is crucial in ensuring the line and system risk constraints are met (and a larger indirect cost contribution).  These system risk constraints capture the risk of a heavily loaded transmission system that the traditional OPF and CC miss that is related to cascading power failure risk in literature.
  
%Combining the system risk measure and random power injects for the joint chance constrained model accounts for many things the traditional OPF model misses.  This model gives prices for system response to aggregated random injects, which is regulation and reserve support for different time periods.  Instead of having an ancillary service model which separates this, the JCC model directly accounts for it.  This allows for valuing random injects, which is important going forward as the grid integrates more uncertain generation (wind turbines) and demand (electric vehicles).


There are many avenues to improve and extend this model.  One first question would be, what does the failure density function look like.  We used a piecewise linear model, but perhaps other models better represent the true physics of the situations, such as a polynomial, ie $x^2$.  In addition, often the effective capacities are correlated.  Effective capacities are related to environmental conditions, such as high wind, which are geographically correlated.  During high wind periods, there is increased wind generation as well as increased effective capacities, which means that when the turbines are producing more energy, the nearby transmission lines are also capable of carrying more energy.  

%Finally, the random injects need to be priced and accounted for.  Random injects add stress to the grid which needs to be secured through ancillary markets such as operating reserves and regulation.

%\newcommand{\mypathslk}{../thesis/slk}
\chapter{Slack Market for Price Risk Reduction}

\section{Slack Market}



\chapter{Conclusion}
\section{Contributions}
contribute 

\section{Future Work}
Two areas of interest going forward are to price the covariance matrix in the JCC model and use the JCC model in the OPA framework to compare what effects it might have on cascading.  

\section*{Thanks}
Thanks for reading, I look forward to the help and feedback.


\bibliographystyle{plain}	
\bibliography{ref,msip-bib,jcc-bib,dfo-bib}		

\theendnotes


\end{document}
''}}
%\tikzset{external/export=false}



%\setlength{\textwidth}{15cm}
\parindent 1cm
\parskip 0.2cm
\topmargin 0.2cm
\oddsidemargin .4cm
\evensidemargin 0.4cm
\textwidth 15cm
\textheight 21cm


\newcommand{\bi}{\begin{itemize}}
\newcommand{\ei}{\end{itemize}}

\newcommand{\norm}[1]{ \left| \left| {#1} \right| \right|}
\newcommand{\magf}{ \left| f \right| }
\newcommand{\magp}{ \left| p \right| }
\newcommand{\magr}{ \left| r \right| }
\newcommand{\magomega}{ \left|  \omega  \right| }
\newcommand{\magE}{ \left| \mathcal{E} \right| }
\newcommand{\magG}{ \left| \mathcal{G} \right| }
\newcommand{\magL}{ \left| \mathcal{L} \right| }
\newcommand{\magM}{ \left| \mathcal{M} \right| }
\newcommand{\magT}{ \left| \mathcal{T} \right| }
\newcommand{\magV}{ \left| \mathcal{V} \right| }

\newcommand{\R}[1]{\mathbb{R}^{#1}}
\def\S{\mathbb{ S}}
\def\I{\mathbb{ I}}


\newcommand{\cA}{\mathcal{A}}
\newcommand{\cB}{\mathcal{B}}
\newcommand{\cD}{\mathcal{D}}
\newcommand{\cCALI}{\mathcal{CALI}}
\newcommand{\cDC}{\mathcal{DC}}
\newcommand{\cGR}{\mathcal{GR}}
\newcommand{\cE}{\mathcal{E}}
\newcommand{\cF}{\mathcal{F}}
\newcommand{\cG}{\mathcal{G}}
\newcommand{\cH}{\mathcal{H}}
\newcommand{\cJ}{\mathcal{J}}
\newcommand{\cK}{\mathcal{K}}
\newcommand{\cL}{\mathcal{L}}
\newcommand{\cM}{\mathcal{M}}
\newcommand{\cN}{\mathcal{N}}
\newcommand{\cO}{\mathcal{O}}
\newcommand{\cS}{\mathcal{S}}
\newcommand{\cT}{\mathcal{T}}
\newcommand{\cU}{\mathcal{U}}
\newcommand{\cV}{\mathcal{V}}
\newcommand{\cW}{\mathcal{W}}
\newcommand{\cX}{\mathcal{X}}




\newcommand{\Expect}{\mathbb{E}}
\newcommand{\Exx}[1]{\mathbb{E}\left[ #1 \right]}
\newcommand{\Var}[1]{Var\left[ #1 \right]}
\newcommand{\CoVar}[2]{CoVar\left[ #1, #2 \right]}
\newcommand{\Prob}{\mathbb{P}}
\newcommand{\cvar}{\mathbb{CV}{\mathsf @}\mathbb{R}}
\newcommand{\grad}{\bigtriangledown}
\newcommand{\lb}{\left\{}
\newcommand{\rb}{\right\}}
\newcommand{\floor}[1]{\lfloor #1 \rfloor}

\newcommand{\defeq}{\stackrel{\rm def}{=}}



\makeatletter
\def\BState{\State\hskip-\ALG@thistlm}
\makeatother


\tikzset{
        hatch distance/.store in=\hatchdistance,
        hatch distance=10pt,
        hatch thickness/.store in=\hatchthickness,
        hatch thickness=2pt
    }

\makeatletter
    \pgfdeclarepatternformonly[\hatchdistance,\hatchthickness]{flexible hatch}
    {\pgfqpoint{0pt}{0pt}}
    {\pgfqpoint{\hatchdistance}{\hatchdistance}}
    {\pgfpoint{\hatchdistance-1pt}{\hatchdistance-1pt}}%
    {
        \pgfsetcolor{\tikz@pattern@color}
        \pgfsetlinewidth{\hatchthickness}
        \pgfpathmoveto{\pgfqpoint{0pt}{0pt}}
        \pgfpathlineto{\pgfqpoint{\hatchdistance}{\hatchdistance}}
        \pgfusepath{stroke}
    }


\usetikzlibrary{arrows,shapes,positioning,shadows,trees,svg.path}

\tikzset{
  basic/.style  = {draw, text width=2cm, drop shadow, font=\sffamily, rectangle},
  root/.style   = {basic, rounded corners=2pt, thin, align=center, fill=green!30, text width=2in},
  level 2/.style = {basic, rounded corners=6pt, thin,align=center, fill=green!45, text width=8em},
  oracle/.style = {basic, rounded corners=6pt, thin, fill=blue!30, text width=10em},
  level 2 oracle/.style = {basic, rounded corners=6pt, thin, fill=blue!10, text width=5.5em},
  level 3/.style = {basic, thin, align=left, fill=pink!60, text width=6.5em}
}


\makeindex
