%\documentclass{standalone}
%\begin{document}

\def \sc     { .65 }
\def \lw     { 1.25pt }

\begin{figure}
\centering
\begin{subfigure}[b]{.46\linewidth}
\begin{tikzpicture}[line width=\lw,scale=\sc]

\node[circle,fill=blue!20] (one) at (3,0) {\small 1};
\node[circle,fill=red!20] (two) at (5,0) {\small 2};
\node[rectangle,fill=green!20] (three) at (7,1) {\small 3};
\node[rectangle,fill=green!20] (four) at (5,1.75) {\small 4};
\node[rectangle,fill=green!20] (five) at (3,1.75) {\small 5};
\node[circle,fill=red!20] (six) at (2,3) {\small 6};
\node[rectangle,fill=green!20] (seven) at (5,3) {\small 7};
\node[circle,fill=red!20] (eight) at (3.85,3.45) {\small 8};
\node[rectangle,fill=green!20] (nine) at (7,3) {\small 9};
\node[rectangle,fill=green!20] (ten) at (5.85,4.1) {\small 10};
\node[rectangle,fill=green!20] (eleven) at (3.75,5) {\small 11};
\node[rectangle,fill=green!20] (twelve) at (2.35,5) {\small 12};
\node[rectangle,fill=green!20] (thirteen) at (.81,4.2) {\small 13};
\node[rectangle,fill=green!20] (fourteen) at (4,6) {\small 14};


\draw (one) -- (two) ;
\draw (one) -- (five);
\draw (two) -- (three) ; 
\draw (two) -- (four) ; 
\draw (two) -- (five) ; 
\draw (three) -- (four) ; 
\draw[red] (four) -- (five) ; 
\draw (four) -- (seven);
\draw (four) -- (nine);
\draw (five) -- (six) ; 
\draw (six) -- (eleven) ; 
\draw (six) -- (twelve) ; 
\draw (six) -- (thirteen) ; 
\draw (seven) -- (eight) ; 
\draw (seven) -- (nine) ; 
\draw (nine) -- (ten) ; 
\draw (nine) .. controls +(up:1.2cm) .. (fourteen) ;
\draw[red] (ten) -- (eleven); 
%\draw[red] (eleven) -- (twelve); 
\draw[red] (twelve) -- (thirteen);
\draw (thirteen) .. controls +(up:1.2cm) .. (fourteen) ; 
\end{tikzpicture}
\caption{Initial Event: The three red lines are outaged and the power flow redistributes.}
\end{subfigure}
\begin{subfigure}[b]{.46\linewidth}
\begin{tikzpicture}[line width=\lw,scale=\sc]

\node[circle,fill=blue!20] (one) at (3,0) {\small 1};
\node[circle,fill=red!20] (two) at (5,0) {\small 2};
\node[rectangle,fill=green!20] (three) at (7,1) {\small 3};
\node[rectangle,fill=green!20] (four) at (5,1.75) {\small 4};
\node[rectangle,fill=green!20] (five) at (3,1.75) {\small 5};
\node[circle,fill=red!20] (six) at (2,3) {\small 6};
\node[rectangle,fill=green!20] (seven) at (5,3) {\small 7};
\node[circle,fill=red!20] (eight) at (3.85,3.45) {\small 8};
\node[rectangle,fill=green!20] (nine) at (7,3) {\small 9};
\node[rectangle,fill=green!20] (ten) at (5.85,4.1) {\small 10};
\node[rectangle,fill=green!20] (eleven) at (3.75,5) {\small 11};
\node[rectangle,fill=green!20] (twelve) at (2.35,5) {\small 12};
\node[rectangle,fill=green!20] (thirteen) at (.81,4.2) {\small 13};
\node[rectangle,fill=green!20] (fourteen) at (4,6) {\small 14};

\draw[red] (one) -- (two) ;
\draw (one) -- (five);
\draw (two) -- (three) ; 
\draw[red] (two) -- (four) ; 
\draw (two) -- (five) ; 
\draw[red] (three) -- (four) ; 
%\draw (four) -- (five) ; 
\draw (four) -- (seven);
\draw (four) -- (nine);
\draw (five) -- (six) ; 
\draw (six) -- (eleven) ; 
\draw (six) -- (twelve) ; 
\draw (six) -- (thirteen) ; 
\draw (seven) -- (eight) ; 
\draw (seven) -- (nine) ; 
\draw (nine) -- (ten) ; 
\draw (nine) .. controls +(up:1.2cm) .. (fourteen) ;
%\draw (ten) -- (eleven); 
%\draw (twelve) -- (thirteen);
\draw (thirteen) .. controls +(up:1.2cm) .. (fourteen) ; 
\end{tikzpicture}
\caption{Stage 1: Lines 1-2, 2-4, and 3-4 are overloaded. Lines 1-2 and 3-4 fail, but line 2-4 remains in operation at an overloaded state. }

\end{subfigure}
\begin{subfigure}[b]{.46\linewidth}

\begin{tikzpicture}[line width=\lw,scale=\sc]

\node[circle,fill=blue!20] (one) at (3,0) {\small 1};
\node[circle,fill=red!20] (two) at (5,0) {\small 2};
\node[rectangle,fill=green!20] (three) at (7,1) {\small 3};
\node[rectangle,fill=green!20] (four) at (5,1.75) {\small 4};
\node[rectangle,fill=green!20] (five) at (3,1.75) {\small 5};
\node[circle,fill=red!20] (six) at (2,3) {\small 6};
\node[rectangle,fill=green!20] (seven) at (5,3) {\small 7};
\node[circle,fill=red!20] (eight) at (3.85,3.45) {\small 8};
\node[rectangle,fill=green!20] (nine) at (7,3) {\small 9};
\node[rectangle,fill=green!20] (ten) at (5.85,4.1) {\small 10};
\node[rectangle,fill=green!20] (eleven) at (3.75,5) {\small 11};
\node[rectangle,fill=green!20] (twelve) at (2.35,5) {\small 12};
\node[rectangle,fill=green!20] (thirteen) at (.81,4.2) {\small 13};
\node[rectangle,fill=green!20] (fourteen) at (4,6) {\small 14};

%\draw[red] (one) -- (two) ;
\draw (one) -- (five);
\draw (two) -- (three) ; 
\draw[red] (two) -- (four) ; 
\draw[red] (two) -- (five) ; 
%\draw[red] (three) -- (four) ; 
%\draw (four) -- (five) ; 
\draw (four) -- (seven);
\draw (four) -- (nine);
\draw (five) -- (six) ; 
\draw (six) -- (eleven) ; 
\draw (six) -- (twelve) ; 
\draw[red] (six) -- (thirteen) ; 
\draw (seven) -- (eight) ; 
\draw[red] (seven) -- (nine) ; 
\draw (nine) -- (ten) ; 
\draw (nine) .. controls +(up:1.2cm) .. (fourteen) ;
%\draw (ten) -- (eleven); 
%\draw (twelve) -- (thirteen);
\draw (thirteen) .. controls +(up:1.2cm) .. (fourteen) ; 
\end{tikzpicture}
\caption{Stage 2: On the new topology, lines 2-5, 6-13, and 7-9 become overloaded.  The cascade progresses by outaging lines 2-4 and 7-9.}
\end{subfigure}
\begin{subfigure}[b]{.46\linewidth}
\begin{tikzpicture}[line width=\lw,scale=\sc]


\node[circle,fill=blue!20] (one) at (3,0) {\small 1};
\node[circle,fill=red!20] (two) at (5,0) {\small 2};
\node[rectangle,fill=green!20] (three) at (7,1) {\small 3};
\node[rectangle,fill=green!20] (four) at (5,1.75) {\small 4};
\node[rectangle,fill=green!20] (five) at (3,1.75) {\small 5};
\node[circle,fill=red!20] (six) at (2,3) {\small 6};
\node[rectangle,fill=green!20] (seven) at (5,3) {\small 7};
\node[circle,fill=red!20] (eight) at (3.85,3.45) {\small 8};
\node[rectangle,fill=green!20] (nine) at (7,3) {\small 9};
\node[rectangle,fill=green!20] (ten) at (5.85,4.1) {\small 10};
\node[rectangle,fill=green!20] (eleven) at (3.75,5) {\small 11};
\node[rectangle,fill=green!20] (twelve) at (2.35,5) {\small 12};
\node[rectangle,fill=green!20] (thirteen) at (.81,4.2) {\small 13};
\node[rectangle,fill=green!20] (fourteen) at (4,6) {\small 14};

%\draw[red] (one) -- (two) ;
\draw (one) -- (five);
\draw (two) -- (three) ; 
%\draw[red] (two) -- (four) ; 
\draw[red] (two) -- (five) ; 
%\draw[red] (three) -- (four) ; 
%\draw (four) -- (five) ; 
\draw[red] (four) -- (seven);
\draw[red] (four) -- (nine);
\draw (five) -- (six) ; 
\draw (six) -- (eleven) ; 
\draw (six) -- (twelve) ; 
\draw[red] (six) -- (thirteen) ; 
\draw (seven) -- (eight) ; 
%\draw[red] (seven) -- (nine) ; 
\draw (nine) -- (ten) ; 
\draw (nine) .. controls +(up:1.2cm) .. (fourteen) ;
%\draw (ten) -- (eleven); 
%\draw (twelve) -- (thirteen);
\draw[red] (thirteen) .. controls +(up:1.2cm) .. (fourteen) ; 
\end{tikzpicture}
\caption{Stage 3: This has the effect of routing all power destined for load 8 through the north passage. Lines 13-14, 4-7, and 4-9 are outaged along the path. }
\end{subfigure}

\begin{subfigure}[b]{.46\linewidth}
\begin{tikzpicture}[line width=\lw,scale=\sc]


\node[circle,fill=blue!20] (one) at (3,0) {\small 1};
\node[circle,fill=red!20] (two) at (5,0) {\small 2};
\node[rectangle,fill=green!20] (three) at (7,1) {\small 3};
\node[rectangle,fill=green!20] (four) at (5,1.75) {\small 4};
\node[rectangle,fill=green!20] (five) at (3,1.75) {\small 5};
\node[circle,fill=red!20] (six) at (2,3) {\small 6};
\node[rectangle,fill=green!20] (seven) at (5,3) {\small 7};
\node[circle,fill=red!20] (eight) at (3.85,3.45) {\small 8};
\node[rectangle,fill=green!20] (nine) at (7,3) {\small 9};
\node[rectangle,fill=green!20] (ten) at (5.85,4.1) {\small 10};
\node[rectangle,fill=green!20] (eleven) at (3.75,5) {\small 11};
\node[rectangle,fill=green!20] (twelve) at (2.35,5) {\small 12};
\node[rectangle,fill=green!20] (thirteen) at (.81,4.2) {\small 13};
\node[rectangle,fill=green!20] (fourteen) at (4,6) {\small 14};


%\draw[red] (one) -- (two) ;
\draw (one) -- (five);
\draw (two) -- (three) ; 
%\draw[red] (two) -- (four) ; 
\draw[red] (two) -- (five) ; 
%\draw[red] (three) -- (four) ; 
%\draw (four) -- (five) ; 
%\draw (four) -- (seven);
%\draw (four) -- (nine);
\draw (five) -- (six) ; 
\draw (six) -- (eleven) ; 
\draw (six) -- (twelve) ; 
\draw (six) -- (thirteen) ; 
\draw (seven) -- (eight) ; 
%\draw[red] (seven) -- (nine) ; 
\draw (nine) -- (ten) ; 
\draw (nine) .. controls +(up:1.2cm) .. (fourteen) ;
%\draw (ten) -- (eleven); 
%\draw (twelve) -- (thirteen);
%\draw (thirteen) .. controls +(up:1.2cm) .. (fourteen) ; 
\end{tikzpicture}
\caption{Stage 4:  Finally, line 2-5 that is still overloaded is outaged. }
\end{subfigure}
\begin{subfigure}[b]{.46\linewidth}
\begin{tikzpicture}[line width=\lw,scale=\sc]


\node[circle,fill=blue!20] (one) at (3,0) {\small 1};
\node[circle,fill=red!20] (two) at (5,0) {\small 2};
\node[rectangle,fill=green!20] (three) at (7,1) {\small 3};
\node[rectangle,fill=green!20] (four) at (5,1.75) {\small 4};
\node[rectangle,fill=green!20] (five) at (3,1.75) {\small 5};
\node[circle,fill=red!20] (six) at (2,3) {\small 6};
\node[rectangle,fill=green!20] (seven) at (5,3) {\small 7};
\node[circle,fill=red!20] (eight) at (3.85,3.45) {\small 8};
\node[rectangle,fill=green!20] (nine) at (7,3) {\small 9};
\node[rectangle,fill=green!20] (ten) at (5.85,4.1) {\small 10};
\node[rectangle,fill=green!20] (eleven) at (3.75,5) {\small 11};
\node[rectangle,fill=green!20] (twelve) at (2.35,5) {\small 12};
\node[rectangle,fill=green!20] (thirteen) at (.81,4.2) {\small 13};
\node[rectangle,fill=green!20] (fourteen) at (4,6) {\small 14};


%\draw[red] (one) -- (two) ;
\draw (one) -- (five);
\draw (two) -- (three) ; 
%\draw (two) -- (four) ; 
%\draw[red] (two) -- (five) ; 
%\draw[red] (three) -- (four) ; 
%\draw (four) -- (five) ; 
%\draw (four) -- (seven);
%\draw (four) -- (nine);
\draw (five) -- (six) ; 
\draw (six) -- (eleven) ; 
\draw (six) -- (twelve) ; 
\draw (six) -- (thirteen) ; 
\draw (seven) -- (eight) ; 
%\draw[red] (seven) -- (nine) ; 
\draw (nine) -- (ten) ; 
\draw (nine) .. controls +(up:1.2cm) .. (fourteen) ;
%\draw (ten) -- (eleven); 
%\draw (twelve) -- (thirteen);
%\draw (thirteen) .. controls +(up:1.2cm) .. (fourteen) ; 
\end{tikzpicture}
\caption{Stage 5:  The system stabilizes into islands with generator 1 serving load 6.  However, loads 2 and 8 are out of service.}
\end{subfigure}
\caption{ \label{fig:cascade-example} \small An example of a cascading power failure. Node 1 is a generator and nodes 2, 6, and 8 are loads.}
\end{figure}

%\end{document}
