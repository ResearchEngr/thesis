\textbf{Chance Constraint on N-k Criteria}

In this model, constraints are used to express the probability that less than a given number of lines are outaged is close to 1.  A typical case in power systems would be that the system should not go beyond 1 contingency in a large percentage of scenarios.
\begin{equation}
P\left\{\text{Line Outage} \le k \right\} \ge 1 - \epsilon
\end{equation}
This can be done by adding another binary variable for each scenario that represents whether or not that scenario has less than $k$ outages.  Then these new binary variables will be summed and constrained by the given probability.
\begin{subequations}
\begin{align}
\sum_i z_{is} &\le	M_s \hat{z_s} + | \xi_s | + k \\
\sum_s \hat{z_s} &\le \epsilon | \cS |
\end{align}
\end{subequations}
where $M_s = | \cE | - | \xi_s | - k$.  The objective of this program would be to minimize load shed as in (\ref{leastcost}) such that the power system does not lose more than $k$ branches.
