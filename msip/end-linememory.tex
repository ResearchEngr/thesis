\textbf{Line Memory}

Bienstock made several modifications \cite{bienstock_2011} to the original OPA model in order to remove some undesirable effects of the simulation, notably the erratic behavior of its output under small changes in the input.  To do this, he introduced the concept of memory to the system.  In order to see if a line is in or near an overloaded state, it uses a running time average of the current state and previous states.
\begin{equation}
\tilde{f}_{et} = \alpha f_{et} + (1-\alpha) \tilde{f}_{e\rho(t)}
\end{equation} 
Here, $f_{et}$ represents the power flow on edge $e$ at time $t$.  Then, $\tilde{f}_{et}$ is used in the overload and outage calculations.

In addition to including a memory, he also smoothed out the definition of an overloaded line by creating a step in between normal and overloaded states in which the failure probability was more than nominal but less than in the overloaded state.  Using $0 \le \epsilon \le 1$, for edge $e$, the following failure model smooths the effects of overloaded lines failing.
\begin{align}
\tilde{f}_{et} \ge (1+\epsilon) U_{et}			&	\hspace{10pt} \mbox{The line outages with certainty} 	\\
(1-\epsilon) U_{et} < \tilde{f}_{et} < (1+\epsilon) U_{et}	&	\hspace{10pt} \mbox{The line outages with probability }\frac{1}{2} 	\\
\tilde{f}_{et} \le (1-\epsilon) U_{et}			&	\hspace{10pt} \mbox{The line remains in operation} 
\end{align}
