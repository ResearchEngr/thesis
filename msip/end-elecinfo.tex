\textbf{Electrical Information}

Complex power has both real and reactive parts.  Alternating currents  on a circuit affect components of energy storage such as inductors (changing the current as opposed by a voltage)  or capacitors (store electrical charge).  Over one full cycle of the electricity changing direction, across any individual element there can be real power transferred.  However, there is also power which is stored and released within one cycle and the net energy transfer of this power is 0.  This is called reactive power and is modelled as the imaginary component of complex power.  Let $S_i$ be the complex power inject at some bus on the grid, that is
\begin{equation}
S_i = P_i + j Q_i = V_i I_i^*
\end{equation}
where $P$ is real power, $Q$ is reactive power, $V_i$ is complex voltage, and $I_i^*$ is the complex conjugate of current.  
	
To model a transmission line, the characteristic impedence is used.  At any point, there is complex current I, in each individual line in the transmission element.  Also, there is a complex voltage V difference between the lines.  The characteristic impedence (generalization of Ohm's Law) is then
\begin{equation} \label{impedence}
V/I = Z_0
\end{equation}

Here is a phasor representation of complex voltage $V_i = | V_i | e^{i (\omega t + \delta_i) }$, where the real voltage would be Re[$V_i$]$=\cos (\omega t + \delta_i)$.  Now, by modeling \ref{impedence} in phasor notation, we have 
\begin{equation}
 | V_i | e^{j (\omega t + \delta_V) } = | I_i | e^{j (\omega t + \delta_I) }  | Z_i | e^{j ( \delta_Z) } = |I||Z|e^{j (\omega t + \delta_I + \delta_Z)}
\end{equation}
For this equation to hold at all times $t$, we know that this must hold $\delta_V = \delta_I + \delta_Z$.  When an element has $\delta_Z = 0$, the current and voltage are exactly in phase and it is a purely resistive load with no reactive production or absorption. 

The lines can be modelled using a pi model (schematic diagram looks like $\pi$), which uses line parameters of resistance $R$, inductance $L$, conductance $G$, and capacitance $C$. 
\begin{equation}
Z_0 = \sqrt{  \frac{R+j \omega L}{G+j\omega C} }
\end{equation}
The system angular frequency is $\omega = 2 \pi f$, where $f$ is frequency here.  

If a transmission element is loaded at its surge impedence loading, it neither creates nor consumes reactive power.  This level is defined by 
\begin{equation}
SIL = V_{LL}^2 / Z_0
\end{equation}
Where $V_{LL}$ is the line to line voltage.  When a transmission line is at a level below this, it will supply reactive power and raise system voltages.  When the loading is above this level, the transmission line consumes reactive power, depressing voltages.
