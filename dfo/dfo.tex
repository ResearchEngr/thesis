\newcommand{\mypathdfo}{../thesis/dfo}
\newcommand{\mypathdfodata}{../thesis/dfo/data}
\newcommand{\scd}{\cD_\oplus}
\newcommand{\btu}{\bigtriangleup}
\chapter{Solving Design Problems}

The cascading model built in chapter \ref{msip-model} is too hard to optimize at the scale we need to model the uncertainty in the cascading process.  In order to work around this difficulty, we will decompose the multi-stage structure of this problem and simulate the entire cascading process.  The primary benefit is removing the temporal linked binary variables for line outages from the master problem allowing sequential evaluation of the decision dependent uncertainty.  This lets us to parallelize the OPA evaluations to increase the computational effort.

In order to optimize design problems using the OPA simulation for risk metrics, we need to first understand the characterestics of the cascading process and the effects that generation, reserves, and line limits have on the resulting load shed distributions.  Then, we can modify existing derivative free frameworks to optimize this simulation and utilize explotatory steps to improve local performance and global strategies to find near optimal decisions.

The OPA simulation provides fundamental difficulties to the optimization process.  The function evaluations can shown to be no convex nor continuously differentiable every.  It is inherently noisy and the load shed distribution is wide, often charactereized by it's power law distribution.  Some of these effects can be smoothed away by employing a wide range of potential initiating events, however it may be important to optimize against a small subset of events that have a higher likihood of occurance and are known to be risky.  In this case, the non convexity and discontinuoity are most apparent.

To begin, an overview of derivative free optimization will be giving.  Three classes will be looked at for their convergence properties and practicle applications.  Then, initial experimentation on the OPA model will be given as well as the implementation and variance reduction strategies.  Then a modified algorithm will be developed and finally compared against vanilla algorithms of DFO.  A primary undercurrent is the use of accessory information in the modified DFO algorithm.

\section{Literature Review}
This section will give an overview to the history and current state of derivative free optimization.  The foundation for direct search and model based search of DFO techniques will be explored to be used in solution methods for the OPA optimization procedure.  The DFO field has been around for some time now and has seen a resurgence over the last decade and a half with the only textbook at less than 5 years old (Conn, Scheinberg, Vicente) \cite{conn_2009}.
\subsection{Derivative Free Optimization}
This class of optimization strategies cover a wide-range of problems and techniques.  In general, the problem has a function $f: \R{n} \rightarrow \R{}$ that takes a decision variable $x \in \R{n}$ and returns a scalar. 
\begin{equation}
\min_{x \in \R{n}} f(x)
\end{equation}
The primary problem attribute that makes DFO a good choice for solution method is that standard gradient or Newton based method will not work.  This can be due to a variety of reasons, a common one being simulation-based optimization.  Here, the derivative is unavailable symbolically and, perhaps due to stochastic or numerical noise, is unable to be calculated with finite difference methods.  Even if the underlying function is smooth, a costly function evaluation may make the finite difference approach undesirable due to the considerable time to calculate full gradients.

While still useful for smooth functions with a Lipschitz continuous derivatives, these methods can shine in nonsmooth and even non-convex application.  Direct search methods work by searching in many directions in order to guarantee a descent direction is chosen.  This has the ability to provide robustness against noise that may mislead gradient based methods using a single search direction.  In addition, by using relatively large step size the trial points provide a smoothing affect to the function which allow it to ignore high-frequency noise until it is close to a lower-frequency, higher amplitude minimum.  Contrary to problem application, we will assume a smooth function with a Lipschitz continuous derivative in order to find convergence results for different algorithms.  No guarantees can be made for the nonsmooth problems, however in practice these techniques are relatively successful for this class of problems

\subsubsection*{History}
\begin{figure}
\centering
\begin{tikzpicture}[
  level 1/.style={sibling distance=40mm},
  edge from parent/.style={->,draw},
  >=latex]


% root of the the initial tree, level 1
\node[root] (root) {\large \textbf{ Derivative Free Optimization}}
% The first level, as children of the initial tree
  child {node[level 2] (c1) {Direct Search}}
  child {node[level 2] (c2) {Model Based}}
  child {node[level 2] (c3) {Heuristics}}
  child {node[level 2] (c4) {Global}};

\node[oracle] (fo) at (5.1,2.1){Function Oracle};
\node[level 2 oracle,below of = fo,xshift=5pt] (o1) {Dimension Reduction};
\node[level 2 oracle,below of = fo,xshift=80pt, yshift=-20pt] (o2) {Variance Reduction};
\node[level 2 oracle,below of = o1,yshift=-10pt] (o3) {Accessory Information};
  \draw[->] (fo.190) |- (o1.west);
  \draw[->] (fo.190) |- (o2.west);
  \draw[->] (fo.190) |- (o3.west);

\draw[<->,very thick,dashed] (root) -- (fo.west);

% The second level, relatively positioned nodes
\begin{scope}[every node/.style={level 3}]
%\node [below of = c1, xshift=15pt,yshift=-5pt] (c11) {Generalized Pattern Search};
\node [below of = c1,xshift=15pt,yshift=-5pt] (c11) {Generating Set Search}; %with bound constraints
\node [below of = c11,yshift=0pt] (c12) {Mesh Adaptive};
\node [below of = c12] (c13) {Geometric, Simplex};

\node [below of = c2, xshift=15pt,yshift=2pt] (c21) {Trust Regions};
\node [below of = c21] (c22) {Implicit \newline Filtering};
\node [below of = c22,yshift=-5pt] (c23) {Stochastic \newline Approximation};

\node [below of = c3,xshift=15pt,yshift=2pt] (c31) {Particle Swarm};
\node [below of = c31] (c32) {Simulated \newline Annealing};
\node [below of = c32,yshift=-5pt] (c33) {Genetic \newline Algorithms};

\node [below of = c4, xshift=15pt,yshift=-5pt] (c41) {Nested \newline Partitioning};
\node [below of = c41, yshift=-2pt] (c42) {Space Filling};
\node [below of = c42,yshift=7pt ] (c43) {Global Model}; %Response Surface, radial basis functions, kriging
\end{scope}

% lines from each level 1 node to every one of its ``children''
\foreach \value in {1,2,3}
  \draw[->] (c1.195) |- (c1\value.west);

\foreach \value in {1,...,3}
  \draw[->] (c2.195) |- (c2\value.west);

\foreach \value in {1,...,3}
  \draw[->] (c3.195) |- (c3\value.west);

\foreach \value in {1,...,3}
  \draw[->] (c4.195) |- (c4\value.west);
\end{tikzpicture}

\caption{DFO methods for continuous variable optimization}
\end{figure}
Derivative-free techniques for optimizing problems have literature dating back to the 60's when Hookes and Jeeves \cite{hooke_1961} proposed a simple search method to find a maximum or minimum when classical methods were unfeasible.  Since then, direct search methods have seen many improvements both to their performance in practice and results in theory.  Modern direct search methods are able to conform to local topology as well as guarantee convergence to stationary points.  New methods are beginning to use probabalistic search procedures to improve in performance as well as increase the classes of problems they are able to solve.

Geometry based methods also started in the 60's with Nelder and Mead \cite{nelder_1965} using the geometric simplex and operations such as reflection, expansion, and contraction.  These methods can be useful as they naturally conform to local topology, however the geometry can often degenerate and converge to a non-optimal point.  Modern methods can use safeguarding techniques to prevent this or restart a simplex when it has deteriorated.  

Heuristic search procedures showed up in the 70's and have continued to be used throughout with many papers in practical application.  Common procedures such as genetic algorithms (1975) \cite{holland_1975}, simulated annealing (1983) \cite{kirkpatrick_1983}, and particle swarm (1995) \cite{eberhart_1995} \cite{kennedy_1995} are used throughout engineering and physic disciplines, partly due to their ease in application and decent practical performance.  While good for practical applications, they lack the more rigorous theory for convergence that is behind direct search and model based methods.

Model-based methods are very good at using function evaluations and spend considerably more effort to choose new points.  These interpolation based strategies where around at least by 1973 \cite{winfield_1973} and assume the model is accurate within its trust-region.  The method proceeds by minimizing its model and using that as a new trial point.  Modern implementations can use linear models \cite{powell_1994}, quadratic models \cite{powell_2002}, or perhaps a good compromising being underdetermined quadratic models.

Finally, global based strategies are used in conjunction with many of the previous models in order to work towards global optima. One primary avenue is partitioning models.  These partions have trial points at the extreme points of the partion (1972)\cite{shubert_1972}, the center point (1993) \cite{jones_1993}, or arbitrary points within (1999) \cite{huyer_1999} and typically work by building underestimators of the function using the Lipschitz constant or an estimate.  Another strategy employed are surrogate functions to opitmize instead of the original function.  Common versions are response surfaces, kriging, and radial basis functions.  

A more complete history and overview of current derivative free methods as well as numerical tests for a large, diverse set of solvers can be found in \cite{rios_2013}.  Now we turn to direct search and model based methods to understand their implementation, performance, and convergence properties .


\subsubsection*{Gradient Line Search}
To start, a quick overview of line search methods and properties to guarentee convergence will be given.  These properties will be used as a comparison for comparable properties used in DFO methods.  Line search methods move in a descent direction and have constraints on step length in order to ensure convergence.  Assuming our function is continuously differentiable and letting $\alpha$ be a step size and $d$ a direction, we have
\begin{equation}
f(x + \alpha d ) = f(x) + \alpha \grad f(x) d  + o (\alpha)
\end{equation}
and by setting $d = -\grad f(x)$ we are guarenteed to reduce $f$ for sufficiently small step sizes $\alpha$.  We can see that any direction that is sufficiently in line with the negative gradient of $f$ will lead to a decrease, that is
\begin{equation}
-\grad f(x_k)^T d_k > 0
\end{equation}
Additionally, a near orthogonal direction to the gradient will lead to poor progress in search and convergence.  The angle between the search direction and the negative gradient needs to be bounded away from 0.  In the typical line search, this angle condition is
\begin{equation}
\frac{-\grad f(x_k)^T d_k}{\norm{\grad f(x_k) } \norm{d_k}} \geq c > 0
\end{equation}
and is automatically satisfied for $d_k = -\grad f(x_k)$.

A simple decrease condition $f(x_{k+1}) < f(x_{k})$ is not enough to guarantee convergence, due to the existence of sequences of points moving in a decent direction with a limit point other than the minimum.  This is related to the step size and both too short and too long of step sizes can cause problems.  The Armijo-Goldstein-Wolfe conditions prevent these problems by constraining the step length 
\begin{align}
f(x_k + \alpha_k d_k) \le f(x_k) + c_1 \alpha_k \grad f(x_k)^T d_k  \label{step_length_1}\\
\grad f (x_k + \alpha_k d_k)^T d_k \ge c_2 \grad f(x_k)^T d_k \label{step_length_2}
\end{align}
with $0 < c_1 < c_2 < 1$.  Typically, one needs to only enforce too long of steps\ref{step_length_2} since there are no too short of steps\ref{step_length_1} due to backtracking schemes.


\subsection{Direct Search}
An overview of the convergence results for a standard direct search method will be given. Positive spanning sets as well as the cosine measure are used to bound the angle between the polling directions and the negative gradient.  Then, using the subsequence of unsuccessful iterates, the trial steps become arbitrarily small and $x$ will approach a limit point.  If $x_k$ enters a neighborhood of a stationarity point with $\grad^2 f(x) > 0$, it will converge to this point.

\begin{figure}
\centering
\begin{tabular}{c c}
\begin{tikzpicture}[ 
   scale=5,
    axis/.style={very thick, ->, >=stealth'},
    important line/.style={thick, color=red, ->},
    dashed line/.style={dashed, thin},
    pile/.style={thick, ->, >=stealth', shorten <=2pt, shorten
    >=2pt},
    every node/.style={color=black}
    ]
    % axis
    \draw[axis] (-0.1,0)  -- (1.1,0) node(xline)[right]
        {$x_1$};
    \draw[axis] (0,-0.1) -- (0,1.1) node(yline)[above] {$x_2$};
    % Lines
    \draw[pile] (.5,.5) coordinate (x) -- (.775,.5)
        coordinate (d2) node[right, text width=5em] {$d_2$};
    \draw[pile] (.5,.5) coordinate (x) -- (.3,.3)
        coordinate (d1) node[right, text width=5em] {$d_3$};
    \draw[pile] (.5,.5) coordinate (x) -- (.3,.7)
        coordinate (d3) node[right, text width=5em] {$d_1$};
    \fill[red]  (x) +(0,0) coordinate (.5,.5) circle (.4pt)
        node[above] {$x$};

    \draw[important line] (.5,.5) coordinate (x) -- (.78,.925)
        coordinate (gf) node[right, text width=5em] {$-\grad f(x)$};

     \draw[dashed line] (d2) -- (.78,.5);
     \draw[dashed line] (gf) -- (.78,.5);

     \path[clip] (.5,.5) -- (.78,.5) -- (.78,.925);

     \node[circle,draw=blue,minimum size=25pt] at (.5,.5) (circ) {};
     \node at(.62,.55) {$\theta$};


\end{tikzpicture}
&
\begin{tikzpicture}[ 
   scale=5,
    axis/.style={very thick, ->, >=stealth'},
    important line/.style={thick, color=red, ->},
    dashed line/.style={dashed, thin},
    pile/.style={thick, ->, >=stealth', shorten <=2pt, shorten
    >=2pt},
    every node/.style={color=black}
    ]
    % axis
    \draw[axis] (-0.1,0)  -- (1.1,0) node(xline)[right]
        {$x_1$};
    \draw[axis] (0,-0.1) -- (0,1.1) node(yline)[above] {$x_2$};
    % Lines
    \draw[pile] (.5,.5) coordinate (x) -- (.75,.5)
        coordinate (d2) node[right, text width=5em] {$d_2$};
    \draw[pile] (.5,.5) coordinate (x) -- (.5,.25)
        coordinate (d3) node[right, text width=5em] {$d_3$};
    \draw[pile] (.5,.5) coordinate (x) -- (.5,.75)
        coordinate (d1) node[ text width=3em] {$d_1$};
    \draw[pile] (.5,.5) coordinate (x) -- (.25,.5)
        coordinate (d4) node[above right, text width=5em] {$d_4$};
    \fill[red]  (x) +(0,0) coordinate (.5,.5) circle (.4pt)
        node[above left] {$x$};

    \draw[important line] (.5,.5) coordinate (x) -- (.78,.925)
        coordinate (gf) node[right, text width=5em] {$-\grad f(x)$};

     \draw[dashed line] (d1) -- (.5,.925);
     \draw[dashed line] (gf) -- (.5,.925);

     \path[clip] (.5,.5) -- (.78,.925) -- (.5,.925);

     \node[circle,draw=blue,minimum size=25pt] at (.5,.5) (circ) {};
     \node at(.535,.625) {$\theta$};
\end{tikzpicture}
\\
$N+1$ points & $2N$ points 
\end{tabular}
%    \draw[important line] (.15,.85) coordinate (C) -- (.85,.15)
%        coordinate (D) node[right, text width=5em] {$\mathit{NX}=x$};
    % Intersection of lines
 %   \fill[red] (intersection cs:
%       first line={(A) -- (B)},
%       second line={(C) -- (D)}) coordinate (E) circle (.4pt)
%       node[above,] {$A$};
    % The E point is placed more or less randomly
%    \fill[red]  (E) +(-.075cm,-.2cm) coordinate (out) circle (.4pt)
%        node[below left] {$B$};
    % Line connecting out and ext balances
%    \draw [pile] (out) -- (intersection of A--B and out--[shift={(0:1pt)}]out)
%        coordinate (extbal);
%    \fill[red] (extbal) circle (.4pt) node[above] {$C$};
    % line connecting  out and int balances
%    \draw [pile] (out) -- (intersection of C--D and out--[shift={(0:1pt)}]out)
%        coordinate (intbal);
%    \fill[red] (intbal) circle (.4pt) node[above] {$D$};
%    % line between out og all balanced out :)
 %   \draw[pile] (out) -- (E);

\caption{Positive spanning sets for $\R{2}$}
\end{figure}

A positively spanning set $\cG$ of $\R{n}$ can write any vector $v \in \R{n}$ as a positive combination of points $d_i \in \cG$,  $\beta_i \geq 0 \forall i$
\begin{equation}
v = \sum_i \beta_i d_i
\end{equation}
Kolda, Lewis, and Torczon \cite{kolda_2003} call this a generating set of $\R{n}$ which makes the foundation for their class of generating set search methods.  This is a large class of problems which generalize lattice methods of Berman \cite{berman_1966} \cite{berman_1969} as well as their own older methods \cite{torczon_1999} \cite{lewis_2000} and include the original Hookes and Jeeves method\cite{hooke_1961}.  Generating sets can be adapted for explicit linear constraints to conform to local topology.\endnote{\textbf{Explicit Linear Constraints}

These constraints can be dealt with in the framework of generating set search.  Here, one needs to concern themselves only with the nearby active constraints and a satisifactory set of polling directions would be a positive spanning set of the tangent cone of the nearby active constraints. 
\begin{figure}
\centering
\begin{tabular}{m{.625\textwidth} m{.3\textwidth}}
\begin{tikzpicture}[ 
   scale=7,
    axis/.style={very thick, ->, >=stealth'},
    locset/.style={very thick},
    important line/.style={thick, color=red, ->},
    dashed line/.style={dashed, thin},
    pile/.style={thick, ->, >=stealth', shorten <=2pt, shorten
    >=2pt},
    every node/.style={color=black}
    ]
    % axis
%  \draw[locset] (.45,0)  -- (.75,.3) -- (.75,.8) -- (.7,.9) --  (0,.9);
\filldraw[fill=green!20,draw=green!50!black] (0,0) -- (.45,0) -- (.75,.3) -- (.75,.7) -- (.55,.9) --  (0,.9) -- cycle; 

    \draw[axis] (-0.1,0)  -- (1.1,0) node(xline)[right]
        {$x_1$};
    \draw[axis] (0,-0.1) -- (0,1.1) node(yline)[above] {$x_2$};


    \fill[red]  (.65,.35) coordinate (x) circle (.4pt) ;

    \node[above left] at (x) {$x$};
     \node[circle,draw=blue,minimum size=75pt] at (x) (circ) {};

%    \draw[important line] (x) -- (1,.725)
%        coordinate (gf) node[right, text width=5em] {$-\grad f(x)$};  %% gradient
     \draw[thin,<->,color=blue] (x) -- node[above right]{$\epsilon$} +(5.38pt,0);  %% radius of circle
     
     \draw[very thick,color=blue!50!black] (.75,.3) -- (.75,.7);
     \draw[very thick,color=blue!50!black] (.45,0) -- (.75,.3);

    \draw[pile]  (x) -- +(0,.25)
        coordinate (d1) node[right, text width=5em] {$d_1$};

    \draw[pile]  (x) -- +(-.15,-.175)
        coordinate (d1) node[right, text width=5em] {$d_2$};


    \draw[pile,color=blue]  (.75,.575) -- +(.17,0)
        coordinate (n1) node[right, text width=5em] {};
    \draw[pile,color=blue]  (.55,.1) -- +(.12,-.12)
        coordinate (n2) node[right, text width=5em] {};

    \draw[thin,dashed,color=blue] (.55,.1) -- +(.04,-.04) -- +(.08,0) -- +(.04,.04);
    \draw[thin,dashed,color=blue] (.75,.575) -- +(.06,0) -- +(.06,.06) -- +(0,.06);

\end{tikzpicture}
&
\begin{tikzpicture}[
    scale=2,
    axis/.style={very thick, <->, >=stealth'}]

    \draw[thin] (-1.1,-1.1) -- (-1.1,1.1) -- (1.1,1.1) -- (1.1,-1.1) -- cycle;
    \filldraw[fill=blue!15!white,draw=blue!50!black] (0,0) -- (1.1,0) --  (1.1,-1.1) --   cycle; 
    \filldraw[fill=green!20,draw=green!50!black] (0,0) -- (0,1.1) -- (-1.1,1.1) --  (-1.1,-1.1) -- cycle; 
    \draw[axis] (-1.1,0)  -- (1.1,0) node(xline)[right] {$x_1$};
    \draw[axis] (0,-1.1) -- (0,1.1) node(yline)[above] {$x_2$};
    
    \node at (-.5,.5) {$\cK^o(x,\epsilon)$};
    \node at (.7,-.325) {$\cK(x,\epsilon)$};



\end{tikzpicture}
\\
\centering Note & \centering Note 2
\end{tabular}

\caption{Generating set for polar cone to conform to local topology}
\end{figure}
The compass search for standard bound constraints conforms to the constraints ideally and no modifications need to be made other than letting $f = \inf $ for $x$ out of bounds and not waste the time on the function evaluation.
}

By searching in all directions of a generating set of $\R{n}$, we a guaranteed to have a direction which is somewhat aligned with the decent direction $f$, if it is smooth and has a lipschitz continous derivative.  To make this matter concrete, the cosine measure of a set is the worst case scenario for the descent direction aligning with any direction of the generating set $\cG$.
\begin{equation}
\kappa
\end{equation}


\begin{algorithm}
\caption{Compass search, a generating set search}\label{dfo_genset}
\begin{algorithmic}
\Procedure{CS}{$f:\R{n} \rightarrow \R{}$}
\State $x_0 \in \R{n}$ Initial guess
\State $\bigtriangleup_{tol} >0$ Termination criteria
\State $\bigtriangleup_0 > \bigtriangleup_{tol} >0$ Initial neighborhood
\BState For each $k=1,2,...$

\State Let $\scd =\left\{ \pm e_i| i=1,...,n\right\}$ be the set of coordinate directions
\If{ $\exists d_k \in \scd$ such that $f(x_k + \btu_k d_k ) < f(x_k)$}
\State $x_{k+1} \gets x_k + \btu_k d_x$
\State $\btu_{k+1} = \btu_k$
\Else
\State $x_{k+1} \gets x_k$
\State $\btu_{k+1} = \frac{1}{2} \btu_k$
\If{$\btu_{k+1} < \btu_{tol}$} terminate \EndIf
\EndIf
\EndProcedure
\end{algorithmic}
\end{algorithm}


rational lattice - only a finite number of trial points can be considered before the step size must contract
%\subsubsection*{Mesh Adaptive}
Conn pg 140
can be transformed linearly, rotated etc.


\begin{figure}
\centering
\begin{tikzpicture}[ 
   scale=5,
    axis/.style={very thick, ->, >=stealth'},
    important line/.style={thick, color=red, ->},
    dashed line/.style={dashed, thin},
    pile/.style={thick, ->, >=stealth', shorten <=2pt, shorten
    >=2pt},
    every node/.style={color=black}
    ]
%  \draw [step=1.0,blue, very thick] (0.5,0.5) grid (5.5,4.5);
\draw [color=blue!40,opacity=.7, step=.1,xshift=0, yshift=0] (0,0) grid +(1.1,1.1);

    % axis
    \draw[axis] (-0.1,0)  -- (1.1,0) node(xline)[right]
        {$x_1$};
    \draw[axis] (0,-0.1) -- (0,1.1) node(yline)[above] {$x_2$};
    % Lines

        
    \fill[red]  (.4,.4) coordinate (x) circle (.4pt)
        node[above left] {$x_k$};

    \fill[blue]  (.4,.4) coordinate (x)+(0,.2) circle (.4pt)
        node[above left] {$d_k^1$};
    \fill[blue]  (.4,.4) coordinate (x)+(.2,0) circle (.4pt)
        node[above left] {$d_k^2$};
    \fill[blue]  (.4,.4) coordinate (x)+(0,-.2) circle (.4pt)
        node[above left] {$d_k^3$};
    \fill[blue]  (.4,.4) coordinate (x)+(-.2,0) circle (.4pt)
        node[above left] {$d_k^4$};

    \fill[brown]  (.4,.4) coordinate (x)+(.6,0) circle (.4pt)
        node[above left] {$h_k^1$};
    \fill[brown]  (.4,.4) coordinate (x)+(.3,.3) circle (.4pt)
        node[above left] {$h_k^2$};


%    \fill[red]  (.6,.4) coordinate (x) circle (.4pt)
%        node[above left] {$x_1$};

%    \fill[red]  (.8,.4) coordinate (x) circle (.4pt)
%        node[above left] {$x_2$};

\end{tikzpicture}

\caption{Polling step with exploratory trial points}
\end{figure}





\subsection{Model Based Search}
trust regions
Stochastic Trust Region \endnote{\textbf{Stochastic Trust Region}	(Chang and Wan) \cite{chang_2009}

\begin{equation}
\min_{x \in \mathbb{X}^p} g(x)
\end{equation}
with $g(x) = \Expect [ G(x,\omega) ]$, 
$G(x, \omega)$ is stochastic response
$\omega$ stochastic effect of system
$x$ is controllable parameter
\begin{equation}
G(x, \omega) = g(x) + \epsilon_x
\end{equation}
$g(x)$ uknown deterministic function
$\epsilon_x$ is random error induced by $\omega$, assume $\epsilon_x \sim F( \dot )$ where 
$F( \dot )$ is a general distribution, mean zero and variance $\sigma_x^2$

Assumptions
$G$ expectation bounded above
$g$ bounded below, twice differentiable, gradient and Hessian uniformly bounded
for a sufficiently small neighborhood, $G$ can be modeled as a quadratic with error $\epsilon_x$ 
--Taylor Teorem (Trench 2003)

Strong Law of Large Numbers
\begin{equation}
\overline{G}_N (x) = \frac{ \sum_{i=1}^N G(x,\omega_i) }{N} \rightarrow g(x) \mbox{    w.p. 1}
\end{equation}
as $N \rightarrow \infty$ for every $x \in \mathbb{X}^p$

Uses 2 stage process
initially, just use first order points
as trust region shrinks, there is transition points
begin to sample more, construct second order model, and reduce trust region at successful iterates
}

\subsubsection*{Approximate Gradient Methods}
Implicit filtering
Simplex gradients
Stochastic Approximation

Stochastic Gradient \endnote{\textbf{Randomized Stochastic Gradient} (Lan)

stochastic descent
zeroth order function values only
gaussian smoothhing function  (nesterov)

-mirror descent Meirovski 23,18
-sample average approximation 16, 36
-This
-subgradient averaging 13, 15, 26

\begin{equation}
f^* := \inf_{x \in \R{n}} f(x)
\end{equation}
$f: \R{n} \rightarrow \R{}$ differentiable
$\bigtriangledown$ Lipschitz continuous
\begin{equation}
|| \bigtriangledown f(y) - \grad f (x) || \le L || y - x ||
\end{equation}

stochastic gradient
$G (x_k, \xi_k$, $\xi_k$ random variable, distribution $P_k$ supported on $\Xi \subset \R{d}$
with following assumptions, get convergence results that are good
\begin{align}
\Expect \left[ G(x_k,\xi_k) \right] = \grad f (x_k)		\\
\Expect \left[ || G(x_k,\xi_k) - \grad f(x)||^2 \right] \le \sigma^2
\end{align}

get $(\epsilon, \Lambda)$ solution, $\epsilon > 0$, $\Lambda \in (0,1)$
st.
Prob $\left\{ || \grad f(\overline{x}) ||^2 \le \epsilon \right\} \ge 1-\Lambda$
is bounded
}

Stochastic Approximation \endnote{\textbf{Stochastic Approximation and Gradient Estimation}

\begin{equation}
x_{n+1} = \prod_\Theta \left[ x_n - a_n \hat{\grad} g(x_n) \right]
\end{equation}
$x_n$ solution at iteration $n$, $\hat\grad g(x_n)$ is estimate of gradient, $\lb a_n \rb$ sequance of positive real numbers called gain sequence, and $\prod_\Theta$ is projection onto feasibly region.

SA alg first proposed by Robbins and Monro (1951) - root finding algorithm
stochastic version of steepest descent algorithm
but gradient in SA is noisy estimate, convergence requires $\Expect \left[ \hat \grad g(x_n) \right] - g(x_n) \rightarrow 0$ at a certain rate
step size is often pre-determined, instead of adaptive
convergence requires $\sum_{n=1}^\infty a_n = \infty$ and $\sum_{n=1}^\infty a_n^2 < \infty$, algorithm sensitive to choice of $\lb a_n \rb$.  $\lb a_n \rb $ too large and $x_n$ oscillatory, $\lb a_n \rb $ too small, $x_n$ barely changes

simulation as black box $\grad g(x)$ by finite difference, use forward estimate with $d+1$ sim evals or central difference with $2d$ sim runs.
when dimension of $x$ large, not efficient due to number of sims required for one gradient evaluation
Spall (1992) proposed simultaneuous pertubation uses 2 sim runs to produce estimate
Spall (1998) gives nice intro to theory and application

When inside structure known, use this to get better gradient estimators
Fu (2008) gives excellent overview, pertubation analysis (PA), likelihood ratio/score function (LR/SF)
PA proposed by Ho and Cao (1983)
\begin{equation}
\grad g(x) = \grad \Expect [ Y(x) ] = \Expect [ \grad Y (x) ] 
\end{equation}
with $Y(x)$ stochastically lipschitz continuous and differentiable with prob 1.
overcome bad properties of $Y(x)$
smoothed pertubabtion analysis Gong and Ho (1987) Fu and Hu (1997)
pathwise method Hong (2009) Hong and Liu (2009)

LR/SF method in Reiman Weiss (1989) Glynn (1990)
represent $Y(x) = h(Z)$, $Z$ is vector of random variables generated in sim and $h(Z)$ is performance measure intereseted in .  $f_z(z,x)$ denote probability density of $Z$.  Note $x$ is only in density function of $Z$ and not $h(\dot)$.
then under some regularity conditoins
\begin{align}
\grad g(x) &= \grad \Expect [ h(Z) ] = \grad \int h(z) f_z(z,x) dz = \int h(z) \grad_x f_z(z,x) dz \\
	&= \Expect [ h(Z) \grad_x \log [ f_z(Z,x) ] ]
\end{align}
if $f_z(z,x)$ is known, then LR SF estimate $\grad g(x)$ by $h(Z) \grad_x \log [ f_z(Z,x) ]$. requires only single simulation run to comput, is unbiased estimaor, and does not require $Y(x)$ to be stochastic lipschitz continuous, applicabiliity often wider than PA, however variance of estimator is often high
}





\section{Optimizing the OPA Simulation}

First, we need to develop an efficient evaluation of cascading power failures for use in the optimization procedure.  For the needed resolution in function value, parallel computing will be used for concurrent function evaluation and common random numbers will be used to reduce the variance of the risk metrics.

After the OPA model is implemented, we will explore the behavior of the simulation with changes to our decision variables (transmission expansion variables from previous chapter).  We will look at different risk measures and compare their features, opting to use primarily expectation and conditional value at risk of load shed.  It is noted that even the expectation, which is typically smoothed under stochastic uncertainty, is non-convex nor monotonic for line capacity additions.  Additionally, initial stage load shed results are poor approximators of total load shed for a given cascade sequence.

Finally, we will determine the performance of direct search and model based methods on the OPA simulation.  A modified method will be developed to take advantage of accessory information within effective DFO strategies.  Accessory information produced by the OPA simulation is used to reduce the search space and find breakpoints in a line search method.  The performance of these methods will be shown and intuition gained from the solutions.

\subsection*{OPA Simulation}

The fast time scale OPA model (given in chapter one \ref{fast_opa} has been shown to have the same power-law distribution seen in real-world blackout data.  This simulation can be seen as a surrogate model for the response of the power system to rare-event stress.  As such, it is useful to explore the effects different parameters can have and even optimize over them to find any characterestics or trends their may be.

The linear program \ref{dcopf_program} is a load shedding versions of the standard DC OPF economic dispatch model.  
\begin{subequations}
\label{dcopf_program}
\begin{alignat}{3}
\min_{\left(x,\beta,l;\theta,y\right)} && \multispan{2}{$\displaystyle\sum_j \left[  c_2 x_j^2  + c_1 x_j + c_0 \right] + W \sum_i l_i$}   \label{jcc_obj}\\
                        && \textstyle \sum_j c^g_{ij} x_j - \sum_j c^b_{ie} y_e   +l_i       &=D_i       && \forall i \label{opf_cons}\\ 
                 && y_e - b_e \textstyle \sum_i c^b_{ie} \theta_i          &=0         && \forall e \label{opf_kcl}\\
                 && l_i &\in \left[ 0, D_i \right] && \forall e \label{opf_loadshed}\\
                 && y_e &\in \left[ -U_e, U_e \right] && \forall e \label{opf_limit}\\
                 && x_j &\in \left[ G^{min}_j, G^{max}_j \right] && \forall j  \label{opf_gen}  
\end{alignat}
\end{subequations}


It will be necessary to calculate the total load shed that results for different stages of the cascade, which is given by
\begin{equation}
D_{tot} = \sum_i \left[ D_i - l_i \right]
\end{equation}


\begin{algorithm}
\caption{OPA Cascading Algorithm}\label{opa_alg}
\begin{algorithmic}
\Procedure{LS}{$x, D, \xi, \omega$}
\State Solve ( \ref{dcopf_program} ) to find base case $D_{tot,0}$
\State $\xi$ occurs and corresponding changes to the grid are made
\State Stage $s \gets 1$
\While{ Not DONE }
\State Solve ( \ref{dcopf_program} ) to find power injects and branch flows for adjusted grid
\State $\mathbb{O}_s \gets \emptyset$
\For{$\forall e \in \cE $}
\State $\mathbb{O}_s = 
\left\{ 
\begin{array}{lr}
  \mathbb{O}_s + \left\{ e \right\} & \mbox{if } \left( y_e = U_e \mbox{ or } -y_e = U_e \right)  \mbox{ w/ prob. } p \\
  \mathbb{O}_s & \mbox{o/w }
\end{array}
\right. $ 
\If{ $\mathbb{O}_s \neq \emptyset$ }
\State Modify Grid with $\mathbb{O}_s, s\gets s+1$
\Else
\State $s^* \gets s,$ calculate $D_{tot,s^*}$, DONE
\EndIf
\EndFor
\EndWhile
\State \label{done}Load Shed $  LS = D_{tot,0} - D_{tot,s^*} $
\EndProcedure
\end{algorithmic}
\end{algorithm}

\subsection*{Implementation}
The simulation is implemented in C++ using CPLEX to solve the linear programs.  At each stage of the cascade, the solution may only require a small amount of pivots so that the individual LP solves are very fast and load shed evaluation's can be found in bulk.  The number of trial, $N_k$, will be selected so that the standard error of the function value is small relative to it's value.  The number of trials per function evaluation can be changed dependent on iteration count for improvements in speed or resolution.

The Center for High Throughput Computing provides computation resources for UW and afiliated researchers.  Jobs can be submitted through Condor \cite{beowulfbook-condor} which manages the collective pool of around 1 million cpu hours per day.  User's submit jobs to the cluster, which assigns resources that process the job.  Requirements can be given to ensure that the resource is capable of performaning the job.  In order to get access to a larger portion of the cluster, low memory and disk requirements help.  Overhead associated with the process, such as being assigned a resources and data transfers can be minimized but not removed. As such, the workflow was designed for job times of around 5-30 minutes and perform all analysis locally with the raw data to reduce network data exchanges.

In order to take advantage of a larger cluster \cite{condor_flock}, the main C++ program was compiled for two primary linux kernals that make up the majority of the cluster using the interactive shell job for Condor.  The memory requirements were set at 500MBs and storage at 3GBs, with plenty of buffer room for program operation.  The large data output from the OPA simulation includes stage by stage details of net power injects, branch flows, and load shed.  This data is then analyzed on the remote resource using python to find the risk metrics of the load shed as well as any accessory information computation.  The output of the analysis is less than 8K for load shed and outage data that is needed for most of the optimization routines.  Additional overhead is used to smooth the job requests and restart hanged jobs. 

\subsubsection*{Common Random Numbers}
The stochastic uncertainty of the OPA process leads to large standard errors for the risk measures.  Variance reduction techniques are important to reduce the computational burden and a common random number scheme was employed.  Common random numbers essentialy gives each test system the same set of experimental conditions.  By doing so, the variance in the difference between two systems is reduced and less computational resources will be needed. \cite{law_2007}

The OPA simulation uses effective capacities to determine whether a line fails for a particular stage and loading.  In order for the alternative configurations to be under similar experimental conditions, a random number seeding strategy was used to ensure alternative configurations would recieve the same luck.  Formally, we can see why this works in some cases.  Suppose we have two systems with expected load shed $L_{ij}$ for systems $i=1,2$ and trials $j=1,2,...N$.  Now lets study the metric $Z_j = L_{1j} - L_{2j}$ and let the true comparison be $\Exx{Z}=\mu_Z$.  In order to make decisions about this, we need to be fairly confident in our estimation $\hat{Z}(n)=\sum_j \frac{Z_j}{n}$ of $\mu_z$.  The standard error of our sample mean $\hat{Z}(n)$ is
\begin{equation}
\Var{ \hat{Z}(n) } = \frac{ \Var{X_1} + \Var{X_2} - \CoVar{X_1}{X_2} }{n}
\end{equation}

In order for this to reduce the variance, we need $\CoVar{X_1}{X_2}>0$.  This makes sense, as a sample path with consistently low effective capacities should do worse in all systems.  In practice, this common random number scheme is very successful for our problem and makes comparing systems less costly.


\subsection{Model Behavior}
To investigate the behavior of the model, we will begin by looking at the results from simple changes to the decision variables.  The risk measures, table \ref{tab:risk}, are calculated for the load shed distribution and the expectation and the conditional value at risk are used for optimization.  The load shed from this cascading power simulation follows a power law distribution which cause tail events to have a relatively large effect.  The expected value and conditional value at risk could be surrogate models of each other as they tend to track in a similar manner. The maximum can display much more erratic behavior and optimizing against this would be costly and restrictive.


\newcommand{\tabheight}{11pt}
\begin{table}
\centering
\begin{tabular}{| c | c | c|}
\hline
& & \\[1pt]
Sample Mean & $f^{ex}$ &$ \hat{Z}(n)=\sum_j \frac{Z_j}{n} $\\[\tabheight]
Sample Variance & $s^2_d$ &$ S^2(n)=\sum_j \frac{\left(Z_j - \hat{Z}(n)\right)^2}{n-1} $\\[\tabheight]
Standard Error & $s^2_e$ &$ \Var{\hat{Z}(n)}= \frac{S^2(n)}{n}$\\[\tabheight]
Confidence Interval& $CI(1-\eta)$  & $f^{ex} \pm z_{1-\eta/2} \sqrt{s^2_e}$ \\[\tabheight]
Value at Risk & $ VaR(\eta)$& $ \lb Z_{a} |  a = \floor{ \eta n }\rb $\\[\tabheight]  %$\inf\left\{l \in \R{} | F_L(l) \geq \eta \right\}$\\[\tabheight]
Conditional Value at Risk & $ CVaR(\eta)$& $\Exx{Z | Z \geq VaR(\eta)}$ \\[\tabheight]
\hline
\end{tabular}
\caption{Risk Measures}\label{tab:risk}
\end{table}

\begin{figure}
\centering
\documentclass{standalone}
\usepackage{tikz}
\usepackage{pgfplots}
\newcommand{\mypathdfodata}{../data}
\begin{document}
	\begin{tikzpicture}
	\begin{axis}[ title=\mbox{ Risk Measures },legend pos=outer north east, scale=1, xlabel=Capacity (MW), ylabel=Load Shed (MW)]			
\addplot+[black,opacity=.5, mark size=.5,only marks] table[x=name, y=ex] {\mypathdfodata/done.dat};
\addlegendentry{Expect}
	\addplot+[blue,opacity=.5, mark size=.5,solid,error bars/.cd, y dir=both, y explicit] table[x=name, y=ex, y error=se] {\mypathdfodata/done.dat};
\addlegendentry{St. Error}
	\addplot+[red,opacity=.15, mark size=.5,solid,error bars/.cd, y dir=both, y explicit] table[x=name, y=ex, y error=st] {\mypathdfodata/done.dat};
\addlegendentry{St. Dev}
	\addplot+[purple,opacity=.715, mark size=.5,solid] table[x=name, y=var] {\mypathdfodata/done.dat};
\addlegendentry{5\% V@R}
	\addplot+[purple,opacity=.815, mark size=.5,solid] table[x=name, y=cvar] {\mypathdfodata/done.dat};
\addlegendentry{5\% CV@R}
	\addplot+[purple,opacity=.8515, mark size=.5,solid] table[x=name, y=max] {\mypathdfodata/done.dat};
\addlegendentry{Maximum}
%		\addlegendentryexpanded{$\i$ - Ex}
	\end{axis}	
	\end{tikzpicture}
\end{document}

\caption{Value at Risk, Conditional Value at Risk, and Maximum risk measures for load shed distribution}
\end{figure}



\subsubsection{Capacity Addition}

%\endnote{\textbf{Double the Capacity}

To begin, we will look at what happens when you double capacity on each line.  (\ref{fig:grad})  Out of the 186 lines, 51 lines resulted in an improved system with respect to the cascading process, 47 were within the margin of error of the nominal system, and 88 reduced the system performance.


\begin{figure}
\begin{center}
	\begin{tikzpicture}
	\begin{axis}[ title=\mbox{ Load Shed for Capacity Addition on Coordinate Directions},legend pos=north east ,scale=1.12, xlabel=Line Number, ylabel=Load Shed (MW)]	
			\addplot+[opacity=.6775, mark size=.915, only marks,error bars/.cd, y dir=both, y explicit] table[x=name, y=ex, y error=se] {./data/grad.dat};
\addlegendentry{Coordinate Directions};
			\addplot+[opacity=.1175, mark size=.915, red, only marks,error bars/.cd, y dir=both, y explicit] table[x=name, y=ex, y error=se] {./data/base.dat};
\addlegendentry{Nominal System}
	\end{axis}	
	\end{tikzpicture}
\caption{Comparing the total load shed for doubling capacity along the coordinate direction.}
\label{fig:grad}
\end{center}

\end{figure}

Lets look along a line in each group.  It varies alot.  Less than 40\% of lines lead to a reduction in expected load shed when their capacity was doubles 


\begin{figure}


\begin{subfigure}[b]{0.3\textwidth}
\centering
	\begin{tikzpicture}
	\begin{axis}[title=\mbox{Line 1},scale=.4568]			\addplot+[opacity=.5, mark size=.5, smooth,solid] table[x=name, y=ex, y error=se] {./data/done.dat};
%		\addlegendentryexpanded{$\i$ - Ex}
	\end{axis}	
	\end{tikzpicture}
\end{subfigure}
\begin{subfigure}[b]{0.3\textwidth}
\centering
	\begin{tikzpicture}
	\begin{axis}[title=\mbox{Line 2}, scale=.4568]			\addplot+[opacity=.5, mark size=.5, smooth,solid] table[x=run, y=ex, y error=se] {./data/clusters/n2.dat};
%		\addlegendentryexpanded{$\i$ - Ex}
	\end{axis}	
	\end{tikzpicture}
\end{subfigure}
\begin{subfigure}[b]{0.3\textwidth}
\begin{center}
	\begin{tikzpicture}
	\begin{axis}[title=\mbox{Line 3},scale=.4568]			\addplot+[opacity=.5, mark size=.5, smooth,solid] table[x=run, y=ex, y error=se] {./data/n1.dat};
%		\addlegendentryexpanded{$\i$ - Ex}
	\end{axis}	
	\end{tikzpicture}
\end{center}
\end{subfigure}

	\caption{Plotting expected load shed for capacity additions along coordinate directions.  }
\end{figure}
}



\begin{figure}
\centering
\newcommand{\numb}{negEX}
\newcommand{\cols}{16}
\begin{tabular}{c c}
\begin{tikzpicture}[scale=.8]
\begin{axis}[
    colormap/jet,
    colorbar,mesh/cols=\cols,
    view={25}{40}]
    \addplot3[surf,shader=flat] table{\mypathdfodata/heatmapnegEX.dat};
\end{axis}
\end{tikzpicture}
&
\begin{tikzpicture}[scale=.8]
\begin{axis}[
    colormap/jet,
    colorbar,mesh/cols=\cols,
    view={25}{90}]
    \addplot3[surf,shader=flat] table{\mypathdfodata/heatmapnegEX.dat};
\end{axis}
\end{tikzpicture}
\\
Surface plot of Ex for 2D search & Heatmap of Ex for 2D search \\
\end{tabular}

\caption{2d heat map}
\end{figure}


Clustering \endnote{\textbf{Clustering}

Lines tend to behave in groups which relate to underlying topology.
Take advantage by looking at correlation between load shed and line outages
\begin{figure}
\centering
\documentclass{standalone}
\usepackage{tikz}
\usepackage{pgfplots}
\newcommand{\mypathdfodata}{../data}
\begin{document}
%\documentclass{standalone}
%\usepackage{tikz}
%\usepackage{pgfplots}
%\begin{document}
\foreach \i in {n3}
{
	\begin{tikzpicture}[scale=1,line width=1]
	\begin{axis}[scale=1.23 , ylabel=Load Shed, xlabel=Capacity , legend style={at={(1.5,.97)},anchor=north east}]

		\addplot+[opacity=1,mark size=2,solid,thick,black] table[x expr=\thisrowno{0}*.1, y=ex] {\mypathdfodata/clusters/m-\i.dat};
	\addlegendentry{Expect}
%		\addlegendentryexpanded{$\i$ - St. Dv.}	
	\end{axis}
	\begin{axis}[ axis y line*=right,title=\mbox{  Line C },scale=1.23, legend style={at={(1.5,.85)},anchor=north east} ,ylabel=Frequency, y label style={at={(1.25,.5)}}, xtick=\empty ]	
	\foreach \j in {163,164,165,166,168,170,171,173,175,176}{
		\addplot+[opacity=.4, mark size=.4, solid] table[x expr=\thisrowno{0}*.1, y=\j] {\mypathdfodata/clusters/\i.ldat};
		\addlegendentryexpanded{Line $\j$}
	}

	\end{axis}	
	\end{tikzpicture}
}
%\end{document}
\end{document}

\caption{A cluster of lines responsible for reducing system performance}
\label{fig:cluster}
\end{figure}

}


\subsubsection{Direct Search Application}

To get an idea of the strengths and weaknesses of direct search on the OPA simulation, the standard compass search algorithm was implemented on a reduced 2 dimensional subspace for your viewing pleasure.  These figures show the rough search space and the need for filtering higher frequency effects.

%Conditional Value at Risk \endnote{\textbf{Conditional Value at Risk}

\begin{figure}
\centering
\documentclass{standalone}
\usepackage{tikz}
\usepackage{pgfplots}
\newcommand{\mypathdfodata}{../dfo/data}
\begin{document}

\begin{tabular}{c c}
\begin{tikzpicture}[scale=.8]
\begin{axis}[
    colormap/jet,
    colorbar,mesh/cols=16,
    view={25}{40}]
    \addplot3[surf,shader=flat] table{\mypathdfodata/heatmapnegCVAR.dat};
\end{axis}
\end{tikzpicture}
&
\begin{tikzpicture}[scale=.8]
\begin{axis}[
    colormap/jet,
    colorbar,mesh/cols=16,
    view={25}{90}]
    \addplot3[surf,shader=flat] table{\mypathdfodata/heatmapnegCVAR.dat};
\end{axis}
\end{tikzpicture}
\\
Surface plot of CVAR for 2D search & Heat map of CVAR for 2D search
\end{tabular}


\end{document}

\caption{2d heat map}
\end{figure}
}



Here we can see the compass search get trapped in a local minima.  If a small step size is used, the search will get trapped in nearby minima when more progress can be made else where.  This highlights the need for a strategy to find local minima that are nearer to the global minimum.
\begin{figure}
\centering
\documentclass{standalone}
\usepackage{tikz}
\usepackage{pgfplots}
\newcommand{\mypathdfodata}{../dfo/data}
\begin{document}
\begin{tikzpicture}
\begin{axis}[
    colormap/jet,colorbar,
    mesh/cols=22,
    view={0}{90},xmin=45,xmax=55,ymin=-10,ymax=10]
    \addplot3[surf,shader=flat,opacity=.25] table{\mypathdfodata/heatmap_trial.dat};
\end{axis}
\begin{axis}[title=Trial Points,legend pos=north east, xmin=45,xmax=55,ymin=-10,ymax=10]
\addplot+[only marks, circle, mark size=1.5, opacity=.6] table[x=x1, y=x2]{\mypathdfodata/trial.dat};
\addlegendentry{Trial}
\addplot+[only marks, red, mark size=1] table[x=x1, y=x2]{\mypathdfodata/select.dat};
\addlegendentry{Selection}
\addplot+[only marks, black,opacity=.5, mark size=2] table[x=x1, y=x2]{\mypathdfodata/reduce.dat};
\addlegendentry{Reduce}
\end{axis}
\end{tikzpicture}
\end{document}

\caption{Trial exploration using standard compass search}
\end{figure}

%\endnote{\textbf{Compass Search}

Implementation of compass search with different initial trial steps.

\begin{figure}
\centering
\begin{tabular}{c c}
\begin{tikzpicture}[scale=.8]
\begin{axis}[title=Trial Points]
\addplot+[only marks, circle, mark size=1, opacity=.95] table[x=num, y=ex]{\mypathdfodata/minobj.dat};
\addlegendentry{Compass Search 1}%$x_0 = (0,0), \Delta_0=25$
\addplot+[only marks, circle, mark size=1, opacity=.95] table[x=num, y=ex]{\mypathdfodata/objmin25.dat};
\addlegendentry{Compass Search 25}%$x_0 = (0,0), \Delta_0=50$
\addplot+[only marks, circle, mark size=1, opacity=.95] table[x=num, y=ex]{\mypathdfodata/minobj2.dat};
\addlegendentry{Compass Search 50}%$x_0 = (0,0), \Delta_0=50$
\addplot+[only marks, circle, mark size=1, opacity=.95] table[x=num, y=ex]{\mypathdfodata/objmin75.dat};
\addlegendentry{Compass Search 75}%$x_0 = (0,0), \Delta_0=50$
\end{axis}
\end{tikzpicture}
&
\begin{tikzpicture}[scale=.8]
\begin{axis}[
    colormap/jet,colorbar,
    mesh/cols=16,
    view={0}{90},xmin=-5,xmax=80,ymin=-5,ymax=80]
    \addplot3[surf,shader=flat,opacity=.25] table{\mypathdfodata/heatmap_opt.dat};
\end{axis}
\begin{axis}[title=Trial Points,legend pos=north east, xmin=-5,xmax=80,ymin=-5,ymax=80]
\addplot+[mark size=4] coordinates { (.969,1.63) };
\addplot+[mark size=4] coordinates { (26.56,37.5) };
\addplot+[mark size=4] coordinates {  (53.32, -1.56) };
\addplot+[mark size=4] coordinates { (74.56,75) };
\end{axis}
\end{tikzpicture}
\\
Note 1 & Note 2
\end{tabular}

\caption{Function Value}
\end{figure}
}







\subsubsection*{1st Stage Approximators}
Is there any good way to approximate the total effect of the cascading process?  One common technique in DFO methods is to find a surrogate function which has similar properties to your real function but is faster to compute.  We looked at two different statistics to see if it was possible to infer how bad a cascade would be in the early stages.  Unfortunately, the initial stages of the cascade were poorly correlated with the total load shed for any given sample path.

Figure \ref{fig:first} shows a scatter plot of load shed in the 1st stage with load shed in the final stage.
%\endnote{\textbf{First Stage Line Outages}

Poor Estimator
\begin{figure}
\centering
\foreach \i in {0, 1, 5,10,15,20}{
	\begin{tikzpicture}
	\begin{axis}[ scale=.6575, xlabel=Load Served (MW) Stage \i , ylabel=Load Served (MW) Final ]	
			\addplot+[opacity=.46775, mark size=.915, only marks] table[x=ls\i, y=ls30] {./data/long.corr};
%		\addlegendentryexpanded{$\i$ - Ex)
	\end{axis}	
	\end{tikzpicture}
 %\label{fig:lmp}
}

 \caption{Load Served at Different Stages}\label{fig:loadserve}
\end{figure}



}
Visual inspection of this for various initial contingencies showed that any calculation using this would be a poor estimator.  This is not surprising since we the rare event cases have a disproportionitly high influence due to the power law distribution.  More stages need to progress to find the highly disruptive cases.  Figure \ref{fig:loadserve} shows the same scatter plot for stages 0, 1, 5, 10, 15, and 20 to show the progression.  Even in later stages, the undetermined cases still vary.

\begin{figure}
\centering
\documentclass{standalone}
\usepackage{tikz}
\usepackage{pgfplots}
\newcommand{\mypathdfodata}{../data}
\begin{document}
	\begin{tikzpicture}
	\begin{axis}[ title=\mbox{ Load Shed versus First Stage Line Outages },xlabel=Number of Failed Lines, ylabel=Load Shed]	
 			\addplot+[opacity=.3875, mark size=.65, only marks] table[x=numlines, y=ex] {\mypathdfodata/done.fst};
	\end{axis}	
	\end{tikzpicture}
\end{document}

\caption{The poor relationship between load shed and the number of lines outaged in the first stage.}
\label{fig:first}
\end{figure}

We can actually find more information in the actual line flow statistics for different stages in the cascade.  Changes with respect to 1st stage line flows can identify changes in final stage load shed, although not whether it will improve or degrade the performance.
Can estimate relative effects of cascade on different systems
\begin{figure}
\centering
%	\begin{tikzpicture}
	\begin{axis}[ title=\mbox{ Line Flow by Line over Scenarios  },legend pos=north east ,scale=1.1572, ymax=500,xlabel=Line Number,ylabel=Flow (MW) ]	
 			\addplot+[opacity=.1297598375, mark size=.67, only marks] table[x=line, y=flow] {./data/small.lfi};
	\addlegendentry{Scene. Flow}
			\addplot+[red,opacity=.995, mark size=.915, only marks] table[x=line, y=flow] {./data/small.lfa};
		\addlegendentry{Avg. Flow}
	\end{axis}	
	\end{tikzpicture}

%	\begin{tikzpicture}
	\begin{axis}[ title=\mbox{ Failure Probability by Line over Scenarios },legend pos=north east ,scale=1.4572, ymax=.55, xlabel=Line Number,ylabel=Proability of Failure ]	
 			\addplot+[opacity=.19758375, mark size=.67, only marks] table[x=line, y=prob] {./data/small.pob};
	\addlegendentry{Scene. Prob}
			\addplot+[red,opacity=.9895, mark size=.915, only marks] table[x=line, y=prob] {./data/small.pba};
		\addlegendentry{Avg. Prob}


	\end{axis}	
	\end{tikzpicture}

 \caption{Line flows and failure probabilities for each scenario and  their averages}
\label{fig:flows}
\end{figure}
%\endnote{\textbf{Failure Probabilities}

\begin{figure}
\centering
%	\begin{tikzpicture}
	\begin{axis}[ title=\mbox{ Line Flow by Line over Scenarios  },legend pos=north east ,scale=1.1572, ymax=500,xlabel=Line Number,ylabel=Flow (MW) ]	
 			\addplot+[opacity=.1297598375, mark size=.67, only marks] table[x=line, y=flow] {./data/small.lfi};
	\addlegendentry{Scene. Flow}
			\addplot+[red,opacity=.995, mark size=.915, only marks] table[x=line, y=flow] {./data/small.lfa};
		\addlegendentry{Avg. Flow}
	\end{axis}	
	\end{tikzpicture}

	\begin{tikzpicture}
	\begin{axis}[ title=\mbox{ Failure Probability by Line over Scenarios },legend pos=north east ,scale=1.4572, ymax=.55, xlabel=Line Number,ylabel=Proability of Failure ]	
 			\addplot+[opacity=.19758375, mark size=.67, only marks] table[x=line, y=prob] {./data/small.pob};
	\addlegendentry{Scene. Prob}
			\addplot+[red,opacity=.9895, mark size=.915, only marks] table[x=line, y=prob] {./data/small.pba};
		\addlegendentry{Avg. Prob}


	\end{axis}	
	\end{tikzpicture}

 \caption{Line flows and failure probabilities for each scenario and  their averages}
\label{fig:flows}
\end{figure}
}



\section{Algorithm Design}
\subsection{Search Direction and Break Points}
most outaged lines
refine with 	-topology information
		-electrical properties
clustering
		-line correlations
\subsection*{Line Search}
breakpoint analsis




This information can be used to tell if two operating points have different cascading properties.  By looking at the distance between two operating points, breakpoints can be found where the distance between operating points risk characteristics are extremely different from another, close, operating point.
%Break Points \endnote{\textbf{Break Points}

\begin{figure}
\centering
\documentclass{standalone}
\usepackage{tikz}
\usepackage{pgfplots}
\newcommand{\mypathdfodata}{../data}
\begin{document}
\newcommand{\inum}{2}         
\begin{tabular}{c c}
         \begin{tikzpicture}
	\begin{axis}[ scale=.8272, ,legend pos=south east, xlabel={\small Capacity (MW)}, ylabel={\small Load Shed (MW)},xmax=95]	
			\addplot+[opacity=.456775, mark size=.915, only marks,error bars/.cd, y dir=both, y explicit] table[x=cap, y=ex, y error=se] {\mypathdfodata/breakpoint/done.dat};
			\addplot+[red,opacity=.995, mark size=.915, line width=1.25] table[x=cap, y=ls] {\mypathdfodata/breakpoint/nbhd\inum.dat};
\addlegendentry{Num. Pts: \inum};

	\end{axis}
	\end{tikzpicture}
\renewcommand{\inum}{3}         
&
         \begin{tikzpicture}
	\begin{axis}[ scale=.8272, ,legend pos=south east, xlabel={\small Capacity (MW)}, ylabel={\small Load Shed (MW)},xmax=95]	
			\addplot+[opacity=.456775, mark size=.915, only marks,error bars/.cd, y dir=both, y explicit] table[x=cap, y=ex, y error=se] {\mypathdfodata/breakpoint/done.dat};
			\addplot+[red,opacity=.995, mark size=.915, line width=1.25] table[x=cap, y=ls] {\mypathdfodata/breakpoint/nbhd\inum.dat};
\addlegendentry{Num. Pts: \inum};

	\end{axis}
	\end{tikzpicture}

\\

\renewcommand{\inum}{5}         
         \begin{tikzpicture}
	\begin{axis}[ scale=.8272, ,legend pos=south east, xlabel={\small Capacity (MW)}, ylabel={\small Load Shed (MW)},xmax=95]	
			\addplot+[opacity=.456775, mark size=.915, only marks,error bars/.cd, y dir=both, y explicit] table[x=cap, y=ex, y error=se] {\mypathdfodata/breakpoint/done.dat};
			\addplot+[red,opacity=.995, mark size=.915, line width=1.25] table[x=cap, y=ls] {\mypathdfodata/breakpoint/nbhd\inum.dat};
\addlegendentry{Num. Pts: \inum};

	\end{axis}
	\end{tikzpicture}
&
\renewcommand{\inum}{10}         
         \begin{tikzpicture}
	\begin{axis}[ scale=.8272, ,legend pos=south east, xlabel={\small Capacity (MW)}, ylabel={\small Load Shed (MW)},xmax=95]	
			\addplot+[opacity=.456775, mark size=.915, only marks,error bars/.cd, y dir=both, y explicit] table[x=cap, y=ex, y error=se] {\mypathdfodata/breakpoint/done.dat};
			\addplot+[red,opacity=.995, mark size=.915, line width=1.25] table[x=cap, y=ls] {\mypathdfodata/breakpoint/nbhd\inum.dat};
\addlegendentry{Num. Pts: \inum};

	\end{axis}
	\end{tikzpicture}

\\

\renewcommand{\inum}{25}         
         \begin{tikzpicture}
	\begin{axis}[ scale=.8272, ,legend pos=south east, xlabel={\small Capacity (MW)}, ylabel={\small Load Shed (MW)},xmax=95]	
			\addplot+[opacity=.456775, mark size=.915, only marks,error bars/.cd, y dir=both, y explicit] table[x=cap, y=ex, y error=se] {\mypathdfodata/breakpoint/done.dat};
			\addplot+[red,opacity=.995, mark size=.915, line width=1.25] table[x=cap, y=ls] {\mypathdfodata/breakpoint/nbhd\inum.dat};
\addlegendentry{Num. Pts: \inum};

	\end{axis}
	\end{tikzpicture}
&
\renewcommand{\inum}{50}         
         \begin{tikzpicture}
	\begin{axis}[ scale=.8272, ,legend pos=south east, xlabel={\small Capacity (MW)}, ylabel={\small Load Shed (MW)},xmax=95]	
			\addplot+[opacity=.456775, mark size=.915, only marks,error bars/.cd, y dir=both, y explicit] table[x=cap, y=ex, y error=se] {\mypathdfodata/breakpoint/done.dat};
			\addplot+[red,opacity=.995, mark size=.915, line width=1.25] table[x=cap, y=ls] {\mypathdfodata/breakpoint/nbhd\inum.dat};
\addlegendentry{Num. Pts: \inum};

	\end{axis}
	\end{tikzpicture}
\end{tabular}
\end{document}

  \caption{Approximation of function by finding breakpoints}
\label{fig:break}
\end{figure}
} choosing N

%\subsection{Local Convergence}
derivative free techniques
find neighborhood with continuity
drive towards local optimal
search basis
exploratory steps

\subsection{Modified Coordinate Search}
All together now

%\section{Computational Results}





%\subsection{MCS Implementation}
%Negative - No local convergence properties

\endnote{\textbf{Brute Force Serach}

Old problem parameters, which is why the scales are so different

\begin{figure}
\begin{center}
\documentclass{standalone}
\usepackage{tikz}
\usepackage{pgfplots}
\newcommand{\mypathdfodata}{../data}
\begin{document}

	\begin{tikzpicture}
	\begin{axis}[ title=\mbox{ Pattern Search Routine },legend pos=outer north east, scale=1, xlabel=Capacity (MW), ylabel=Load Shed (MW)]			
\addplot+[black,opacity=.5, mark size=.5,only marks,smooth,solid] table[x=run, y=ex] {\mypathdfodata/opt1.dat};
\addlegendentry{Expect}

	\addplot+[blue,opacity=.5, mark size=.5,solid,smooth,error bars/.cd, y dir=both, y explicit] table[x=run, y=ex, y error=sd] {\mypathdfodata/opt1.dat};
\addlegendentry{St. Dev}
	\addplot+[purple,opacity=.615, mark size=.5,solid,smooth] table[x=run, y=var] {\mypathdfodata/opt1.dat};
\addlegendentry{5\% V@R}
	\addplot+[purple,opacity=.815, mark size=.5,solid,smooth] table[x=run, y=cvar] {\mypathdfodata/opt1.dat};
\addlegendentry{5\% CV@R}
	\addplot+[purple,opacity=1, mark size=.5,solid,smooth] table[x=run, y=max] {\mypathdfodata/opt1.dat};
\addlegendentry{Maximum}
%		\addlegendentryexpanded{$\i$ - Ex}
	\end{axis}	
	\end{tikzpicture}

\end{document}


\bc{
	\begin{tikzpicture}
	\begin{axis}[ title=\mbox{  Pattern Search Routine - Expected Load Shed } ,xlabel=Iteration,ylabel=Load Shed (MW) ]	
 	
		\addplot+[opacity=.5, mark size=.5, smooth,solid,thick,black, error bars/.cd, y dir=both, y explicit] table[x=run, y=ex, y error=se] {\mypathdfodata/opt1.dat};
		\addlegendentry{Ex}
		\addplot+[opacity=.5,mark size=.75, smooth,solid] table[x=run, y=max] {\mypathdfodata/opt1.dat};
		\addlegendentry{Max}	
		\addplot+[opacity=.5,mark size=.75, smooth,solid] table[x=run, y=var] {\mypathdfodata/opt1.dat};
		\addlegendentry{VaR 5\%}	
%		\addplot+[opacity=.5,mark size=.75, smooth,solid, green] table[x=run, y=sd] {\mypathdfodata/opt1.dat};
%		\addlegendentry{St. Dv.}	
%		\addlegendentry{Expectation}


	\end{axis}	
\end{tikzpicture}
\begin{tikzpicture}
	\begin{axis}[ title=\mbox{Pattern Search Routine - Tail Measures},xlabel=Iteration]
%
		\addplot+[opacity=.5,mark size=.75, smooth,solid] table[x=run, y=max] {\mypathdfodata/opt1.dat};
		\addlegendentry{Max}	
		\addplot+[opacity=.5,mark size=.75, smooth,solid] table[x=run, y=var] {\mypathdfodata/opt1.dat};
		\addlegendentry{VaR 5\%}	
	\end{axis}


	\end{tikzpicture}



	\addplot+[blue,opacity=.5, mark size=.5,solid,smooth,error bars/.cd, y dir=both, y explicit] table[x=run, y=ex, y error=se] {\mypathdfodata/opt1.dat};
\addlegendentry{St. Error}

}

 \caption{Brute force search procedure along coordinate directions in the null space}
 \label{fig:opt1}
\end{center}
\end{figure}
}


\section{Conclusion}


\theendnotes
\setcounter{endnote}{0}

