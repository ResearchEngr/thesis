\documentclass[11pt]{report}
\usepackage{standalone}
\usepackage{amsmath,amssymb,amsfonts} % Typical maths resource packages
\usepackage{graphicx}                 % Packages to allow inclusion of graphics
\usepackage{color}                    % For creating coloured text and background
\usepackage[table]{xcolor}
\usepackage{hyperref}                 % For creating hyperlinks in cross references
\usepackage{algorithm}
\usepackage{algorithmic}
\usepackage{longtable}
\usepackage{import}
\usepackage{lscape}
\usepackage{booktabs,colortbl}
\usepackage{tikz}
\usetikzlibrary{patterns}
\usepackage{pgfplots}
\usepackage{pgfplotstable}

\usepackage{cleveref}
\usepackage{caption}
\usepackage{subcaption}
    
\graphicspath{{./img/}}



%\setlength{\textwidth}{15cm}
\parindent 1cm
\parskip 0.2cm
\topmargin 0.2cm
\oddsidemargin 1cm
\evensidemargin 0.5cm
\textwidth 15cm
\textheight 21cm
\newtheorem{theorem}{Theorem}[section]
\newtheorem{proposition}[theorem]{Proposition}
\newtheorem{corollary}[theorem]{Corollary}
\newtheorem{lemma}[theorem]{Lemma}
\newtheorem{remark}[theorem]{Remark}
\newtheorem{definition}[theorem]{Definition}


%\newcommand{\mag}[1][mag]{ \left| {#1} \right| }
\newcommand{\magf}{ \left| f \right| }
\newcommand{\magp}{ \left| p \right| }
\newcommand{\magr}{ \left| r \right| }
\newcommand{\magomega}{ \left|  \omega  \right| }
\newcommand{\magE}{ \left| \mathcal{E} \right| }
\newcommand{\magG}{ \left| \mathcal{G} \right| }
\newcommand{\magL}{ \left| \mathcal{L} \right| }
\newcommand{\magM}{ \left| \mathcal{M} \right| }
\newcommand{\magT}{ \left| \mathcal{T} \right| }
\newcommand{\magV}{ \left| \mathcal{V} \right| }

\newcommand{\R}[1]{\mathbb{R}^{#1}}
\def\S{\mathbb{ S}}
\def\I{\mathbb{ I}}


\newcommand{\cA}{\mathcal{A}}
\newcommand{\cB}{\mathcal{B}}
\newcommand{\cD}{\mathcal{D}}
\newcommand{\cCALI}{\mathcal{CALI}}
\newcommand{\cDC}{\mathcal{DC}}
\newcommand{\cGR}{\mathcal{GR}}
\newcommand{\cE}{\mathcal{E}}
\newcommand{\cF}{\mathcal{F}}
\newcommand{\cG}{\mathcal{G}}
\newcommand{\cL}{\mathcal{L}}
\newcommand{\cN}{\mathcal{N}}
\newcommand{\cM}{\mathcal{M}}
\newcommand{\cO}{\mathcal{O}}
\newcommand{\cS}{\mathcal{S}}
\newcommand{\cT}{\mathcal{T}}
\newcommand{\cV}{\mathcal{V}}
\newcommand{\cX}{\mathcal{X}}
\newcommand{\cU}{\mathcal{U}}

\newcommand{\Expect}{\mathbb{E}}
\newcommand{\Prob}{\mathbb{P}}
\newcommand{\cvar}{\mathbb{CV}{\mathsf @}\mathbb{R}}
\newcommand{\grad}{\bigtriangledown}
\newcommand{\lb}{\left\{}
\newcommand{\rb}{\right\}}

\newcommand{\defeq}{\stackrel{\rm def}{=}}




\tikzset{
        hatch distance/.store in=\hatchdistance,
        hatch distance=10pt,
        hatch thickness/.store in=\hatchthickness,
        hatch thickness=2pt
    }

\makeatletter
    \pgfdeclarepatternformonly[\hatchdistance,\hatchthickness]{flexible hatch}
    {\pgfqpoint{0pt}{0pt}}
    {\pgfqpoint{\hatchdistance}{\hatchdistance}}
    {\pgfpoint{\hatchdistance-1pt}{\hatchdistance-1pt}}%
    {
        \pgfsetcolor{\tikz@pattern@color}
        \pgfsetlinewidth{\hatchthickness}
        \pgfpathmoveto{\pgfqpoint{0pt}{0pt}}
        \pgfpathlineto{\pgfqpoint{\hatchdistance}{\hatchdistance}}
        \pgfusepath{stroke}
    }

\makeindex


\title{ Reliability Model }
\begin{document}

This model will attempt to move away from these critical points for a given budget.  A critical point is when the system is at a maximum throughput operating point, that is, it is constrained by line limits or generator limits.  Power systems tend towards these spots due to economic pressures.  Critical points are the cheapest way to provide power if there were no concern of outages or uncertainty.

Two types of critical points
\begin{itemize}
\item Inadequate generation capacity - leads to a lot of smaller blackouts
\item Inadequate transmission system capacity - leads to few large blackouts.
\end{itemize}

First solve economic dispatch model subject to power flow constraints to find least cost dispatch at cost $B^*$.

Use reliability focused objective and move cost function to constraint.  Now the system can move away from the least cost operating point along some ``reliability frontier" for an increased budget.
\begin{subequations} \label{rel}
\begin{align} 
\max	& \hspace{10px} \lambda r_g(g) + (1-\lambda) r_f(f) \\
\mbox{s.t.}	& \hspace{10px}c(g) \le B^* + \delta	\\
  &	\hspace{10px}\mbox{ power flow constraints} 
\end{align}
\end{subequations}
Here $\lambda$ will be used to express the trade off between lots of small outages or few large outages.  $\delta$ represents the increased budget to find a new operating point.  $c(g)$ is the cost function for the generator dispatch $g$.  As $\delta \rightarrow 0$, \ref{rel} will approach the economic dispatch point. $f$ is power flows on the transmission lines.  

An example of possible reliability functions may be to maximize a weighted excess capacity for generators and transmission lines.
\begin{align}
r_g(g) &= \sum_i \left[ \alpha_i (g_i^+ - g_i ) + \beta_i (g_i - g_i^-) \right]\\
r_f(f) &= \sum_{ij} \gamma_{ij} \left( \frac{U_{ij} - f_{ij}}{U_{ij}} \right)
\end{align}
Here $\alpha$ and $\beta$ will represent the relative need for up-ramping and down-ramping capabilities as well as the desired geographical location of ramping rates through the relative size of $\alpha_i$ at any node $i$.  $\gamma_{ij}$ can be developed using electrical information of the transmission line to ensure that the core network is more stable.  More weight can be point on critical lines to the power grid.


\end{document}
