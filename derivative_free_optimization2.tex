\subsection{1st Stage Approximators}
Any good way to approximate the total effect of the cascading process?

Correlation between number of lines outaged in 1st stage and the total load shed.
\begin{figure}
\begin{center}
	\begin{tikzpicture}
	\begin{axis}[ title=\mbox{ Load Shed versus First Stage Line Outages },xlabel=Number of Lines Outaged, ylabel=Load Shed]	
 			\addplot+[opacity=.3875, mark size=.65, only marks] table[x=numlines, y=ex] {./data/done.fst};
	\end{axis}	
	\end{tikzpicture}
\caption{Relationship between load shed and the number of lines outaged in the first stage.}
\label{fig:first}
\end{center}
\end{figure}

Correlation between

\begin{figure}
\begin{center}
\foreach \i in {0, 1, 5,10,15,20}{
	\begin{tikzpicture}
	\begin{axis}[ scale=.6575, xlabel=Load Served (MW) Stage \i , ylabel=Load Served (MW) Final ] %, xmin=1600,xmax=3700,ymin=1600,ymax=3400]	
			\addplot+[opacity=.46775, mark size=.915, only marks] table[x=ls\i, y=ls30] {./data/long.corr};
%		\addlegendentryexpanded{$\i$ - Ex)
	\end{axis}	
	\end{tikzpicture}
%  \caption{Locational Marginal Prices (LMP) for MG\&E serving Madison, Wisconsin.  Wednesday, Aug 4th, 2010. }
 %\label{fig:lmp}
}

\end{center}
\end{figure}

Good news, can estimate relative effects of cascade on different systems
1st stage line flows
\begin{figure}
\begin{subfigure}[b]{\textwidth}
\centering
	\begin{tikzpicture}
	\begin{axis}[ title=\mbox{ Line Flow by Line over Scenarios  },legend pos=north east ,scale=1.4572, ymax=500,xlabel=Line Number,ylabel=Flow (MW) ]	
 			\addplot+[opacity=.1297598375, mark size=.67, only marks] table[x=line, y=flow] {./data/small.lfi};
	\addlegendentry{Scene. Flow}
			\addplot+[red,opacity=.995, mark size=.915, only marks] table[x=line, y=flow] {./data/small.lfa};
		\addlegendentry{Avg. Flow}
	\end{axis}	
	\end{tikzpicture}
\end{subfigure}

\begin{subfigure}[b]{\textwidth}
\begin{center}
	\begin{tikzpicture}
	\begin{axis}[ title=\mbox{ Failure Probability by Line over Scenarios },legend pos=north east ,scale=1.4572, ymax=.55, xlabel=Line Number,ylabel=Proability of Failure ]	
 			\addplot+[opacity=.19758375, mark size=.67, only marks] table[x=line, y=prob] {./data/small.pob};
	\addlegendentry{Scene. Prob}
			\addplot+[red,opacity=.9895, mark size=.915, only marks] table[x=line, y=prob] {./data/small.pba};
		\addlegendentry{Avg. Prob}


	\end{axis}	
	\end{tikzpicture}

\end{center}
\end{subfigure}
 \caption{Line flows and failure probabilities for each scenario and  their averages}
\label{fig:flows}
\end{figure}

This information can be used to tell if two operating points have different cascading properties.  By looking at the distance between two operating points, breakpoints can be found where the distance between operating points risk characteristics are extremely different from another, close, operating point.

\section{Algorithm Design}
\subsection{Search Direction}
most outaged lines
refine with 	-topology information
		-electrical properties
clustering
		-line correlations
\subsection{Line Search}
breakpoint analsis


\begin{figure}
\begin{center}
\foreach \i in {2, 3, 5,10,25,50}{
	\begin{tikzpicture}
	\begin{axis}[ scale=.8272, ,legend pos=south east, xlabel={\small Capacity (MW)}, ylabel={\small Load Shed (MW)},xmax=95]	
			\addplot+[opacity=.456775, mark size=.915, only marks,error bars/.cd, y dir=both, y explicit] table[x=cap, y=ex, y error=se] {./data/breakpoint/done.dat};
			\addplot+[red,opacity=.995, mark size=.915, line width=1.25] table[x=cap, y=ls] {./data/breakpoint/nbhd\i.dat};
\addlegendentry{Num. Pts: \i};

	\end{axis}
	\end{tikzpicture}
}
  \caption{Approximation of function by finding breakpoints}
\label{fig:break}

\end{center}
\end{figure}



choosing N
\subsection{Local Convergence}
derivative free techniques
find neighborhood with continuity
drive towards local optimal
search basis
exploratory steps

\subsection{Algorithm}
All together now

\section{Analysis of Model Output}

\subsection{Modified Coordinate Search}
Negative - No local convergence properties
Old problem parameters, which is why the scales are so different

\begin{figure}
\begin{center}
	\begin{tikzpicture}
	\begin{axis}[ title=\mbox{  Opt Routine }  ]	
 	
		\addplot+[opacity=.5, mark size=.5, smooth,solid,thick,black, error bars/.cd, y dir=both, y explicit] table[x=run, y=ex, y error=se] {./data/opt1.dat};
		\addlegendentry{Ex}
		\addplot+[opacity=.5,mark size=.75, smooth,solid, green] table[x=run, y=sd] {./data/opt1.dat};
		\addlegendentry{St. Dv.}	
%		\addlegendentry{Expectation}


	\end{axis}	
	\begin{axis}[ axis y line*=right, legend pos=outer north east]
%
		\addplot+[opacity=.5,mark size=.75, smooth,solid] table[x=run, y=max] {./data/opt1.dat};
		\addlegendentry{Max}	
		\addplot+[opacity=.5,mark size=.75, smooth,solid] table[x=run, y=var] {./data/opt1.dat};
		\addlegendentry{VAR}	
	\end{axis}


	\end{tikzpicture}
 \caption{Brute force search procedure along coordinate directions in the null space}
 \label{fig:opt1}
\end{center}
\end{figure}


%\end{document}

