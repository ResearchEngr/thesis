
\section{Simulation Based Analysis}
	This section will develop the statistical tools to understand the output of this random process.  The mean load shed, $\hat l $, and the $\alpha$ value at risk, $l_\alpha$, will be calculated based on a set of trials.  In order to find the confidence in the metric, sets of the metrics from many experiments will be used.

\subsection{Load Shed}

	Let $\cL$ be a set of output from the OPA algorithm \ref{opa_alg}. The the load shed of the experiment is  
\begin{equation}
	\hat l = \frac{\sum_{i \in \cL} l_i }{ \magL }   
\end{equation}

	The $\alpha$ value at risk is also a useful quantity to measure.  This value represents the $(1-\alpha)$ percentile of trials in a given experiment.  Since blackouts occur with a power law distribution, it will be important to calculate the value of the rare events.
\begin{equation}
	VaR_{\alpha}(\cL) = \mbox{max} \left\{  l \in \R : \sum_i 1_{l_i \leq l } \leq ( 1 - \alpha ) \magL \right\}  
\end{equation}

	The conditional value at risk is the weighted sum of all trials with a load shed equal to or greater than the $VaR_{\alpha}$.
\begin{equation}
		CVaR_{alpha}(\cL) = \sum_{ l \in \cL : l \geq VaR_{\alpha}(\cL)} l
\end{equation}


\subsection{Sample Variance}

For two sets of metrics, a refrence experiment, $\cM_0$, and a designed system experiment, $\cM$.
\begin{equation}
	\hat m = \frac{\sum_{m \in \cM, m_0 \in \cM_0} ( m - m_0 )}{ \magM }
\end{equation}

The standard deviation can then be calculated and an $alpha \%$ confidence interval can be found.
\begin{equation}
 StDv( \cM ) = \sqrt{ \frac{\sum  \left( m - m_0 - \hat{m} \right)^2 }{  \left( \magM-1 \right)  }} 
\end{equation}

With the following bounds, $ \underline{ \hat{m} } = \hat{m} - t_{\alpha,\magM-1} \frac{S}{\sqrt{\magM}} $, 
			$ \overline{ \hat{m} } = \hat{m} + t_{\alpha,\magM-1} \frac{S}{\sqrt{\magM}} $ \newline
The following statement holds true.
\begin{equation}
 P (  \underline{ \hat{m} } \leq m \leq  \overline{ \hat{m} } ) \geq (1-\alpha) 
\end{equation}


\subsection{Common Random Number}\label{crn}
 To ensure that $\omega$ is calculated under similiar experimental conditions, we employ a common random number approach.  For each trial $t$, every branch $k$ has a seed for a stream of random numbers.  In round $r$, the $r$th random number in the $k$th random number stream should be used.\\
 \begin{algorithmic}
 \FOR{ $k \in \mathbb{K} $}
 		\IF{ $f_k = F_k \mbox{ or } f_k = -F_k $}
 					\STATE Random Number Seed $ \gets RNS_{t,k} $
 					\FOR{ $temp \in 1,2, ..., r -1 $}
 							\STATE Generate Random Number
 					\ENDFOR
 					\STATE $p \gets U \left[ 0, 1\right] $
 					\STATE ${k} \mbox{ added to outage if } Fail(p) = 1 $
 		\ENDIF
 \ENDFOR
 \end{algorithmic}


It seems as if regardless of our approach, it may be very useful to
read and understand the seminal works by Calafirori and Campi and
Luedtke on the topic of SAA for chance constraints.
