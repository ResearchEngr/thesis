
\subsection{Calibrating against $N-1$}
In order to create a useful data set, we take advantage of the knowledge that power grids are required, and thus engineered, to withstand any $N$ - 1 contingency, that is, any line or generator can be taken out of the system and it is still possible to meet all demand.  The following algorithm ensures that the grid is capable of withstanding any continigency, $\xi$ in the $N$ - 1 set. $\gamma$ represents the typical execess capacity on a power line.
\begin{equation}
g \prime = \cCALI ( g, d, \gamma ) 
\end{equation}
where $g \prime $ is the parameters that are robust to the demand profile $d$.  

Given nominal grid $g_{nom} = g_{0,0} $, repeat the following until the system stabilizes, that is $g_{e,t} = g_{ e, \rho (t) } $.
For each $e \in \cE$, perform the following while keeping the relation $g_{e,0} = g_{ e-1, t^* }$.
\begin{align}
	( f , d ) = \cDC (g_{e,t}, d_{nom} )  \\
\mbox{ if } \left| f_i \right| = u_i, 
	\mbox{ then }  u_i = \gamma f_i
\end{align}
which expands the transmission capacity in order to not be overloaded if line e fails.  