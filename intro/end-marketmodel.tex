\textbf{Market Model Shortcomings}

The market model should put a price on destabilizing effects.  In order to illustrate some of the shortcomings of the current market, two examples will be used.  The first example will look at generators and highlight the fact that they are not getting paid according to their value to the system.  The second will look at demand and highlight that all demands are not equal and thus should not pay the same marginal price.

The first example is of 2 generators at the same bus of a power system and they have equal marginal cost at some point in time.  Generator A is outputting at its maximum output level but generator B is below its maximum output level.  Generator A is only able to ramp down but generator B is able to ramp up or ramp down.  Generator B provides additional flexibility to the system in order to find a least cost power flow and also increase stability of the system.  In today's market, this generator could bid this capacity into regulation services and get paid.  However, if these generators are not used for ancillary services, they get paid at an equal rate despite offering unequal services.   Furthermore, the regulation and reserve markets are set up so that you can bid into the markets as long as you meet some minimum requirements.  This puts unequal services on the same level and when you are finding a least cost solution, you are likely to deploy inferior quality services.  The market should be able to place the correct value on the true varying levels of service.

The second example is of two different demand profiles at the same node and their demand does not affect the locational marginal price (LMP).  Demand A is a fixed demand and doesn't fluctuate at all.  Demand B has an average demand equal to demand A, but half the time is at twice A's level and half the time at 0.  Since they both consume the same amount of power, these demands both pay the same rate.  But these demands place unequal stress on the system.  If all demand was type B, additional resources would need to be put on the grid to meet demand.  Type B demand also increases the regulation needed to ensure reliability of the grid.  Currently, the regulation services bought to support type B demand is paid for by all demand equally.  The market should be able to place the correct price on the varying levels of stress demand puts on the grid.     

The market model needs to account for the cost of volatility to the system.  Currently, this cost is socialized to the whole system through the cost of ancillary services such as reserves and regulation support instead of being bore by those that cause the instabilities.  In order to properly account for these costs and reward those offering these services, a new market model will be made.  The strengths and weakness of this model will be showed, in particular the effects it has on the likelihood of  cascading power failures.

%A key tool in building the market model will be critical points. The reliability of the electricity grid can be controlled by varying the distance between the operating positions and various critical points.  Using a list of contingencies, costs associated with each contingency can be measured by its new distance to bad operating points.  Once we have a measure, we will have a value to place on entities causing stress to the system and reward those providing the services.  This market will be able to take over both the regulation and reserve markets.  Regulation and reserve markets are incapable of properly pricing reliability issues, which is exactly what they are designed to do.  Instead, due to having set points for how much reserve and regulation they have, there is no ability to toggle how reliable the system is.  However, we know that the reliability of power systems fluctuates seasonally and daily, but the level of support the system is getting is not.  In addition, with the ability to have a tangible measure for reliability, the consumers of electricity will have a better quality product.  The power systems will now have the ability to adjust the cost of electricity for a trade-off in reliability.  This will link the economic and reliability issues of the power grid.  It will allow a uniform system to apply to various tough problems in electricity markets such as demand response and energy storage.

%If we look at an uncertain demand and note it has some average consumption rate over a time interval and a certain amount of variance from its mean, we can see that a load with zero variance will be the cheapest possible load.  Any positive amount of variance will move the system closer to critical points at some point in time over the given interval.  Loads with extremely high variance will have to pay even more as they will move the system even closer to the bad operating points.  Energy storage devices and demand response will be able to play an important role in these markets as well.  It is unlikely that we stumbled onto the reliability frontier and this model should help the system move closer.  

