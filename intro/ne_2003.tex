
\subsection{Northeast 2003 Blackout} 

%This account of the Northeast Blackout is drawn from the \textit{Final Report on the August 14, 2003 Blackout in the United States and Canada} by the U.S.-Canada Power System Outage Task Force.\cite{northeast_2003}  

\subsubsection{Historical Problems}
The blackout began in the Cleveland-Akron area of the Eastern Interconnection.  This area had a history of problems of low voltages due to the relatively high amount of imported power.  In 1998, the system was becoming unstable and everything besides load shed was done to fix the problem.  It was shown to not be within the reliability standards, and regulations were loosened instead of addressing the problem.  A transformer problem led American Electric Power (AEP), a control area entity, to perform a reliability study on the neighboring control area operated by First Energy (FE).  The study again showed voltage instability problems.

The summer of 2003 was fairly typical with less than historical peak load.  While the load was less than historal peak, it was consistently greater than the forecasted load.  High temperatures creating a greater air conditioning load contributed to reactive power shortages.  The voltages were low in Cleveland-Akron during the week but within the new operating limits (92\% to 105\%).  A group of shunt capacitors out of service for planned maintenance further reduce reactive power supply.  Also, a large nuclear reactor, Davis-Besse, was in a long term forced outage state.  This plant is normally able to provide a large amount of real power as well as reactive power support.

\subsubsection{Normal Day Turns Bad}
A series of initiating events as well as the failure of FE's control server led to an unstable system and the inability for operators to do anything about it.  In addition, neighboring regional reliability coordinators (Midwest Independent System Operator, MISO, and Pennsylvania-New Jersey-Maryland Interconnect, PJM) had bad information on the state of the system and were unable to provide support.  The operators knew about voltage concern and were trying to get additional reactive power online.  The shunt capacitors were unable to return to operation.

The following initiating events didn't cause the system to collapse, but moved it into an unreliable state, which allowed the collapse to take place. The Stuart-Atlanta 345 kV line tripped due to contact with surrounding vegetation (area voltage at 98.5\%) at less than 100\% capacity.  In addition to the loss of the line, MISO was unaware, which led to unusable output and the inability to provide support.  An important generating unit, Eastlake 5, attempts to increase reactive power output, but the internal protection trips the generator offline (area voltage 97.3\%).  This led to real power imports rising, which increased the need for reactive power as well.  Another transmission line loss of Harding-Chamberlin 345 kV continued to depress the voltage (95.9\%).  Around this time, the underlying 138 kV network began to fail.   MISO was unable to perform N-1 contingency analysis and the FE reliability charts flatlined due to the server problems.

Ultimately, it was in reliable state before 15:05, however, after all of these events, the system was no longer N-1 stable.  In addition, reliability coordinators were using separate state information due to poor communication.  

\subsubsection{System Becomes Unreliable}

A total of 3 lines tripped at below operating capacity due to tree contact starting with Harding-Chamberlin (44\% of capacity) at 15:05.  Then the Hanna-Juniper line tripped (88\% normal and emergency rating) at 15:32.  The Star-South Canton had multiple tree contacts (55\%) starting at 14:27, and finally tripped off for good (93\% emergency rating)  at 15:41.  There was no prior sustained outages from tree contact in the previous years and FE used vegetation management practices consistent with the industry.  When designing line limits, the thermal ratings are based on many variables including ambient temperatures and wind speeds.  A combination of higher than nominal ambient temperature and lower wind speeds reduced the cooling capacity of lines.

Perry plant, the largest local generation, was getting voltage spikes at levels close to tripping the generator.  They notified FE of the problems, but the operators were unable to recover from this precarious situation.  The task force notes that load shed here of 1500MW may have saved the system by increasing voltages and stopping the ultimate trip of Sammis Star line.

\subsubsection{High Speed Cascading Failures}
At 16:05 the final straw was pulled by the tripping of the Sammis-Star 345kV line seperating geographical regions of the grid.  Unplanned power shifts across regions caused phase 3 operations tripping lines far away from the problem area.  The grid continued to stabilize after each one of these individual outages.  However, by 16:10, the north and south separated (AEP separates from FE), and the high generation area was no longer directly connected to the high load center.  This caused a massive power surge from PJM through New York and Ontario counter clockwise around the great lake to load centers in Michigan. This surge caused the Northeast to separate from rest of Eastern Interconnect (EI).

There was insufficient generation in the newly formed islands to support the load.  The frequency in remaining parts of EI rose to 60.3\%, representing 3700 MW of excess generation.  The Northeast grid kept breaking apart until island were formed in which an equillibrium between load and generation could be made.  Under-frequency and under-voltage load shedding helped to stabilize the system within the islands.  New York dropped 10,648 MW through automatic load shedding and Ontario dropped 7800 MW.  Some generators tripped off at unreasonable levels making island stabilization more difficult.  Thousands of events occured between 16:10 and 16:13.  

The cascade spread not due to voltage problems but dyanmic power swings which caused system instability.  The voltage instability was only companion to the angle instability (represents large real power flow swings) which tripped generators and lines.  The large power swings come from imbalance between generation and load across regions and the electrical separation of these two areas.  The inherent weak points are lines with the highest impedence, which trip off relatively early due to protective relay settings.  These are typically long over-head lines with high loadings.


\subsubsection{The Blackout Results}
In the United States, 45 million people lost power totaling 61,800 MW in Ohio, Michigan, Pennsylvania, New York, Vermont, Massachusetts, Connecticut, and New Jersey.  The US loss estimate was between \$4 billion and \$10 billion.  Another 10 million people from Canada lost power leading to an estimated 18.9 million lost man hours and \$2.3 billion in lost manufacturing shipments.   There were at least 11 fatalities, power took up to 4 days to return, and rolling blackouts continued in Ontario for the following week.\cite{northeast_2003}

The formation of a large island, based off of hydro plants in western New York and Canada, was the basis for system restoration.


