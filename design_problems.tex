\subsection{Design Problems}
These are examples of design problems, which would aim to make the system more robust for a given budget.

\subsubsection{Transmission Expansion}
Add or expand capacity on transmission system

\subsubsection{Vegetation Management}
Find important areas to be extra vigilent with vegetation practices
\cite{qi_2013}

\subsubsection{Allocating Reserves}

REGULATION AND RESERVES

In this case an operating configuration need not be given.  The program can either search for an operating configuration as well as reserves or given an operating configuration, allocate the reserves.  These operating reserves determine where the system is able to relieve congestion in a contingency.

\subsubsection{Generator Redispatch}
This model tries to find a operating configuration that is within a given distance from the current operating regime and minimizes the worst case scenario outage.  The input for this model is the current operating configuration $(p_0, d_0)$ as well as a distance vector $\delta$ that represents how far each generator can move from its current output levels.  In this case, the root node of the stochastic tree will have variables that represent initial generator output levels and the child trees will be constrained from how far they can move from this.  Also, a continuous variable will be added that represents the amount of load shed in the worst case scenario.  The objective will be to minimize the load shed in the worst case scenario.  
This program could be used when the system is becoming close to unstable and cost concernes become less of a priority than system stability.

\subsubsection{Energy Storage}
The ability to store energy and dispatch it creates an arbitrage opportunity on the time-varying price of electricity.  Arbitrage is the process of taking advantage of price differences in different markets.  Here, the different markets are simply the points in time during the day, at which each point must have a supply of electricity greater than demand.  The demand for electricity has both daily and seasonal variation.  The price of electricity tends to be low during the night and winter and high during the day and particularly summer.  To engage in arbitrage over the daily variation of electricity prices, one would store energy during the night when the prices are low and then sell during the day when the prices are high.  

TYPES OF ENERGY STORAGE

Chemical, Thermal, Electrical, Momentum

There are two forms of arbitrage availble based on the electricity markets.  The first type is a risk-free arbitrage that earns a profit and no possibility of negative cash flows by operating strictly on the day-ahead market.  The other type is a statistical arbitrage which utilizes the day-ahead and real-time market and refers to a positive expected profit.    

The operation and placement of these devices could be done in such a way as to improve the reliability of the system.  Since we know that outages are more likely near stressed points, by reducing the volatility of the system, we will have lowered peak loadings and be moving away from critical points.
