
\subsubsection{Pseudo-Topological Measures}

Electrical structure
Data about topological structure are not sufficient to describe the performance of power networks (24)
connectivity between components
-Sensitivity matrix
--Power transfer distribution factor matricies(41)
-Electrical distances (42) - earliest work
Lagonotte (1989) Structural analysis of the electrical system: Application to secondary voltage control in france
--application to reliability and economic power system problems (43-46)
Lu (1995) A new formulation of generator penalty factors
Hang (2000) A fast voltage security assessment method using adaptive bounding
Zhong (2004) Localized reactive power markets using the concept of voltage control areas
Hines (2008) A centrality measure for electrical networks
Wang (2010) Electrical centrallity measures for electric power grid vulnerability analysis

-Voltage phase angles between areas as measure of stress in power networks (47)
Power Flow Jacobian matricies
$ \Delta P = \frac{ \partial P }{ \partial \theta} \Delta \theta + \frac{ \partial P }{ \partial | V | } \Delta | V |$
assume voltages held constant
$\frac{ \partial P }{ \partial \theta} $ is a Laplacian matrix
Set $G = \frac{ \partial P }{ \partial \theta} $
$e(a,b) = g_{a,a}^{-1} + g_{b,b}^{-1} - g_{a,b}^{-1} - g_{b,a}^{-1}$
$E$ satisfies properties of distance matrix under dc power flow assumption, and empiraclly held otherwise
analogus to node degree
$e_a = \sum_{b=1}^n \frac{e_{ab}}{n-1}$
$c_a = e_a^{-1}$ centrality

topological distances have exponential tail, electrical distances have power-law tails
weak correlation between two types of distances
picture of electric centrality seem to point out very well importance of each node to grid stability
$E$ is weighted and fully connected $n(n-1)$ links
$R$ with $r_{ab} = 1 $ if $e_{ab} < t$, with $t $ adjusted to produce $m$ links
$R$ has lots of nodes with no connection, with the interpretation that few nodes have disproportionate influence on a large portion of the nework

MAY BE ABLE TO USE THIS TO FIND AREAS TO IMPROVE THE NETWORK!!!!!


%%%%%%%%%%%%%%%%%%%%%%%%%%%%%%%%%%%%%%%%%%%%%%


Ohm and Kirchoff law not captured well in simple topological models
relationship between physical properties and topo metrics (17,18,10) sometimes correlate to performance

GOAL: Compare vulnerability conclusions from topo measure of vulnerability with more realistic model of power network failure


USed cascading simulation to compare results to some common topological models.  The topo models appear to have erroneous results


Critical slowing down - half of paper, may be interesting to write some up for more depth of generator dynamics


%%%%%%%%%%%%%%%%%%%%%%%%%%%%%%%%%%


  Similar papers have used similar model (4) (19) (24)  (OPA, IMPROVED OPA)
