
\chapter{Conclusion}
In this thesis we explored rare event risk on the bulk power system.  From the recent literature of cascading power failures, we built upon the OPA cascading power failure simulation.  We started by modeling this process as a multi-stage stochastic program with integer variables in \Cref{msip-model}.  We introduced the concept of effective capacity to model the decision-dependent uncertain inherent in this cascading process.  This model allows for the flexibility of using the cascading process as a sub-problem in long term design problems.  We then turn to Monte Carlo simulation in order to parallelize the computational process to get better resolution of the cascade evolution uncertainty in \Cref{dfo-chapter}.  We look at the effects of transmission expansion on the OPA simulation and optimize this using derivative free optimization techniques and large computational resources through HTCondor.  In the second half of the thesis, we switch to real-time dispatch models.  In \Cref{jcc-chapter} we develop a system risk measure that constrains the probability of one or more lines failing.  Also, we model uncertainty in generation and demand and translate this to uncertainty in branch flows by calculating the branch covariance matrix.  Using these two models, we approximate this system risk measure under uncertainty and use a cutting plane algorithm to solve this convex problem.  Finally, in \Cref{ch:jccow} we combine the rare event risk model of the OPA simulation with the real-time line failure dispatch model.  We extend the joint chance constraint to the N-1 contingencies and use these contingencies to seed the OPA model.  Based off of these results, we develop a linear weighting system to approximate how important each line is with respect to the OPA cascade simulation.  We use this linear weighting system and a cutting plane algorithm to solve this convex optimization problem.







\section{Contributions}
\subsubsection{Modeling Casacading Power Failures}
In \Cref{msip-model}, we explore the models of power flow over topological networks, cascading risk models, and a system equilibrium model (OPA) in which there is a balance found between economic efficiency and reliability.  For economic reasons, power systems move towards critical points, which are points of maximum throughput for the given infrastructure and are characterized by power flow being limited by either transmission constraints or generation capacity limits.  A critical point primarily due to transmission constraints lead to larger blackouts, however infrequently, and a critical point primarily due to generation capacity lead to smaller blackouts that are more frequent.  In the OPA model, there is an engineering response to these blackouts to improve the infrastructure.  An equilibrium is found in the distribution of blackout size such that the load shed for these events follow a power tail distribution.  This equilibrium has been seen in the real world, the distribution of load shed events follow a power tail distribution with exponent $-1.3 \pm .2$ and has stayed the same for 30 years.  We model the short term cascading process as a multi-stage stochastic program that can be embedded in design problems.

The primary contribution was to model the existing cascading power failure simulation as a multi-stage stochastic program using mixed integer variables.  The primary difficulty was overcoming the decision dependent uncertainty, which was done by using the concept of effective capacity distribution and a priori sampling.  While the computational difficulty made practical use of this impossible for realistically-sized power systems.  The analytical modeling of the simulation process and decision dependent uncertainty will hopefully find application in other power system problems.

\subsubsection{Using the OPA Simulation for Transmission Expansion}
In \Cref{dfo-chapter}, we decompose the MSIP model by scenario and use a fast Monte Carlo simulation approach that is parallelizable to evaluate the load shed distribution given demand and contingency scenarios.  Here we aim to improve the system before an event occurs.  As we saw in the Northeast Blackout of 2003, once things go wrong, events that the system is typically robust to can compound and eventually lead to large scale blackout.  For example, after losing the Stuart-Atlanta 345 kV line, the fact that MISO was unaware of this event, led to the inability to provide support in dealing with this problem.  One way to reduce the likelihood of these rare events is to reduce the likelihood of the initiating contingencies.  We proceeded to explore this load shed distribution by looking at the design problem of increasing capacity on transmission elements and the effects it has on the OPA simulation.  This OPA simulation has a natural connection with our risk model that we explore in this chapter.

The primary contribution was to optimize over the functional landscape defined by OPA simulation efficiently.  We looked at the design problem of transmission expansion and explored the characteristics of this risk function.  We developed an efficient implementation with a common random number scheme to reduce the variance of function estimates and effective software engineering to allow for massively parallel solution methodology.  In addition, DFO techniques were used to filter the high frequency noise and exploratory steps using accessory information from the OPA simulation enhanced the solution time.  A computational experiment demonstrating how the system could effectively find system configurations improving load shed was performed.

\subsubsection{Line Failure Risk Models for Real-Time Dispatch}
In \Cref{jcc-chapter}, we developed a model to constrain the probability that one or more lines fail to be small.  This is in line with the idea in \Cref{dfo-chapter} where we would like to reduce the likelihood of initiating events occuring.  To start, we noted that due to the increased penetration in renewables and modern technologies of electric vehicles and energy storage, it is vital to include uncertainty of net injections into the model.  We then look at the chance constraint models in literature that deal with this uncertainty in net injections.  Another interesting class of models in literature approached the reliability problem from a system perspective.  Since we don't want any lines to fail, we develop a joint chance constraint on the probability that one or more lines fail.  In order to model this, we assume that the failure density function of an individual line is a piecewise linear function and that line failure probabilities are independent of each other given the branch flow.  However, the branch flow is certainly correlated, as it is the flows on fixed topological structure with fixed power flow parameters.  For the DC approximation of power flows, the branch flows follow a linear relationship with net injections.  If we assume that the net injections are a multivariate Gaussian with a known covariance matrix it follows that the branch flows are multivariate Gaussian and we can calculate its covariance matrix.  We now make our first approximation by taking a linearization ofthesystem risk measure using a Taylor expansion and dropping higher order terms.  This is a very good approximation for small failure probabilities and it allows us to pass this multivariate Gaussian through our piecewise linear failure density function.  We finish this model by calculating the expected probability of a line failing for each branch and then sum over all branches to get our system risk measure, which is constrained.

The primary contribution was to develop a dispatch model that included a system risk constraint to control endogenous risk from line loading.  This system risk measure was modeled as a joint chance constraint that we can solve  exactly when there is no uncertainty in generation and demand.  When there is net injection uncertainty, we approximate the system risk measure and solve the model approximately using a cutting plane algorithm.  The cost-risk frontier was explored and the new model was compared to the standard chance constrained model used in recent literature.  

\subsubsection{Reducing Cascading Risk Through Real-Time Dispatch}
In \Cref{ch:jccow}, we extended our endogenous system risk model to the N-1 contingencies.  We also evaluate the N-1 contingencies using OPA to evaluate rare event risk.  We then develop a linear weighting system to account for this rare event risk and constrain it in our real time dispatch model.  Computational experiments suggest that the JCC N-1 with OPA weighting model may have more desirable load-shed distribution.


\section{Future Work}
The \Cref{msip-model,dfo-chapter} can be improved by incorporating a more detailed simulation of cascading power failures including risk from voltage collapse, power transients, and other physical risk leading to large load shed events.  \Cref{jcc-chapter,ch:jccow} can be improved by incorporating covariance between effective capacities and demand uncertainty.  Geographically correlated weather has an effect on both transmission capacities as well as generation and demand. Additionally, the real time model should include contingencies of generator failures as well.  The model is dependent on the chosen line failure density model and it will be useful to better understand this function. Determining approximate values for the function parameters that represent real world conditions will be useful.

\bibliographystyle{plain}	
\bibliography{ref,msip-bib,jcc-bib,dfo-bib}		

\clearpage

\begin{appendices}
\chapter{Code for Computational Models}

\section{Multi Stage Stochastic Program}
This code is publicly available on GitHub at
\url{https://github.com/eanderson4/msip}

\section{HTCondor Parallelization Code}

\linespread{1}
\scriptsize
\lstinputlisting[language=Python,label={procpy},caption={proc.py: HTCondor Queue and Log File Reader}]{\mypathdfocode/Proc.py}

\lstinputlisting[language=bash,label={runit},caption={runit: Main Process Flow}]{\mypathdfocode/runit.sh}

\lstinputlisting[language=Python,label={powerpy},caption={power.py: Power Class}]{\mypathdfocode/Powerin/power.py}
\lstinputlisting[language=Python,label={toolspy},caption={tools.py: Common Functions}]{\mypathdfocode/Powerin/tools.py}
%\lstinputlisting[language=Python,label={cappy},caption={cap.py: Capacity file analysis functions}]{\mypathdfocode/Powerin/cap.py}
\lstinputlisting[language=Python,label={allocatepy},caption={allocate.py: Direct Search Pattern}]{\mypathdfocode/Powerin/allocate.py}
\lstinputlisting[language=Python,label={consubpy},caption={consub.py: Build HTCondor Submit Structure}]{\mypathdfocode/Powerin/consub.py}
\lstinputlisting[language=Python,label={loadshedpy},caption={loadshed.py: Take raw load shed and summarize}]{\mypathdfocode/Powerin/loadShed.py}
\lstinputlisting[language=Python,label={countlinespy},caption={countLines.py: Take raw line out info and summarize}]{\mypathdfocode/Powerin/countLines.py}

\lstinputlisting[language=bash,label={optimalpointlao},caption={point.lao: Line outage file from chosen design}]{\mypathdfocode/opt/point.lao}

\linespread{2}


\section{Joint Chance Constraint Model}
This code is publicly available on GitHub at
\url{https://github.com/eanderson4/pow-opt}

\section{Joint Chance Constraint with OPA Weighting}
This code is publicly available on GitHub at
\url{https://github.com/eanderson4/opt-opa}


\end{appendices}

%\clearpage

%\theendnotes
