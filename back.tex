
\chapter{Conclusion}
In this thesis we explored rare event risk on the bulk power system.  From the recent literature of cascading power failures, we built upon the OPA cascading power failure simulation.  We started by modeling this process as a multi-stage stochastic program with integer variables in \cref{msip-model}.  We introduced the concept of effective capacity to model the decision-dependent uncertain inherent in this cascading process.  This model allows for the flexibility of using the cascading process as a sub-problem in long term design problems.  We then turn to Monte Carlo simulation of this process in order to parallelize the computational process to get better resolution of the cascade evolution uncertainty in \cref{dfo-chapter}.  We look at the effects of transmission expansion on the OPA simulation and optimize this using derivative free optimization techniques and large computational resources through Condor.  In the second half of the thesis, we switch to real-time dispatch models.  In \cref{jcc-chapter} we develop a system risk measure that constrains the probability of one or more lines failing.  Also, we model uncertainty in generation and demand and translate this to uncertainty in branch flows by calculating the branch covariance matrix.  Using these two models, we approximate this system risk measure under uncertainty and use a cutting plane algorithm to solve this convex problem.  Finally, in \cref{ch:jccow} we combine the rare event risk model of the OPA simulation with the real-time line failure dispatch model.  We extend the joint chance constraint to the N-1 contingencies and use these contingencies to seed the OPA model.  Based off of these results, we develop a linear weighting system to approximate how important each line is with respect to the OPA cascade simulation.  We use this linear weighting system and a cutting plane algorithm to solve this convex optimization problem.
\section{Contributions}
In \cref{msip-model}, the primary contribution was to model the existing cascading power failure simulation as a multi-stage stochastic program using mixed integer variables.  The primary difficulty was overcoming the decision dependent uncertainty, which was done by using the concept of effective capacity distribution and a priori sampling.  While the computational difficulty made practical use of this impossible, if highly probable sequences of failures were known in advance, this model could be used to protect against them.

In \cref{dfo-chapter}, the primary contribution was to optimize over the OPA simulation efficiently.  We looked at the design problem of transmission expansion and explored the characteristics of this risk function.  We developed an efficient implementation was made with a common random number scheme to reduce variance and low system requirements to allow for massively parallel solution methodology.  In addition, DFO techniques were used to filter the high frequency noise and exploratory steps using accessory information from the OPA simulation enhanced the solution time.

In \cref{jcc-chapter}, the primary contribution was to develop a dispatch model that included a system risk constraint to control endogenous risk from line loading.  This system risk measure was modeled as a joint chance constraint we solve this exactly when there is no uncertainty in generation and demand.  When there is net injection uncertainty, we approximate this system risk measure and solve approximately using a cutting plane algorithm..  The cost-risk frontier was explored and the new model was compared to the standard chance constraint model used in recent literature.  

In \cref{ch:jccow}, we extended our endogenous system risk model to the N-1 contingencies.  We also evaluate the N-1 contingencies using OPA to evaluate rare event risk.  We then develop a linear weighting system to account for this rare event risk and constrain it in our real time dispatch model.

\bibliographystyle{plain}	
\bibliography{ref,msip-bib,jcc-bib,dfo-bib}		

\clearpage

\begin{appendices}
\chapter{Code for Computational Models}

\section{Multi Stage Stochastic Program}
This code is publicly available on GitHub at
\url{https://github.com/eanderson4/msip}

\section{Condor Parallelization Code}

\linespread{1}
\scriptsize
\lstinputlisting[language=Python,label={procpy},caption={proc.py: Condor Queue and Log File Reader}]{\mypathdfocode/Proc.py}

\lstinputlisting[language=bash,label={runit},caption={runit: Main Process Flow}]{\mypathdfocode/runit.sh}

\lstinputlisting[language=Python,label={powerpy},caption={power.py: Power Class}]{\mypathdfocode/Powerin/power.py}
\lstinputlisting[language=Python,label={toolspy},caption={tools.py: Common Functions}]{\mypathdfocode/Powerin/tools.py}
%\lstinputlisting[language=Python,label={cappy},caption={cap.py: Capacity file analysis functions}]{\mypathdfocode/Powerin/cap.py}
\lstinputlisting[language=Python,label={allocatepy},caption={allocate.py: Direct Search Pattern}]{\mypathdfocode/Powerin/allocate.py}
\lstinputlisting[language=Python,label={consubpy},caption={consub.py: Build Condor Submit Structure}]{\mypathdfocode/Powerin/consub.py}
\lstinputlisting[language=Python,label={loadshedpy},caption={loadshed.py: Take raw load shed and summarize}]{\mypathdfocode/Powerin/loadShed.py}
\lstinputlisting[language=Python,label={countlinespy},caption={countLines.py: Take raw line out info and summarize}]{\mypathdfocode/Powerin/countLines.py}

\lstinputlisting[language=bash,label={optimalpointlao},caption={point.lao: Line outage file from chosen design}]{\mypathdfocode/opt/point.lao}

\linespread{2}


\section{Joint Chance Constraint Model}
This code is publicly available on GitHub at
\url{https://github.com/eanderson4/pow-opt}

\section{Joint Chance Constraint with OPA Weighting}
This code is publicly available on GitHub at
\url{https://github.com/eanderson4/opt-opa}


\end{appendices}

%\clearpage

%\theendnotes
